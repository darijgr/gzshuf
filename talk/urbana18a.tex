\documentclass{beamer}
\usepackage{amsmath}
\usepackage{amssymb}
\usepackage{array}
\usepackage{setspace}
\usepackage{graphicx}
%\usepackage{tikz}
%\usetikzlibrary{matrix,arrows,backgrounds,shapes.misc,shapes.geometric,fit}
\usepackage{etex}
\usepackage{amsthm}
\usepackage{color}
\usepackage{wasysym}
\usepackage{allrunes}
\usepackage[all]{xy}
\usepackage{textpos}
\usepackage{ytableau}
\usepackage{stmaryrd}
%\usetikzlibrary{calc,through,backgrounds}
%\CompileMatrices

\definecolor{grau}{rgb}{.5 , .5 , .5}
\definecolor{dunkelgrau}{rgb}{.35 , .35 , .35}
\definecolor{schwarz}{rgb}{0 , 0 , 0}
\definecolor{violet}{RGB}{143,0,255}
\definecolor{forestgreen}{RGB}{34, 100, 34}

\newcommand{\red}{\color{red}}
\newcommand{\grey}{\color{grau}}
\newcommand{\green}{\color{forestgreen}}
\newcommand{\violet}{\color{violet}}
\newcommand{\blue}{\color{blue}}

\newcommand{\bIf}{\textbf{If} }
\newcommand{\bif}{\textbf{if} }
\newcommand{\bthen}{\textbf{then} }

\newcommand{\EE}{{\mathbf{E}}}
\newcommand{\ZZ}{{\mathbb Z}}
\newcommand{\NN}{{\mathbb N}}
\newcommand{\QQ}{{\mathbb Q}}
\newcommand{\kk}{{\mathbf k}}
\newcommand{\OO}{\operatorname {O}}
\newcommand{\Nm}{\operatorname {N}}
\newcommand{\Par}{\operatorname{Par}}
\newcommand{\Comp}{\operatorname{Comp}}
\newcommand{\Stab}{\operatorname {Stab}}
\newcommand{\id}{\operatorname{id}}
\newcommand{\ev}{\operatorname{ev}}
\newcommand{\Sym}{\operatorname{Sym}}
\newcommand{\Lpk}{\operatorname{Lpk}}
\newcommand{\lpk}{\operatorname{lpk}}
\newcommand{\Rpk}{\operatorname{Rpk}}
\newcommand{\rpk}{\operatorname{rpk}}
\newcommand{\Pk}{\operatorname{Pk}}
\newcommand{\Epk}{\operatorname{Epk}}
\newcommand{\epk}{\operatorname{epk}}
\newcommand{\Des}{\operatorname{Des}}
\newcommand{\des}{\operatorname{des}}
\newcommand{\inv}{\operatorname{inv}}
\newcommand{\maj}{\operatorname{maj}}
\newcommand{\Val}{\operatorname{Val}}
\newcommand{\pk}{\operatorname{pk}}
\newcommand{\st}{\operatorname{st}}
\newcommand{\QSym}{\operatorname{QSym}}
\newcommand{\NSym}{\operatorname{NSym}}
\newcommand{\Mat}{\operatorname{M}}
\newcommand{\PS}{\operatorname{PS}}
\newcommand{\bk}{\mathbf{k}}
\newcommand{\Nplus}{\mathbb{N}_{+}}
\newcommand\arxiv[1]{\href{http://www.arxiv.org/abs/#1}{\texttt{arXiv:#1}}}
\newcommand{\Orb}{{\mathcal O}}
\newcommand{\GL}{\operatorname {GL}}
\newcommand{\SL}{\operatorname {SL}}
\newcommand{\Or}{\operatorname {O}}
\newcommand{\im}{\operatorname {Im}}
\newcommand{\Iso}{\operatorname {Iso}}
\newcommand{\Adm}{\operatorname{Adm}}
\newcommand{\Supp}{\operatorname{Supp}}
\newcommand{\Powser}{\QQ\left[\left[x_1,x_2,x_3,\ldots\right]\right]}
\newcommand{\rad}{\operatorname {rad}}
\newcommand{\zero}{\mathbf{0}}
\newcommand{\xx}{\mathbf{x}}
\newcommand{\ord}{\operatorname*{ord}}
\newcommand{\bbK}{{\mathbb{K}}}
\newcommand{\whP}{{\widehat{P}}}
%\newcommand{\Trop}{\operatorname*{Trop}}
%\newcommand{\TropZ}{{\operatorname*{Trop}\mathbb{Z}}}
\newcommand{\rato}{\dashrightarrow}
\newcommand{\lcm}{\operatorname*{lcm}}
\newcommand{\tlab}{\operatorname*{tlab}}
\newcommand{\tvi}{\left. \textarm{\tvimadur} \right.}
\newcommand{\bel}{\left. \textarm{\belgthor} \right.}
\newcommand{\calK}{\mathcal{K}}
\newcommand{\calE}{\mathcal{E}}
\newcommand{\calL}{\mathcal{L}}
\newcommand{\calN}{\mathcal{N}}
\newcommand{\calZ}{\mathcal{Z}}

\newcommand{\fti}[1]{\frametitle{\ \ \ \ \ #1}}
\newenvironment{iframe}[1][]{\begin{frame} \fti{[#1]} \begin{itemize}}{\end{frame}\end{itemize}}

\newcommand{\are}{\ar@{-}}
\newcommand{\arinj}{\ar@{_{(}->}}
\newcommand{\arsurj}{\ar@{->>}}

\newcommand{\set}[1]{\left\{ #1 \right\}}
% $\set{...}$ yields $\left\{ ... \right\}$.
\newcommand{\abs}[1]{\left| #1 \right|}
% $\abs{...}$ yields $\left| ... \right|$.
\newcommand{\tup}[1]{\left( #1 \right)}
% $\tup{...}$ yields $\left( ... \right)$.
\newcommand{\ive}[1]{\left[ #1 \right]}
% $\ive{...}$ yields $\left[ ... \right]$.
\newcommand{\verts}[1]{\operatorname{V}\left( #1 \right)}
% $\verts{...}$ yields $\operatorname{V}\left( ... \right)$.
\newcommand{\edges}[1]{\operatorname{E}\left( #1 \right)}
% $\edges{...}$ yields $\operatorname{E}\left( ... \right)$.
\newcommand{\arcs}[1]{\operatorname{A}\left( #1 \right)}
% $\arcs{...}$ yields $\operatorname{A}\left( ... \right)$.
\newcommand{\underbrack}[2]{\underbrace{#1}_{\substack{#2}}}
% $\underbrack{...1}{...2}$ yields
% $\underbrace{...1}_{\substack{...2}}$. This is useful for doing
% local rewriting transformations on mathematical expressions with
% justifications.

\setbeamertemplate{itemize/enumerate body begin}{\large}
\setbeamertemplate{itemize/enumerate subbody begin}{\large}
\setbeamertemplate{itemize/enumerate subsubbody begin}{\large}

\usepackage{url}%this line and the next are related to hyperlinks
%\usepackage[colorlinks=true, pdfstartview=FitV, linkcolor=blue, citecolor=blue, urlcolor=blur]{hyperref}

\usepackage{color}

%\usetheme{Antibes}
%\usetheme{Bergen}
%\usetheme{Berkeley}
%\usetheme{Berlin}
%\usetheme{Boadilla}
%\usetheme{Copenhagen}
%\usetheme{Darmstadt}
%\usetheme{Dresden}
\usetheme{Frankfurt}
%\usetheme{Goettingen}
%\usetheme{Hannover}
%\usetheme{Ilmenau}
%\usetheme{JuanLesPins}
%\usetheme{Luebeck}
%\usetheme{Madrid}
%\usetheme{Malmoe}
%\usetheme{Marburg}
%\usetheme{Montpellier}
%\usetheme{PaloAlto}
%\usetheme{Pittsburgh}
%\usetheme{Rochester}
%\usetheme{Singapore}
%\usetheme{Szeged}
%\usetheme{Warsaw}

\usefonttheme[onlylarge]{structurebold}
\setbeamerfont*{frametitle}{size=\normalsize,series=\bfseries}
\setbeamertemplate{navigation symbols}{}
\setbeamertemplate{footline}[frame number]
\setbeamertemplate{itemize/enumerate body begin}{}
\setbeamertemplate{itemize/enumerate subbody begin}{\normalsize}
%\setbeamertemplate{section in head/foot shaded}[default][60]
%\setbeamertemplate{subsection in head/foot shaded}[default][60]
\beamersetuncovermixins{\opaqueness<1>{0}}{\opaqueness<2->{15}}

%\usepackage{beamerthemesplit}
\usepackage{epsfig,amsfonts,bbm,mathrsfs}
\usepackage{verbatim} 

% Dark red emphasis
\definecolor{darkred}{rgb}{0.7,0,0} % darkred color
\newcommand{\defn}[1]{{\color{darkred}\emph{#1}}} % emphasis of a definition



\newcommand{\STRUT}{\vrule width 0pt depth 8pt height 0pt}
\newcommand{\ASTRUT}{\vrule width 0pt depth 0pt height 11pt}


\theoremstyle{plain}
\newtheorem{conj}[theorem]{Conjecture}


\setbeamertemplate{headline}{}
%This removes a black stripe from the top of the slides.


\author{Darij Grinberg (UMN)
}
\title[Shuffle-compatibility of the descent set]{Shuffle-compatibility of the descent set}
% Pre-talk, expository.
\date{8 March 2018 \\ University of Illinois at Urbana-Champaign}

\begin{document}

\frame{\titlepage
\textbf{slides: \red \url{http://www.cip.ifi.lmu.de/~grinberg/algebra/urbana18a.pdf}} \\
\textbf{paper: \red \url{http://www.cip.ifi.lmu.de/~grinberg/algebra/gzshuf2.pdf}} \\
\textbf{project: \red \url{https://github.com/darijgr/gzshuf}}
}

\begin{frame}
\fti{Introduction}

\begin{itemize}

\item This is an \textbf{expository} talk on a little part of the paper:
        \begin{itemize}
        \item \href{https://arxiv.org/abs/1706.00750}{\red Ira M. Gessel, Yan Zhuang, \textit{Shuffle-compatible permutation statistics}, arXiv:1706.00750}.
        \end{itemize}
      Nothing here is my invention. \pause
      For my own work, see the {\red \href{http://www.cip.ifi.lmu.de/~grinberg/algebra/urbana18b.pdf}{next talk}}.

\pause

\item I will sketch the proofs of Theorem 2.8 and of
      Theorem 6.1 from their paper.

\item Unlike that paper, I will avoid any extraneous notation
      and theory here.

\end{itemize}

\end{frame}

\begin{frame}
\fti{Permutations and descents}

\begin{itemize}

\item Let $\NN = \set{0, 1, 2, \ldots}$.

\item For $n \in \NN$, an \defn{$n$-permutation} means a tuple of $n$
distinct positive integers. \\
Example: $\tup{3, 1, 7}$ is a $3$-permutation, but
         $\tup{2, 1, 2}$ is not.
      \\
      (\textbf{Caveat lector:} Not the usual meaning of ``permutation''.) \\

\pause

\item If $\pi$ is an $n$-permutation and $i \in \set{1, 2, \ldots, n}$,
      then \defn{$\pi_i$} denotes the $i$-th entry of $\pi$.

\pause

\item If $\pi$ is an $n$-permutation, then a \defn{descent} of $\pi$
      means an $i \in \set{1, 2, \ldots, n-1}$ such that
      $\pi_i > \pi_{i+1}$.

\item The \defn{descent set $\Des \pi$} of an $n$-permutation $\pi$ is
      the set of all descents of $\pi$. \\
      \textbf{Example:} $\Des \tup{3, 1, 5, 2, 4} = \set{1, 3}$.

\end{itemize}
\end{frame}

\begin{frame}
\fti{Shuffles of permutations}

\begin{itemize}

\item Let $m \in \NN$, and let $\pi$ be an $m$-permutation. \\
      Let $n \in \NN$, and let $\sigma$ be an $n$-permutation.

\item We say that $\pi$ and $\sigma$ are \defn{disjoint} if they have
      no letter in common.

\pause

\item Assume that $\pi$ and $\sigma$ are disjoint.
      An $\tup{m+n}$-permutation $\tau$ is called a \defn{shuffle} of
      $\pi$ and $\sigma$ if both $\pi$ and $\sigma$ appear as
      subsequences of $\tau$. \\
      (And thus, no other letters can appear in $\tau$.)

\item \textbf{Example:}
      The shuffles of ${\red \tup{4, 1}}$ and ${\blue \tup{2, 5}}$
      are
      \begin{align*}
      &     \tup{ {\red 4}, {\red 1}, {\blue 2}, {\blue 5} } ,
            \tup{ {\red 4}, {\blue 2}, {\red 1}, {\blue 5} } ,
            \tup{ {\red 4}, {\blue 2}, {\blue 5}, {\red 1} } , \\
      &
            \tup{ {\blue 2}, {\red 4}, {\red 1}, {\blue 5} } ,
            \tup{ {\blue 2}, {\red 4}, {\blue 5}, {\red 1} } ,
            \tup{ {\blue 2}, {\blue 5}, {\red 4}, {\red 1} } .
      \end{align*}

\pause

\item Observe that $\pi$ and $\sigma$ have $\dbinom{m+n}{m}$
      shuffles, in bijection with $m$-element subsets of
      $\set{1, 2, \ldots, m+n}$.

\end{itemize}
\end{frame}

\begin{frame}
\fti{Weak compositions}

\begin{itemize}

\item The set $\NN^k$ of $k$-tuples is an additive monoid. \\
      (Keep in mind: $0 \in \NN$.)

\item If $\tup{a_1, a_2, \ldots, a_k} \in \NN^k$, then
      \defn{$\abs{\alpha}$} is defined to be
      $a_1 + a_2 + \cdots + a_k$.

\pause

\item For any $\tup{a_1, a_2, \ldots, a_k} \in \NN^k$, we
      define a set
      \defn{$\PS \tup{a_1, a_2, \ldots, a_k}$} to be
      \begin{align*}
      & \set{a_1 + a_2 + \cdots + a_i \ \mid  1 \leq i \leq k-1 } \\
      & = \set{a_1, a_1 + a_2, \ldots, a_1 + a_2 + \cdots + a_{k-1}} .
      \end{align*}
      ($\PS$ stands for ``partial sums''.)
      \pause \\
      (\textbf{Note:}
      $\PS \tup{a_1, a_2, \ldots, a_k}
      \subseteq \set{0, 1, \ldots, \abs{\alpha}}$.)

\pause

\item Let $n \in \NN$.
      A \defn{weak composition of $n$} means an
      $\alpha \in \NN^k$ satisfying $\abs{\alpha} = n$.

\end{itemize}

\end{frame}

\begin{frame}
\fti{Shuffle-compatibility of $\Des$: statement}

\begin{itemize}

\item Let $m \in \NN$, and let $\pi$ be an $m$-permutation. \\
      Let $n \in \NN$, and let $\sigma$ be an $n$-permutation. \\
      Assume that $\pi$ and $\sigma$ are disjoint.

\pause

\item Let $A$ be a subset of $\ive{m+n-1}$. \\
      Here, \defn{$\ive{k}$} means $\set{1, 2, \ldots, k}$ for
      each $k \in \NN$.

\pause

\only<3-4>{
\item How many shuffles $\tau$ of $\pi$ and $\sigma$ satisfy
      $\Des \tau \subseteq A$ ?

\pause

\item The following theorem by Gessel and Zhuang gives the
      answer.
}

\pause

\item Let $L$ be a weak composition of $m+n$ such that
      $\PS\tup{L} = A$.
      %\only<5>{
      \\ (Such $L$ can easily be constructed.)
      %}
      \\ Let $k$ be such that $L \in \NN^k$.

\pause

\item \textbf{Theorem (Gessel \& Zhuang, {\red \arxiv{1706.00750}}, Theorem 2.8).} \\
      The number of shuffles $\tau$ of $\pi$ and $\sigma$ satisfying
      $\Des \tau \subseteq A$ \\
      \textbf{equals} the number of pairs $\tup{J, K} \in \NN^k \times \NN^k$ such that
      \begin{itemize}
      \item $J$ is a weak composition of $m$ satisfying $\Des \pi \subseteq \PS \tup{J}$;
      \item $K$ is a weak composition of $n$ satisfying $\Des \sigma \subseteq \PS \tup{K}$;
      \item we have $J + K = L$ (in the monoid $\NN^k$).
      \end{itemize}

\end{itemize}

\vspace{10cm}

\end{frame}

\begin{frame}
\fti{Shuffle-compatibility of $\Des$: example 1}

\begin{itemize}

\item \textbf{Example:}
      Let ${\red m = 2}$ and ${\red \pi = \tup{4, 1}}$. \\
      Let ${\blue n = 2}$ and ${\blue \sigma = \tup{2, 5}}$. \\
      \only<1-4>{
      The shuffles $\tau$ of $\pi$ and $\sigma$ are
      \begin{align*}
      &     \tup{ {\red 4}, {\red 1}, {\blue 2}, {\blue 5} } ,
            \tup{ {\red 4}, {\blue 2}, {\red 1}, {\blue 5} } ,
            \tup{ {\red 4}, {\blue 2}, {\blue 5}, {\red 1} } , \\
      &
            \tup{ {\blue 2}, {\red 4}, {\red 1}, {\blue 5} } ,
            \tup{ {\blue 2}, {\red 4}, {\blue 5}, {\red 1} } ,
            \tup{ {\blue 2}, {\blue 5}, {\red 4}, {\red 1} } .
      \end{align*}
      \pause
      Their descent sets $\Des \tau$ are
      \begin{align*}
      &     \set{1} , \qquad \set{1, 2}, \qquad \set{1, 3}, \\
      &     \set{2}, \qquad  \set{2, 3}, \qquad \set{3} .
      \end{align*}
      \pause
      Pick $A = \set{3}$.
      Then, the number of shuffles $\tau$ of $\pi$ and $\sigma$ satisfying
      $\Des \tau \subseteq A$ is $1$. \\
      What about the other number?
      \pause
      We must pick a weak composition $L$ of $m+n = 4$ such that
      $\PS\tup{L} = A = \set{3}$.
      \\
      We can take $L = \tup{3, 1}$
      (or $L = \tup{3, 0, 0, \ldots, 0, 1}$ for any number of $0$'s).
      Let's pick $L = \tup{3, 1}$.
      }
      \only<5->{
      So we have $A = \set{3}$ and $L = \tup{3, 1}$.
      \\
      We want to find the number of pairs $\tup{J, K}$ such that
      \begin{itemize}
      \item $J$ is a weak composition of $m$ satisfying $\Des \pi \subseteq \PS \tup{J}$;
      \item $K$ is a weak composition of $n$ satisfying $\Des \sigma \subseteq \PS \tup{K}$;
      % \item $J$ is a weak composition of $m$ satisfying \only<5>{$\Des \pi \subseteq \PS \tup{J}$} \only<6->{$\set{1} \subseteq \PS \tup{J}$};
      % \item $K$ is a weak composition of $n$ satisfying \only<5>{$\Des \sigma \subseteq \PS \tup{K}$} \only<6->{$\set{} \subseteq \PS \tup{K}$};
      \item we have $J + K = L$ (in the monoid $\NN^k$).
      \end{itemize}
      \pause
      Let's solve this:
      \only<6>{
        \[
        \text{%
        \begin{tabular}
        [c]{r|ll|l}
        &  &  & requirements\\\hline
        $J$ & $?$ & $?$ & $\left\vert J\right\vert =m,\ \operatorname*{PS}%
        J\supseteq\operatorname*{Des}\pi$\\
        $+\ K$ & $?$ & $?$ & $\left\vert K\right\vert =n,\ \operatorname*{PS}%
        K\supseteq\operatorname*{Des}\sigma$\\\hline
        $=L$ & $3$ & $1$ &
        \end{tabular}
        }%
        \]
      }
      \only<7>{
        \[
        \text{%
        \begin{tabular}
        [c]{r|ll|l}
        &  &  & requirements\\\hline
        $J$ & $?$ & $?$ & $\left\vert J\right\vert =2,\ \ \operatorname*{PS}%
        J\supseteq\set{1}\ \ \ $\\
        $+\ K$ & $?$ & $?$ & $\left\vert K\right\vert =2,\ \ \operatorname*{PS}%
        K\supseteq\set{}\ \ \ $\\\hline
        $=L$ & $3$ & $1$ &
        \end{tabular}
        }%
        \]
      }
      \only<8>{
        \[
        \text{%
        \begin{tabular}
        [c]{r|ll|l}
        &  &  & requirements\\\hline
        $J$ & $1$ & $1$ & $\left\vert J\right\vert =2,\ \ \operatorname*{PS}%
        J\supseteq\set{1}\ \ \ $\\
        $+\ K$ & $?$ & $?$ & $\left\vert K\right\vert =2,\ \ \operatorname*{PS}%
        K\supseteq\set{}\ \ \ $\\\hline
        $=L$ & $3$ & $1$ &
        \end{tabular}
        }%
        \]
      }
      \only<9->{
        \[
        \text{%
        \begin{tabular}
        [c]{r|ll|l}
        &  &  & requirements\\\hline
        $J$ & $1$ & $1$ & $\left\vert J\right\vert =2,\ \ \operatorname*{PS}%
        J\supseteq\set{1}\ \ \ $\\
        $+\ K$ & $2$ & $0$ & $\left\vert K\right\vert =2,\ \ \operatorname*{PS}%
        K\supseteq\set{}\ \ \ $\\\hline
        $=L$ & $3$ & $1$ &
        \end{tabular}
        }%
        \]
      }
      \only<10>{
      Thus, there is exactly $1$ solution, as the Theorem predicts.
      }
      }

\end{itemize}
\vspace{10cm}
\end{frame}

\begin{frame}
\fti{Shuffle-compatibility of $\Des$: example 2}

\begin{itemize}

\item \textbf{Example:}
      Let ${\red m = 2}$ and ${\red \pi = \tup{4, 1}}$. \\
      Let ${\blue n = 2}$ and ${\blue \sigma = \tup{2, 5}}$. \\
      \only<1-4>{
      The shuffles $\tau$ of $\pi$ and $\sigma$ are
      \begin{align*}
      &     \tup{ {\red 4}, {\red 1}, {\blue 2}, {\blue 5} } ,
            \tup{ {\red 4}, {\blue 2}, {\red 1}, {\blue 5} } ,
            \tup{ {\red 4}, {\blue 2}, {\blue 5}, {\red 1} } , \\
      &
            \tup{ {\blue 2}, {\red 4}, {\red 1}, {\blue 5} } ,
            \tup{ {\blue 2}, {\red 4}, {\blue 5}, {\red 1} } ,
            \tup{ {\blue 2}, {\blue 5}, {\red 4}, {\red 1} } .
      \end{align*}
      \pause
      Their descent sets $\Des \tau$ are
      \begin{align*}
      &     \set{1} , \qquad \set{1, 2}, \qquad \set{1, 3}, \\
      &     \set{2}, \qquad  \set{2, 3}, \qquad \set{3} .
      \end{align*}
      \pause
      Pick $A = \set{2, 3}$.
      Then, the number of shuffles $\tau$ of $\pi$ and $\sigma$ satisfying
      $\Des \tau \subseteq A$ is $3$. \\
      What about the other number?
      \pause
      We must pick a weak composition $L$ of $m+n = 4$ such that
      $\PS\tup{L} = A = \set{2, 3}$.
      \\
      We can take $L = \tup{2, 1, 1}$
      (or $L = \tup{2, 0, 0, \ldots, 0, 1, 0, 0, \ldots, 0, 1}$ for any number of $0$'s).
      Let's pick $L = \tup{2, 1, 1}$.
      }
      \only<5->{
      So we have $A = \set{2, 3}$ and $L = \tup{2, 1, 1}$.
      \\
      We want to find the number of pairs $\tup{J, K}$ such that
      \begin{itemize}
      \item $J$ is a weak composition of $m$ satisfying $\Des \pi \subseteq \PS \tup{J}$;
      \item $K$ is a weak composition of $n$ satisfying $\Des \sigma \subseteq \PS \tup{K}$;
      \item we have $J + K = L$ (in the monoid $\NN^k$).
      \end{itemize}
      \pause
      Let's solve this:
      \only<6>{
        \[
        \text{%
        \begin{tabular}
        [c]{r|lll|l}
        &  &  &  & requirements\\\hline
        $J$ & $?$ & $?$ & $?$ & $\left\vert J\right\vert =m,\ \operatorname*{PS}%
        J\supseteq\operatorname*{Des}\pi$\\
        $+\ K$ & $?$ & $?$ & $?$ & $\left\vert K\right\vert =n,\ \operatorname*{PS}%
        K\supseteq\operatorname*{Des}\sigma$\\\hline
        $=L$ & $2$ & $1$ & $1$ &
        \end{tabular}
        }%
        \]
      }
      \only<7>{
        \[
        \text{%
        \begin{tabular}
        [c]{r|lll|l}
        &  &  &  & requirements\\\hline
        $J$ & $?$ & $?$ & $?$ & $\left\vert J\right\vert =2,\ \ \operatorname*{PS}%
        J\supseteq\set{1}\ \ \ $\\
        $+\ K$ & $?$ & $?$ & $?$ & $\left\vert K\right\vert =2,\ \ \operatorname*{PS}%
        K\supseteq\set{}\ \ \ $\\\hline
        $=L$ & $2$ & $1$ & $1$ &
        \end{tabular}
        }%
        \]
      }
      \only<8>{
        \[
        \text{%
        \begin{tabular}
        [c]{r|lll|l}
        &  &  &  & requirements\\\hline
        $J$ & $1$ & $1$ & $0$ & $\left\vert J\right\vert =2,\ \ \operatorname*{PS}%
        J\supseteq\set{1}\ \ \ $\\
        $+\ K$ & $1$ & $0$ & $1$ & $\left\vert K\right\vert =2,\ \ \operatorname*{PS}%
        K\supseteq\set{}\ \ \ $\\\hline
        $=L$ & $2$ & $1$ & $1$ &
        \end{tabular}
        }%
        \]
      }
      \only<9>{
        \[
        \text{%
        \begin{tabular}
        [c]{r|lll|l}
        &  &  &  & requirements\\\hline
        $J$ & $1$ & $0$ & $1$ & $\left\vert J\right\vert =2,\ \ \operatorname*{PS}%
        J\supseteq\set{1}\ \ \ $\\
        $+\ K$ & $1$ & $1$ & $0$ & $\left\vert K\right\vert =2,\ \ \operatorname*{PS}%
        K\supseteq\set{}\ \ \ $\\\hline
        $=L$ & $2$ & $1$ & $1$ &
        \end{tabular}
        }%
        \]
      }
      \only<10->{
        \[
        \text{%
        \begin{tabular}
        [c]{r|lll|l}
        &  &  &  & requirements\\\hline
        $J$ & $0$ & $1$ & $1$ & $\left\vert J\right\vert =2,\ \ \operatorname*{PS}%
        J\supseteq\set{1}\ \ \ $\\
        $+\ K$ & $2$ & $0$ & $0$ & $\left\vert K\right\vert =2,\ \ \operatorname*{PS}%
        K\supseteq\set{}\ \ \ $\\\hline
        $=L$ & $2$ & $1$ & $1$ &
        \end{tabular}
        }%
        \]
      }
      \only<11>{
      Thus, there are $3$ solutions, as the Theorem predicts.
      }
      }

\end{itemize}
\vspace{10cm}
\end{frame}

\begin{frame}
\fti{Shuffle-compatibility of $\Des$: consequence}

\begin{itemize}

\item Let $m \in \NN$, and let $\pi$ be an $m$-permutation. \\
      Let $n \in \NN$, and let $\sigma$ be an $n$-permutation. \\
      Assume that $\pi$ and $\sigma$ are disjoint.

\item Let $A$ be a subset of $\ive{m+n-1}$. \\

\item Let $L$ be a weak composition of $m+n$ such that
      $\PS\tup{L} = A$.
      \\ Let $k$ be such that $L \in \NN^k$.

\only<1>{
\item \textbf{Theorem (Gessel \& Zhuang, from previous slide).} \\
      The number of shuffles $\tau$ of $\pi$ and $\sigma$ satisfying
      $\Des \tau \subseteq A$ \\
      \textbf{equals} the number of pairs $\tup{J, K} \in \NN^k \times \NN^k$ such that
      \begin{itemize}
      \item $J$ is a weak composition of $m$ satisfying $\Des \pi \subseteq \PS \tup{J}$;
      \item $K$ is a weak composition of $n$ satisfying $\Des \sigma \subseteq \PS \tup{K}$;
      \item we have $J + K = L$ (in the monoid $\NN^k$).
      \end{itemize}
}
\pause

\item \textbf{Corollary.} \\
      The number of shuffles $\tau$ of $\pi$ and $\sigma$ satisfying
      $\Des \tau \subseteq A$
      depends only on $m$, $n$, $\Des \pi$, $\Des \sigma$
      and $A$ (but not on $\pi$ and $\sigma$ themselves).
      \pause

\item \textbf{Corollary.} \\
      The number of shuffles $\tau$ of $\pi$ and $\sigma$ satisfying
      $\Des \tau = A$
      depends only on $m$, $n$, $\Des \pi$, $\Des \sigma$
      and $A$ (but not on $\pi$ and $\sigma$ themselves).
      \\
      (Follows from previous corollary by induction on
      $\abs{A}$.)
      \pause
      \\ Gessel and Zhuang say that
      this makes $\Des$ \defn{shuffle-compatible}.
      See the {\red \href{http://www.cip.ifi.lmu.de/~grinberg/algebra/urbana18b.pdf}{next talk}}
      for more about this.
      
\end{itemize}

\vspace{10cm}

\end{frame}

\begin{frame}
\fti{Shuffle-compatibility of $\Des$: proof, 1}

\begin{itemize}

\item Let $m \in \NN$, and let $\pi$ be an $m$-permutation. \\
      Let $n \in \NN$, and let $\sigma$ be an $n$-permutation. \\
      Assume that $\pi$ and $\sigma$ are disjoint.

\item Let $A$ be a subset of $\ive{m+n-1}$. \\

\item Let $L$ be a weak composition of $m+n$ such that
      $\PS\tup{L} = A$.
      \\ Let $k$ be such that $L \in \NN^k$.

\only<1>{
\item To prove the Theorem, let us restate it using shorthands:
}
\pause

\item A \defn{good shuffle} shall mean a
      shuffles $\tau$ of $\pi$ and $\sigma$ satisfying
      $\Des \tau \subseteq A$.

\item A \defn{good pair} shall mean a pair
      $\tup{J, K} \in \NN^k \times \NN^k$ such that
      \begin{itemize}
      \item $J$ is a weak composition of $m$ satisfying $\Des \pi \subseteq \PS \tup{J}$;
      \item $K$ is a weak composition of $n$ satisfying $\Des \sigma \subseteq \PS \tup{K}$;
      \item we have $J + K = L$ (in the monoid $\NN^k$).
      \end{itemize}

\item \textbf{Theorem (Gessel \& Zhuang, from previous slide).} \\
      The number of good shuffles equals the number of good pairs.

\pause

\item For a proof, we need bijections
      \begin{align*}
               \set{\text{good shuffles}}
               &\rightleftarrows
               \set{\text{good pairs}} .
      \end{align*}

\end{itemize}

\vspace{10cm}

\end{frame}

\begin{frame}
\fti{Shuffle-compatibility of $\Des$: proof, 2: $\leftarrow$}

\begin{itemize}

\item We construct the map
      $\set{\text{good pairs}} \to \set{\text{good shuffles}}$:

\item Let $\tup{J, K}$ be a good pair. Thus,
      $\tup{J, K} \in \NN^k \times \NN^k$ and
      \begin{itemize}
      \item $J$ is a weak composition of $m$ satisfying $\Des \pi \subseteq \PS \tup{J}$;
      \item $K$ is a weak composition of $n$ satisfying $\Des \sigma \subseteq \PS \tup{K}$;
      \item we have $J + K = L$ (in the monoid $\NN^k$).
      \end{itemize}

\pause

\item Write $J$ as $J = \tup{j_1, j_2, \ldots, j_k}$, \\
      and $K$ as $K = \tup{k_1, k_2, \ldots, k_k}$ (sorry).

\pause

\item For each $p \in \ive{k-1}$, insert a bar (``$\mid$'') between
      the $\tup{j_1+j_2+\cdots+j_p}$-th letter of $\pi$
      and the next one.
        \only<3>{\\
        \textbf{Example:}
        If $m = 8$ and $J = \tup{3, 2, 0, 2, 1, 0}$, then we
        get $\pi_1 \pi_2 \pi_3 \mid \pi_4 \pi_5 \mid \mid \pi_6 \pi_7 \mid \pi_8 \mid$.
        }

\pause

\item These bars subdivide $\pi$ into $k$ blocks (some empty),
      each increasing (since $\Des \pi \subseteq \PS \tup{J}$).

\pause

\item Similarly, subdivide $\sigma$ into $k$ increasing blocks
      using $K$.

\pause

\item Now, for each $i \in \ive{k}$, let
      \begin{itemize}
      \item $\pi^{(i)}$ be the $i$-th block of $\pi$;
      \item $\sigma^{(i)}$ be the $i$-th block of $\sigma$;
      \item $\tau^{(i)}$ be the unique increasing shuffle of
            $\pi^{(i)}$ and $\sigma^{(i)}$.
      \end{itemize}
      \pause
      Then, the concatenation $\pi^{(1)} \pi^{(2)} \cdots \pi^{(k)}$
      is a good shuffle.
      \pause \\
      So we have found a map
      $\set{\text{good pairs}} \to \set{\text{good shuffles}}$.

\end{itemize}

\vspace{10cm}
\end{frame}

\begin{frame}
\fti{Shuffle-compatibility of $\Des$: proof, 2: $\rightarrow$}

\begin{itemize}

\item We now construct the map
      $\set{\text{good shuffles}} \to \set{\text{good pairs}}$:

\item Let $\tau$ be a good shuffle. Thus,
      $\tau$ is a shuffle of $\pi$ and $\sigma$ satisfying
      $\Des \tau \subseteq A$.

\pause

\item Write $L$ as $L = \tup{l_1, l_2, \ldots, l_k}$.

\pause

\item For each $p \in \ive{k-1}$, insert a bar (``$\mid$'') between
      the $\tup{l_1+l_2+\cdots+l_p}$-th letter of $\tau$
      and the next one.
        \only<3>{\\
        (The positions of these bars are the elements of $A$,
        though they might have multiplicities.)
        }

\pause

\item These bars subdivide $\tau$ into $k$ blocks (some empty),
      each increasing (since $\Des \tau \subseteq A = \PS \tup{L}$).

\pause

\item Let $J = \tup{j_1, j_2, \ldots, j_k}$, where $j_p$ is
      the number of letters in the $p$-th block of $\tau$
      that come from $\pi$.

\pause

\item Similarly define $K$.

\pause

\item Then, $\tup{J, K}$ is a good pair.
      \pause \\
      So we have found a map
      $\set{\text{good shuffles}} \to \set{\text{good pairs}}$.

\pause

\item The two maps constructed are mutually inverse bijections
      \begin{align*}
               \set{\text{good shuffles}}
               &\rightleftarrows
               \set{\text{good pairs}} ;
      \end{align*}
      so the theorem is proven.

\end{itemize}

\vspace{10cm}
\end{frame}

\begin{frame}
\fti{The hollowed-out descent sets $\Des_{i,j} \pi$}

\begin{itemize}

\item Fix $i \in \NN$ and $j \in \NN$. \\
      For any $n$ and any $n$-permutation $\pi$, we define
      the \defn{hollowed-out descent set $\Des_{i,j} \pi$} by
      \[
      \Des_{i,j} \pi
      = \tup{\Des \pi}
      \cap
      \tup{ \set{1, 2, \ldots, i} \cup \set{n-1, n-2, \ldots, n-j} } .
      \]
      \pause
      Thus, $\Des_{i,j} \pi$ is the set of all descents of
      $\pi$ that are among the $i$ first or $j$ last possible
      positions for a descent to be in.

\end{itemize}
\end{frame}

\begin{frame}
\fti{Shuffle-compatibility of $\Des_{i,j}$: statement}

\begin{itemize}

\item Let $m \in \NN$, and let $\pi$ be an $m$-permutation. \\
      Let $n \in \NN$, and let $\sigma$ be an $n$-permutation. \\
      Assume that $\pi$ and $\sigma$ are disjoint.

\item Let $B$ be a subset of $\set{1, 2, \ldots, i} \cup \set{m+n-1, m+n-2, \ldots, m+n-j}$. \\

\item Let $A = B \cup \set{i+1, i+2, \ldots, m+n-j-1}$.

\item Let $L$ be a weak composition of $m+n$ such that
      $\PS\tup{L} = A$.
      \\ Let $k$ be such that $L \in \NN^k$.

\item \textbf{Theorem (Gessel \& Zhuang, {\red \arxiv{1706.00750}}, Theorem 6.1).} \\
      The number of shuffles $\tau$ of $\pi$ and $\sigma$ satisfying
      $\Des_{i, j} \tau \subseteq B$ \\
      \textbf{equals} the number of pairs $\tup{J, K} \in \NN^k \times \NN^k$ such that
      \begin{itemize}
      \item $J$ is a weak composition of $m$ satisfying $\Des_{i, j} \pi \subseteq \PS \tup{J}$;
      \item $K$ is a weak composition of $n$ satisfying $\Des_{i, j} \sigma \subseteq \PS \tup{K}$;
      \item we have $J + K = L$ (in the monoid $\NN^k$).
      \end{itemize}

\end{itemize}

\vspace{10cm}

\end{frame}

\begin{frame}
\fti{Shuffle-compatibility of $\Des_{i,j}$: proof}

\begin{itemize}

\item We can derive this Theorem from the previous Theorem. \\
      This relies on the following three observations:
      \begin{itemize}
      \item We have $\Des_{i, j} \tau \subseteq B$ if and only if
            $\Des \tau \subseteq A$.
      \item For any weak composition $J$ of $m$ satisfying
            $J \leq L$ (that is, each entry of $J$ is $\leq$
            to the corresponding entry of $L$),
            we have
            $\Des_{i, j} \pi \subseteq \PS \tup{J}$
            if and only if
            $\Des \pi \subseteq \PS \tup{J}$.
      \item A similar statement about weak compositions $K$ of $n$.
      \end{itemize}
      
\pause
\only<2>{
\item The first observation is obvious.
}
\pause
\item Proof of the second observation: \\
      Since $\PS \tup{L} = A \supseteq \set{i+1, i+2, \ldots, m+n-j-1}$,
      the composition $L$ has the form
      \begin{align*}
      L &= \big( \tup{\text{some numbers with sum } \leq i+1} , \\
                &\qquad \ \tup{\text{a sequence of }0\text{'s and }1\text{'s}}, \\
                &\qquad \ \tup{\text{some numbers with sum }\leq j+1 } \big) .
      \end{align*}
      Since $J \leq L$, it follows that $J$ also has this form.
      In other words,
      $\PS \tup{J} \supseteq \set{i+1, i+2, \ldots, m-j-1}$.
      Hence, the second observation follows.

\end{itemize}

\vspace{10cm}

\end{frame}

% \begin{frame}
% \fti{}

% \begin{itemize}

% \item 
      

% \item 
      

% \item 
      

% \item 
      

% \item 
      

% \end{itemize}

% \end{frame}


\begin{frame}
\fti{Thanks}

\textbf{Thanks} to Ira Gessel and Yan Zhuang for initiating this direction
(and for helpful discussions), and to
Alex Yong for an invitation to UIUC. \\
And thanks to you for attending!

\vspace{3cm}

\textbf{slides: \red \url{http://www.cip.ifi.lmu.de/~grinberg/algebra/urbana18a.pdf}} \\
\textbf{paper: \red \url{http://www.cip.ifi.lmu.de/~grinberg/algebra/gzshuf2.pdf}} \\
\textbf{project: \red \url{https://github.com/darijgr/gzshuf}}

\end{frame}

\end{document}
