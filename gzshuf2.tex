% -------------------------------------------------------------
% NOTE ON THE DETAILED AND SHORT VERSIONS:
% -------------------------------------------------------------
% This paper comes in two versions, a detailed and a short one.
% The short version should be more than sufficient for any
% reasonable use; the detailed one was written purely to
% convince the author of its correctness.
% To switch between the two versions, find the line containing
% "\newenvironment{noncompile}{}{}" in this LaTeX file.
% Look at the two lines right beneath this line.
% To compile the detailed version, they should be as follows:
%   \includecomment{verlong}
%   \excludecomment{vershort}
% To compile the short version, they should be as follows:
%   \excludecomment{verlong}
%   \includecomment{vershort}
% As a rule, the line
%   \excludecomment{noncompile}
% should stay as it is.
% -------------------------------------------------------------
% NOTES ON SOME HACKS USED IN THIS FILE:
% -------------------------------------------------------------
% One of my pet peeves with amsthm is its use of italics in the theorem and
% proposition environments; this makes math and text indistinguishable in said
% enviroments. To avoid this, I redefine the enviroments to use the standard
% font and to use a hanging indent, along with a bold vertical bar to its
% left, to distinguish these environments from surrounding text. (Along with
% the advantage of distinguishing math from text, this also allows nesting
% several such environments inside each other, like a definition inside a
% remark. I'm not sure how good of an idea this is, though. There are also
% downsides related to the hanging indentation, such as footnotes out of it
% being painful to do right.) This is done starting from the line
%   \theoremstyle{definition}
% and until the line
%   {\end{leftbar}\end{exmp}}

\documentclass[numbers=enddot,12pt,final,onecolumn,notitlepage]{scrartcl}%
\usepackage[headsepline,footsepline,manualmark]{scrlayer-scrpage}
\usepackage[all,cmtip]{xy}
\usepackage{amsfonts}
\usepackage{amssymb}
\usepackage{framed}
\usepackage{amsmath}
\usepackage{comment}
\usepackage{color}
\usepackage[breaklinks=true]{hyperref}
\usepackage[sc]{mathpazo}
\usepackage[T1]{fontenc}
\usepackage{amsthm}
\usepackage{needspace}
\usepackage{allrunes}
%TCIDATA{OutputFilter=latex2.dll}
%TCIDATA{Version=5.50.0.2960}
%TCIDATA{LastRevised=Monday, May 28, 2018 17:01:02}
%TCIDATA{SuppressPackageManagement}
%TCIDATA{<META NAME="GraphicsSave" CONTENT="32">}
%TCIDATA{<META NAME="SaveForMode" CONTENT="1">}
%TCIDATA{BibliographyScheme=Manual}
%BeginMSIPreambleData
\providecommand{\U}[1]{\protect\rule{.1in}{.1in}}
%EndMSIPreambleData
\theoremstyle{definition}
\newtheorem{theo}{Theorem}[section]
\newenvironment{theorem}[1][]
{\begin{theo}[#1]\begin{leftbar}}
{\end{leftbar}\end{theo}}
\newtheorem{lem}[theo]{Lemma}
\newenvironment{lemma}[1][]
{\begin{lem}[#1]\begin{leftbar}}
{\end{leftbar}\end{lem}}
\newtheorem{prop}[theo]{Proposition}
\newenvironment{proposition}[1][]
{\begin{prop}[#1]\begin{leftbar}}
{\end{leftbar}\end{prop}}
\newtheorem{defi}[theo]{Definition}
\newenvironment{definition}[1][]
{\begin{defi}[#1]\begin{leftbar}}
{\end{leftbar}\end{defi}}
\newtheorem{remk}[theo]{Remark}
\newenvironment{remark}[1][]
{\begin{remk}[#1]\begin{leftbar}}
{\end{leftbar}\end{remk}}
\newtheorem{coro}[theo]{Corollary}
\newenvironment{corollary}[1][]
{\begin{coro}[#1]\begin{leftbar}}
{\end{leftbar}\end{coro}}
\newtheorem{conv}[theo]{Convention}
\newenvironment{convention}[1][]
{\begin{conv}[#1]\begin{leftbar}}
{\end{leftbar}\end{conv}}
\newtheorem{quest}[theo]{Question}
\newenvironment{question}[1][]
{\begin{quest}[#1]\begin{leftbar}}
{\end{leftbar}\end{quest}}
\newtheorem{warn}[theo]{Warning}
\newenvironment{warning}[1][]
{\begin{warn}[#1]\begin{leftbar}}
{\end{leftbar}\end{warn}}
\newtheorem{conj}[theo]{Conjecture}
\newenvironment{conjecture}[1][]
{\begin{conj}[#1]\begin{leftbar}}
{\end{leftbar}\end{conj}}
\newtheorem{exmp}[theo]{Example}
\newenvironment{example}[1][]
{\begin{exmp}[#1]\begin{leftbar}}
{\end{leftbar}\end{exmp}}
\newenvironment{statement}{\begin{quote}}{\end{quote}}
\iffalse
\newenvironment{proof}[1][Proof]{\noindent\textbf{#1.} }{\ \rule{0.5em}{0.5em}}
\newenvironment{convention}[1][Convention]{\noindent\textbf{#1.} }{\ \rule{0.5em}{0.5em}}
\newenvironment{warning}[1][Warning]{\noindent\textbf{#1.} }{\ \rule{0.5em}{0.5em}}
\newenvironment{question}[1][Question]{\noindent\textbf{#1.} }{\ \rule{0.5em}{0.5em}}
\fi
\newenvironment{verlong}{}{}
\newenvironment{vershort}{}{}
\newenvironment{noncompile}{}{}
\excludecomment{verlong}
\includecomment{vershort}
\excludecomment{noncompile}
\newcommand{\id}{\operatorname{id}}
\newcommand{\ev}{\operatorname{ev}}
\newcommand{\Comp}{\operatorname{Comp}}
\newcommand{\QSym}{\operatorname{QSym}}
\newcommand{\bdd}{\operatorname{bdd}}
\newcommand{\Bdd}{\Powser_{\bdd}}
\newcommand{\bD}{\mathbf{D}}
\newcommand{\bk}{\mathbf{k}}
\newcommand{\Nplus}{\mathbb{N}_{+}}
\newcommand{\NN}{\mathbb{N}}
\newcommand{\ZZ}{\mathbb{Z}}
\newcommand{\xx}{\mathbf{x}}
\newcommand{\tvi}{\left. \textarm{\tvimadur} \right.}
\newcommand{\bel}{\left. \textarm{\belgthor} \right.}
\newcommand{\are}{\ar@{-}}
\newcommand{\arinj}{\ar@{_{(}->}}
\newcommand{\arsurj}{\ar@{->>}}
\iffalse
\NOEXPAND{\tvi}{\left. \textarm{\tvimadur} \right.}
\NOEXPAND{\bel}{\left. \textarm{\belgthor} \right.}
\fi
\newcommand\arxiv[1]{\href{http://www.arxiv.org/abs/#1}{\texttt{arXiv:#1}}}
\let\sumnonlimits\sum
\let\prodnonlimits\prod
\renewcommand{\sum}{\sumnonlimits\limits}
\renewcommand{\prod}{\prodnonlimits\limits}
\setlength\textheight{22.5cm}
\setlength\textwidth{15cm}
\begin{document}

\title{Shuffle-compatible permutation statistics II: the exterior peak set}
\author{Darij Grinberg}
\date{\today}
\maketitle
\tableofcontents

\section*{***}

This paper is a continuation of the work \cite{part1} by Gessel and Zhuang
(but can be read independently from the latter). It is devoted to the study of
shuffle-compatibility of permutation statistics -- a concept introduced in
\cite{part1}, although various instances of it have appeared throughout the
literature before.

In Section \ref{sect.notations}, we introduce the notations that we will need
throughout this paper. In Section \ref{sect.Zenri}, we prove that the exterior
peak set statistic $\operatorname*{Epk}$ is shuffle-compatible (Theorem
\ref{thm.Epk.sh-co-a}), as conjectured by Gessel and Zhuang in \cite{part1}.
In Section \ref{sect.LR}, we introduce the concept of an \textquotedblleft
LR-shuffle-compatible\textquotedblright\ statistic, which is stronger than
shuffle-compatibility. We give a sufficient criterion for it and use it to
show that $\operatorname*{Epk}$ and some other statistics are
LR-shuffle-compatible.

\begin{vershort}
The last three sections relate all of this to quasisymmetric functions; these
sections are only brief summaries, and we refer to \cite{verlong} for the
details. In Section \ref{sect.Descent}, we recall the concept of descent
statistics introduced in \cite{part1} and its connection to quasisymmetric
functions. Motivated by this connection, in Section \ref{sect.kernel}, we
define the kernel of a descent statistic, and study this kernel for
$\operatorname*{Epk}$, giving two explicit generating sets for this kernel. In
Section \ref{sect.dendri}, we extend the quasisymmetric functions connection
to the concept of LR-shuffle-compatible statistics, and relate it to
dendriform algebras.
\end{vershort}

\begin{verlong}
The last three sections relate all of this to quasisymmetric functions: In
Section \ref{sect.Descent}, we recall the concept of descent statistics
introduced in \cite{part1} and its connection to quasisymmetric functions.
Motivated by this connection, in Section \ref{sect.kernel}, we define the
kernel of a descent statistic, and study this kernel for $\operatorname*{Epk}%
$, giving two explicit generating sets for this kernel. In Section
\ref{sect.dendri}, we extend the quasisymmetric functions connection to the
concept of LR-shuffle-compatible statistics, and relate it to dendriform algebras.
\end{verlong}

\begin{noncompile}
Much of the following is a rough draft, and some proofs (particularly in
Section \ref{sect.dendri}) are merely outlined.
\end{noncompile}

\subsection*{Acknowledgments}

We thank Yan Zhuang and Ira Gessel for helpful conversations and a few
corrections. The SageMath computer algebra system \cite{SageMath} has been
used in finding some of the results below.

\subsection{Remark on alternative versions}

\begin{vershort}
This paper also has a detailed version \cite{verlong}, which includes some
proofs that have been omitted from the present version as well as more details
on some other proofs and further results in Sections \ref{sect.Descent},
\ref{sect.kernel} and \ref{sect.dendri}.
\end{vershort}

\begin{verlong}
You are reading the detailed version of this paper. For the standard version
(which is shorter by virtue of omitting some proofs and even some results),
see \cite{vershort}.
\end{verlong}

\section{\label{sect.notations}Notations and definitions}

Let us first introduce the definitions and notations that we will use in the
rest of this paper. Many of these definitions appear in \cite{part1} already;
we have tried to deviate from the notations of \cite{part1} as little as possible.

\subsection{Permutations and other basic concepts}

\begin{definition}
We let $\mathbb{N}=\left\{  0,1,2,3,\ldots\right\}  $ and $\mathbb{P}=\left\{
1,2,3,\ldots\right\}  $. Both of these sets are understood to be equipped with
their standard total order.
\end{definition}

\begin{definition}
Let $n\in\mathbb{Z}$. We shall use the notation $\left[  n\right]  $ for the
totally ordered set $\left\{  1,2,\ldots,n\right\}  $ (with the usual order
relation inherited from $\mathbb{Z}$). Note that $\left[  n\right]
=\varnothing$ when $n\leq0$.
\end{definition}

\begin{definition}
\label{def.perm}Let $n\in\mathbb{N}$. An $n$\textit{-permutation} shall mean a
word with $n$ letters, which are distinct and belong to $\mathbb{P}$.
Equivalently, an $n$-permutation shall be regarded as an injective map
$\left[  n\right]  \rightarrow\mathbb{P}$ (the image of $i$ under this map
being the $i$-th letter of the word).
\end{definition}

For example, $\left(  3,6,4\right)  $ and $\left(  9,1,2\right)  $ are
$3$-permutations, but $\left(  2,1,2\right)  $ is not.

\begin{definition}
A \textit{permutation} is defined to be an $n$-permutation for some
$n\in\mathbb{N}$. If $\pi$ is an $n$-permutation for some $n\in\mathbb{N}$,
then the number $n$ is called the \textit{size} of the
permutation $\pi$ and is denoted by $\left\vert \pi\right\vert $. A
permutation is said to be \textit{nonempty} if it is nonempty as a word (i.e.,
if its size is $>0$).
\end{definition}

Note that the meaning of \textquotedblleft permutation\textquotedblright\ we
have just defined is unusual (most authors define a permutation to be a
bijection from a set to itself); we are following \cite{part1} in defining
permutations this way.

\begin{definition}
Let $n\in\mathbb{N}$. Two $n$-permutations $\alpha$ and $\beta$ are said to be
\textit{order-equivalent} if they have the following property: For every two
integers $i,j\in\left[  n\right]  $, we have $\alpha\left(  i\right)
<\alpha\left(  j\right)  $ if and only if $\beta\left(  i\right)
<\beta\left(  j\right)  $.
\end{definition}

\begin{definition}
\textbf{(a)} A \textit{permutation statistic} is a map $\operatorname*{st}$
from the set of all permutations to an arbitrary set that has the following
property: Whenever $\alpha$ and $\beta$ are two order-equivalent permutations,
we have $\operatorname*{st}\left(  \alpha\right)  =\operatorname*{st}\left(
\beta\right)  $.

\textbf{(b)} Let $\operatorname*{st}$ be a permutation statistic. Two
permutations $\alpha$ and $\beta$ are said to be $\operatorname*{st}%
$\textit{-equivalent} if they satisfy $\left\vert \alpha\right\vert
=\left\vert \beta\right\vert $ and $\operatorname*{st}\alpha
=\operatorname*{st}\beta$. The relation \textquotedblleft$\operatorname*{st}%
$-equivalent\textquotedblright\ is an equivalence relation; its equivalence
classes are called $\operatorname*{st}$\textit{-equivalence classes}.
\end{definition}

\begin{remark}
Let $n\in\mathbb{N}$. Let us call an $n$-permutation $\pi$ \textit{standard}
if its letters are $1,2,\ldots,n$ (in some order). The standard $n$%
-permutations are in bijection with the $n!$ permutations of the set $\left\{
1,2,\ldots,n\right\}  $ in the usual sense of this word (i.e., the bijections
from this set to itself).

It is easy to see that for each $n$-permutation $\sigma$, there exists a
\textbf{unique} standard $n$-permutation $\pi$ order-equivalent to $\sigma$.
Thus, a permutation statistic is uniquely determined by its values on standard
permutations. Consequently, we can view permutation statistics as statistics
defined on standard permutations, i.e., on permutations in the usual sense of
the word.
\end{remark}

The word \textquotedblleft permutation statistic\textquotedblright\ is often
abbreviated as \textquotedblleft statistic\textquotedblright.

\subsection{Some examples of permutation statistics}

\begin{definition}
\label{def.Des-et-al}Let $n\in\mathbb{N}$. Let $\pi=\left(  \pi_{1},\pi
_{2},\ldots,\pi_{n}\right)  $ be an $n$-permutation.

\textbf{(a)} The \textit{descents} of $\pi$ are the elements $i\in\left[
n-1\right]  $ satisfying $\pi_{i}>\pi_{i+1}$.

\textbf{(b)} The \textit{descent set} of $\pi$ is defined to be the set of all
descents of $\pi$. This set is denoted by $\operatorname*{Des}\pi$, and is
always a subset of $\left[  n-1\right]  $.

\textbf{(c)} The \textit{peaks} of $\pi$ are the elements $i\in\left\{
2,3,\ldots,n-1\right\}  $ satisfying $\pi_{i-1}<\pi_{i}>\pi_{i+1}$.

\textbf{(d)} The \textit{peak set} of $\pi$ is defined to be the set of all
peaks of $\pi$. This set is denoted by $\operatorname*{Pk}\pi$, and is always
a subset of $\left\{  2,3,\ldots,n-1\right\}  $.

\textbf{(e)} The \textit{left peaks} of $\pi$ are the elements $i\in\left[
n-1\right]  $ satisfying $\pi_{i-1}<\pi_{i}>\pi_{i+1}$, where we set $\pi
_{0}=0$.

\textbf{(f)} The \textit{left peak set} of $\pi$ is defined to be the set of
all left peaks of $\pi$. This set is denoted by $\operatorname*{Lpk}\pi$, and
is always a subset of $\left[  n-1\right]  $. It is easy to see that (for
$n\geq2$) we have%
\[
\operatorname*{Lpk}\pi=\operatorname*{Pk}\pi\cup\left\{  1\ \mid\ \pi_{1}%
>\pi_{2}\right\}  .
\]
(Of course, $\left\{  1\ \mid\ \pi_{1}>\pi_{2}\right\}  $ is the $1$-element
set $\left\{  1\right\}  $ if $\pi_{1}>\pi_{2}$, and the empty set
$\varnothing$ otherwise.)

\textbf{(g)} The \textit{right peaks} of $\pi$ are the elements $i\in\left\{
2,3,\ldots,n\right\}  $ satisfying $\pi_{i-1}<\pi_{i}>\pi_{i+1}$, where we set
$\pi_{n+1}=0$.

\textbf{(h)} The \textit{right peak set} of $\pi$ is defined to be the set of
all right peaks of $\pi$. This set is denoted by $\operatorname*{Rpk}\pi$, and
is always a subset of $\left\{  2,3,\ldots,n\right\}  $. It is easy to see
that (for $n\geq2$) we have%
\[
\operatorname*{Rpk}\pi=\operatorname*{Pk}\pi\cup\left\{  n\ \mid\ \pi
_{n-1}<\pi_{n}\right\}  .
\]


\textbf{(i)} The \textit{exterior peaks} of $\pi$ are the elements
$i\in\left[  n\right]  $ satisfying $\pi_{i-1}<\pi_{i}>\pi_{i+1}$, where we
set $\pi_{0}=0$ and $\pi_{n+1}=0$.

\textbf{(j)} The \textit{exterior peak set} of $\pi$ is defined to be the set
of all exterior peaks of $\pi$. This set is denoted by $\operatorname*{Epk}%
\pi$, and is always a subset of $\left[  n\right]  $. It is easy to see that
(for $n\geq2$) we have%
\begin{align*}
\operatorname*{Epk}\pi &  =\operatorname*{Pk}\pi\cup\left\{  1\ \mid\ \pi
_{1}>\pi_{2}\right\}  \cup\left\{  n\ \mid\ \pi_{n-1}<\pi_{n}\right\} \\
&  =\operatorname*{Lpk}\pi\cup\operatorname*{Rpk}\pi.
\end{align*}
(For $n=1$, we have $\operatorname*{Epk}\pi=\left\{  1\right\}  $.)
\end{definition}

For example, the $6$-permutation $\pi=\left(  4,1,3,9,6,8\right)  $ has%
\begin{align*}
\operatorname*{Des}\pi &  =\left\{  1,4\right\}
,\ \ \ \ \ \ \ \ \ \ \operatorname*{Pk}\pi=\left\{  4\right\}  ,\\
\operatorname*{Lpk}\pi &  =\left\{  1,4\right\}
,\ \ \ \ \ \ \ \ \ \ \operatorname*{Rpk}\pi=\left\{  4,6\right\}
,\ \ \ \ \ \ \ \ \ \ \operatorname*{Epk}\pi=\left\{  1,4,6\right\}  .
\end{align*}
For another example, the $6$-permutation $\pi=\left(  1,4,3,2,9,8\right)  $
has%
\begin{align*}
\operatorname*{Des}\pi &  =\left\{  2,3,5\right\}
,\ \ \ \ \ \ \ \ \ \ \operatorname*{Pk}\pi=\left\{  2,5\right\}  ,\\
\operatorname*{Lpk}\pi &  =\left\{  2,5\right\}
,\ \ \ \ \ \ \ \ \ \ \operatorname*{Rpk}\pi=\left\{  2,5\right\}
,\ \ \ \ \ \ \ \ \ \ \operatorname*{Epk}\pi=\left\{  2,5\right\}  .
\end{align*}


Notice that Definition \ref{def.Des-et-al} actually defines several
permutation statistics. For example, Definition \ref{def.Des-et-al}
\textbf{(b)} defines the permutation statistic $\operatorname*{Des}$, whose
codomain is the set of all subsets of $\mathbb{P}$. Likewise, Definition
\ref{def.Des-et-al} \textbf{(d)} defines the permutation statistic
$\operatorname*{Pk}$, and Definition \ref{def.Des-et-al} \textbf{(f)} defines
the permutation statistic $\operatorname*{Lpk}$, whereas Definition
\ref{def.Des-et-al} \textbf{(h)} defines the permutation statistic
$\operatorname*{Rpk}$. The main permutation statistic that we will study in
this paper is $\operatorname*{Epk}$, which is defined in Definition
\ref{def.Des-et-al} \textbf{(j)}; its codomain is the set of all subsets of
$\mathbb{P}$.

The following simple fact expresses the set $\operatorname*{Epk}\pi$
corresponding to an $n$-permutation $\pi$ in terms of $\operatorname*{Des}\pi$:

\begin{proposition}
\label{prop.Epk.through-Des}Let $n$ be a positive integer. Let $\pi$ be an
$n$-permutation. Then,%
\[
\operatorname*{Epk}\pi=\left(  \operatorname*{Des}\pi\cup\left\{  n\right\}
\right)  \setminus\left(  \operatorname*{Des}\pi+1\right)  ,
\]
where $\operatorname*{Des}\pi+1$ denotes the set $\left\{  i+1\ \mid
\ i\in\operatorname*{Des}\pi\right\}  $.
\end{proposition}

\begin{vershort}
\begin{proof}
[Proof of Proposition \ref{prop.Epk.through-Des}.]The rather easy proof can be
found in the detailed version \cite{verlong} of this paper.
\end{proof}
\end{vershort}

\begin{verlong}
\begin{proof}
[Proof of Proposition \ref{prop.Epk.through-Des}.]Write $\pi$ in the form
$\pi=\left(  \pi_{1},\pi_{2},\ldots,\pi_{n}\right)  $. Set $\pi_{0}=0$ and
$\pi_{n+1}=0$. Recall that $\operatorname*{Des}\pi$ is defined as the set of
all descents of $\pi$. In other words,%
\[
\operatorname*{Des}\pi=\left(  \text{the set of all descents of }\pi\right)
=\left\{  i\in\left[  n-1\right]  \ \mid\ \pi_{i}>\pi_{i+1}\right\}
\]
(because the descents of $\pi$ are defined to be the $i\in\left[  n-1\right]
$ satisfying $\pi_{i}>\pi_{i+1}$).

But $\left(  \pi_{1},\pi_{2},\ldots,\pi_{n}\right)  =\pi$ is an $n$%
-permutation, and thus has no equal entries. Hence, for each $i\in\left[
n-1\right]  $, we have $\pi_{i}\neq\pi_{i+1}$. Thus, for each $i\in\left[
n-1\right]  $, we have the equivalence $\left(  \pi_{i}\geq\pi_{i+1}\right)
\Longleftrightarrow\left(  \pi_{i}>\pi_{i+1}\right)  $. Therefore,%
\[
\left\{  i\in\left[  n-1\right]  \ \mid\ \pi_{i}\geq\pi_{i+1}\right\}
=\left\{  i\in\left[  n-1\right]  \ \mid\ \pi_{i}>\pi_{i+1}\right\}
=\operatorname*{Des}\pi.
\]


On the other hand, $\pi_{n}\in\left[  n\right]  $, so that $\pi_{n}%
>0=\pi_{n+1}$. Hence, $n$ is an element of the set $\left\{  i\in\left\{
n\right\}  \ \mid\ \pi_{i}>\pi_{i+1}\right\}  $. Clearly, this set cannot have
any other element (since it is a subset of $\left\{  n\right\}  $); thus,
$\left\{  i\in\left\{  n\right\}  \ \mid\ \pi_{i}>\pi_{i+1}\right\}  =\left\{
n\right\}  $.

But $\left[  n\right]  =\left[  n-1\right]  \cup\left\{  n\right\}  $, so
that
\begin{align}
&  \left\{  i\in\left[  n\right]  \ \mid\ \pi_{i}>\pi_{i+1}\right\}
\nonumber\\
&  =\left\{  i\in\left[  n-1\right]  \cup\left\{  n\right\}  \ \mid\ \pi
_{i}>\pi_{i+1}\right\} \nonumber\\
&  =\underbrace{\left\{  i\in\left[  n-1\right]  \ \mid\ \pi_{i}>\pi
_{i+1}\right\}  }_{=\operatorname*{Des}\pi}\cup\underbrace{\left\{
i\in\left\{  n\right\}  \ \mid\ \pi_{i}>\pi_{i+1}\right\}  }_{=\left\{
n\right\}  }=\operatorname*{Des}\pi\cup\left\{  n\right\}  .
\label{pf.prop.Epk.through-Des.Desun}%
\end{align}


On the other hand, $\pi_{1}\in\left[  n\right]  $, so that $\pi_{1}>0=\pi_{0}%
$. Hence, we do not have $\pi_{0}\geq\pi_{1}$. Thus, $0$ is not an element of
the set $\left\{  i\in\left\{  0\right\}  \ \mid\ \pi_{i}\geq\pi
_{i+1}\right\}  $. Clearly, this set cannot have any other element (since it a
subset of $\left\{  0\right\}  $); thus, $\left\{  i\in\left\{  0\right\}
\ \mid\ \pi_{i}\geq\pi_{i+1}\right\}  =\varnothing$.

But $\left\{  0,1,\ldots,n-1\right\}  =\left\{  0\right\}  \cup\left[
n-1\right]  $, so that
\begin{align*}
&  \left\{  i\in\left\{  0,1,\ldots,n-1\right\}  \ \mid\ \pi_{i}\geq\pi
_{i+1}\right\} \\
&  =\left\{  i\in\left\{  0\right\}  \cup\left[  n-1\right]  \ \mid\ \pi
_{i}\geq\pi_{i+1}\right\} \\
&  =\underbrace{\left\{  i\in\left\{  0\right\}  \ \mid\ \pi_{i}\geq\pi
_{i+1}\right\}  }_{=\varnothing}\cup\underbrace{\left\{  i\in\left[
n-1\right]  \ \mid\ \pi_{i}\geq\pi_{i+1}\right\}  }_{=\operatorname*{Des}\pi
}=\varnothing\cup\operatorname*{Des}\pi=\operatorname*{Des}\pi.
\end{align*}
Hence,%
\[
\operatorname*{Des}\pi=\left\{  i\in\left\{  0,1,\ldots,n-1\right\}
\ \mid\ \pi_{i}\geq\pi_{i+1}\right\}  ,
\]
so that%
\begin{align}
\operatorname*{Des}\pi+1  &  =\left\{  i+1\ \mid\ i\in\left\{  0,1,\ldots
,n-1\right\}  \ \text{satisfies}\ \pi_{i}\geq\pi_{i+1}\right\} \nonumber\\
&  =\left\{  j\in\left[  n\right]  \ \mid\ \pi_{j-1}\geq\pi_{j}\right\}
\nonumber\\
&  =\left\{  i\in\left[  n\right]  \ \mid\ \pi_{i-1}\geq\pi_{i}\right\}
\label{pf.prop.Epk.through-Des.Des+1}%
\end{align}
(here, we have renamed the index $j$ as $i$).

But $\operatorname*{Epk}\pi$ is the set of all exterior peaks of $\pi$ (by the
definition of $\operatorname*{Epk}\pi$). Thus,%
\begin{align*}
\operatorname*{Epk}\pi &  =\left(  \text{the set of all exterior peaks of }%
\pi\right) \\
&  =\left\{  i\in\left[  n\right]  \ \mid\ \underbrace{\pi_{i-1}<\pi_{i}%
>\pi_{i+1}}_{\Longleftrightarrow\ \left(  \pi_{i}>\pi_{i+1}\text{ and }%
\pi_{i-1}<\pi_{i}\right)  }\right\} \\
&  \ \ \ \ \ \ \ \ \ \ \left(  \text{by the definition of an \textquotedblleft
exterior peak\textquotedblright\ of }\pi\right) \\
&  =\left\{  i\in\left[  n\right]  \ \mid\ \pi_{i}>\pi_{i+1}\text{ and
}\underbrace{\pi_{i-1}<\pi_{i}}_{\Longleftrightarrow\ \left(  \text{not }%
\pi_{i-1}\geq\pi_{i}\right)  }\right\} \\
&  =\left\{  i\in\left[  n\right]  \ \mid\ \pi_{i}>\pi_{i+1}\text{ and not
}\pi_{i-1}\geq\pi_{i}\right\} \\
&  =\underbrace{\left\{  i\in\left[  n\right]  \ \mid\ \pi_{i}>\pi
_{i+1}\right\}  }_{\substack{=\operatorname*{Des}\pi\cup\left\{  n\right\}
\\\text{(by (\ref{pf.prop.Epk.through-Des.Desun}))}}}\setminus
\underbrace{\left\{  i\in\left[  n\right]  \ \mid\ \pi_{i-1}\geq\pi
_{i}\right\}  }_{\substack{=\operatorname*{Des}\pi+1\\\text{(by
(\ref{pf.prop.Epk.through-Des.Des+1}))}}}\\
&  =\left(  \operatorname*{Des}\pi\cup\left\{  n\right\}  \right)
\setminus\left(  \operatorname*{Des}\pi+1\right)  .
\end{align*}
This proves Proposition \ref{prop.Epk.through-Des}.
\end{proof}
\end{verlong}

\subsection{Shuffles and shuffle-compatibility}

\begin{definition}
\label{def.shuffles}Let $\pi$ and $\sigma$ be two permutations.

\textbf{(a)} We say that $\pi$ and $\sigma$ are \textit{disjoint} if no letter
appears in both $\pi$ and $\sigma$.

\textbf{(b)} Assume that $\pi$ and $\sigma$ are disjoint. Set $m=\left\vert
\pi\right\vert $ and $n=\left\vert \sigma\right\vert $. Let $\tau$ be an
$\left(  m+n\right)  $-permutation. Then, we say that $\tau$ is a
\textit{shuffle} of $\pi$ and $\sigma$ if both $\pi$ and $\sigma$ are
subsequences of $\tau$.

\textbf{(c)} We let $S\left(  \pi,\sigma\right)  $ be the set of all shuffles
of $\pi$ and $\sigma$.
\end{definition}

For example, the permutations $\left(  3,1\right)  $ and $\left(
6,2,9\right)  $ are disjoint, whereas the permutations $\left(  3,1,2\right)
$ and $\left(  6,2,9\right)  $ are not. The shuffles of the two disjoint
permutations $\left(  3,1\right)  $ and $\left(  2,6\right)  $ are%
\begin{align*}
&  \left(  3,1,2,6\right)  ,\ \ \ \ \ \ \ \ \ \ \left(  3,2,1,6\right)
,\ \ \ \ \ \ \ \ \ \ \left(  3,2,6,1\right)  ,\\
&  \left(  2,3,1,6\right)  ,\ \ \ \ \ \ \ \ \ \ \left(  2,3,6,1\right)
,\ \ \ \ \ \ \ \ \ \ \left(  2,6,3,1\right)  .
\end{align*}


If $\pi$ and $\sigma$ are two disjoint permutations, then $S\left(  \pi
,\sigma\right)  =S\left(  \sigma,\pi\right)  $ is an $\dbinom{m+n}{m}$-element
set, where $m=\left\vert \pi\right\vert $ and $n=\left\vert \sigma\right\vert
$.

Definition \ref{def.shuffles} \textbf{(b)} is used, e.g., in \cite{Greene88}.
From the point of view of combinatorics on words, it is somewhat naive, as it
fails to properly generalize to the case when the words $\pi$ and $\sigma$ are
no longer disjoint\footnote{In this general case, it is best to define a
shuffle of two words $\pi=\left(  \pi_{1},\pi_{2},\ldots,\pi_{m}\right)  $ and
$\sigma=\left(  \sigma_{1},\sigma_{2},\ldots,\sigma_{n}\right)  $ as a word of
the form $\left(  \gamma_{\eta\left(  1\right)  },\gamma_{\eta\left(
2\right)  },\ldots,\gamma_{\eta\left(  m+n\right)  }\right)  $, where $\left(
\gamma_{1},\gamma_{2},\ldots,\gamma_{m+n}\right)  $ is the word $\left(
\pi_{1},\pi_{2},\ldots,\pi_{m},\sigma_{1},\sigma_{2},\ldots,\sigma_{n}\right)
$, and where $\eta$ is some permutation of the set $\left\{  1,2,\ldots
,m+n\right\}  $ (that is, a bijection from this set to itself) satisfying
$\eta^{-1}\left(  1\right)  <\eta^{-1}\left(  2\right)  <\cdots<\eta
^{-1}\left(  m\right)  $ (this causes the letters $\pi_{1},\pi_{2},\ldots
,\pi_{m}$ to appear in the word $\left(  \gamma_{\eta\left(  1\right)
},\gamma_{\eta\left(  2\right)  },\ldots,\gamma_{\eta\left(  m+n\right)
}\right)  $ in this order) and $\eta^{-1}\left(  m+1\right)  <\eta^{-1}\left(
m+2\right)  <\cdots<\eta^{-1}\left(  m+n\right)  $ (this causes the letters
$\sigma_{1},\sigma_{2},\ldots,\sigma_{n}$ to appear in the word $\left(
\gamma_{\eta\left(  1\right)  },\gamma_{\eta\left(  2\right)  },\ldots
,\gamma_{\eta\left(  m+n\right)  }\right)  $ in this order). Furthermore, the
proper generalization of $S\left(  \pi,\sigma\right)  $ to this case would be
a multiset, not a mere set.}. But we will not be considering this general
case, since our results do not seem to straightforwardly extend to it
(although we might have to look more closely); thus, Definition
\ref{def.shuffles} will suffice for us.

\begin{definition}
\textbf{(a)} If $a_{1},a_{2},\ldots,a_{k}$ are finitely many arbitrary
objects, then $\left\{  a_{1},a_{2},\ldots,a_{k}\right\}
_{\operatorname*{multi}}$ denotes the \textbf{multiset} whose elements are
$a_{1},a_{2},\ldots,a_{k}$ (each appearing with the multiplicity with which it
appears in the list $\left(  a_{1},a_{2},\ldots,a_{k}\right)  $).

\textbf{(b)} Let $\left(  a_{i}\right)  _{i\in I}$ be a finite family of
arbitrary objects. Then, $\left\{  a_{i}\ \mid\ i\in I\right\}
_{\operatorname*{multi}}$ denotes the \textbf{multiset} whose elements are the
elements of this family (each appearing with the multiplicity with which it
appears in the family).
\end{definition}

For example, $\left\{  k^{2}\ \mid\ k\in\left\{  -2,-1,0,1,2\right\}
\right\}  _{\operatorname*{multi}}$ is the multiset that contains the element
$4$ twice, the element $1$ twice, and the element $0$ once (and no other
elements). This multiset can also be written in the form $\left\{
4,1,0,1,4\right\}  _{\operatorname*{multi}}$, or in the form $\left\{
0,1,1,4,4\right\}  _{\operatorname*{multi}}$.

\begin{definition}
Let $\operatorname*{st}$ be a permutation statistic. We say that
$\operatorname*{st}$ is \textit{shuffle-compatible} if and only if it has the
following property: For any two disjoint permutations $\pi$ and $\sigma$, the
multiset%
\[
\left\{  \operatorname*{st}\left(  \tau\right)  \ \mid\ \tau\in S\left(
\pi,\sigma\right)  \right\}  _{\operatorname*{multi}}%
\]
depends only on $\operatorname*{st}\left(  \pi\right)  $, $\operatorname*{st}%
\left(  \sigma\right)  $, $\left\vert \pi\right\vert $ and $\left\vert
\sigma\right\vert $.
\end{definition}

In other words, a permutation statistic $\operatorname*{st}$ is
shuffle-compatible if and only if it has the following property:

\begin{itemize}
\item If $\pi$ and $\sigma$ are two disjoint permutations, and if $\pi
^{\prime}$ and $\sigma^{\prime}$ are two disjoint permutations, and if these
permutations satisfy%
\begin{align*}
\operatorname*{st}\left(  \pi\right)   &  =\operatorname*{st}\left(
\pi^{\prime}\right)  ,\ \ \ \ \ \ \ \ \ \ \operatorname*{st}\left(
\sigma\right)  =\operatorname*{st}\left(  \sigma^{\prime}\right)  ,\\
\left\vert \pi\right\vert  &  =\left\vert \pi^{\prime}\right\vert
\ \ \ \ \ \ \ \ \ \ \text{and}\ \ \ \ \ \ \ \ \ \ \left\vert \sigma\right\vert
=\left\vert \sigma^{\prime}\right\vert ,
\end{align*}
then
\[
\left\{  \operatorname*{st}\left(  \tau\right)  \ \mid\ \tau\in S\left(
\pi,\sigma\right)  \right\}  _{\operatorname*{multi}}=\left\{
\operatorname*{st}\left(  \tau\right)  \ \mid\ \tau\in S\left(  \pi^{\prime
},\sigma^{\prime}\right)  \right\}  _{\operatorname*{multi}}.
\]

\end{itemize}

The notion of a shuffle-compatible permutation statistic was coined by Gessel
and Zhuang in \cite{part1}, where various statistics were analyzed for their
shuffle-compatibility. In particular, it was shown in \cite{part1} that the
statistics $\operatorname*{Des}$, $\operatorname*{Pk}$, $\operatorname*{Lpk}$
and $\operatorname*{Rpk}$ are shuffle-compatible. Our next goal is to prove
the same for the statistic $\operatorname*{Epk}$.

\section{\label{sect.Zenri}Extending enriched $P$-partitions and the exterior
peak set}

We are going to define \textit{$\mathcal{Z}$-enriched }$P$\textit{-partitions}%
, which are a straightforward generalization of the notions of
\textquotedblleft$P$-partitions\textquotedblright\ \cite{Stanle72},
\textquotedblleft enriched $P$-partitions\textquotedblright\ \cite[\S 2]{Stembr97}
and \textquotedblleft left enriched $P$-partitions\textquotedblright
\ \cite{Peters05}. We will then consider a new particular case
of this notion, which leads to a proof of the shuffle-compatibility of
$\operatorname*{Epk}$ conjectured in \cite{part1}.

\subsection{\label{subsect.Zenri.gen}$\mathcal{Z}$-enriched $\left(
P,\gamma\right)  $-partitions}

\begin{convention}
By abuse of notation, we will often use the same notation for a poset
$P=\left(  X,\leq\right)  $ and its ground set $X$ when there is no danger of
confusion. In particular, if $x$ is some object, then \textquotedblleft$x\in
P$\textquotedblright\ shall mean \textquotedblleft$x\in X$\textquotedblright.
\end{convention}

\begin{definition}
A \textit{labeled poset} means a pair $\left(  P,\gamma\right)  $ consisting
of a finite poset $P=\left(  X,\leq\right)  $ and an injective map
$\gamma:X\rightarrow A$ for some totally ordered set $A$. The injective map
$\gamma$ is called the \textit{labeling} of the labeled poset $\left(
P,\gamma\right)  $. The poset $P$ is called the \textit{ground poset} of the
labeled poset $\left(  P,\gamma\right)  $.
\end{definition}

\begin{convention}
Let $\mathcal{N}$ be a totally ordered set, whose (strict) order relation will
be denoted by $\prec$. Let $+$ and $-$ be two distinct symbols. Let
$\mathcal{Z}$ be a subset of the set $\mathcal{N}\times\left\{  +,-\right\}
$. For each $q=\left(  n,s\right)  \in\mathcal{Z}$, we denote the element
$n\in\mathcal{N}$ by $\left\vert q\right\vert $, and we call the element
$s\in\left\{  +,-\right\}  $ the \textit{sign} of $q$. If $n\in\mathcal{N}$,
then we will denote the two elements $\left(  n,+\right)  $ and $\left(
n,-\right)  $ of $\mathcal{N}\times\left\{  +,-\right\}  $ by $+n$ and $-n$, respectively.

We equip the set $\mathcal{Z}$ with a total order, whose (strict) order
relation $\prec$ is defined by%
\[
\left(  n,s\right)  \prec\left(  n^{\prime},s^{\prime}\right)  \text{ if and
only if either }n\prec n^{\prime}\text{ or }\left(  n=n^{\prime}\text{ and
}s=-\text{ and }s^{\prime}=+\right)  .
\]
Let $\operatorname*{Pow}\mathcal{N}$ be the ring of all formal power series
over $\mathbb{Q}$ in the indeterminates $x_{n}$ for $n\in\mathcal{N}$.

We fix $\mathcal{N}$ and $\mathcal{Z}$ throughout Subsection
\ref{subsect.Zenri.gen}. That is, any result in this subsection is tacitly
understood to begin with \textquotedblleft Let $\mathcal{N}$ be a totally
ordered set, whose (strict) order relation will be denoted by $\prec$, and let
$\mathcal{Z}$ be a subset of the set $\mathcal{N}\times\left\{  +,-\right\}
$\textquotedblright; and the notations of this convention shall always be
in place throughout this Subsection.

Whenever $\prec$ denotes some strict order, the corresponding weak order will
be denoted by $\preccurlyeq$. (Thus, $a \preccurlyeq b$ means ``$a \prec b$ or
$a = b$''.)
\end{convention}

\begin{definition}
\label{def.ambivPp}Let $\left(  P,\gamma\right)  $ be a labeled poset. A
\textit{$\mathcal{Z}$-enriched }$\left(  P,\gamma\right)  $\textit{-partition}
means a map $f:P\rightarrow\mathcal{Z}$ such that for all $x<y$ in $P$, the
following conditions hold:

\begin{enumerate}
\item[\textbf{(i)}] We have $f\left(  x\right)  \preccurlyeq f\left(
y\right)  $.

\item[\textbf{(ii)}] If $f\left(  x\right)  =f\left(  y\right)  =+n$ for some
$n\in\mathcal{N}$, then $\gamma\left(  x\right)  <\gamma\left(  y\right)  $.

\item[\textbf{(iii)}] If $f\left(  x\right)  =f\left(  y\right)  =-n$ for some
$n\in\mathcal{N}$, then $\gamma\left(  x\right)  >\gamma\left(  y\right)  $.
\end{enumerate}

(Of course, this concept depends on $\mathcal{N}$ and $\mathcal{Z}$, but these
will always be clear from the context.)
\end{definition}

\Needspace{4cm}

\begin{example}
\label{exa.ambivPp.diamond}Let $P$ be the poset with the following Hasse
diagram:%
\[%
%TCIMACRO{\TeXButton{Hasse diagram}{\xymatrix{
%& b \are[dl] \are[dr] \\
%c \are[dr] & & d \are[dl] \\
%& a
%}}}%
%BeginExpansion
\xymatrix{
& b \are[dl] \are[dr] \\
c \are[dr] & & d \are[dl] \\
& a
}%
%EndExpansion
\]
(that is, the ground set of $P$ is $\left\{ a, b, c, d \right\}$, and
its order relation is given by $a < c < b$ and $a < d < b$).
Let $\gamma:P\rightarrow\mathbb{Z}$ be a set that satisfies
$\gamma\left(  a\right)  <\gamma\left(  b\right)  <\gamma\left(  c\right)
<\gamma\left(  d\right)  $ (for example, $\gamma$ could be the map that sends
$a,b,c,d$ to $2,3,5,7$, respectively).
Then, $\left(P, \gamma\right)$ is a labeled poset.
A $\mathcal{Z}$-enriched $\left(
P,\gamma\right)  $-partition is a map $f:P\rightarrow\mathcal{Z}$ satisfying
the following conditions:

\begin{enumerate}
\item[\textbf{(i)}] We have $f\left(  a\right)  \preccurlyeq f\left(
c\right)  \preccurlyeq f\left(  b\right)  $ and $f\left(  a\right)
\preccurlyeq f\left(  d\right)  \preccurlyeq f\left(  b\right)  $.

\item[\textbf{(ii)}] We cannot have $f\left(  c\right)  =f\left(  b\right)
=+n$ with $n\in\mathcal{N}$. \newline
We cannot have $f\left(  d\right)
=f\left(  b\right)  =+n$ with $n\in\mathcal{N}$.

\item[\textbf{(iii)}] We cannot have $f\left(  a\right)  =f\left(  c\right)
=-n$ with $n\in\mathcal{N}$. \newline
We cannot have $f\left(  a\right)
=f\left(  d\right)  =-n$ with $n\in\mathcal{N}$.
\end{enumerate}

For example, if $\mathcal{N}=\mathbb{P}$ (the totally ordered set of positive
integers, with its usual ordering) and $\mathcal{Z}=\mathcal{N}\times\left\{
+,-\right\}  $, then the map $f:P\rightarrow\mathcal{Z}$ sending $a,b,c,d$ to
$+2,-3,+2,-3$ (respectively) is a $\mathcal{Z}$-enriched $\left(
P,\gamma\right)  $-partition. Notice that the total ordering on $\mathcal{Z}$
in this case is given by%
\[
-1\prec+1\prec-2\prec+2\prec-3\prec+3\prec\cdots,
\]
rather than by the familiar total order on $\mathbb{Z}$.
\end{example}

The concept of a \textquotedblleft$\mathcal{Z}$-enriched $\left(
P,\gamma\right)  $-partition\textquotedblright\ generalizes three notions in
existing literature: that of a \textquotedblleft$\left(  P,\gamma\right)
$-partition\textquotedblright, that of an \textquotedblleft enriched $\left(
P,\gamma\right)  $-partition\textquotedblright, and that of a
\textquotedblleft left enriched $\left(  P,\gamma\right)  $%
-partition\textquotedblright\footnote{The ideas behind these three concepts
are due to Stanley \cite{Stanle72}, Stembridge \cite[\S 2]{Stembr97} and
Petersen \cite{Peters05}, respectively, but the precise definitions are not
standardized across the literature. We define a \textquotedblleft$\left(
P,\gamma\right)  $-partition\textquotedblright\ as in \cite[\S 1.1]{Stembr97};
this definition differs noticeably from Stanley's (in particular, Stanley
requires $f\left(  x\right)  \succcurlyeq f\left(  y\right)  $ instead of
$f\left(  x\right)  \preccurlyeq f\left(  y\right)  $, but the differences do
not end here). We define an \textquotedblleft enriched $\left(  P,\gamma
\right)  $-partition\textquotedblright\ as in \cite[\S 2]{Stembr97}. Finally,
we define a \textquotedblleft left enriched $\left(  P,\gamma\right)
$-partition\textquotedblright\ to be a $\mathcal{Z}$-enriched $\left(
P,\gamma\right)  $-partition where $\mathcal{N}=\mathbb{N}$ and $\mathcal{Z}%
=\left(  \mathcal{N}\times\left\{  +,-\right\}  \right)  \setminus\left\{
-0\right\}  $; this definition is equivalent to Petersen's
\cite[Definition 3.4.1]{Peters06} up to some differences of notation
(in particular, Petersen assumes that the ground set of $P$ is already
a subset of $\mathbb{P}$, and that the labelling $\gamma$ is the
canonical inclusion map $ P \to \mathbb{P}$; also, he identifies the elements
$+0, -1, +1, -2, +2, \ldots$ of
$\left(  \mathcal{N}\times\left\{  +,-\right\}  \right)  \setminus\left\{
-0\right\}  $ with the integers
$0, -1, +1, -2, +2, \ldots$, respectively).
Note that the definition Petersen gives in \cite[Definition 4.1]{Peters05}
is incorrect, and the one in \cite[Definition 3.4.1]{Peters06} is probably
his intent.}:

\begin{example}
\label{exa.ambivPp.abc}\textbf{(a)} If $\mathcal{N}=\mathbb{P}$ (the totally
ordered set of positive integers) and $\mathcal{Z}=\mathcal{N}\times\left\{
+\right\}  =\left\{  +n\ \mid\ n\in\mathcal{N}\right\}  $, then the
$\mathcal{Z}$-enriched $\left(  P,\gamma\right)  $-partitions are simply the
$\left(  P,\gamma\right)  $-partitions into $\mathcal{N}$, composed with the
canonical bijection $\mathcal{N}\rightarrow\mathcal{Z},\ n\mapsto\left(
+n\right)  $.

\textbf{(b)} If $\mathcal{N}=\mathbb{P}$ (the totally ordered set of positive
integers) and $\mathcal{Z}=\mathcal{N}\times\left\{  +,-\right\}  $, then the
$\mathcal{Z}$-enriched $\left(  P,\gamma\right)  $-partitions are the enriched
$\left(  P,\gamma\right)  $-partitions.

\textbf{(c)} If $\mathcal{N}=\mathbb{N}$ (the totally ordered set of
nonnegative integers) and $\mathcal{Z}=\left(  \mathcal{N}\times\left\{
+,-\right\}  \right)  \setminus\left\{  -0\right\}  $, then the $\mathcal{Z}%
$-enriched $\left(  P,\gamma\right)  $-partitions are the left enriched
$\left(  P,\gamma\right)  $-partitions. Note that $+0$ and $-0$ here stand for
the pairs $\left(  0,+\right)  $ and $\left(  0,-\right)  $; thus, they are
not equal.
\end{example}

\begin{definition}
If $\left(  P,\gamma\right)  $ is a labeled poset, then $\mathcal{E}\left(
P,\gamma\right)  $ shall denote the set of all $\mathcal{Z}$-enriched $\left(
P,\gamma\right)  $-partitions.
\end{definition}

\begin{definition}
Let $P$ be any finite poset. Then, $\mathcal{L}\left(  P\right)  $ shall
denote the set of all linear extensions of $P$. A linear extension of $P$
shall be understood simultaneously as a totally ordered set extending $P$ and
as a list $\left(  w_{1},w_{2},\ldots,w_{n}\right)  $ of all elements of $P$
such that no two integers $i<j$ satisfy $w_{i}\geq w_{j}$ in $P$.
\end{definition}

Let us prove some basic facts about $\mathcal{Z}$-enriched $\left(
P,\gamma\right)  $-partitions, straightforwardly generalizing classical
results proven by Stanley and Gessel (for the case of \textquotedblleft
plain\textquotedblright\ $\left(  P,\gamma\right)  $-partitions), Stembridge
(for enriched $\left(  P,\gamma\right)  $-partitions) and Petersen (for left
enriched $\left(  P,\gamma\right)  $-partitions):

\begin{proposition}
\label{prop.fund-lem}For any labeled poset $\left(  P,\gamma\right)  $, we
have%
\[
\mathcal{E}\left(  P,\gamma\right)  =\bigsqcup_{w\in\mathcal{L}\left(
P\right)  }\mathcal{E}\left(  w,\gamma\right)  .
\]

\end{proposition}

\begin{proof}
[Proof of Proposition \ref{prop.fund-lem}.]This is analogous to the proof of
\cite[Lemma 2.1]{Stembr97}.
\end{proof}

\begin{definition}
\label{def.GammaZ}Let $\left(  P,\gamma\right)  $ be a labeled poset. We
define a power series $\Gamma_{\mathcal{Z}}\left(  P,\gamma\right)
\in\operatorname*{Pow}\mathcal{N}$ by%
\[
\Gamma_{\mathcal{Z}}\left(  P,\gamma\right)  =\sum_{f\in\mathcal{E}\left(
P,\gamma\right)  }\prod_{p\in P}x_{\left\vert f\left(  p\right)  \right\vert
}.
\]
This is easily seen to be convergent in the usual topology on
$\operatorname*{Pow}\mathcal{N}$.
\end{definition}

\begin{corollary}
\label{cor.fund-lem}For any labeled poset $\left(  P,\gamma\right)  $, we have%
\[
\Gamma_{\mathcal{Z}}\left(  P,\gamma\right)  =\sum_{w\in\mathcal{L}\left(
P\right)  }\Gamma_{\mathcal{Z}}\left(  w,\gamma\right)  .
\]

\end{corollary}

\begin{proof}
[Proof of Corollary \ref{cor.fund-lem}.]Follows straight from Proposition
\ref{prop.fund-lem}.
\end{proof}

\begin{definition}
Let $P$ be any set. Let $A$ be a totally ordered set. Let $\gamma:P\rightarrow
A$ and $\delta:P\rightarrow A$ be two maps. We say that $\gamma$ and $\delta$
are \textit{order-equivalent} if the following holds: For every pair $\left(
p,q\right)  \in P\times P$, we have $\gamma\left(  p\right)  \leq\gamma\left(
q\right)  $ if and only if $\delta\left(  p\right)  \leq\delta\left(
q\right)  $.
\end{definition}

\begin{proposition}
\label{prop.prod1}Let $\left(  P,\gamma\right)  $ and $\left(  Q,\delta
\right)  $ be two labeled posets. Let $\left(  P\sqcup Q,\varepsilon\right)  $
be a labeled poset whose ground poset $P\sqcup Q$ is the disjoint union of $P$
and $Q$, and whose labeling $\varepsilon$ is such that the restriction of
$\varepsilon$ to $P$ is order-equivalent to $\gamma$ and such that the
restriction of $\varepsilon$ to $Q$ is order-equivalent to $\delta$. Then,%
\[
\Gamma_{\mathcal{Z}}\left(  P,\gamma\right)  \Gamma_{\mathcal{Z}}\left(
Q,\delta\right)  =\Gamma_{\mathcal{Z}}\left(  P\sqcup Q,\varepsilon\right)  .
\]

\end{proposition}

\begin{proof}
[Proof of Proposition \ref{prop.prod1}.]We WLOG assume that the ground sets
$P$ and $Q$ are disjoint; thus, we can identify $P\sqcup Q$ with the union
$P\cup Q$. The map%
\begin{align}
\mathcal{E}\left(  P\sqcup Q,\varepsilon\right)   &  \rightarrow
\mathcal{E}\left(  P,\gamma\right)  \times\mathcal{E}\left(  Q,\delta\right)
,\nonumber\\
f  &  \mapsto\left(  f\mid_{P},f\mid_{Q}\right)  \label{pf.prop.prod1.bij}%
\end{align}
is a bijection (this is easy to see). Now,%
\begin{align*}
\Gamma_{\mathcal{Z}}\left(  P\sqcup Q,\varepsilon\right)   &  =\sum
_{f\in\mathcal{E}\left(  P\sqcup Q,\varepsilon\right)  }\underbrace{\prod
_{p\in P\sqcup Q}x_{\left\vert f\left(  p\right)  \right\vert }}_{=\left(
\prod_{p\in P}x_{\left\vert f\left(  p\right)  \right\vert }\right)  \left(
\prod_{p\in Q}x_{\left\vert f\left(  p\right)  \right\vert }\right)  }\\
&  =\sum_{f\in\mathcal{E}\left(  P\sqcup Q,\varepsilon\right)  }\left(
\prod_{p\in P}x_{\left\vert f\left(  p\right)  \right\vert }\right)  \left(
\prod_{p\in Q}x_{\left\vert f\left(  p\right)  \right\vert }\right) \\
&  =\underbrace{\left(  \sum_{g\in\mathcal{E}\left(  P,\gamma\right)  }%
\prod_{p\in P}x_{\left\vert g\left(  p\right)  \right\vert }\right)
}_{=\Gamma_{\mathcal{Z}}\left(  P,\gamma\right)  }\underbrace{\left(
\sum_{h\in\mathcal{E}\left(  Q,\delta\right)  }\prod_{p\in Q}x_{\left\vert
h\left(  p\right)  \right\vert }\right)  }_{=\Gamma_{\mathcal{Z}}\left(
Q,\delta\right)  }\\
&  \ \ \ \ \ \ \ \ \ \ \left(
\begin{array}
[c]{c}%
\text{here, we have substituted }\left(  g,h\right)  \text{ for }\left(
f\mid_{P},f\mid_{Q}\right)  \text{, since}\\
\text{the map (\ref{pf.prop.prod1.bij}) is a bijection}%
\end{array}
\right) \\
&  =\Gamma_{\mathcal{Z}}\left(  P,\gamma\right)  \Gamma_{\mathcal{Z}}\left(
Q,\delta\right)  .
\end{align*}

\end{proof}

\begin{definition}
Let $n\in\mathbb{N}$. Let $\pi$ be any $n$-permutation. (Recall that we have
defined the concept of an \textquotedblleft$n$-permutation\textquotedblright%
\ in Definition \ref{def.perm}.) Then, $\left(  \left[  n\right]  ,\pi\right)
$ is a labeled poset (in fact, $\pi$ is an injective map $\left[  n\right]
\rightarrow\left\{  1,2,3,\ldots\right\}  $, and thus can be considered a
labeling). We define $\Gamma_{\mathcal{Z}}\left(  \pi\right)  $ to be the
power series $\Gamma_{\mathcal{Z}}\left(  \left[  n\right]  ,\pi\right)  $.
\end{definition}

We shall now prove two simple facts of auxiliary use:

\begin{proposition}
\label{prop.Gamma=Gamma}Let $w$ be a finite totally ordered set with ground
set $W$. Let $n=\left\vert W\right\vert $. Let $\overline{w}$ be the unique
poset isomorphism $w\rightarrow\left[  n\right]  $. Let $\gamma:W\rightarrow
\left\{  1,2,3,\ldots\right\}  $ be any injective map. Then, $\Gamma
_{\mathcal{Z}}\left(  w,\gamma\right)  =\Gamma_{\mathcal{Z}}\left(
\gamma\circ\overline{w}^{-1}\right)  $.
\end{proposition}

\begin{proof}
[Proof of Proposition \ref{prop.Gamma=Gamma}.]The map $\gamma\circ\overline
{w}^{-1}:\left[  n\right]  \rightarrow\left\{  1,2,3,\ldots\right\}  $ is an
injective map, thus an $n$-permutation. Hence, $\Gamma_{\mathcal{Z}}\left(
\gamma\circ\overline{w}^{-1}\right)  $ is well-defined, and in fact we have
$\Gamma_{\mathcal{Z}}\left(  \gamma\circ\overline{w}^{-1}\right)
=\Gamma_{\mathcal{Z}}\left(  \left[  n\right]  ,\gamma\circ\overline{w}%
^{-1}\right)  $. Now, the map%
\begin{align*}
\mathcal{E}\left(  w,\gamma\right)   &  \rightarrow\mathcal{E}\left(  \left[
n\right]  ,\gamma\circ\overline{w}^{-1}\right)  ,\\
f  &  \mapsto f\circ\overline{w}^{-1}%
\end{align*}
is a bijection. Furthermore, it satisfies $\prod_{p\in w}x_{\left\vert
f\left(  p\right)  \right\vert }=\prod_{p\in\left[  n\right]  }x_{\left\vert
\left(  f\circ\overline{w}^{-1}\right)  \left(  p\right)  \right\vert }$ for
each $f\in\mathcal{E}\left(  w,\gamma\right)  $. Hence, $\Gamma_{\mathcal{Z}%
}\left(  w,\gamma\right)  =\Gamma_{\mathcal{Z}}\left(  \left[  n\right]
,\gamma\circ\overline{w}^{-1}\right)  =\Gamma_{\mathcal{Z}}\left(  \gamma
\circ\overline{w}^{-1}\right)  $.
\end{proof}

\begin{corollary}
\label{cor.fund-lem2}Let $\left(  P,\gamma\right)  $ be a labeled poset. Let
$n=\left\vert P\right\vert $. Then,%
\[
\Gamma_{\mathcal{Z}}\left(  P,\gamma\right)  =\sum_{\substack{x:P\rightarrow
\left[  n\right]  \\\text{bijective poset}\\\text{homomorphism}}%
}\Gamma_{\mathcal{Z}}\left(  \gamma\circ x^{-1}\right)  .
\]

\end{corollary}

\begin{proof}
[Proof of Corollary \ref{cor.fund-lem2}.]For each totally ordered set $w$ with
ground set $P$, we let $\overline{w}$ be the unique poset isomorphism
$w\rightarrow\left[  n\right]  $.

Corollary \ref{cor.fund-lem} yields%
\begin{equation}
\Gamma_{\mathcal{Z}}\left(  P,\gamma\right)  =\sum_{w\in\mathcal{L}\left(
P\right)  }\underbrace{\Gamma_{\mathcal{Z}}\left(  w,\gamma\right)
}_{\substack{=\Gamma_{\mathcal{Z}}\left(  \gamma\circ\overline{w}^{-1}\right)
\\\text{(by Proposition \ref{prop.Gamma=Gamma})}}}=\sum_{w\in\mathcal{L}%
\left(  P\right)  }\Gamma_{\mathcal{Z}}\left(  \gamma\circ\overline{w}%
^{-1}\right)  . \label{pf.cor.fund-lem2.1}%
\end{equation}
But the linear extensions of $P$ are in bijection with the bijective poset
homomorphisms $x:P\rightarrow\left[  n\right]  $; the bijection sends a linear
extension $w$ of $P$ to the bijective poset homomorphism $\overline
{w}:P\rightarrow\left[  n\right]  $. Thus, we can substitute $x$ for $w$ in
the sum $\sum_{w\in\mathcal{L}\left(  P\right)  }\Gamma_{\mathcal{Z}}\left(
\gamma\circ\overline{w}^{-1}\right)  $, obtaining%
\[
\sum_{w\in\mathcal{L}\left(  P\right)  }\Gamma_{\mathcal{Z}}\left(
\gamma\circ\overline{w}^{-1}\right)  =\sum_{\substack{x:P\rightarrow\left[
n\right]  \\\text{bijective poset}\\\text{homomorphism}}}\Gamma_{\mathcal{Z}%
}\left(  \gamma\circ x^{-1}\right)  .
\]
Combining this with (\ref{pf.cor.fund-lem2.1}), we end up with the claim of
Corollary \ref{cor.fund-lem2}.
\end{proof}

\begin{corollary}
\label{cor.prod2}Let $n\in\mathbb{N}$ and $m\in\mathbb{N}$. Let $\pi$ be an
$n$-permutation and let $\sigma$ be an $m$-permutation such that $\pi$ and
$\sigma$ are disjoint. Then,%
\[
\Gamma_{\mathcal{Z}}\left(  \pi\right)  \Gamma_{\mathcal{Z}}\left(
\sigma\right)  =\sum_{\tau\in S\left(  \pi,\sigma\right)  }\Gamma
_{\mathcal{Z}}\left(  \tau\right)  .
\]

\end{corollary}

\begin{proof}
[Proof of Corollary \ref{cor.prod2}.]Let $\varepsilon$ be the map $\left[
n\right]  \sqcup\left[  m\right]  \rightarrow\left\{  1,2,3,\ldots\right\}  $
which restricts to $\pi$ on the $\left[  n\right]  $ part and restricts to
$\sigma$ on the $\left[  m\right]  $ part. This map $\varepsilon$ is an
$\left(  n+m\right)  $-permutation, since $\pi$ and $\sigma$ are disjoint.

Let $\rho:\left[  n\right]  \sqcup\left[  m\right]  \rightarrow\left[
n+m\right]  $ be the strictly order-preserving bijection which sends the
elements of $\left[  n\right]  $ to $1,2,\ldots,n$ and sends the elements of
$\left[  m\right]  $ to $n+1,n+2,\ldots,n+m$.

The definitions of $\Gamma_{\mathcal{Z}}\left(  \pi\right)  $ and
$\Gamma_{\mathcal{Z}}\left(  \sigma\right)  $ yield%
\begin{align*}
\Gamma_{\mathcal{Z}}\left(  \pi\right)  \Gamma_{\mathcal{Z}}\left(
\sigma\right)   &  =\Gamma_{\mathcal{Z}}\left(  \left[  n\right]  ,\pi\right)
\Gamma_{\mathcal{Z}}\left(  \left[  m\right]  ,\sigma\right) \\
&  =\Gamma_{\mathcal{Z}}\left(  \left[  n\right]  \sqcup\left[  m\right]
,\varepsilon\right)  \ \ \ \ \ \ \ \ \ \ \left(  \text{by Proposition
\ref{prop.prod1}}\right) \\
&  =\sum_{\substack{x:\left[  n\right]  \sqcup\left[  m\right]  \rightarrow
\left[  n+m\right]  \\\text{bijective poset}\\\text{homomorphism}}%
}\Gamma_{\mathcal{Z}}\left(  \varepsilon\circ x^{-1}\right)
\ \ \ \ \ \ \ \ \ \ \left(  \text{by Corollary \ref{cor.fund-lem2}}\right) \\
&  =\sum_{\tau\in S\left(  \sigma,\pi\right)  }\Gamma_{\mathcal{Z}}\left(
\tau\right)  .
\end{align*}
Here, the last equality sign makes use of the (easy) fact that the map%
\begin{align*}
\left\{  \text{bijective poset homomorphisms }x:\left[  n\right]
\sqcup\left[  m\right]  \rightarrow\left[  n+m\right]  \right\}   &
\rightarrow S\left(  \sigma,\pi\right)  ,\\
x  &  \mapsto\varepsilon\circ x^{-1}%
\end{align*}
is a well-defined bijection.
\end{proof}

\subsection{Exterior peaks}

So far we have been doing general nonsense. Let us now specialize to a
situation that is connected to exterior peaks.

\begin{convention}
From now on, we set $\mathcal{N}=\left\{  0,1,2,\ldots\right\}  \cup\left\{
\infty\right\}  $, with total order given by $0\prec1\prec2\prec\cdots
\prec\infty$, and we set
\begin{align*}
\mathcal{Z}  &  =\left(  \mathcal{N}\times\left\{  +,-\right\}  \right)
\setminus\left\{  -0,+\infty\right\} \\
&  =\left\{  +0\right\}  \cup\left\{  +n\ \mid\ n\in\left\{  1,2,3,\ldots
\right\}  \right\}  \cup\left\{  -n\ \mid\ n\in\left\{  1,2,3,\ldots\right\}
\right\}  \cup\left\{  -\infty\right\}  .
\end{align*}
Recall that the total order on $\mathcal{Z}$ has%
\[
+0\prec-1\prec+1\prec-2\prec+2\prec\cdots\prec-\infty.
\]

\end{convention}

\begin{definition}
A map $\chi$ from a subset $S$ of $\mathbb{Z}$ to a totally ordered set $K$ is
said to be \textit{V-shaped} if there exists some $t\in S$ such that the map
$\chi\mid_{\left\{  s\in S\ \mid\ s\leq t\right\}  }$ is strictly decreasing
while the map $\chi\mid_{\left\{  s\in S\ \mid\ s\geq t\right\}  }$ is
strictly increasing. Notice that this $t\in S$ is uniquely determined in this
case; namely, it is the unique $k\in S$ that minimizes $\chi\left(  k\right)
$.
\end{definition}

Thus, roughly speaking, a map from a totally ordered set is \textit{V-shaped}
if and only if it is strictly decreasing up until a certain value of its
argument, and then strictly increasing afterwards.

\begin{definition}
Let $n\in\mathbb{N}$.

\begin{enumerate}
\item[\textbf{(a)}] Let $f:\left[  n\right]  \rightarrow\mathcal{Z}$ be any
map. Then, $\left\vert f\right\vert $ shall denote the map $\left[  n\right]
\rightarrow\mathcal{N},\ i\mapsto\left\vert f\left(  i\right)  \right\vert $.

\item[\textbf{(b)}] Let $g:\left[  n\right]  \rightarrow\mathcal{N}$ be any
map. Then, we define a monomial $\mathbf{x}_{g}$ in $\operatorname*{Pow}%
\mathcal{N}$ by $\mathbf{x}_{g}=\prod_{i=1}^{n}x_{g\left(  i\right)  }$. Note
that this allows us to rewrite the definition of $\Gamma_{\mathcal{Z}}\left(
\pi\right)  $ as follows: If $\pi$ is any $n$-permutation, then%
\begin{equation}
\Gamma_{\mathcal{Z}}\left(  \pi\right)  =\sum_{f\in\mathcal{E}\left(  \left[
n\right]  ,\pi\right)  }\prod_{p\in\left[  n\right]  }x_{\left\vert f\left(
p\right)  \right\vert }=\sum_{f\in\mathcal{E}\left(  \left[  n\right]
,\pi\right)  }\mathbf{x}_{\left\vert f\right\vert }. \label{eq.Gamma.rewr}%
\end{equation}


\item[\textbf{(c)}] Let $g:\left[  n\right]  \rightarrow\mathcal{N}$ be any
map. Let $\pi$ be an $n$-permutation. We shall say that $g$ is $\pi
$\textit{-amenable} if it has the following properties:

\begin{enumerate}
\item[\textbf{(i')}] The map $\pi\mid_{g^{-1}\left(  0\right)  }$ is strictly
increasing. (This allows the case when $g^{-1}\left(  0\right)  =\varnothing$.)

\item[\textbf{(ii')}] For each $h\in g\left(  \left[  n\right]  \right)
\cap\left\{  1,2,3,\ldots\right\}  $, the map $\pi\mid_{g^{-1}\left(
h\right)  }$ is V-shaped.

\item[\textbf{(iii')}] The map $\pi\mid_{g^{-1}\left(  \infty\right)  }$ is
strictly decreasing. (This allows the case when $g^{-1}\left(  \infty\right)
=\varnothing$.)

\item[\textbf{(iv')}] The map $g$ is weakly increasing.
\end{enumerate}
\end{enumerate}
\end{definition}

\begin{proposition}
\label{prop.Epk-formula}Let $n\in\mathbb{N}$. Let $\pi$ be any $n$%
-permutation. Then,%
\[
\Gamma_{\mathcal{Z}}\left(  \pi\right)  =\sum_{\substack{g:\left[  n\right]
\rightarrow\mathcal{N}\\\text{is }\pi\text{-amenable}}}2^{\left\vert g\left(
\left[  n\right]  \right)  \cap\left\{  1,2,3,\ldots\right\}  \right\vert
}\mathbf{x}_{g}.
\]

\end{proposition}

\begin{proof}
[Proof of Proposition \ref{prop.Epk-formula}.]The claim will immediately
follow from (\ref{eq.Gamma.rewr}) once we have shown the following two observations:

\begin{statement}
\textit{Observation 1:} If $f\in\mathcal{E}\left(  \left[  n\right]
,\pi\right)  $, then the map $\left\vert f\right\vert :\left[  n\right]
\rightarrow\mathcal{N}$ is $\pi$-amenable.
\end{statement}

\begin{statement}
\textit{Observation 2:} If $g:\left[  n\right]  \rightarrow\mathcal{N}$ is a
$\pi$-amenable map, then there exist precisely $2^{\left\vert g\left(  \left[
n\right]  \right)  \cap\left\{  1,2,3,\ldots\right\}  \right\vert }$ maps
$f\in\mathcal{E}\left(  \left[  n\right]  ,\pi\right)  $ satisfying
$\left\vert f\right\vert =g$.
\end{statement}

But both of these observations are easy:

[\textit{Proof of Observation 1:} This is a simple consequence of the
definition of a $\mathcal{Z}$-enriched $\left(  \left[  n\right]  ,\pi\right)
$-partition. Let me spell it out: Let $f\in\mathcal{E}\left(  \left[
n\right]  ,\pi\right)  $. Thus, $f$ is an $\mathcal{Z}$-enriched $\left(
\left[  n\right]  ,\pi\right)  $-partition. In other words, $f$ is a map
$\left[  n\right]  \rightarrow\mathcal{Z}$ such that for all $x<y$ in $\left[
n\right]  $, the following conditions hold:

\begin{enumerate}
\item[\textbf{(i)}] We have $f\left(  x\right)  \preccurlyeq f\left(
y\right)  $.

\item[\textbf{(ii)}] If $f\left(  x\right)  =f\left(  y\right)  =+h$ for some
$h\in\mathcal{N}$, then $\pi\left(  x\right)  <\pi\left(  y\right)  $.

\item[\textbf{(iii)}] If $f\left(  x\right)  =f\left(  y\right)  =-h$ for some
$h\in\mathcal{N}$, then $\pi\left(  x\right)  >\pi\left(  y\right)  $.
\end{enumerate}

Condition \textbf{(i)} shows that the map $f$ is weakly increasing. Condition
\textbf{(ii)} shows that for each $h\in\mathcal{N}$, the map $\pi\mid
_{f^{-1}\left(  +h\right)  }$ is strictly increasing. Condition \textbf{(iii)}
shows that for each $h\in\mathcal{N}$, the map $\pi\mid_{f^{-1}\left(
-h\right)  }$ is strictly decreasing.

Now, set $g=\left\vert f\right\vert $. Then, $g^{-1}\left(  0\right)
=f^{-1}\left(  +0\right)  $ (since $-0\notin\mathcal{Z}$). But the map
$\pi\mid_{f^{-1}\left(  +0\right)  }$ is strictly increasing\footnote{because
for each $h\in\mathcal{N}$, the map $\pi\mid_{f^{-1}\left(  +h\right)  }$ is
strictly increasing}. Thus, the map $\pi\mid_{g^{-1}\left(  0\right)  }$ is
strictly increasing (since $g^{-1}\left(  0\right)  =f^{-1}\left(  +0\right)
$). Hence, Condition \textbf{(i')} in the definition of \textquotedblleft$\pi
$-amenable\textquotedblright\ holds. Similarly, Condition \textbf{(iii')} in
that definition also holds.

Now, fix $h\in g\left(  \left[  n\right]  \right)  \cap\left\{  1,2,3,\ldots
\right\}  $. Then, the set $g^{-1}\left(  h\right)  $ is nonempty (since $h\in
g\left(  \left[  n\right]  \right)  $), and can be written as the union of its
two disjoint subsets $f^{-1}\left(  +h\right)  $ and $f^{-1}\left(  -h\right)
$. Furthermore, each element of $f^{-1}\left(  -h\right)  $ is smaller than
each element of $f^{-1}\left(  +h\right)  $ (since $f$ is weakly increasing),
and we know that the map $\pi\mid_{f^{-1}\left(  -h\right)  }$ is strictly
decreasing while the map $\pi\mid_{f^{-1}\left(  +h\right)  }$ is strictly
increasing. Hence, the map $\pi\mid_{g^{-1}\left(  h\right)  }$ is strictly
decreasing up until some value of its argument, and then strictly increasing
afterwards. In other words, the map $\pi\mid_{g^{-1}\left(  h\right)  }$ is
V-shaped. Thus, Condition \textbf{(ii')} in the definition of
\textquotedblleft$\pi$-amenable\textquotedblright\ holds. Finally, Condition
\textbf{(iv')} in the definition of \textquotedblleft$\pi$%
-amenable\textquotedblright\ holds because $f$ is weakly increasing. We have
hence checked all four conditions; thus, $g$ is $\pi$-amenable. This proves
Observation 1.]

[\textit{Proof of Observation 2:} Let $g:\left[  n\right]  \rightarrow
\mathcal{N}$ be a $\pi$-amenable map. Consider a map $f\in\mathcal{E}\left(
\left[  n\right]  ,\pi\right)  $ satisfying $\left\vert f\right\vert =g$. We
are wondering to what extent the map $f$ is determined by $g$ and $\pi$.

Everything that we said in the proof of Observation 1 is true.

In order to determine the map $f$, it clearly suffices to determine the sets
$f^{-1}\left(  q\right)  $ for all $q\in\mathcal{Z}$. In other words, it
suffices to determine the set $f^{-1}\left(  +0\right)  $, the set
$f^{-1}\left(  -\infty\right)  $ and the sets $f^{-1}\left(  +h\right)  $ and
$f^{-1}\left(  -h\right)  $ for all $h\in\left\{  1,2,3,\ldots\right\}  $.

Recall from the proof of Observation 1 that $g^{-1}\left(  0\right)
=f^{-1}\left(  +0\right)  $. Thus, $f^{-1}\left(  +0\right)  $ is uniquely
determined by $g$. Similarly, $f^{-1}\left(  -\infty\right)  $ is uniquely
determined by $g$. Thus, we can focus on the remaining sets $f^{-1}\left(
+h\right)  $ and $f^{-1}\left(  -h\right)  $ for $h\in\left\{  1,2,3,\ldots
\right\}  $.

Fix $h\in\left\{  1,2,3,\ldots\right\}  $. Recall that the set $g^{-1}\left(
h\right)  $ is the union of its two disjoint subsets $f^{-1}\left(  +h\right)
$ and $f^{-1}\left(  -h\right)  $. Thus, $f^{-1}\left(  +h\right)  $ and
$f^{-1}\left(  -h\right)  $ are complementary subsets of $g^{-1}\left(
h\right)  $. If $g^{-1}\left(  h\right)  =\varnothing$, then this uniquely
determines $f^{-1}\left(  +h\right)  $ and $f^{-1}\left(  -h\right)  $. Thus,
we focus only on the case when $g^{-1}\left(  h\right)  \neq\varnothing$.

So assume that $g^{-1}\left(  h\right)  \neq\varnothing$. Hence, $h\in
g\left(  \left[  n\right]  \right)  $, so that $h\in g\left(  \left[
n\right]  \right)  \cap\left\{  1,2,3,\ldots\right\}  $. Since the map $g$ is
$\pi$-amenable, we thus conclude that the map $\pi\mid_{g^{-1}\left(
h\right)  }$ is V-shaped (by Condition \textbf{(ii')} in the definition of
\textquotedblleft$\pi$-amenable\textquotedblright).

The map $g$ is weakly increasing (by Condition \textbf{(iv')} in the
definition of \textquotedblleft$\pi$-amenable\textquotedblright). Hence,
$g^{-1}\left(  h\right)  $ is an interval of $\left[  n\right]  $. Let
$\alpha\in\mathbb{Z}$ and $\gamma\in\mathbb{Z}$ be such that $g^{-1}\left(
h\right)  =\left[  \alpha,\gamma\right]  $ (where $\left[  \alpha
,\gamma\right]  $ means the interval $\left\{  \alpha,\alpha+1,\ldots
,\gamma\right\}  $).

As in the proof of Observation 1, we can see that each element of
$f^{-1}\left(  -h\right)  $ is smaller than each element of $f^{-1}\left(
+h\right)  $. Since the union of $f^{-1}\left(  -h\right)  $ and
$f^{-1}\left(  +h\right)  $ is $g^{-1}\left(  h\right)  =\left[  \alpha
,\gamma\right]  $, we thus conclude that there exists some $\beta\in\left[
\alpha-1,\gamma\right]  $ such that $f^{-1}\left(  -h\right)  =\left[
\alpha,\beta\right]  $ and $f^{-1}\left(  +h\right)  =\left[  \beta
+1,\gamma\right]  $. Consider this $\beta$. Clearly, $f^{-1}\left(  -h\right)
$ and $f^{-1}\left(  +h\right)  $ are uniquely determined by $\beta$; we just
need to find out which values $\beta$ can take.

As in the proof of Observation 1, we can see that the map $\pi\mid
_{f^{-1}\left(  -h\right)  }$ is strictly decreasing while the map $\pi
\mid_{f^{-1}\left(  +h\right)  }$ is strictly increasing. Let $k$ be the
element of $g^{-1}\left(  h\right)  $ minimizing $\pi\left(  k\right)  $.
Then, the map $\pi$ is strictly decreasing on the set $\left\{  u\in
g^{-1}\left(  h\right)  \ \mid\ u\leq k\right\}  $ and strictly increasing on
the set $\left\{  u\in g^{-1}\left(  h\right)  \ \mid\ u\geq k\right\}  $
(since the map $\pi\mid_{g^{-1}\left(  h\right)  }$ is V-shaped).

The map $\pi\mid_{f^{-1}\left(  -h\right)  }$ is strictly decreasing. In other
words, the map $\pi$ is strictly decreasing on the set $f^{-1}\left(
-h\right)  =\left[  \alpha,\beta\right]  $. On the other hand, the map $\pi$
is strictly increasing on the set $\left\{  u\in g^{-1}\left(  h\right)
\ \mid\ u\geq k\right\}  $. Hence, the two sets $\left[  \alpha,\beta\right]
$ and $\left\{  u\in g^{-1}\left(  h\right)  \ \mid\ u\geq k\right\}  $ cannot
have more than one point in common (since $\pi$ is strictly decreasing on one
and strictly increasing on the other). Thus, $k\geq\beta$. A similar argument
shows that $k\leq\beta+1$. Combining these inequalities, we obtain
$k\in\left\{  \beta,\beta+1\right\}  $, so that $\beta\in\left\{
k,k-1\right\}  $. This shows that $\beta$ can take only two values: $k$ and
$k-1$.

Now, let us take a bird's eye view. We have shown that for each $h\in g\left(
\left[  n\right]  \right)  \cap\left\{  1,2,3,\ldots\right\}  $, the sets
$f^{-1}\left(  +h\right)  $ and $f^{-1}\left(  -h\right)  $ are uniquely
determined once the integer $\beta$ is chosen, and that this integer $\beta$
can be chosen in two ways. (As we have seen, all other values of $h$ do not
matter.) Thus, in total, the map $f$ is uniquely determined up to $\left\vert
g\left(  \left[  n\right]  \right)  \cap\left\{  1,2,3,\ldots\right\}
\right\vert $ decisions, where each decision allows choosing from two values.
Thus, there are at most $2^{\left\vert g\left(  \left[  n\right]  \right)
\cap\left\{  1,2,3,\ldots\right\}  \right\vert }$ maps $f\in\mathcal{E}\left(
\left[  n\right]  ,\pi\right)  $ satisfying $\left\vert f\right\vert =g$.
Working the above argument backwards, we see that each possible decision
actually leads to a map $f\in\mathcal{E}\left(  \left[  n\right]  ,\pi\right)
$ satisfying $\left\vert f\right\vert =g$; thus, there are \textbf{exactly}
$2^{\left\vert g\left(  \left[  n\right]  \right)  \cap\left\{  1,2,3,\ldots
\right\}  \right\vert }$ maps $f\in\mathcal{E}\left(  \left[  n\right]
,\pi\right)  $ satisfying $\left\vert f\right\vert =g$. This proves
Observation 2.]
\end{proof}

Now, let us observe that if $g:\left[  n\right]  \rightarrow\mathcal{N}$ is a
weakly increasing map (for some $n\in\mathbb{N}$), then the fibers of $g$
(that is, the subsets $g^{-1}\left(  h\right)  $ of $\left[  n\right]  $ for
various $h\in\mathcal{N}$) are intervals of $\left[  n\right]  $ (possibly
empty). Of course, when these fibers are nonempty, they have smallest elements
and largest elements. We shall next study these elements more closely.

\begin{definition}
\label{def.fiberends}Let $n\in\mathbb{N}$. Let $g:\left[  n\right]
\rightarrow\mathcal{N}$ be any map. We define a subset $\operatorname*{FE}%
\left(  g\right)  $ of $\left[  n\right]  $ as follows:%
\begin{align*}
\operatorname*{FE}\left(  g\right)   &  =\left\{  \min\left(  g^{-1}\left(
h\right)  \right)  \ \mid\ h\in\left\{  1,2,3,\ldots,\infty\right\}  \text{
with }g^{-1}\left(  h\right)  \neq\varnothing\right\} \\
&  \ \ \ \ \ \ \ \ \ \ \cup\left\{  \max\left(  g^{-1}\left(  h\right)
\right)  \ \mid\ h\in\left\{  0,1,2,3,\ldots\right\}  \text{ with }%
g^{-1}\left(  h\right)  \neq\varnothing\right\}  .
\end{align*}
In other words, $\operatorname*{FE}\left(  g\right)  $ is the set comprising
the smallest elements of all nonempty fibers of $g$ except for $g^{-1}\left(
0\right)  $ as well as the largest elements of all nonempty fibers of $g$
except for $g^{-1}\left(  \infty\right)  $. We shall refer to the elements of
$\operatorname*{FE}\left(  g\right)  $ as the \textit{fiber-ends} of $g$.
\end{definition}

We can rewrite Proposition \ref{prop.Epk-formula} as follows, exhibiting its
analogy with \cite[Proposition 2.2]{Stembr97}:

\begin{proposition}
\label{prop.Epk-formula2}Let $n\in\mathbb{N}$. Let $\pi$ be any $n$%
-permutation. Then,%
\[
\Gamma_{\mathcal{Z}}\left(  \pi\right)  =\sum_{\substack{g:\left[  n\right]
\rightarrow\mathcal{N}\text{ is}\\\text{weakly increasing;}%
\\\operatorname*{Epk}\pi\subseteq\operatorname*{FE}\left(  g\right)
}}2^{\left\vert g\left(  \left[  n\right]  \right)  \cap\left\{
1,2,3,\ldots\right\}  \right\vert }\mathbf{x}_{g}.
\]

\end{proposition}

\begin{proof}
[Proof of Proposition \ref{prop.Epk-formula2}.]This will follow from
Proposition \ref{prop.Epk-formula} once we know that the $\pi$-amenable maps
$g:\left[  n\right]  \rightarrow\mathcal{N}$ are precisely the weakly
increasing maps $g:\left[  n\right]  \rightarrow\mathcal{N}$ satisfying
$\operatorname*{Epk}\pi\subseteq\operatorname*{FE}\left(  g\right)  $. But
this is easy to check. (The main idea: If $g:\left[  n\right]  \rightarrow
\mathcal{N}$ is a weakly increasing map, then all nonempty fibers
$g^{-1}\left(  h\right)  $ of $g$ are intervals, and Conditions \textbf{(i')},
\textbf{(ii')} and \textbf{(iii')} in the definition of \textquotedblleft$\pi
$-amenable\textquotedblright\ say precisely that no peak of $\pi$ can appear
in the interior of any such fiber; i.e., all peaks must appear at fiber-ends.
Of course, this needs some exceptions for $g^{-1}\left(  0\right)  $ and
$g^{-1}\left(  \infty\right)  $, but this is all straightforward.)
\end{proof}

\begin{definition}
\label{def.KnL}Let $n\in\mathbb{N}$. If $\Lambda$ is any subset of $\left[
n\right]  $, then we define a power series $K_{n,\Lambda}^{\mathcal{Z}}%
\in\operatorname*{Pow}\mathcal{N}$ by%
\begin{equation}
K_{n,\Lambda}^{\mathcal{Z}}=\sum_{\substack{g:\left[  n\right]  \rightarrow
\mathcal{N}\text{ is}\\\text{weakly increasing;}\\\Lambda\subseteq
\operatorname*{FE}\left(  g\right)  }}2^{\left\vert g\left(  \left[  n\right]
\right)  \cap\left\{  1,2,3,\ldots\right\}  \right\vert }\mathbf{x}_{g}.
\label{eq.def.KnL.1}%
\end{equation}
Thus, if $\Lambda=\operatorname*{Epk}\pi$ for some $n$-permutation $\pi$, then
Proposition \ref{prop.Epk-formula2} shows that%
\begin{equation}
\Gamma_{\mathcal{Z}}\left(  \pi\right)  =K_{n,\Lambda}^{\mathcal{Z}}.
\label{eq.def.KnL.2}%
\end{equation}

\end{definition}

Corollary \ref{cor.prod2} now leads directly to the following multiplication rule:

\begin{corollary}
\label{cor.KnEpk-prodrule}Let $n\in\mathbb{N}$ and $m\in\mathbb{N}$. Let $\pi$
be an $n$-permutation. Let $\sigma$ be an $m$-permutation. Then,%
\[
K_{n,\operatorname*{Epk}\pi}^{\mathcal{Z}}\cdot K_{m,\operatorname*{Epk}%
\sigma}^{\mathcal{Z}}=\sum_{\tau\in S\left(  \pi,\sigma\right)  }%
K_{n+m,\operatorname*{Epk}\tau}^{\mathcal{Z}}.
\]

\end{corollary}

\begin{proof}
[Proof of Corollary \ref{cor.KnEpk-prodrule}.]From (\ref{eq.def.KnL.2}), we
obtain $\Gamma_{\mathcal{Z}}\left(  \pi\right)  =K_{n,\Lambda}^{\mathcal{Z}}$.
Similarly, $\Gamma_{\mathcal{Z}}\left(  \sigma\right)
=K_{m,\operatorname*{Epk}\sigma}^{\mathcal{Z}}$. Multiplying these two
equalities, we obtain $\Gamma_{\mathcal{Z}}\left(  \pi\right)  \cdot
\Gamma_{\mathcal{Z}}\left(  \sigma\right)  =K_{n,\operatorname*{Epk}\pi
}^{\mathcal{Z}}\cdot K_{m,\operatorname*{Epk}\sigma}^{\mathcal{Z}}$. Hence,%
\begin{align*}
K_{n,\operatorname*{Epk}\pi}^{\mathcal{Z}}\cdot K_{m,\operatorname*{Epk}%
\sigma}^{\mathcal{Z}}  &  =\Gamma_{\mathcal{Z}}\left(  \pi\right)  \cdot
\Gamma_{\mathcal{Z}}\left(  \sigma\right)  =\sum_{\tau\in S\left(  \pi
,\sigma\right)  }\underbrace{\Gamma_{\mathcal{Z}}\left(  \tau\right)
}_{\substack{=K_{n+m,\operatorname*{Epk}\tau}^{\mathcal{Z}}\\\text{(by
(\ref{eq.def.KnL.2}))}}}\ \ \ \ \ \ \ \ \ \ \left(  \text{by Corollary
\ref{cor.prod2}}\right) \\
&  =\sum_{\tau\in S\left(  \pi,\sigma\right)  }K_{n+m,\operatorname*{Epk}\tau
}^{\mathcal{Z}}.
\end{align*}
This proves Corollary \ref{cor.KnEpk-prodrule}.
\end{proof}

\begin{definition}
A set $S$ of integers is said to be \textit{lacunar} if each $s\in S$
satisfies $s+1\notin S$.
\end{definition}

The following observation is easy:

\begin{proposition}
\label{prop.Epk-lac}Let $n$ be a positive integer. Let $\pi$ be an
$n$-permutation. Then, $\operatorname*{Epk}\pi$ is a lacunar and nonempty
subset of $\left[  n\right]  $.
\end{proposition}

\begin{proof}
[Proof of Proposition \ref{prop.Epk-lac}.]The set $\operatorname*{Epk}\pi$ is
lacunar (since two consecutive integers cannot both be exterior peaks of $\pi
$), and is also nonempty (since $\pi^{-1}\left(  n\right)  $ is an exterior
peak of $\pi$).

\begin{verlong}
[An alternative reason for the nonemptiness of $\operatorname*{Epk}\pi$ is the
following: If $\operatorname*{Epk}\pi$ was empty, then $\pi$ would have no
peaks, so that the sequence $\left(  \pi\left(  1\right)  ,\pi\left(
2\right)  ,\ldots,\pi\left(  n\right)  \right)  $ would be strictly decreasing
up to a certain point and then strictly increasing from there on; but then,
either $n$ or $1$ would be an exterior peak of $\pi$, which would contradict
the emptiness of $\operatorname*{Epk}\pi$.]
\end{verlong}

This proves Proposition \ref{prop.Epk-lac}.
\end{proof}

Proposition \ref{prop.Epk-lac} actually has a sort of converse:

\begin{proposition}
\label{prop.when-Epk}Let $n$ be a positive integer. Let $\Lambda$ be a subset
of $\left[  n\right]  $. Then, there exists an $n$-permutation $\pi$
satisfying $\Lambda=\operatorname*{Epk}\pi$ if and only if $\Lambda$ is
lacunar and nonempty.
\end{proposition}

\begin{vershort}


\begin{proof}
[Proof of Proposition \ref{prop.when-Epk}.]Omitted; see \cite{verlong} for a proof.
\end{proof}
\end{vershort}

\begin{verlong}


\begin{proof}
[Proof of Proposition \ref{prop.when-Epk}.]$\Longrightarrow:$ We need to prove
that for any $n$-permutation $\pi$, the set $\operatorname*{Epk}\pi$ is
lacunar and nonempty. But this follows immediately from Proposition
\ref{prop.Epk-lac}. This proves the $\Longrightarrow$ direction of Proposition
\ref{prop.when-Epk}.

$\Longleftarrow:$ Assume that $\Lambda$ is lacunar and nonempty. We must prove
that there exists an $n$-permutation $\pi$ satisfying $\Lambda
=\operatorname*{Epk}\pi$. Such an $n$-permutation $\pi$ can be constructed as follows:

\begin{itemize}
\item Write the set $\Lambda$ in the form $\Lambda=\left\{  u_{1}<u_{2}%
<\cdots<u_{\ell}\right\}  $ (where $\ell=\left\vert \Lambda\right\vert $).
Thus, $\ell\geq1$ (since $\Lambda$ is nonempty), and we can represent the set
$\left[  n\right]  \setminus\Lambda$ as a union of disjoint intervals as
follows:
\begin{align*}
&  \left[  n\right]  \setminus\Lambda\\
&  =\left[  1,u_{1}-1\right]  \cup\left[  u_{1}+1,u_{2}-1\right]  \cup\left[
u_{2}+1,u_{3}-1\right]  \cup\cdots\cup\left[  u_{\ell-1}+1,u_{\ell}-1\right]
\\
&  \ \ \ \ \ \ \ \ \ \ \cup\left[  u_{\ell}+1,n\right]  .
\end{align*}


\item Let $\pi$ take the values $n,n-1,\ldots,n-\ell+1$ on the elements of
$\Lambda$. (For example, this can be achieved by setting $\pi\left(
u_{i}\right)  =n+1-i$ for each $i\in\left[  \ell\right]  $.)

\item Let $\pi$ take the values $1,2,\ldots,n-\ell$ on the elements of
$\left[  n\right]  \setminus\Lambda$ in such a way that:

\begin{itemize}
\item[(A)] on each of the intervals $\left[  1,u_{1}-1\right]  ,\left[
u_{1}+1,u_{2}-1\right]  ,\left[  u_{2}+1,u_{3}-1\right]  ,\ldots,$%
\newline$\left[  u_{\ell-1}+1,u_{\ell}-1\right]  ,\left[  u_{\ell}+1,n\right]
$, the map $\pi$ is either strictly increasing or strictly decreasing;

\item[(B)] if the interval $\left[  1,u_{1}-1\right]  $ is nonempty, then the
map $\pi$ is strictly increasing on this interval;

\item[(C)] if the interval $\left[  u_{\ell}+1,n\right]  $ is nonempty, then
the map $\pi$ is strictly decreasing on this interval.
\end{itemize}

(This is indeed possible, because if the two intervals $\left[  1,u_{1}%
-1\right]  $ and $\left[  u_{\ell}+1,n\right]  $ are both nonempty, then they
are distinct (since $\ell\geq1$).)
\end{itemize}

Any $n$-permutation $\pi$ constructed in this way will satisfy $\Lambda
=\operatorname*{Epk}\pi$. Indeed, it is clear that $\pi$ satisfies%
\[
\pi\left(  u\right)  >\pi\left(  v\right)  \ \ \ \ \ \ \ \ \ \ \text{for all
}u\in\Lambda\text{ and }v\in\left[  n\right]  \setminus\Lambda.
\]
Hence, any element of $\Lambda$ is an exterior peak of $\pi$. Conversely, an
element of $\left[  n\right]  \setminus\Lambda$ cannot be an exterior peak of
$\pi$ (because our construction of $\pi$ guarantees that any $s\in\left[
n\right]  \setminus\Lambda$ satisfies either $\left(  s-1\in\left[  n\right]
\text{ and }\pi\left(  s-1\right)  >\pi\left(  s\right)  \right)  $ or
$\left(  s+1\in\left[  n\right]  \text{ and }\pi\left(  s+1\right)
>\pi\left(  s\right)  \right)  $). Thus, the exterior peaks of $\pi$ are
precisely the elements of $\Lambda$; in other words, we have $\Lambda
=\operatorname*{Epk}\pi$. This proves the $\Longleftarrow$ direction of
Proposition \ref{prop.when-Epk}.
\end{proof}
\end{verlong}

The following lemma is a variant of the principle of inclusion and exclusion
tailored to our setting:

\begin{lemma}
\label{lem.KnL.iex}Let $n\in\mathbb{N}$. For each subset $\Lambda$ of $\left[
n\right]  $, define a power series $L_{n,\Lambda}^{\mathcal{Z}}\in
\operatorname*{Pow}\mathcal{N}$ by%
\begin{equation}
L_{n,\Lambda}^{\mathcal{Z}}=\sum_{\substack{g:\left[  n\right]  \rightarrow
\mathcal{N}\text{ is}\\\text{weakly increasing;}\\\Lambda\cap
\operatorname*{FE}\left(  g\right)  =\varnothing}}2^{\left\vert g\left(
\left[  n\right]  \right)  \cap\left\{  1,2,3,\ldots\right\}  \right\vert
}\mathbf{x}_{g}. \label{eq.lem.KnL.iex.defL}%
\end{equation}
Then:

\textbf{(a)} For each subset $\Lambda$ of $\left[  n\right]  $, we have%
\[
K_{n,\Lambda}^{\mathcal{Z}}=\sum_{Q\subseteq\Lambda}\left(  -1\right)
^{\left\vert Q\right\vert }L_{n,Q}^{\mathcal{Z}}.
\]


\textbf{(b)} For each subset $\Lambda$ of $\left[  n\right]  $, we have%
\[
L_{n,\Lambda}^{\mathcal{Z}}=\sum_{Q\subseteq\Lambda}\left(  -1\right)
^{\left\vert Q\right\vert }K_{n,Q}^{\mathcal{Z}}.
\]

\end{lemma}

\begin{proof}
[Proof of Lemma \ref{lem.KnL.iex}.]Recall the following known fact: If $R$ is
a finite set, then%
\begin{equation}
\sum_{Q\subseteq R}\left(  -1\right)  ^{\left\vert Q\right\vert }=%
\begin{cases}
1, & \text{if }R=\varnothing;\\
0, & \text{if }R\neq\varnothing
\end{cases}
. \label{pf.lem.KnL.iex.1}%
\end{equation}


Let $\Lambda$ be a subset of $\left[  n\right]  $. The subsets $Q$ of
$\Lambda$ satisfying $Q\cap\operatorname*{FE}\left(  g\right)  =\varnothing$
are precisely the subsets of $\Lambda\setminus\operatorname*{FE}\left(
g\right)  $. Therefore,
\begin{align}
\sum_{\substack{Q\subseteq\Lambda;\\Q\cap\operatorname*{FE}\left(  g\right)
=\varnothing}}\left(  -1\right)  ^{\left\vert Q\right\vert }  &
=\sum_{Q\subseteq\Lambda\setminus\operatorname*{FE}\left(  G\right)  }\left(
-1\right)  ^{\left\vert Q\right\vert }=%
\begin{cases}
1, & \text{if }\Lambda\setminus\operatorname*{FE}\left(  G\right)
=\varnothing;\\
0, & \text{if }\Lambda\setminus\operatorname*{FE}\left(  G\right)
\neq\varnothing
\end{cases}
\ \ \ \ \ \ \ \ \ \ \left(  \text{by (\ref{pf.lem.KnL.iex.1})}\right)
\nonumber\\
&  =%
\begin{cases}
1, & \text{if }\Lambda\subseteq\operatorname*{FE}\left(  G\right)  ;\\
0, & \text{if }\Lambda\not \subseteq \operatorname*{FE}\left(  G\right)
\end{cases}
. \label{pf.lem.KnL.iex.2}%
\end{align}
Also, the subsets $Q$ of $\Lambda$ satisfying $Q\subseteq\operatorname*{FE}%
\left(  g\right)  $ are precisely the subsets of $\Lambda\cap
\operatorname*{FE}\left(  g\right)  $. Thus,%
\begin{equation}
\sum_{\substack{Q\subseteq\Lambda;\\Q\subseteq\operatorname*{FE}\left(
g\right)  }}\left(  -1\right)  ^{\left\vert Q\right\vert }=\sum_{Q\subseteq
\Lambda\cap\operatorname*{FE}\left(  G\right)  }\left(  -1\right)
^{\left\vert Q\right\vert }=%
\begin{cases}
1, & \text{if }\Lambda\cap\operatorname*{FE}\left(  G\right)  =\varnothing;\\
0, & \text{if }\Lambda\cap\operatorname*{FE}\left(  G\right)  \neq\varnothing
\end{cases}
\label{pf.lem.KnL.iex.3}%
\end{equation}
(by (\ref{pf.lem.KnL.iex.1})).

Now, forget that we fixed $\Lambda$. We thus have proven
(\ref{pf.lem.KnL.iex.2}) and (\ref{pf.lem.KnL.iex.3}) for each subset
$\Lambda$ of $\left[  n\right]  $.

\textbf{(a)} Let $\Lambda$ be a subset of $\left[  n\right]  $. Then,%
\begin{align*}
&  \sum_{Q\subseteq\Lambda}\left(  -1\right)  ^{\left\vert Q\right\vert
}\underbrace{L_{n,Q}^{\mathcal{Z}}}_{\substack{=\sum_{\substack{g:\left[
n\right]  \rightarrow\mathcal{N}\text{ is}\\\text{weakly increasing;}%
\\Q\cap\operatorname*{FE}\left(  g\right)  =\varnothing}}2^{\left\vert
g\left(  \left[  n\right]  \right)  \cap\left\{  1,2,3,\ldots\right\}
\right\vert }\mathbf{x}_{g}\\\text{(by the definition of }L_{n,Q}%
^{\mathcal{Z}}\text{)}}}\\
&  =\sum_{Q\subseteq\Lambda}\left(  -1\right)  ^{\left\vert Q\right\vert }%
\sum_{\substack{g:\left[  n\right]  \rightarrow\mathcal{N}\text{
is}\\\text{weakly increasing;}\\Q\cap\operatorname*{FE}\left(  g\right)
=\varnothing}}2^{\left\vert g\left(  \left[  n\right]  \right)  \cap\left\{
1,2,3,\ldots\right\}  \right\vert }\mathbf{x}_{g}\\
&  =\sum_{\substack{g:\left[  n\right]  \rightarrow\mathcal{N}\text{
is}\\\text{weakly increasing}}}\underbrace{\left(  \sum_{\substack{Q\subseteq
\Lambda;\\Q\cap\operatorname*{FE}\left(  g\right)  =\varnothing}}\left(
-1\right)  ^{\left\vert Q\right\vert }\right)  }_{\substack{=%
\begin{cases}
1, & \text{if }\Lambda\subseteq\operatorname*{FE}\left(  G\right)  ;\\
0, & \text{if }\Lambda\not \subseteq \operatorname*{FE}\left(  G\right)
\end{cases}
\\\text{(by (\ref{pf.lem.KnL.iex.2}))}}}2^{\left\vert g\left(  \left[
n\right]  \right)  \cap\left\{  1,2,3,\ldots\right\}  \right\vert }%
\mathbf{x}_{g}\\
&  =\sum_{\substack{g:\left[  n\right]  \rightarrow\mathcal{N}\text{
is}\\\text{weakly increasing}}}%
\begin{cases}
1, & \text{if }\Lambda\subseteq\operatorname*{FE}\left(  G\right)  ;\\
0, & \text{if }\Lambda\not \subseteq \operatorname*{FE}\left(  G\right)
\end{cases}
2^{\left\vert g\left(  \left[  n\right]  \right)  \cap\left\{  1,2,3,\ldots
\right\}  \right\vert }\mathbf{x}_{g}\\
&  =\sum_{\substack{g:\left[  n\right]  \rightarrow\mathcal{N}\text{
is}\\\text{weakly increasing;}\\\Lambda\subseteq\operatorname*{FE}\left(
g\right)  }}2^{\left\vert g\left(  \left[  n\right]  \right)  \cap\left\{
1,2,3,\ldots\right\}  \right\vert }\mathbf{x}_{g}=K_{n,\Lambda}^{\mathcal{Z}}%
\end{align*}
(by (\ref{eq.def.KnL.1})). This proves Lemma \ref{lem.KnL.iex} \textbf{(a)}.

\begin{vershort}
\textbf{(b)} The proof of Lemma \ref{lem.KnL.iex} \textbf{(b)} is analogous,
except that the roles of $K_{n,Q}^{\mathcal{Z}}$ and $K_{n,Q}^{\Lambda}$ are
interchanged and we need to use (\ref{pf.lem.KnL.iex.3}) instead of
(\ref{pf.lem.KnL.iex.2}).
\end{vershort}

\begin{verlong}
\textbf{(b)} Let $\Lambda$ be a subset of $\left[  n\right]  $. Then,%
\begin{align*}
&  \sum_{Q\subseteq\Lambda}\left(  -1\right)  ^{\left\vert Q\right\vert
}\underbrace{K_{n,Q}^{\mathcal{Z}}}_{\substack{=\sum_{\substack{g:\left[
n\right]  \rightarrow\mathcal{N}\text{ is}\\\text{weakly increasing;}%
\\Q\subseteq\operatorname*{FE}\left(  g\right)  }}2^{\left\vert g\left(
\left[  n\right]  \right)  \cap\left\{  1,2,3,\ldots\right\}  \right\vert
}\mathbf{x}_{g}\\\text{(by the definition of }K_{n,Q}^{\mathcal{Z}}\text{)}%
}}\\
&  =\sum_{Q\subseteq\Lambda}\left(  -1\right)  ^{\left\vert Q\right\vert }%
\sum_{\substack{g:\left[  n\right]  \rightarrow\mathcal{N}\text{
is}\\\text{weakly increasing;}\\Q\subseteq\operatorname*{FE}\left(  g\right)
}}2^{\left\vert g\left(  \left[  n\right]  \right)  \cap\left\{
1,2,3,\ldots\right\}  \right\vert }\mathbf{x}_{g}\\
&  =\sum_{\substack{g:\left[  n\right]  \rightarrow\mathcal{N}\text{
is}\\\text{weakly increasing}}}\underbrace{\left(  \sum_{\substack{Q\subseteq
\Lambda;\\Q\subseteq\operatorname*{FE}\left(  g\right)  }}\left(  -1\right)
^{\left\vert Q\right\vert }\right)  }_{\substack{=%
\begin{cases}
1, & \text{if }\Lambda\cap\operatorname*{FE}\left(  G\right)  =\varnothing;\\
0, & \text{if }\Lambda\cap\operatorname*{FE}\left(  G\right)  \neq\varnothing
\end{cases}
\\\text{(by (\ref{pf.lem.KnL.iex.3}))}}}2^{\left\vert g\left(  \left[
n\right]  \right)  \cap\left\{  1,2,3,\ldots\right\}  \right\vert }%
\mathbf{x}_{g}\\
&  =\sum_{\substack{g:\left[  n\right]  \rightarrow\mathcal{N}\text{
is}\\\text{weakly increasing}}}%
\begin{cases}
1, & \text{if }\Lambda\cap\operatorname*{FE}\left(  G\right)  =\varnothing;\\
0, & \text{if }\Lambda\cap\operatorname*{FE}\left(  G\right)  \neq\varnothing
\end{cases}
2^{\left\vert g\left(  \left[  n\right]  \right)  \cap\left\{  1,2,3,\ldots
\right\}  \right\vert }\mathbf{x}_{g}\\
&  =\sum_{\substack{g:\left[  n\right]  \rightarrow\mathcal{N}\text{
is}\\\text{weakly increasing;}\\\Lambda\cap\operatorname*{FE}\left(  g\right)
=\varnothing}}2^{\left\vert g\left(  \left[  n\right]  \right)  \cap\left\{
1,2,3,\ldots\right\}  \right\vert }\mathbf{x}_{g}=L_{n,\Lambda}^{\mathcal{Z}}%
\end{align*}
(by (\ref{eq.lem.KnL.iex.defL})). This proves Lemma \ref{lem.KnL.iex}
\textbf{(b)}.
\end{verlong}
\end{proof}

\begin{proposition}
\label{prop.KnL.lindep}Let $n\in\mathbb{N}$. Then, the family%
\[
\left(  K_{n,\Lambda}^{\mathcal{Z}}\right)  _{\Lambda\subseteq\left[
n\right]  \text{ is lacunar and nonempty}}%
\]
is $\mathbb{Q}$-linearly independent.
\end{proposition}

\begin{proof}
[Proof of Proposition \ref{prop.KnL.lindep}.]Let $\Omega$ be the subset
$\left\{  1,3,5,\ldots\right\}  \cap\left[  n\right]  =\left\{  i\in\left[
n\right]  \ \mid\ i\text{ is odd}\right\}  $ of $\left[  n\right]  $. This is
clearly a lacunar subset of $\left[  n\right]  $.

We are going to prove the following claim:

\begin{statement}
\textit{Claim 1:} \textbf{(a)} If $n$ is odd, then the only syzygy\footnote{If
$\left(  v_{h}\right)  _{h\in H}$ is a family of vectors in a vector space
over a field $\mathbb{F}$, then a \textit{syzygy} of this family $\left(
v_{h}\right)  _{h\in H}$ means a family $\left(  \lambda_{h}\right)  _{h\in
H}\in\mathbb{F}^{H}$ of scalars in $\mathbb{F}$ satisfying $\sum_{h\in
H}\lambda_{h}v_{h}=0$.
\par
Thus, a syzygy is what is commonly called a \textquotedblleft linear
dependence relation\textquotedblright\ (at least when the scalars $\lambda
_{h}$ are not all $0$). By abuse of notation, we shall speak of the
\textquotedblleft syzygy $\sum_{h\in H}\lambda_{h}v_{h}=0$\textquotedblright%
\ meaning not the equality $\sum_{h\in H}\lambda_{h}v_{h}=0$ but the family of
coefficients $\left(  \lambda_{h}\right)  _{h\in H}$.
\par
When we say \textquotedblleft the only syzygy\textquotedblright, we mean
\textquotedblleft the only nonzero syzygy up to scalar
multiples\textquotedblright.} of the family $\left(  K_{n,\Lambda
}^{\mathcal{Z}}\right)  _{\Lambda\subseteq\left[  n\right]  \text{ is
lacunar}}$ is $\sum_{\Lambda\subseteq\Omega}\left(  -1\right)  ^{\left\vert
\Lambda\right\vert }K_{n,\Lambda}^{\mathcal{Z}}=0$.

\textbf{(b)} If $n$ is even, then the family $\left(  K_{n,\Lambda
}^{\mathcal{Z}}\right)  _{\Lambda\subseteq\left[  n\right]  \text{ is
lacunar}}$ is $\mathbb{Q}$-linearly independent.
\end{statement}

For each subset $\Lambda$ of $\left[  n\right]  $, define a power series
$L_{n,\Lambda}^{\mathcal{Z}}\in\operatorname*{Pow}\mathcal{N}$ by
(\ref{eq.lem.KnL.iex.defL}). Then, for each lacunar subset $\Lambda$ of
$\left[  n\right]  $, we have%
\begin{align*}
K_{n,\Lambda}^{\mathcal{Z}}  &  =\sum_{Q\subseteq\Lambda}\left(  -1\right)
^{\left\vert Q\right\vert }L_{n,Q}^{\mathcal{Z}}\ \ \ \ \ \ \ \ \ \ \left(
\text{by Lemma \ref{lem.KnL.iex} \textbf{(a)}}\right)  \text{ and}\\
L_{n,\Lambda}^{\mathcal{Z}}  &  =\sum_{Q\subseteq\Lambda}\left(  -1\right)
^{\left\vert Q\right\vert }K_{n,Q}^{\mathcal{Z}}\ \ \ \ \ \ \ \ \ \ \left(
\text{by Lemma \ref{lem.KnL.iex} \textbf{(b)}}\right)  .
\end{align*}
Hence, the two families $\left(  K_{n,\Lambda}^{\mathcal{Z}}\right)
_{\Lambda\subseteq\left[  n\right]  \text{ is lacunar}}$ and $\left(
L_{n,\Lambda}^{\mathcal{Z}}\right)  _{\Lambda\subseteq\left[  n\right]  \text{
is lacunar}}$ can be obtained from each other by a unitriangular transition
matrix (unitriangular with respect to inclusion\footnote{We are using the fact
that a subset of a lacunar subset is lacunar.}). Thus, the syzygies of these
two families are in bijection with each other. Hence, in order to prove Claim
1, it suffices to prove the following claim:

\begin{statement}
\textit{Claim 2:} \textbf{(a)} If $n$ is odd, then the only syzygy of the
family $\left(  L_{n,\Lambda}^{\mathcal{Z}}\right)  _{\Lambda\subseteq\left[
n\right]  \text{ is lacunar}}$ is $L_{n,\Omega}^{\mathcal{Z}}=0$.

\textbf{(b)} If $n$ is even, then the family $\left(  L_{n,\Lambda
}^{\mathcal{Z}}\right)  _{\Lambda\subseteq\left[  n\right]  \text{ is
lacunar}}$ is $\mathbb{Q}$-linearly independent.
\end{statement}

Let $G$ be the set of all weakly increasing maps $g:\left[  n\right]
\rightarrow\mathcal{N}$. Let $R$ be the free $\mathbb{Q}$-vector space with
basis $G$; its standard basis will be denoted by $\left(  \left[  g\right]
\right)  _{g\in G}$. We define a $\mathbb{Q}$-linear map%
\begin{align*}
\Phi:R  &  \rightarrow\operatorname*{Pow}\mathcal{N},\\
\left[  g\right]   &  \mapsto2^{\left\vert g\left(  \left[  n\right]  \right)
\cap\left\{  1,2,3,\ldots\right\}  \right\vert }\mathbf{x}_{g}.
\end{align*}
This map $\Phi$ is easily seen to be injective (since the maps $g\in G$ are
weakly increasing, and thus can be uniquely recovered from the monomials
$\mathbf{x}_{g}$).

For each subset $\Lambda$ of $\left[  n\right]  $, we define an element
$\widetilde{L}_{\Lambda}$ of $R$ by%
\[
\widetilde{L}_{\Lambda}=\sum_{\substack{g\in G;\\\Lambda\cap\operatorname*{FE}%
\left(  g\right)  =\varnothing}}\left[  g\right]  .
\]
Then, each subset $\Lambda$ of $\left[  n\right]  $ satisfies%
\begin{align*}
L_{n,\Lambda}^{\mathcal{Z}}  &  =\underbrace{\sum_{\substack{g:\left[
n\right]  \rightarrow\mathcal{N}\text{ is}\\\text{weakly increasing;}%
\\\Lambda\cap\operatorname*{FE}\left(  g\right)  =\varnothing}}}%
_{\substack{=\sum_{\substack{g\in G;\\\Lambda\cap\operatorname*{FE}\left(
g\right)  =\varnothing}}\\\text{(by the definition of }G\text{)}%
}}\underbrace{2^{\left\vert g\left(  \left[  n\right]  \right)  \cap\left\{
1,2,3,\ldots\right\}  \right\vert }\mathbf{x}_{g}}_{\substack{=\Phi\left(
\left[  g\right]  \right)  \\\text{(by the definition of }\Phi\text{)}}%
}=\sum_{\substack{g\in G;\\\Lambda\cap\operatorname*{FE}\left(  g\right)
=\varnothing}}\Phi\left(  \left[  g\right]  \right) \\
&  =\Phi\left(  \underbrace{\sum_{\substack{g\in G;\\\Lambda\cap
\operatorname*{FE}\left(  g\right)  =\varnothing}}\left[  g\right]
}_{=\widetilde{L}_{\Lambda}}\right)  =\Phi\left(  \widetilde{L}_{\Lambda
}\right)  .
\end{align*}
Hence, the family $\left(  L_{n,\Lambda}^{\mathcal{Z}}\right)  _{\Lambda
\subseteq\left[  n\right]  \text{ is lacunar}}$ is the image of the family
$\left(  \widetilde{L}_{\Lambda}\right)  _{\Lambda\subseteq\left[  n\right]
\text{ is lacunar}}$ under the map $\Phi$. Thus, the syzygies of the two
families are in bijection (since the map $\Phi$ is injective\footnote{We are
here using the following obvious fact:
\par
Let $V$ and $W$ be two vector spaces over a field $\mathbb{F}$. Let $\left(
v_{h}\right)  _{h\in H}\in V^{H}$ be a family of vectors in $V$. Let
$\phi:V\rightarrow W$ be an injective $\mathbb{F}$-linear map. Then, the
syzygies of the families $\left(  v_{h}\right)  _{h\in H}\in V^{H}$ and
$\left(  f\left(  v_{h}\right)  \right)  _{h\in H}\in W^{H}$ are in bijection.
(Actually, these syzygies, when regarded as families of scalars, are literally
the same.) In particular, the family $\left(  v_{h}\right)  _{h\in H}$ is
$\mathbb{F}$-linearly independent if and only if the family $\left(  f\left(
v_{h}\right)  \right)  _{h\in H}$ is $\mathbb{F}$-linearly independent.}).
Hence, in order to prove Claim 2, it suffices to prove the following claim:

\begin{statement}
\textit{Claim 3:} \textbf{(a)} If $n$ is odd, then the only syzygy of the
family $\left(  \widetilde{L}_{\Lambda}\right)  _{\Lambda\subseteq\left[
n\right]  \text{ is lacunar}}$ is $\widetilde{L}_{\Omega}=0$.

\textbf{(b)} If $n$ is even, then the family $\left(  \widetilde{L}_{\Lambda
}\right)  _{\Lambda\subseteq\left[  n\right]  \text{ is lacunar}}$ is
$\mathbb{Q}$-linearly independent.
\end{statement}

Let us agree that if $g\in G$, then we will set $g\left(  0\right)  =0$ and
$g\left(  n+1\right)  =\infty$. Hence, $g\left(  i\right)  $ will be a
well-defined value of $\mathcal{N}$ for each $i\in\left\{  0,1,\ldots
,n+1\right\}  $.

If $g\in G$, then we let $\operatorname*{Stag}\left(  g\right)  $ be the
subset $\left\{  i\in\left[  n+1\right]  \ \mid\ g\left(  i\right)  =g\left(
i-1\right)  \right\}  $ of $\left[  n+1\right]  $. It is easy to see that the
family $\left(  \sum_{\substack{g\in G;\\\operatorname*{Stag}\left(  g\right)
=T}}\left[  g\right]  \right)  _{T\subseteq\left[  n+1\right]  ;\ T\neq\left[
n+1\right]  }$ of elements of $R$ is $\mathbb{Q}$-linearly
independent\footnote{\textit{Proof.} Clearly, any two elements of this family
are supported on different basis elements (i.e., any $\left[  g\right]  $
appearing in one of them cannot appear in any other). It thus remains to show
that these elements are $\neq0$. In other words, it remains to show that for
any proper subset $T$ of $\left[  n+1\right]  $, we have $\sum_{\substack{g\in
G;\\\operatorname*{Stag}\left(  g\right)  =T}}\left[  g\right]  \neq0$. But
this is easy: Just construct some $g\in G$ satisfying $\operatorname*{Stag}%
\left(  g\right)  =T$.}. Hence, the family $\left(  \sum_{\substack{g\in
G;\\\operatorname*{Stag}\left(  g\right)  \supseteq T}}\left[  g\right]
\right)  _{T\subseteq\left[  n+1\right]  ;\ T\neq\left[  n+1\right]  }$ of
elements of $R$ is $\mathbb{Q}$-linearly independent, too (because this family
is obtained from the previous family $\left(  \sum_{\substack{g\in
G;\\\operatorname*{Stag}\left(  g\right)  =T}}\left[  g\right]  \right)
_{T\subseteq\left[  n+1\right]  ;\ T\neq\left[  n+1\right]  }$ via a
unitriangular change-of-basis matrix\footnote{unitriangular with respect to
the reverse inclusion order (notice that $\sum_{\substack{g\in
G;\\\operatorname*{Stag}\left(  g\right)  =T}}\left[  g\right]  =0$ for
$T=\left[  n+1\right]  $, so the exclusion of $\left[  n+1\right]  $ makes
sense and does not mess up our computations)}). Therefore, the only syzygy of
the family \newline$\left(  \sum_{\substack{g\in G;\\\operatorname*{Stag}%
\left(  g\right)  \supseteq T}}\left[  g\right]  \right)  _{T\subseteq\left[
n+1\right]  }$ is $\sum_{\substack{g\in G;\\\operatorname*{Stag}\left(
g\right)  \supseteq\left[  n+1\right]  }}\left[  g\right]  =0$ (since it is
easy to see that no $g\in G$ satisfies $\operatorname*{Stag}\left(  g\right)
\supseteq\left[  n+1\right]  $, which is why $\sum_{\substack{g\in
G;\\\operatorname*{Stag}\left(  g\right)  \supseteq\left[  n+1\right]
}}\left[  g\right]  $ is indeed $0$).

But if $\Lambda$ is a lacunar subset of $\left[  n\right]  $, and if $g\in G$,
then we have the following logical equivalence:%
\begin{align*}
&  \ \left(  \Lambda\cap\operatorname*{FE}\left(  g\right)  =\varnothing
\right) \\
&  \Longleftrightarrow\ \left(  \text{no }i\in\Lambda\text{ satisfies }%
i\in\operatorname*{FE}\left(  g\right)  \right) \\
&  \Longleftrightarrow\ \left(  \text{each }i\in\Lambda\text{ satisfies
}i\notin\operatorname*{FE}\left(  g\right)  \right) \\
&  \Longleftrightarrow\ \left(  \text{each }i\in\Lambda\text{ satisfies
}\underbrace{i\neq\min\left(  g^{-1}\left(  h\right)  \right)  \text{ for all
}h\in\left\{  1,2,3,\ldots,\infty\right\}  }_{\substack{\Longleftrightarrow
\ \left(  g\left(  i\right)  =g\left(  i-1\right)  \right)  \\\text{(since
}g\text{ is weakly increasing, and }g\left(  0\right)  =0\text{)}}}\right. \\
&  \ \ \ \ \ \ \ \ \ \ \ \ \ \ \ \ \ \ \ \ \left.  \text{and }%
\underbrace{i\neq\max\left(  g^{-1}\left(  h\right)  \right)  \ \text{for
all}\ h\in\left\{  0,1,2,3,\ldots\right\}  }_{\substack{\Longleftrightarrow
\ \left(  g\left(  i\right)  =g\left(  i+1\right)  \right)  \\\text{(since
}g\text{ is weakly increasing, and }g\left(  n+1\right)  =\infty\text{)}%
}}\right) \\
&  \Longleftrightarrow\ \left(  \text{each }i\in\Lambda\text{ satisfies
}\underbrace{g\left(  i\right)  =g\left(  i-1\right)  }_{\Longleftrightarrow
\ \left(  i\in\operatorname*{Stag}\left(  g\right)  \right)  }\text{ and
}\underbrace{g\left(  i\right)  =g\left(  i+1\right)  }_{\Longleftrightarrow
\ \left(  i+1\in\operatorname*{Stag}\left(  g\right)  \right)  }\right) \\
&  \Longleftrightarrow\ \left(  \text{each }i\in\Lambda\text{ satisfies }%
i\in\operatorname*{Stag}\left(  g\right)  \text{ and }i+1\in
\operatorname*{Stag}\left(  g\right)  \right) \\
&  \Longleftrightarrow\ \left(  \Lambda\cup\left(  \Lambda+1\right)
\subseteq\operatorname*{Stag}\left(  g\right)  \right)  \ \Longleftrightarrow
\ \left(  \operatorname*{Stag}\left(  g\right)  \supseteq\Lambda\cup\left(
\Lambda+1\right)  \right)  .
\end{align*}
Hence, if $\Lambda$ is a lacunar subset of $\left[  n\right]  $, then the
condition $\Lambda\cap\operatorname*{FE}\left(  g\right)  =\varnothing$ is
equivalent to the condition $\operatorname*{Stag}\left(  g\right)
\supseteq\Lambda\cup\left(  \Lambda+1\right)  $. Thus, for each lacunar subset
$\Lambda$ of $\left[  n\right]  $, the definition of $\widetilde{L}_{\Lambda}$
becomes%
\[
\widetilde{L}_{\Lambda}=\sum_{\substack{g\in G;\\\Lambda\cap\operatorname*{FE}%
\left(  g\right)  =\varnothing}}\left[  g\right]  =\sum_{\substack{g\in
G;\\\operatorname*{Stag}\left(  g\right)  \supseteq\Lambda\cup\left(
\Lambda+1\right)  }}\left[  g\right]  .
\]
Hence, the family $\left(  \widetilde{L}_{\Lambda}\right)  _{\Lambda
\subseteq\left[  n\right]  \text{ is lacunar}}$ is a subfamily of the family
$\left(  \sum_{\substack{g\in G;\\\operatorname*{Stag}\left(  g\right)
\supseteq T}}\left[  g\right]  \right)  _{T\subseteq\left[  n+1\right]  }$
(because if $\Lambda$ is a lacunar subset of $\left[  n\right]  $, then
$\Lambda\cup\left(  \Lambda+1\right)  $ is a well-defined subset of $\left[
n+1\right]  $, and moreover $\Lambda$ can be uniquely recovered from
$\Lambda\cup\left(  \Lambda+1\right)  $\ \ \ \ \footnote{This takes a bit of
thought to check. You need to show that if $\Lambda_{1}$ and $\Lambda_{2}$ are
two lacunar subsets of $\left[  n\right]  $ satisfying $\Lambda_{1}\cup\left(
\Lambda_{1}+1\right)  =\Lambda_{2}\cup\left(  \Lambda_{2}+1\right)  $, then
$\Lambda_{1}=\Lambda_{2}$. In order to prove this, assume the contrary, and
conclude that there is a smallest element $h$ of the symmetric difference
$\Lambda_{1}\bigtriangleup\Lambda_{2}$. WLOG assume that $h\in\Lambda
_{1}\setminus\Lambda_{2}$, and argue that $h$ belongs to $\Lambda_{1}%
\cup\left(  \Lambda_{1}+1\right)  $ but not to $\Lambda_{2}\cup\left(
\Lambda_{2}+1\right)  $, which contradicts $\Lambda_{1}\cup\left(  \Lambda
_{1}+1\right)  =\Lambda_{2}\cup\left(  \Lambda_{2}+1\right)  $.}). The further
argument depends on the parity of $n$:

\begin{itemize}
\item If $n$ is odd, then the vanishing element $\sum_{\substack{g\in
G;\\\operatorname*{Stag}\left(  g\right)  \supseteq\left[  n+1\right]
}}\left[  g\right]  $ does appear in the family $\left(  \widetilde{L}%
_{\Lambda}\right)  _{\Lambda\subseteq\left[  n\right]  \text{ is lacunar}}$,
because there exists a lacunar subset $\Lambda$ of $\left[  n\right]  $
satisfying $\Lambda\cup\left(  \Lambda+1\right)  =\left[  n+1\right]  $:
Namely, this $\Lambda$ is $\Omega$. Thus, the only syzygy of the family
$\left(  \widetilde{L}_{\Lambda}\right)  _{\Lambda\subseteq\left[  n\right]
\text{ is lacunar}}$ is $\widetilde{L}_{\Omega}=0$ (since the only syzygy of
the family $\left(  \sum_{\substack{g\in G;\\\operatorname*{Stag}\left(
g\right)  \supseteq T}}\left[  g\right]  \right)  _{T\subseteq\left[
n+1\right]  }$ is $\sum_{\substack{g\in G;\\\operatorname*{Stag}\left(
g\right)  \supseteq\left[  n+1\right]  }}\left[  g\right]  =0$).

\item If $n$ is even, then the vanishing element $\sum_{\substack{g\in
G;\\\operatorname*{Stag}\left(  g\right)  \supseteq\left[  n+1\right]
}}\left[  g\right]  $ does not appear in the family $\left(  \widetilde{L}%
_{\Lambda}\right)  _{\Lambda\subseteq\left[  n\right]  \text{ is lacunar}}$,
since no lacunar subset $\Lambda$ of $\left[  n\right]  $ satisfies
$\Lambda\cup\left(  \Lambda+1\right)  =\left[  n+1\right]  $. Hence, the
syzygy $\sum_{\substack{g\in G;\\\operatorname*{Stag}\left(  g\right)
\supseteq\left[  n+1\right]  }}\left[  g\right]  =0$ of the family $\left(
\sum_{\substack{g\in G;\\\operatorname*{Stag}\left(  g\right)  \supseteq
T}}\left[  g\right]  \right)  _{T\subseteq\left[  n+1\right]  }$ disappears
when we pass to the subfamily $\left(  \widetilde{L}_{\Lambda}\right)
_{\Lambda\subseteq\left[  n\right]  \text{ is lacunar}}$. Consequently, the
subfamily $\left(  \widetilde{L}_{\Lambda}\right)  _{\Lambda\subseteq\left[
n\right]  \text{ is lacunar}}$ is $\mathbb{Q}$-linearly independent.
\end{itemize}

This proves Claim 3. As explained above, this yields Claim 2, hence also Claim
1, and thus completes the proof of Proposition \ref{prop.KnL.lindep}.
\end{proof}

\begin{corollary}
\label{cor.KnL.lindep-all}The family%
\[
\left(  K_{n,\Lambda}^{\mathcal{Z}}\right)  _{n>0;\ \Lambda\subseteq\left[
n\right]  \text{ is lacunar and nonempty}}\cup\left(  K_{0,\varnothing
}^{\mathcal{Z}}\right)
\]
(where $\left(  K_{0,\varnothing}^{\mathcal{Z}}\right)  $ is a $1$-element
family) is $\mathbb{Q}$-linearly independent.
\end{corollary}

\begin{vershort}
\begin{proof}
[Proof of Corollary \ref{cor.KnL.lindep-all}.]Follows from Proposition
\ref{prop.KnL.lindep} using gradedness; see \cite{verlong} for details.
\end{proof}
\end{vershort}

\begin{verlong}
\begin{proof}
[Proof of Corollary \ref{cor.KnL.lindep-all}.]For each $n\in\mathbb{N}$, the
family $\left(  K_{n,\Lambda}^{\mathcal{Z}}\right)  _{\Lambda\subseteq\left[
n\right]  \text{ is lacunar and nonempty}}$ is $\mathbb{Q}$-linearly
independent (by Proposition \ref{prop.KnL.lindep}). Furthermore, these
families for varying $n>0$ live in linearly disjoint subspaces of
$\operatorname*{Pow}\mathcal{N}$ (because for each $n>0$ and $\Lambda
\subseteq\left[  n\right]  $, the power series $K_{n,\Lambda}^{\mathcal{Z}}$
is homogeneous of degree $n$). Thus, the union $\left(  K_{n,\Lambda
}^{\mathcal{Z}}\right)  _{n>0;\ \Lambda\subseteq\left[  n\right]  \text{ is
lacunar and nonempty}}$ of all these families must also be $\mathbb{Q}%
$-linearly independent. Adding the $1$-element family $\left(
K_{0,\varnothing}^{\mathcal{Z}}\right)  $ to this union preserves the
$\mathbb{Q}$-linear independence, since it lives in yet another linearly
disjoint subspace of $\operatorname*{Pow}\mathcal{N}$ (namely, in its $0$-th
graded component). This proves Corollary \ref{cor.KnL.lindep-all}.
\end{proof}
\end{verlong}

We can now finally prove what we came here for:

\begin{theorem}
\label{thm.Epk.sh-co-a}The permutation statistic $\operatorname*{Epk}$ is shuffle-compatible.
\end{theorem}

\begin{proof}
[Proof of Theorem \ref{thm.Epk.sh-co-a}.]We must prove that
$\operatorname*{Epk}$ is shuffle-compatible. In other words, we must prove
that for any two disjoint permutations $\pi$ and $\sigma$, the multiset
$\left\{  \operatorname*{Epk}\left(  \tau\right)  \ \mid\ \tau\in S\left(
\pi,\sigma\right)  \right\}  _{\operatorname*{multi}}$ depends only on
$\operatorname*{Epk}\left(  \pi\right)  $, $\operatorname*{Epk}\left(
\sigma\right)  $, $\left\vert \pi\right\vert $ and $\left\vert \sigma
\right\vert $. In other words, we must prove that if $\pi$ and $\sigma$ are
two disjoint permutations, and if $\pi^{\prime}$ and $\sigma^{\prime}$ are two
disjoint permutations satisfying $\operatorname*{Epk}\left(  \pi\right)
=\operatorname*{Epk}\left(  \pi^{\prime}\right)  $, $\operatorname*{Epk}%
\left(  \sigma\right)  =\operatorname*{Epk}\left(  \sigma^{\prime}\right)  $,
$\left\vert \pi\right\vert =\left\vert \pi^{\prime}\right\vert $ and
$\left\vert \sigma\right\vert =\left\vert \sigma^{\prime}\right\vert $, then
the multiset $\left\{  \operatorname*{Epk}\left(  \tau\right)  \ \mid\ \tau\in
S\left(  \pi,\sigma\right)  \right\}  _{\operatorname*{multi}}$ equals the
multiset $\left\{  \operatorname*{Epk}\left(  \tau\right)  \ \mid\ \tau\in
S\left(  \pi^{\prime},\sigma^{\prime}\right)  \right\}
_{\operatorname*{multi}}$.

So let $\pi$ and $\sigma$ be two disjoint permutations, and let $\pi^{\prime}$
and $\sigma^{\prime}$ be two disjoint permutations satisfying
$\operatorname*{Epk}\left(  \pi\right)  =\operatorname*{Epk}\left(
\pi^{\prime}\right)  $, $\operatorname*{Epk}\left(  \sigma\right)
=\operatorname*{Epk}\left(  \sigma^{\prime}\right)  $, $\left\vert
\pi\right\vert =\left\vert \pi^{\prime}\right\vert $ and $\left\vert
\sigma\right\vert =\left\vert \sigma^{\prime}\right\vert $.

Define $n\in\mathbb{N}$ by $n=\left\vert \pi\right\vert =\left\vert
\pi^{\prime}\right\vert $ (this is well-defined, since $\left\vert
\pi\right\vert =\left\vert \pi^{\prime}\right\vert $). Likewise, define
$m\in\mathbb{N}$ by $m=\left\vert \sigma\right\vert =\left\vert \sigma
^{\prime}\right\vert $. Thus, $\pi$ is an $n$-permutation, while $\sigma$ is
an $m$-permutation.

WLOG assume that $n+m>0$ (since otherwise, both $n$ and $m$ are $0$, which
causes all four permutations $\pi$, $\pi^{\prime}$, $\sigma$ and
$\sigma^{\prime}$ to be empty; but then our goal is obviously satisfied).
Corollary \ref{cor.KnEpk-prodrule} yields
\begin{align*}
&  K_{n,\operatorname*{Epk}\left(  \pi\right)  }^{\mathcal{Z}}\cdot
K_{m,\operatorname*{Epk}\left(  \sigma\right)  }^{\mathcal{Z}}\\
&  =\sum_{\tau\in S\left(  \pi,\sigma\right)  }K_{n+m,\operatorname*{Epk}\tau
}^{\mathcal{Z}}=\sum_{\substack{\Lambda\subseteq\left[  n+m\right]  \text{ is
lacunar}\\\text{and nonempty}}}\underbrace{\sum_{\substack{\tau\in S\left(
\pi,\sigma\right)  ;\\\operatorname*{Epk}\tau=\Lambda}}K_{n+m,\Lambda
}^{\mathcal{Z}}}_{=\left\vert \left\{  \tau\in S\left(  \pi,\sigma\right)
\ \mid\ \operatorname*{Epk}\tau=\Lambda\right\}  \right\vert K_{n+m,\Lambda
}^{\mathcal{Z}}}\\
&  \ \ \ \ \ \ \ \ \ \ \left(
\begin{array}
[c]{c}%
\text{because if }\tau\in S\left(  \pi,\sigma\right)  \text{, then
}\operatorname*{Epk}\tau\text{ is a}\\
\text{lacunar and nonempty subset of }\left[  n+m\right] \\
\text{(by Proposition \ref{prop.Epk-lac}, applied to }n+m\text{ and }\tau\\
\text{instead of }n\text{ and }\pi\text{)}%
\end{array}
\right) \\
&  =\sum_{\substack{\Lambda\subseteq\left[  n+m\right]  \text{ is
lacunar}\\\text{and nonempty}}}\left\vert \left\{  \tau\in S\left(  \pi
,\sigma\right)  \ \mid\ \operatorname*{Epk}\tau=\Lambda\right\}  \right\vert
K_{n+m,\Lambda}^{\mathcal{Z}}.
\end{align*}
Similarly,%
\[
K_{n,\operatorname*{Epk}\left(  \pi^{\prime}\right)  }^{\mathcal{Z}}\cdot
K_{m,\operatorname*{Epk}\left(  \sigma^{\prime}\right)  }^{\mathcal{Z}}%
=\sum_{\substack{\Lambda\subseteq\left[  n+m\right]  \text{ is lacunar}%
\\\text{and nonempty}}}\left\vert \left\{  \tau\in S\left(  \pi^{\prime
},\sigma^{\prime}\right)  \ \mid\ \operatorname*{Epk}\tau=\Lambda\right\}
\right\vert K_{n+m,\Lambda}^{\mathcal{Z}}.
\]
The left-hand sides of these two equalities are identical (since
$\operatorname*{Epk}\left(  \pi\right)  =\operatorname*{Epk}\left(
\pi^{\prime}\right)  $ and $\operatorname*{Epk}\left(  \sigma\right)
=\operatorname*{Epk}\left(  \sigma^{\prime}\right)  $). Thus, their right-hand
sides must also be identical. In other words, we have%
\begin{align*}
&  \sum_{\substack{\Lambda\subseteq\left[  n+m\right]  \text{ is
lacunar}\\\text{and nonempty}}}\left\vert \left\{  \tau\in S\left(  \pi
,\sigma\right)  \ \mid\ \operatorname*{Epk}\tau=\Lambda\right\}  \right\vert
K_{n+m,\Lambda}^{\mathcal{Z}}\\
&  =\sum_{\substack{\Lambda\subseteq\left[  n+m\right]  \text{ is
lacunar}\\\text{and nonempty}}}\left\vert \left\{  \tau\in S\left(
\pi^{\prime},\sigma^{\prime}\right)  \ \mid\ \operatorname*{Epk}\tau
=\Lambda\right\}  \right\vert K_{n+m,\Lambda}^{\mathcal{Z}}.
\end{align*}
Since the family $\left(  K_{n+m,\Lambda}^{\mathcal{Z}}\right)  _{\Lambda
\subseteq\left[  n+m\right]  \text{ is lacunar and nonempty}}$ is $\mathbb{Q}%
$-linearly independent (by Proposition \ref{prop.KnL.lindep}), this shows that%
\[
\left\vert \left\{  \tau\in S\left(  \pi,\sigma\right)  \ \mid
\ \operatorname*{Epk}\tau=\Lambda\right\}  \right\vert =\left\vert \left\{
\tau\in S\left(  \pi^{\prime},\sigma^{\prime}\right)  \ \mid
\ \operatorname*{Epk}\tau=\Lambda\right\}  \right\vert
\]
for each lacunar nonempty subset $\Lambda$ of $\left[  n+m\right]  $. In other
words, the multiset $\left\{  \operatorname*{Epk}\left(  \tau\right)
\ \mid\ \tau\in S\left(  \pi,\sigma\right)  \right\}  _{\operatorname*{multi}%
}$ equals the multiset $\left\{  \operatorname*{Epk}\left(  \tau\right)
\ \mid\ \tau\in S\left(  \pi^{\prime},\sigma^{\prime}\right)  \right\}
_{\operatorname*{multi}}$ (because both of these multisets consist of lacunar
nonempty subsets $\Lambda$ of $\left[  n+m\right]  $, and the previous
sentence shows that these subsets appear with equal multiplicities in them).
This completes our proof of Theorem \ref{thm.Epk.sh-co-a}.
\end{proof}

We end this section with a final remark that uses some notations from
\cite{part1}:

\begin{remark}
\label{rmk.Epk.equivalents}The permutation statistics $\left(
\operatorname*{Lpk},\operatorname*{val}\right)  $, $\left(
\operatorname*{Lpk},\operatorname*{udr}\right)  $ and $\left(
\operatorname*{Pk},\operatorname*{udr}\right)  $ are equivalent to
$\operatorname*{Epk}$, and therefore are shuffle-compatible. (See \cite{part1}
for the definitions of all language left undefined here.)
\end{remark}

\begin{proof}
[Proof of Remark \ref{rmk.Epk.equivalents} (sketched).]If $\operatorname*{st}%
\nolimits_{1}$ and $\operatorname*{st}\nolimits_{2}$ are two permutation
statistics, then we shall write $\operatorname*{st}\nolimits_{1}%
\sim\operatorname*{st}\nolimits_{2}$ to mean \textquotedblleft%
$\operatorname*{st}\nolimits_{1}$ is equivalent to $\operatorname*{st}%
\nolimits_{2}$\textquotedblright.

The permutation statistic $\operatorname*{val}$ is equivalent to
$\operatorname*{epk}$, because of \cite[Lemma 2.1 \textbf{(e)}]{part1}. In
other words, $\operatorname*{val}\sim\operatorname*{epk}$. Hence, $\left(
\operatorname*{Lpk},\operatorname*{val}\right)  \sim\left(
\operatorname*{Lpk},\operatorname*{epk}\right)  $. But if $\pi$ is an
$n$-permutation, then $\operatorname*{Epk}\pi$ can be computed from the
knowledge of $\operatorname*{Lpk}\pi$ and $\operatorname*{epk}\pi$ (indeed,
$\operatorname*{Epk}\pi$ differs from $\operatorname*{Lpk}\pi$ only in the
possible element $n$, so that
\[
\operatorname*{Epk}\pi=%
\begin{cases}
\operatorname*{Lpk}\pi, & \text{if }\operatorname*{epk}\pi=\left\vert
\operatorname*{Lpk}\pi\right\vert ;\\
\operatorname*{Lpk}\pi\cup\left\{  n\right\}  , & \text{if }%
\operatorname*{epk}\pi\neq\left\vert \operatorname*{Lpk}\pi\right\vert
\end{cases}
\]
) and vice versa (since $\operatorname*{Lpk}\pi=\left(  \operatorname*{Epk}%
\pi\right)  \setminus\left\{  n\right\}  $ and $\operatorname*{epk}%
\pi=\left\vert \operatorname*{Epk}\pi\right\vert $). Thus, $\left(
\operatorname*{Lpk},\operatorname*{epk}\right)  \sim\operatorname*{Epk}$.
Hence, altogether, we obtain $\left(  \operatorname*{Lpk},\operatorname*{val}%
\right)  \sim\left(  \operatorname*{Lpk},\operatorname*{epk}\right)
\sim\operatorname*{Epk}$. In other words, $\left(  \operatorname*{Lpk}%
,\operatorname*{val}\right)  $ is equivalent to $\operatorname*{Epk}$.

Moreover, \cite[Theorem 3.2]{part1} shows that for any permutation $\pi$, the
knowledge of $\operatorname*{Lpk}\pi$ allows us to compute
$\operatorname*{udr}\pi$ from $\operatorname*{val}\pi$ and vice versa. Hence,
$\left(  \operatorname*{Lpk},\operatorname*{udr}\right)  \sim\left(
\operatorname*{Lpk},\operatorname*{val}\right)  \sim\operatorname*{Epk}$. In
other words, $\left(  \operatorname*{Lpk},\operatorname*{udr}\right)  $ is
equivalent to $\operatorname*{Epk}$.

On the other hand, $\left(  \operatorname*{Pk},\operatorname*{lpk}\right)
\sim\operatorname*{Lpk}$. (This is proven similarly to our proof of $\left(
\operatorname*{Lpk},\operatorname*{epk}\right)  \sim\operatorname*{Epk}$.)

Also, $\operatorname*{udr}\sim\left(  \operatorname*{lpk},\operatorname*{val}%
\right)  $ (indeed, \cite[Lemma 2.2 \textbf{(b)} and \textbf{(c)}]{part1} show
how the value $\left(  \operatorname*{lpk},\operatorname*{val}\right)  \left(
\pi\right)  $ can be computed from $\operatorname*{udr}\pi$, whereas
\cite[Lemma 2.2 \textbf{(a)}]{part1} shows the opposite direction). Hence,
$\left(  \operatorname*{Pk},\operatorname*{udr}\right)  \sim\left(
\operatorname*{Pk},\operatorname*{lpk},\operatorname*{val}\right)  \sim\left(
\operatorname*{Lpk},\operatorname*{val}\right)  $ (since $\left(
\operatorname*{Pk},\operatorname*{lpk}\right)  \sim\operatorname*{Lpk}$).
Therefore, $\left(  \operatorname*{Pk},\operatorname*{udr}\right)  \sim\left(
\operatorname*{Lpk},\operatorname*{val}\right)  \sim\operatorname*{Epk}$. In
other words, $\left(  \operatorname*{Pk},\operatorname*{udr}\right)  $ is
equivalent to $\operatorname*{Epk}$.

We have now shown that the statistics $\left(  \operatorname*{Lpk}%
,\operatorname*{val}\right)  $, $\left(  \operatorname*{Lpk}%
,\operatorname*{udr}\right)  $ and $\left(  \operatorname*{Pk}%
,\operatorname*{udr}\right)  $ are equivalent to $\operatorname*{Epk}$. Thus,
\cite[Theorem 3.2]{part1} shows that they are shuffle-compatible (since
$\operatorname*{Epk}$ is shuffle-compatible). This proves Remark
\ref{rmk.Epk.equivalents}.
\end{proof}

\begin{question}
Our concept of a \textquotedblleft$\mathcal{Z}$-enriched $\left(
P,\gamma\right)  $-partition\textquotedblright\ generalizes the concept of an
\textquotedblleft enriched $\left(  P,\gamma\right)  $%
-partition\textquotedblright\ by restricting ourselves to a subset
$\mathcal{Z}$ of $\mathcal{N}\times\left\{  +,-\right\}  $. (This does not
sound like much of a generalization when stated like this, but as we have seen
the behavior of the power series $\Gamma_{\mathcal{Z}}\left(  P,\gamma\right)
$ depends strongly on what $\mathcal{Z}$ is, and is not all anticipated by the
$\mathcal{Z}=\mathcal{N}\times\left\{  +,-\right\}  $ case.) A different
generalization of enriched $\left(  P,\gamma\right)  $-partitions (introduced
by Hsiao and Petersen in \cite{HsiPet10}) are the \textit{colored }$\left(
P,\gamma\right)  $\textit{-partitions}, where the two-element set $\left\{
+,-\right\}  $ is replaced by the set $\left\{  1,\omega,\ldots,\omega
^{m-1}\right\}  $ of all $m$-th roots of unity (where $m$ is a chosen positive
integer, and $\omega$ is a fixed primitive $m$-th root of unity). We can play
various games with this concept. The most natural thing to do seems to be to
consider $m$ arbitrary total orders $<_{0},<_{1},\ldots,<_{m-1}$ on the
codomain $A$ of the labeling $\gamma$ (perhaps with some nice properties such
as all intervals being finite) and an arbitrary subset $\mathcal{Z}$ of
$\mathcal{N}\times\left\{  1,\omega,\ldots,\omega^{m-1}\right\}  $, and define
a $\mathcal{Z}$-enriched colored $\left(  P,\gamma\right)  $-partition to be a
map $f:P\rightarrow\mathcal{Z}$ such that every $x<y$ in $P$ satisfy the
following conditions:

\begin{enumerate}
\item[\textbf{(i)}] We have $f\left(  x\right)  \preccurlyeq f\left(
y\right)  $. (Here, the total order on $\mathcal{N}\times\left\{
1,\omega,\ldots,\omega^{m-1}\right\}  $ is defined by%
\[
\left(  n,\omega^{i}\right)  \prec\left(  n^{\prime},\omega^{i^{\prime}%
}\right)  \text{ if and only if either }n\prec n^{\prime}\text{ or }\left(
n=n^{\prime}\text{ and }i<i^{\prime}\right)
\]
(for $i,i^{\prime}\in\left\{  0,1,\ldots,m-1\right\}  $).)

\item[\textbf{(ii)}] If $f\left(  x\right)  =f\left(  y\right)  =\left(
n,\omega^{i}\right)  $ for some $n\in\mathcal{N}$ and $i\in\left\{
0,1,\ldots,m-1\right\}  $, then $\gamma\left(  x\right)  <_{i}\gamma\left(
y\right)  $.
\end{enumerate}

Is this a useful concept, and can it be used to study permutation statistics?
\end{question}

\section{\label{sect.LR}LR-shuffle-compatibility}

In this section, we shall introduce the concept of \textquotedblleft
LR-shuffle-compatibility\textquotedblright\ (short for \textquotedblleft
left-and-right-shuffle-compatibility\textquotedblright), which is stronger
than usual shuffle-compatibility. We shall prove that $\operatorname*{Epk}$
still is LR-shuffle-compatible, and study some other statistics that are and
some that are not.

\subsection{Left and right shuffles}

We begin by introducing \textquotedblleft left shuffles\textquotedblright\ and
\textquotedblleft right shuffles\textquotedblright. There is a well-known
notion of left and right shuffles of words (see, e.g., the operations $\prec$
and $\succ$ in \cite[Example 1]{EbMaPa07}). Specialized to permutations, it
can be defined in the following simple way:

\begin{definition}
\label{def.LRshuf}Let $\pi$ and $\sigma$ be two disjoint permutations. Then:

\begin{itemize}
\item A \textit{left shuffle} of $\pi$ and $\sigma$ means a shuffle $\tau$ of
$\pi$ and $\sigma$ such that the first letter of $\tau$ is the first letter of
$\pi$. (This makes sense only when $\pi$ is nonempty. Otherwise, there are no
left shuffles of $\pi$ and $\sigma$.)

\item A \textit{right shuffle} of $\pi$ and $\sigma$ means a shuffle $\tau$ of
$\pi$ and $\sigma$ such that the first letter of $\tau$ is the first letter of
$\sigma$. (This makes sense only when $\sigma$ is nonempty. Otherwise, there
are no right shuffles of $\pi$ and $\sigma$.)

\item We let $S_{\prec}\left(  \pi,\sigma\right)  $ denote the set of all left
shuffles of $\pi$ and $\sigma$.

\item We let $S_{\succ}\left(  \pi,\sigma\right)  $ denote the set of all
right shuffles of $\pi$ and $\sigma$.
\end{itemize}
\end{definition}

For example, the left shuffles of the two disjoint permutations $\left(
3,1\right)  $ and $\left(  2,6\right)  $ are%
\[
\left(  3,1,2,6\right)  ,\ \ \ \ \ \ \ \ \ \ \left(  3,2,1,6\right)
,\ \ \ \ \ \ \ \ \ \ \left(  3,2,6,1\right)  ,
\]
whereas their right shuffles are%
\[
\left(  2,3,1,6\right)  ,\ \ \ \ \ \ \ \ \ \ \left(  2,3,6,1\right)
,\ \ \ \ \ \ \ \ \ \ \left(  2,6,3,1\right)  .
\]
The permutations $\left(  {}\right)  $ and $\left(  1,3\right)  $ have only
one right shuffle, which is $\left(  1,3\right)  $, and they have no left shuffles.

Clearly, if $\pi$ and $\sigma$ are two disjoint permutations such that at
least one of $\pi$ and $\sigma$ is nonempty, then the two sets $S_{\prec
}\left(  \pi,\sigma\right)  $ and $S_{\succ}\left(  \pi,\sigma\right)  $ are
disjoint and their union is $S\left(  \pi,\sigma\right)  $ (because every
shuffle of $\pi$ and $\sigma$ is either a left shuffle or a right shuffle, but
not both).

Left and right shuffles have a recursive structure that makes them amenable to
inductive arguments. To state it, we need one more definition:

\begin{definition}
\label{def.LR.pi1}Let $n\in\mathbb{N}$. Let $\pi$ be an $n$-permutation.

\textbf{(a)} For each $i\in\left\{  1,2,\ldots,n\right\}  $, we let $\pi_{i}$
denote the $i$-th entry of $\pi$. Thus, $\pi=\left(  \pi_{1},\pi_{2}%
,\ldots,\pi_{n}\right)  $.

\textbf{(b)} If $a$ is a positive integer that does not appear in $\pi$, then
$a:\pi$ denotes the $\left(  n+1\right)  $-permutation $\left(  a,\pi_{1}%
,\pi_{2},\ldots,\pi_{n}\right)  $.

\textbf{(c)} If $n>0$, then $\pi_{\sim1}$ denotes the $\left(  n-1\right)
$-permutation $\left(  \pi_{2},\pi_{3},\ldots,\pi_{n}\right)  $.
\end{definition}

\begin{proposition}
\label{prop.LR.rec}Let $\pi$ and $\sigma$ be two disjoint permutations.

\textbf{(a)} We have $S_{\prec}\left(  \pi,\sigma\right)  =S_{\succ}\left(
\sigma,\pi\right)  $.

\textbf{(b)} If $\pi$ is nonempty, then the permutations $\pi_{\sim1}$ and
$\pi_{1}:\sigma$ are well-defined and disjoint, and satisfy $S_{\prec}\left(
\pi,\sigma\right)  =S_{\succ}\left(  \pi_{\sim1},\pi_{1}:\sigma\right)  $.

\textbf{(c)} If $\sigma$ is nonempty, then the permutations $\sigma_{\sim1}$
and $\sigma_{1}:\pi$ are well-defined and disjoint, and satisfy $S_{\succ
}\left(  \pi,\sigma\right)  =S_{\prec}\left(  \sigma_{1}:\pi,\sigma_{\sim
1}\right)  $.
\end{proposition}

\begin{vershort}
\begin{proof}
[Proof of Proposition \ref{prop.LR.rec}.]The fairly simple proof is left to
the reader, who can also find it in \cite{verlong}.
\end{proof}
\end{vershort}

\begin{verlong}
\begin{proof}
[Proof of Proposition \ref{prop.LR.rec}.]\textbf{(a)} The definition of left
shuffles shows that the left shuffles of $\pi$ and $\sigma$ are the shuffles
$\tau$ of $\pi$ and $\sigma$ such that the first letter of $\tau$ is the first
letter of $\pi$. Meanwhile, the definition of right shuffles shows that the
right shuffles of $\sigma$ and $\pi$ are the shuffles $\tau$ of $\sigma$ and
$\pi$ such that the first letter of $\tau$ is the first letter of $\pi$.
Comparing these two descriptions, we conclude that the left shuffles of $\pi$
and $\sigma$ are the same as the right shuffles of $\sigma$ and $\pi$ (since
the shuffles of $\pi$ and $\sigma$ are the same as the shuffles of $\sigma$
and $\pi$). In other words, $S_{\prec}\left(  \pi,\sigma\right)  =S_{\succ
}\left(  \sigma,\pi\right)  $. This proves Proposition \ref{prop.LR.rec}
\textbf{(a)}.

\textbf{(b)} We first make a simple observation:

\begin{statement}
\textit{Claim 1:} Let $\alpha$ and $\beta$ be two permutations such that
$\beta$ is nonempty. Assume that $\alpha$ is a subsequence of $\beta$,
but does not contain the letter $\beta_{1}$. Then, the permutation $\beta
_{1}:\alpha$ also is a subsequence of $\beta$.
\end{statement}

[\textit{Proof of Claim 1:} The letter $\beta_{1}$ does not appear in $\alpha$
(since $\alpha$ does not contain $\beta_{1}$). Thus, $\beta_{1}:\alpha$ is a
well-defined permutation.

We have assumed that $\alpha$ is a subsequence of $\beta$. In other
words, $\alpha=\left(  \beta_{i_{1}},\beta_{i_{2}},\ldots,\beta_{i_{k}%
}\right)  $ for some $k\in\mathbb{N}$ and some positive integers $i_{1}%
,i_{2},\ldots,i_{k}$ satisfying $i_{1}<i_{2}<\cdots<i_{k}$. Consider these
$i_{1},i_{2},\ldots,i_{k}$. From $\alpha=\left(  \beta_{i_{1}},\beta_{i_{2}%
},\ldots,\beta_{i_{k}}\right)  $, we obtain $\beta_{1}:\alpha=\left(
\beta_{1},\beta_{i_{1}},\beta_{i_{2}},\ldots,\beta_{i_{k}}\right)  $.

Let $g\in\left\{  1,2,\ldots,k\right\}  $. Then, $\alpha=\left(  \beta_{i_{1}%
},\beta_{i_{2}},\ldots,\beta_{i_{k}}\right)  $ clearly contains the letter
$\beta_{i_{g}}$. If we had $i_{g}=1$, then this would yield that $\alpha$
contains the letter $\beta_{1}$ (since $i_{g}=1$); but this would contradict
the assumption that $\alpha$ does not contain the letter $\beta_{1}$. Hence,
we cannot have $i_{g}=1$. Thus, $i_{g}>1$, so that $1<i_{g}$.

Now, forget that we fixed $g$. We thus have shown that $1<i_{g}$ for each
$g\in\left\{  1,2,\ldots,k\right\}  $. Combining this with $i_{1}<i_{2}%
<\cdots<i_{k}$, we obtain $1<i_{1}<i_{2}<\cdots<i_{k}$. Hence, $\left(
\beta_{1},\beta_{i_{1}},\beta_{i_{2}},\ldots,\beta_{i_{k}}\right)  $ is a
subsequence of $\beta$. In view of $\beta_{1}:\alpha=\left(  \beta_{1}%
,\beta_{i_{1}},\beta_{i_{2}},\ldots,\beta_{i_{k}}\right)  $, this rewrites as
follows: $\beta_{1}:\alpha$ is a subsequence of $\beta$. This proves Claim 1.]

Assume that $\pi$ is nonempty. The first letter of $\pi$ does not appear in
$\sigma$ (since $\pi$ and $\sigma$ are disjoint). In other words, the letter
$\pi_{1}$ does not appear in $\sigma$. Thus, the permutation $\pi_{1}:\sigma$
is well-defined. The permutation $\pi_{\sim1}$ is clearly well-defined.
Furthermore, the permutations $\pi_{\sim1}$ and $\pi_{1}:\sigma$ are
disjoint\footnote{\textit{Proof.} Let $\ell$ be a letter that appears in both
$\pi_{\sim1}$ and $\pi_{1}:\sigma$. We shall derive a contradiction.
\par
The permutations $\pi$ and $\sigma$ are disjoint; thus, no letter appears in
both $\pi$ and $\sigma$. In other words, a letter that appears in $\pi$ cannot
appear in $\sigma$. But the letter $\ell$ appears in $\pi_{\sim1}$, thus in
$\pi$ (since $\pi_{\sim1}$ is a subsequence of $\pi$). Hence, $\ell$ does not
appear in $\sigma$ (since a letter that appears in $\pi$ cannot appear in
$\sigma$). But $\ell$ appears in $\pi_{1}:\sigma$. But the only letter that
appears in $\pi_{1}:\sigma$ but not in $\sigma$ is the letter $\pi_{1}$ (due
to the construction of $\pi_{1}:\sigma$). Thus, the letter $\ell$ must be
$\pi_{1}$ (since $\ell$ is a letter that appears in $\pi_{1}:\sigma$ but not
in $\sigma$). In other words, $\ell=\pi_{1}$. Hence, the letter $\pi_{1}$
appears in $\pi_{\sim1}$ (since the letter $\ell$ appears in $\pi_{\sim1}$).
\par
But $\pi$ is a permutation, and thus the letters of $\pi$ are distinct. Hence,
the letter $\pi_{1}$ does not appear in $\pi_{\sim1}$. This contradicts the
fact that the letter $\pi_{1}$ appears in $\pi_{\sim1}$.
\par
Now, forget that we fixed $\ell$. We thus have found a contradiction for each
letter $\ell$ that appears in both $\pi_{\sim1}$ and $\pi_{1}:\sigma$. Hence,
no letter appears in both $\pi_{\sim1}$ and $\pi_{1}:\sigma$. In other words,
the permutations $\pi_{\sim1}$ and $\pi_{1}:\sigma$ are disjoint.}. It thus
remains to show that $S_{\prec}\left(  \pi,\sigma\right)  =S_{\succ}\left(
\pi_{\sim1},\pi_{1}:\sigma\right)  $.

Set $m=\left\vert \pi\right\vert $ and $n=\left\vert \sigma\right\vert $.
Thus, $\pi_{\sim1}$ is an $\left(  m-1\right)  $-permutation, whereas $\pi
_{1}:\sigma$ is an $\left(  n+1\right)  $-permutation.

Now, we shall prove that%
\begin{equation}
S_{\prec}\left(  \pi,\sigma\right)  \subseteq S_{\succ}\left(  \pi_{\sim1}%
,\pi_{1}:\sigma\right)  \label{pf.prop.LR.rec.b.1}%
\end{equation}
and%
\begin{equation}
S_{\succ}\left(  \pi_{\sim1},\pi_{1}:\sigma\right)  \subseteq S_{\prec}\left(
\pi,\sigma\right)  . \label{pf.prop.LR.rec.b.2}%
\end{equation}


[\textit{Proof of (\ref{pf.prop.LR.rec.b.1}):} Let $\tau\in S_{\prec}\left(
\pi,\sigma\right)  $. We shall show that $\tau\in S_{\succ}\left(  \pi_{\sim
1},\pi_{1}:\sigma\right)  $.

We have $\tau\in S_{\prec}\left(  \pi,\sigma\right)  $. In other words, $\tau$
is a left shuffle of $\pi$ and $\sigma$ (by the definition of $S_{\prec
}\left(  \pi,\sigma\right)  $). In other words, $\tau$ is a shuffle of $\pi$
and $\sigma$ such that the first letter of $\tau$ is the first letter of $\pi$
(by the definition of a left shuffle).

So the first letter of $\tau$ is the first letter of $\pi$. In other words,
$\tau_{1}=\pi_{1}$. Thus, the permutation $\tau$ is nonempty.

Also, $\tau$ is a shuffle of $\pi$ and $\sigma$. In other words, $\tau$ is an
$\left(  m+n\right)  $-permutation such that both $\pi$ and $\sigma$ are
subsequences of $\tau$ (by the definition of a shuffle).

So $\pi$ is a subsequence of $\tau$. Thus, $\pi_{\sim1}$ is a
subsequence of $\tau$ as well (since $\pi_{\sim1}$ is a subsequence of $\pi$).

Also, $\sigma$ is a subsequence of $\tau$, but does not contain the
letter $\tau_{1}$ (because $\tau_{1}=\pi_{1}$, which does not appear in
$\sigma$). Therefore, Claim 1 (applied to $\alpha=\sigma$ and $\beta=\tau$)
yields that $\tau_{1}:\sigma$ also is a subsequence of $\tau$. In
other words, $\pi_{1}:\sigma$ is a subsequence of $\tau$ (since
$\tau_{1}=\pi_{1}$).

Finally, recall that $\pi_{\sim1}$ is an $\left(  m-1\right)  $-permutation,
whereas $\pi_{1}:\sigma$ is an $\left(  n+1\right)  $-permutation. But $\tau$
is an $\left(  m+n\right)  $-permutation, hence an $\left(  \left(
m-1\right)  +\left(  n+1\right)  \right)  $-permutation (since $m+n=\left(
m-1\right)  +\left(  n+1\right)  $). So we know that $\tau$ is an
\newline$\left(  \left(  m-1\right)  +\left(  n+1\right)  \right)
$-permutation such that both $\pi_{\sim1}$ and $\pi_{1}:\sigma$ are
subsequences of $\tau$. In other words, $\tau$ is a shuffle of $\pi_{\sim1}$
and $\pi_{1}:\sigma$ (by the definition of a shuffle). Since the first letter
of $\tau$ is the first letter of $\pi_{1}:\sigma$ (because the first letter of
$\tau$ is $\tau_{1}=\pi_{1}$, whereas the first letter of $\pi_{1}:\sigma$ is
$\pi_{1}$ as well), we thus conclude that $\tau$ is a right shuffle of
$\pi_{\sim1}$ and $\pi_{1}:\sigma$ (by the definition of a right shuffle). In
other words, $\tau\in S_{\succ}\left(  \pi_{\sim1},\pi_{1}:\sigma\right)  $
(by the definition of $S_{\succ}\left(  \pi_{\sim1},\pi_{1}:\sigma\right)  $).

Since we have proven this for all $\tau\in S_{\prec}\left(  \pi,\sigma\right)
$, we thus conclude that $S_{\prec}\left(  \pi,\sigma\right)  \subseteq
S_{\succ}\left(  \pi_{\sim1},\pi_{1}:\sigma\right)  $. This proves
(\ref{pf.prop.LR.rec.b.1}).]

[\textit{Proof of (\ref{pf.prop.LR.rec.b.2}):} Let $\tau\in S_{\succ}\left(
\pi_{\sim1},\pi_{1}:\sigma\right)  $. We shall show that $\tau\in S_{\prec
}\left(  \pi,\sigma\right)  $.

We have $\tau\in S_{\succ}\left(  \pi_{\sim1},\pi_{1}:\sigma\right)  $. In
other words, $\tau$ is a right shuffle of $\pi_{\sim1}$ and $\pi_{1}:\sigma$
(by the definition of $S_{\succ}\left(  \pi_{\sim1},\pi_{1}:\sigma\right)  $).
In other words, $\tau$ is a shuffle of $\pi_{\sim1}$ and $\pi_{1}:\sigma$ such
that the first letter of $\tau$ is the first letter of $\pi_{1}:\sigma$ (by
the definition of a right shuffle).

So the first letter of $\tau$ is the first letter of $\pi_{1}:\sigma$. In
other words, the first letter of $\tau$ is $\pi_{1}$ (since the first letter
of $\pi_{1}:\sigma$ is $\pi_{1}$). In other words, $\tau_{1}=\pi_{1}$. Note
that the entries of $\pi$ are distinct (since $\pi$ is a permutation); thus,
the letter $\pi_{1}$ does not appear in $\pi_{\sim1}$. In other words, the
letter $\tau_{1}$ does not appear in $\pi_{\sim1}$ (since $\tau_{1}=\pi_{1}$).
Also, the permutation $\tau$ is nonempty (since $\tau_{1}$ exists).

The definitions of $\pi_{1}$, $\pi_{\sim1}$ and $\pi_{1}:\pi_{\sim1}$ yield
$\pi_{1}:\pi_{\sim1}=\pi$. In view of $\tau_{1}=\pi_{1}$, this rewrites as
$\tau_{1}:\pi_{\sim1}=\pi$.

Also, $\tau$ is a shuffle of $\pi_{\sim1}$ and $\pi_{1}:\sigma$. In other
words, $\tau$ is an $\left(  \left(  m-1\right)  +\left(  n+1\right)  \right)
$-permutation such that both $\pi_{\sim1}$ and $\pi_{1}:\sigma$ are
subsequences of $\tau$ (by the definition of a shuffle).

So $\pi_{1}:\sigma$ is a subsequence of $\tau$. Thus, $\sigma$ is
a subsequence of $\tau$ as well (since $\sigma$ is a subsequence of
$\pi_{1}:\sigma$).

Also, $\pi_{\sim1}$ is a subsequence of $\tau$, but does not contain
the letter $\tau_{1}$ (since the letter $\tau_{1}$ does not appear in
$\pi_{\sim1}$). Therefore, Claim 1 (applied to $\alpha=\pi_{\sim1}$ and
$\beta=\tau$) yields that $\tau_{1}:\pi_{\sim1}$ also is a subsequence
of $\tau$. In other words, $\pi$ also is a subsequence of $\tau$
(since $\tau_{1}:\pi_{\sim1}=\pi$).

Finally, recall that $\tau$ is an $\left(  \left(  m-1\right)  +\left(
n+1\right)  \right)  $-permutation, hence an $\left(  m+n\right)
$-permutation (since $\left(  m-1\right)  +\left(  n+1\right)  =m+n$). So we
know that $\tau$ is an $\left(  m+n\right)  $-permutation such that both $\pi$
and $\sigma$ are subsequences of $\tau$. In other words, $\tau$ is a
shuffle of $\pi$ and $\sigma$ (by the definition of a shuffle). Since the
first letter of $\tau$ is the first letter of $\pi$ (because the first letter
of $\tau$ is $\tau_{1}=\pi_{1}$), we thus conclude that $\tau$ is a left
shuffle of $\pi$ and $\sigma$ (by the definition of a left shuffle). In other
words, $\tau\in S_{\prec}\left(  \pi,\sigma\right)  $ (by the definition of
$S_{\prec}\left(  \pi,\sigma\right)  $).

Since we have proven this for all $\tau\in S_{\succ}\left(  \pi_{\sim1}%
,\pi_{1}:\sigma\right)  $, we thus conclude that $S_{\succ}\left(  \pi_{\sim
1},\pi_{1}:\sigma\right)  \subseteq S_{\prec}\left(  \pi,\sigma\right)  $.
This proves (\ref{pf.prop.LR.rec.b.2}).]

Combining (\ref{pf.prop.LR.rec.b.1}) with (\ref{pf.prop.LR.rec.b.2}), we
obtain $S_{\prec}\left(  \pi,\sigma\right)  =S_{\succ}\left(  \pi_{\sim1}%
,\pi_{1}:\sigma\right)  $. This completes the proof of Proposition
\ref{prop.LR.rec} \textbf{(b)}.

\textbf{(c)} Assume that $\sigma$ is nonempty. Proposition \ref{prop.LR.rec}
\textbf{(a)} (applied to $\sigma$ and $\pi$ instead of $\pi$ and $\sigma$)
yields $S_{\prec}\left(  \sigma,\pi\right)  =S_{\succ}\left(  \pi
,\sigma\right)  $.

Proposition \ref{prop.LR.rec} \textbf{(b)} (applied to $\sigma$ and $\pi$
instead of $\pi$ and $\sigma$) yields that the permutations $\sigma_{\sim1}$
and $\sigma_{1}:\pi$ are well-defined and disjoint, and satisfy $S_{\prec
}\left(  \sigma,\pi\right)  =S_{\succ}\left(  \sigma_{\sim1},\sigma_{1}%
:\pi\right)  $. Thus, Proposition \ref{prop.LR.rec} \textbf{(a)} (applied to
$\sigma_{1}:\pi$ and $\sigma_{\sim1}$ instead of $\pi$ and $\sigma$) yields
$S_{\prec}\left(  \sigma_{1}:\pi,\sigma_{\sim1}\right)  =S_{\succ}\left(
\sigma_{\sim1},\sigma_{1}:\pi\right)  $. Combining all the equalities we have
now proven, we obtain%
\[
S_{\succ}\left(  \pi,\sigma\right)  =S_{\prec}\left(  \sigma,\pi\right)
=S_{\succ}\left(  \sigma_{\sim1},\sigma_{1}:\pi\right)  =S_{\prec}\left(
\sigma_{1}:\pi,\sigma_{\sim1}\right)  .
\]
This completes the proof of Proposition \ref{prop.LR.rec} \textbf{(c)}.
\end{proof}
\end{verlong}

\subsection{LR-shuffle-compatibility}

We shall use the so-called \textit{Iverson bracket notation} for truth values:

\begin{definition}
\label{def.iverson}If $\mathcal{A}$ is any logical statement, then we define
an integer $\left[  \mathcal{A}\right]  \in\left\{  0,1\right\}  $ by%
\[
\left[  \mathcal{A}\right]  =%
\begin{cases}
1, & \text{if }\mathcal{A}\text{ is true};\\
0, & \text{if }\mathcal{A}\text{ is false}%
\end{cases}
.
\]
This integer $\left[  \mathcal{A}\right]  $ is known as the \textit{truth
value} of $\mathcal{A}$.
\end{definition}

Thus, for example, $\left[  4>2\right]  =1$ whereas $\left[  2>4\right]  =0$.

We can now define a notion similar to shuffle-compatibility:

\begin{definition}
\label{def.LRcomp}Let $\operatorname*{st}$ be a permutation statistic. We say
that $\operatorname*{st}$ is \textit{LR-shuffle-compatible} if and only if it
has the following property: For any two disjoint nonempty permutations $\pi$
and $\sigma$, the multisets%
\[
\left\{  \operatorname*{st}\left(  \tau\right)  \ \mid\ \tau\in S_{\prec
}\left(  \pi,\sigma\right)  \right\}  _{\operatorname*{multi}}
\ \ \ \ \ \ \ \ \ \ \text{and}\ \ \ \ \ \ \ \ \ \ \left\{  \operatorname*{st}%
\left(  \tau\right)  \ \mid\ \tau\in S_{\succ}\left(  \pi,\sigma\right)
\right\}  _{\operatorname*{multi}}
\]
depend only on $\operatorname*{st}\left(  \pi\right)  $, $\operatorname*{st}%
\left(  \sigma\right)  $, $\left\vert \pi\right\vert $, $\left\vert
\sigma\right\vert $ and $\left[  \pi_{1}>\sigma_{1}\right]  $.
\end{definition}

In other words, a permutation statistic $\operatorname*{st}$ is
LR-shuffle-compatible if and only if every two disjoint nonempty permutations
$\pi$ and $\sigma$ and every two disjoint nonempty permutations $\pi^{\prime}$
and $\sigma^{\prime}$ satisfying%
\begin{align*}
\operatorname*{st}\left(  \pi\right)   &  =\operatorname*{st}\left(
\pi^{\prime}\right)  ,\ \ \ \ \ \ \ \ \ \ \operatorname*{st}\left(
\sigma\right)  =\operatorname*{st}\left(  \sigma^{\prime}\right)  ,\\
\left\vert \pi\right\vert  &  =\left\vert \pi^{\prime}\right\vert
,\ \ \ \ \ \ \ \ \ \ \left\vert \sigma\right\vert =\left\vert \sigma^{\prime
}\right\vert \ \ \ \ \ \ \ \ \ \ \text{and}\ \ \ \ \ \ \ \ \ \ \left[  \pi
_{1}>\sigma_{1}\right]  =\left[  \pi_{1}^{\prime}>\sigma_{1}^{\prime}\right]
\end{align*}
satisfy
\begin{align*}
\left\{  \operatorname*{st}\left(  \tau\right)  \ \mid\ \tau\in S_{\prec
}\left(  \pi,\sigma\right)  \right\}  _{\operatorname*{multi}}  &  =\left\{
\operatorname*{st}\left(  \tau\right)  \ \mid\ \tau\in S_{\prec}\left(
\pi^{\prime},\sigma^{\prime}\right)  \right\}  _{\operatorname*{multi}%
}\ \ \ \ \ \ \ \ \ \ \text{and}\\
\left\{  \operatorname*{st}\left(  \tau\right)  \ \mid\ \tau\in S_{\succ
}\left(  \pi,\sigma\right)  \right\}  _{\operatorname*{multi}}  &  =\left\{
\operatorname*{st}\left(  \tau\right)  \ \mid\ \tau\in S_{\succ}\left(
\pi^{\prime},\sigma^{\prime}\right)  \right\}  _{\operatorname*{multi}}.
\end{align*}


For example, the permutation statistic $\operatorname*{Pk}$ is not
LR-shuffle-compatible. Indeed, if we take $\pi=\left(  4,2,3\right)  $,
$\sigma=\left(  1\right)  $, $\pi^{\prime}=\left(  2,3,4\right)  $ and
$\sigma^{\prime}=\left(  1\right)  $, then the equalities%
\begin{align*}
\operatorname*{Pk}\left(  \pi\right)   &  =\operatorname*{Pk}\left(
\pi^{\prime}\right)  ,\ \ \ \ \ \ \ \ \ \ \operatorname*{Pk}\left(
\sigma\right)  =\operatorname*{Pk}\left(  \sigma^{\prime}\right)  ,\\
\left\vert \pi\right\vert  &  =\left\vert \pi^{\prime}\right\vert
,\ \ \ \ \ \ \ \ \ \ \left\vert \sigma\right\vert =\left\vert \sigma^{\prime
}\right\vert \ \ \ \ \ \ \ \ \ \ \text{and}\ \ \ \ \ \ \ \ \ \ \left[  \pi
_{1}>\sigma_{1}\right]  =\left[  \pi_{1}^{\prime}>\sigma_{1}^{\prime}\right]
\end{align*}
are all satisfied, but%
\[
\left\{  \operatorname*{Pk}\left(  \tau\right)  \ \mid\ \tau\in S_{\succ
}\left(  \pi,\sigma\right)  \right\}  _{\operatorname*{multi}}=\left\{
\underbrace{\operatorname*{Pk}\left(  1,4,2,3\right)  }_{=\left\{  2\right\}
}\right\}  _{\operatorname*{multi}}=\left\{  \left\{  2\right\}  \right\}
_{\operatorname*{multi}}%
\]
is not the same as%
\[
\left\{  \operatorname*{Pk}\left(  \tau\right)  \ \mid\ \tau\in S_{\succ
}\left(  \pi^{\prime},\sigma^{\prime}\right)  \right\}
_{\operatorname*{multi}}=\left\{  \underbrace{\operatorname*{Pk}\left(
1,2,3,4\right)  }_{=\varnothing}\right\}  _{\operatorname*{multi}}=\left\{
\varnothing\right\}  _{\operatorname*{multi}}.
\]
Similarly, the permutation statistic $\operatorname*{Rpk}$ is not
LR-shuffle-compatible. As we will see in Theorem \ref{thm.LRcomp.Pks} further
below, the three statistics $\operatorname*{Des}$, $\operatorname*{Lpk}$ and
$\operatorname*{Epk}$ are LR-shuffle-compatible.

\subsection{Head-graft-compatibility}

We shall now define another compatibility concept for a permutation statistic,
which will later prove a useful stepping stone for checking the
LR-shuffle-compatibility of this statistic.

\begin{definition}
\label{def.head-comp}Let $\operatorname*{st}$ be a permutation statistic. We
say that $\operatorname*{st}$ is \textit{head-graft-compatible} if and only if
it has the following property: For any nonempty permutation $\pi$ and any
letter $a$ that does not appear in $\pi$, the element $\operatorname*{st}%
\left(  a:\pi\right)  $ depends only on $\operatorname*{st}\left(  \pi\right)
$, $\left\vert \pi\right\vert $ and $\left[  a>\pi_{1}\right]  $.
\end{definition}

In other words, a permutation statistic $\operatorname*{st}$ is
head-graft-compatible if and only if every nonempty permutation $\pi$, every
letter $a$ that does not appear in $\pi$, every nonempty permutation
$\pi^{\prime}$ and every letter $a^{\prime}$ that does not appear in
$\pi^{\prime}$ satisfying%
\[
\operatorname*{st}\left(  \pi\right)  =\operatorname*{st}\left(  \pi^{\prime
}\right)  ,\ \ \ \ \ \ \ \ \ \ \left\vert \pi\right\vert =\left\vert
\pi^{\prime}\right\vert \ \ \ \ \ \ \ \ \ \ \text{and}%
\ \ \ \ \ \ \ \ \ \ \left[  a>\pi_{1}\right]  =\left[  a^{\prime}>\pi
_{1}^{\prime}\right]
\]
satisfy $\operatorname*{st}\left(  a:\pi\right)  =\operatorname*{st}\left(
a^{\prime}:\pi^{\prime}\right)  $.

For example, the permutation statistic $\operatorname*{Pk}$ is not
head-graft-compatible, because if we take $\pi=\left(  3,1\right)  $, $a=2$,
$\pi^{\prime}=\left(  1,3\right)  $ and $a^{\prime}=2$, then we do have%
\[
\operatorname*{Pk}\left(  \pi\right)  =\operatorname*{Pk}\left(  \pi^{\prime
}\right)  ,\ \ \ \ \ \ \ \ \ \ \left\vert \pi\right\vert =\left\vert
\pi^{\prime}\right\vert \ \ \ \ \ \ \ \ \ \ \text{and}%
\ \ \ \ \ \ \ \ \ \ \left[  a>\pi_{1}\right]  =\left[  a^{\prime}>\pi
_{1}^{\prime}\right]
\]
but we don't have $\operatorname*{Pk}\left(  a:\pi\right)  =\operatorname*{Pk}%
\left(  a^{\prime}:\pi^{\prime}\right)  $ (in fact, $\operatorname*{Pk}\left(
a:\pi\right)  =\operatorname*{Pk}\left(  2,3,1\right)  =\left\{  2\right\}  $
whereas $\operatorname*{Pk}\left(  a^{\prime}:\pi^{\prime}\right)
=\operatorname*{Pk}\left(  2,1,3\right)  =\varnothing$). Similarly, it can be
shown that $\operatorname*{Rpk}$ is not head-graft-compatible. As we will see
below (in Proposition \ref{prop.head-comp.Pks}), the permutation statistics
$\operatorname*{Des}$, $\operatorname*{Lpk}$ and $\operatorname*{Epk}$ are
head-graft-compatible; we will analyze a few other statistics in Subsection
\ref{subsect.LR.others}.

\begin{remark}
Let $\operatorname*{st}$ be a head-graft-compatible permutation statistic.
Then, it is easy to see that%
\[
\operatorname*{st}\left(  3,1,2\right)  =\operatorname*{st}\left(
2,1,3\right)  \ \ \ \ \ \ \ \ \ \ \text{and}%
\ \ \ \ \ \ \ \ \ \ \operatorname*{st}\left(  2,3,1\right)
=\operatorname*{st}\left(  1,3,2\right)  .
\]
Moreover, these are the only restrictions that head-graft-compatibility places
on the values of $\operatorname*{st}$ at $3$-permutations.
\end{remark}

It is usually easy to check if a given permutation statistic is
head-graft-compatible. For example:

\begin{proposition}
\label{prop.head-comp.Pks}\textbf{(a)} The permutation statistic
$\operatorname*{Des}$ is head-graft-compatible.

\textbf{(b)} The permutation statistic $\operatorname*{Lpk}$ is head-graft-compatible.

\textbf{(c)} The permutation statistic $\operatorname*{Epk}$ is head-graft-compatible.
\end{proposition}

\begin{proof}
[Proof of Proposition \ref{prop.head-comp.Pks}.]In this proof, we shall use
the following notation: If $S$ is a set of integers, and $p$ is an integer,
then $S+p$ shall denote the set $\left\{  s+p\ \mid\ s\in S\right\}  $.

\textbf{(a)} Let $\pi$ be a nonempty permutation. Let $a$ be a letter that
does not appear in $\pi$. Let $n=\left\vert \pi\right\vert $. We shall express
the element $\operatorname*{Des}\left(  a:\pi\right)  $ in terms of
$\operatorname*{Des}\pi$, $n$ and $\left[  a>\pi_{1}\right]  $.

Since $n=\left\vert \pi\right\vert $, we have $\pi=\left(  \pi_{1},\pi
_{2},\ldots,\pi_{n}\right)  $. Thus, $a:\pi=\left(  a,\pi_{1},\pi_{2}%
,\ldots,\pi_{n}\right)  $. Hence, the descents of $a:\pi$ are obtained as follows:

\begin{itemize}
\item The number $1$ is a descent of $a:\pi$ if and only if $a>\pi_{1}$.

\item Adding $1$ to each descent of $\pi$ yields a descent of $a:\pi$. (That
is, if $i$ is a descent of $\pi$, then $i+1$ is a descent of $a:\pi$.)
\end{itemize}

These are all the descents of $a:\pi$. Thus,%
\begin{equation}
\operatorname*{Des}\left(  a:\pi\right)  =\left\{  1\ \mid\ a>\pi_{1}\right\}
\cup\left(  \operatorname*{Des}\pi+1\right)  .
\label{pf.prop.head-comp.Pks.a.1}%
\end{equation}
(The strange notation \textquotedblleft$\left\{  1\ \mid\ a>\pi_{1}\right\}
$\textquotedblright\ means exactly what it says: It is the set of all numbers
$1$ satisfying $a>\pi_{1}$. In other words, it is $\left\{  1\right\}  $ if
$a>\pi_{1}$, and $\varnothing$ otherwise.)

The equality (\ref{pf.prop.head-comp.Pks.a.1}) shows that $\operatorname*{Des}%
\left(  a:\pi\right)  $ depends only on $\operatorname*{Des}\pi$, $\left\vert
\pi\right\vert $ and $\left[  a>\pi_{1}\right]  $ (indeed, the truth value
$\left[  a>\pi_{1}\right]  $ determines whether $a>\pi_{1}$ is true). In other
words, $\operatorname*{Des}$ is head-graft-compatible (by the definition of
\textquotedblleft head-graft-compatible\textquotedblright). This proves
Proposition \ref{prop.head-comp.Pks} \textbf{(a)}.

\textbf{(b)} Let $\pi$ be a nonempty permutation. Let $a$ be a letter that
does not appear in $\pi$. Let $n=\left\vert \pi\right\vert $. We shall express
the element $\operatorname*{Lpk}\left(  a:\pi\right)  $ in terms of
$\operatorname*{Lpk}\pi$, $n$ and $\left[  a>\pi_{1}\right]  $.

Notice first that $a\neq\pi_{1}$ (since $a$ does not appear in $\pi$). Thus,
$a<\pi_{1}$ is true if and only if $a>\pi_{1}$ is false.

Since $n=\left\vert \pi\right\vert $, we have $\pi=\left(  \pi_{1},\pi
_{2},\ldots,\pi_{n}\right)  $. Thus, $a:\pi=\left(  a,\pi_{1},\pi_{2}%
,\ldots,\pi_{n}\right)  $. Hence, the left peaks of $a:\pi$ are obtained as follows:

\begin{itemize}
\item The number $1$ is a left peak of $a:\pi$ if and only if $a>\pi_{1}$.

\item Adding $1$ to each left peak $i$ of $\pi$ yields a left peak $i+1$ of
$a:\pi$, except for the case when $i=1$ (in which case $i+1=2$ is a left peak
of $a:\pi$ only if $a<\pi_{1}$).
\end{itemize}

These are all the left peaks of $a:\pi$. Thus,%
\begin{equation}
\operatorname*{Lpk}\left(  a:\pi\right)  =\left\{  1\ \mid\ a>\pi_{1}\right\}
\cup%
\begin{cases}
\operatorname*{Lpk}\pi+1, & \text{if }a<\pi_{1};\\
\left(  \operatorname*{Lpk}\pi+1\right)  \setminus\left\{  2\right\}  , &
\text{if not }a<\pi_{1}%
\end{cases}
. \label{pf.prop.head-comp.Pks.b.1}%
\end{equation}


This equality shows that $\operatorname*{Lpk}\left(  a:\pi\right)  $ depends
only on $\operatorname*{Lpk}\pi$, $\left\vert \pi\right\vert $ and $\left[
a>\pi_{1}\right]  $ (indeed, the truth value $\left[  a>\pi_{1}\right]  $
determines whether $a>\pi_{1}$ is true and also determines whether $a<\pi_{1}$
is true\footnote{Indeed, $a<\pi_{1}$ is true if and only if $a>\pi_{1}$ is
false.}). In other words, $\operatorname*{Lpk}$ is head-graft-compatible (by
the definition of \textquotedblleft head-graft-compatible\textquotedblright).
This proves Proposition \ref{prop.head-comp.Pks} \textbf{(b)}.

\textbf{(c)} To obtain a proof of Proposition \ref{prop.head-comp.Pks}
\textbf{(c)}, it suffices to take our above proof of Proposition
\ref{prop.head-comp.Pks} \textbf{(b)} and replace every appearance of
\textquotedblleft left peak\textquotedblright\ and \textquotedblleft%
$\operatorname*{Lpk}$\textquotedblright\ by \textquotedblleft exterior
peak\textquotedblright\ and \textquotedblleft$\operatorname*{Epk}%
$\textquotedblright.
\end{proof}

\subsection{Proving LR-shuffle-compatibility}

Let us now state a sufficient criterion for the LR-shuffle-compatibility of a statistic:

\begin{theorem}
\label{thm.head-comp.LRcomp}Let $\operatorname*{st}$ be a permutation
statistic that is both shuffle-compatible and head-graft-compatible. Then,
$\operatorname*{st}$ is LR-shuffle-compatible.
\end{theorem}

Before we prove this theorem, let us introduce some terminology and state an
almost-trivial fact:

\begin{definition}
\textbf{(a)} If $A$ is a finite multiset, and if $g$ is any object, then
$\left\vert A\right\vert _{g}$ means the multiplicity of $g$ in $A$.

\textbf{(b)} If $A$ and $B$ are two finite multisets, then we say that
$B\subseteq A$ if and only if each object $g$ satisfies $\left\vert
B\right\vert _{g}\leq\left\vert A\right\vert _{g}$.

\textbf{(c)} If $A$ and $B$ are two finite multisets satisfying $B\subseteq
A$, then $A-B$ shall denote the \textquotedblleft multiset
difference\textquotedblright\ of $A$ and $B$; this is the multiset $C$ such
that each object $g$ satisfies $\left\vert C\right\vert _{g}=\left\vert
A\right\vert _{g}-\left\vert B\right\vert _{g}$.
\end{definition}

For example, $\left\{  2,3,3\right\}  _{\operatorname*{multi}}\subseteq
\left\{  1,2,2,3,3\right\}  _{\operatorname*{multi}}$ and $\left\{
1,2,2,3,3\right\}  _{\operatorname*{multi}}-\left\{  2,3,3\right\}
_{\operatorname*{multi}}=\left\{  1,2\right\}  _{\operatorname*{multi}}$.

\begin{lemma}
\label{lem.LR.difference}Let $\pi$ and $\sigma$ be two disjoint permutations
such that at least one of $\pi$ and $\sigma$ is nonempty. Let
$\operatorname*{st}$ be any permutation statistic. Then:

\textbf{(a)} We have%
\begin{align*}
&  \left\{  \operatorname*{st}\left(  \tau\right)  \ \mid\ \tau\in S_{\prec
}\left(  \pi,\sigma\right)  \right\}  _{\operatorname*{multi}}\\
&  =\left\{  \operatorname*{st}\left(  \tau\right)  \ \mid\ \tau\in S\left(
\pi,\sigma\right)  \right\}  _{\operatorname*{multi}}-\left\{
\operatorname*{st}\left(  \tau\right)  \ \mid\ \tau\in S_{\succ}\left(
\pi,\sigma\right)  \right\}  _{\operatorname*{multi}}.
\end{align*}


\textbf{(b)} We have%
\begin{align*}
&  \left\{  \operatorname*{st}\left(  \tau\right)  \ \mid\ \tau\in S_{\succ
}\left(  \pi,\sigma\right)  \right\}  _{\operatorname*{multi}}\\
&  =\left\{  \operatorname*{st}\left(  \tau\right)  \ \mid\ \tau\in S\left(
\pi,\sigma\right)  \right\}  _{\operatorname*{multi}}-\left\{
\operatorname*{st}\left(  \tau\right)  \ \mid\ \tau\in S_{\prec}\left(
\pi,\sigma\right)  \right\}  _{\operatorname*{multi}}.
\end{align*}

\end{lemma}

\begin{proof}
[Proof of Lemma \ref{lem.LR.difference}.]Recall that the two sets $S_{\prec
}\left(  \pi,\sigma\right)  $ and $S_{\succ}\left(  \pi,\sigma\right)  $ are
disjoint and their union is $S\left(  \pi,\sigma\right)  $. Thus, $S_{\succ
}\left(  \pi,\sigma\right)  \subseteq S\left(  \pi,\sigma\right)  $ and
$S_{\prec}\left(  \pi,\sigma\right)  =S\left(  \pi,\sigma\right)  \setminus
S_{\succ}\left(  \pi,\sigma\right)  $. Hence,%
\begin{align*}
&  \left\{  \operatorname*{st}\left(  \tau\right)  \ \mid\ \tau\in S_{\prec
}\left(  \pi,\sigma\right)  \right\}  _{\operatorname*{multi}}\\
&  =\left\{  \operatorname*{st}\left(  \tau\right)  \ \mid\ \tau\in S\left(
\pi,\sigma\right)  \right\}  _{\operatorname*{multi}}-\left\{
\operatorname*{st}\left(  \tau\right)  \ \mid\ \tau\in S_{\succ}\left(
\pi,\sigma\right)  \right\}  _{\operatorname*{multi}}.
\end{align*}
This proves Lemma \ref{lem.LR.difference} \textbf{(a)}. The proof of Lemma
\ref{lem.LR.difference} \textbf{(b)} is analogous.
\end{proof}

\begin{proof}
[Proof of Theorem \ref{thm.head-comp.LRcomp}.]We shall first show the following:

\begin{statement}
\textit{Claim 1:} Let $\pi$, $\pi^{\prime}$ and $\sigma$ be three nonempty
permutations. Assume that $\pi$ and $\sigma$ are disjoint. Assume that
$\pi^{\prime}$ and $\sigma$ are disjoint. Assume furthermore that%
\[
\operatorname*{st}\left(  \pi\right)  =\operatorname*{st}\left(  \pi^{\prime
}\right)  ,\ \ \ \ \ \ \ \ \ \ \left\vert \pi\right\vert =\left\vert
\pi^{\prime}\right\vert \ \ \ \ \ \ \ \ \ \ \text{and}%
\ \ \ \ \ \ \ \ \ \ \left[  \pi_{1}>\sigma_{1}\right]  =\left[  \pi
_{1}^{\prime}>\sigma_{1}\right]  .
\]
Then,
\begin{align}
&  \left\{  \operatorname*{st}\left(  \tau\right)  \ \mid\ \tau\in S_{\prec
}\left(  \pi,\sigma\right)  \right\}  _{\operatorname*{multi}}\nonumber\\
&  =\left\{  \operatorname*{st}\left(  \tau\right)  \ \mid\ \tau\in S_{\prec
}\left(  \pi^{\prime},\sigma\right)  \right\}  _{\operatorname*{multi}}
\label{pf.thm.head-comp.LRcomp.c1.eq3}%
\end{align}
and%
\begin{align}
&  \left\{  \operatorname*{st}\left(  \tau\right)  \ \mid\ \tau\in S_{\succ
}\left(  \pi,\sigma\right)  \right\}  _{\operatorname*{multi}}\nonumber\\
&  =\left\{  \operatorname*{st}\left(  \tau\right)  \ \mid\ \tau\in S_{\succ
}\left(  \pi^{\prime},\sigma\right)  \right\}  _{\operatorname*{multi}}.
\label{pf.thm.head-comp.LRcomp.c1.eq4}%
\end{align}

\end{statement}

[\textit{Proof of Claim 1:} We shall prove Claim 1 by induction on $\left\vert
\sigma\right\vert $:

\textit{Induction base:} The case $\left\vert \sigma\right\vert =0$ cannot
happen (because $\sigma$ is assumed to be nonempty). Thus, Claim 1 is true in
the case $\left\vert \sigma\right\vert =0$. This completes the induction base.

\textit{Induction step:} Let $N$ be a positive integer. Assume (as the
induction hypothesis) that Claim 1 holds when $\left\vert \sigma\right\vert
=N-1$. We must now prove that Claim 1 holds when $\left\vert \sigma\right\vert
=N$.

Indeed, let $\pi$, $\sigma$ and $\pi^{\prime}$ be as in Claim 1, and assume
that $\left\vert \sigma\right\vert =N$. We must prove
(\ref{pf.thm.head-comp.LRcomp.c1.eq3}) and
(\ref{pf.thm.head-comp.LRcomp.c1.eq4}).

Proposition \ref{prop.LR.rec} \textbf{(c)} yields that the permutations
$\sigma_{\sim1}$ and $\sigma_{1}:\pi$ are well-defined and disjoint, and
satisfy%
\begin{equation}
S_{\succ}\left(  \pi,\sigma\right)  =S_{\prec}\left(  \sigma_{1}:\pi
,\sigma_{\sim1}\right)  . \label{pf.thm.head-comp.LRcomp.c1.pf.1}%
\end{equation}
Furthermore, $\left\vert \sigma_{\sim1}\right\vert =\left\vert \sigma
\right\vert -1=N-1$ (since $\left\vert \sigma\right\vert =N$).

Proposition \ref{prop.LR.rec} \textbf{(c)} (applied to $\pi^{\prime}$ instead
of $\pi$) yields that the permutations $\sigma_{\sim1}$ and $\sigma_{1}%
:\pi^{\prime}$ are well-defined and disjoint, and satisfy%
\begin{equation}
S_{\succ}\left(  \pi^{\prime},\sigma\right)  =S_{\prec}\left(  \sigma_{1}%
:\pi^{\prime},\sigma_{\sim1}\right)  . \label{pf.thm.head-comp.LRcomp.c1.pf.2}%
\end{equation}


The letter $\sigma_{1}$ does not appear in the permutation $\pi$ (since $\pi$
and $\sigma$ are disjoint). Similarly, the letter $\sigma_{1}$ does not appear
in the permutation $\pi^{\prime}$. Also, $\left\vert \sigma_{1}:\pi\right\vert
=\underbrace{\left\vert \pi\right\vert }_{=\left\vert \pi^{\prime}\right\vert
}+1=\left\vert \pi^{\prime}\right\vert +1=\left\vert \sigma_{1}:\pi^{\prime
}\right\vert $.

We have $\sigma_{1}\neq\pi_{1}$ (since $\pi$ and $\sigma$ are disjoint). Thus,
the statement $\left(  \sigma_{1}>\pi_{1}\right)  $ is equivalent to $\left(
\text{not }\pi_{1}>\sigma_{1}\right)  $. Hence, $\left[  \sigma_{1}>\pi
_{1}\right]  =\left[  \text{not }\pi_{1}>\sigma_{1}\right]  =1-\left[  \pi
_{1}>\sigma_{1}\right]  $. Similarly, $\left[  \sigma_{1}>\pi_{1}^{\prime
}\right]  =1-\left[  \pi_{1}^{\prime}>\sigma_{1}\right]  $. Hence,%
\[
\left[  \sigma_{1}>\pi_{1}\right]  =1-\underbrace{\left[  \pi_{1}>\sigma
_{1}\right]  }_{=\left[  \pi_{1}^{\prime}>\sigma_{1}\right]  }=1-\left[
\pi_{1}^{\prime}>\sigma_{1}\right]  =\left[  \sigma_{1}>\pi_{1}^{\prime
}\right]  .
\]


Both permutations $\sigma_{1}:\pi$ and $\sigma_{1}:\pi^{\prime}$ begin with
the letter $\sigma_{1}$. Thus, both $\left(  \sigma_{1}:\pi\right)  _{1}$ and
$\left(  \sigma_{1}:\pi^{\prime}\right)  _{1}$ equal $\sigma_{1}$. Hence,
$\left(  \sigma_{1}:\pi\right)  _{1}=\left(  \sigma_{1}:\pi^{\prime}\right)
_{1}$.

The statistic $\operatorname*{st}$ is head-graft-compatible. In other words,
for any nonempty permutation $\varphi$ and any letter $a$ that does not appear
in $\varphi$, the element $\operatorname*{st}\left(  a:\varphi\right)  $
depends only on $\operatorname*{st}\left(  \varphi\right)  $, $\left\vert
\varphi\right\vert $ and $\left[  a>\varphi_{1}\right]  $ (by the definition
of \textquotedblleft head-graft-compatible\textquotedblright). Hence,
if $\varphi$ and $\varphi^{\prime}$ are two nonempty permutations, and if $a$
is any letter that does not appear in $\varphi$ and does not appear in
$\varphi^{\prime}$, and if we have $\operatorname*{st}\left(  \varphi\right)
=\operatorname*{st}\left(  \varphi^{\prime}\right)  $ and $\left\vert
\varphi\right\vert =\left\vert \varphi^{\prime}\right\vert $ and $\left[
a>\varphi_{1}\right]  =\left[  a>\varphi_{1}^{\prime}\right]  $, then
$\operatorname*{st}\left(  a:\varphi\right)  =\operatorname*{st}\left(
a:\varphi^{\prime}\right)  $. Applying this to $a=\sigma_{1}$, $\varphi=\pi$
and $\varphi^{\prime}=\pi^{\prime}$, we obtain
\[
\operatorname*{st}\left(  \sigma_{1}:\pi\right)  =\operatorname*{st}\left(
\sigma_{1}:\pi^{\prime}\right)
\]
(since $\operatorname*{st}\left(  \pi\right)  =\operatorname*{st}\left(
\pi^{\prime}\right)  $ and $\left\vert \pi\right\vert =\left\vert \pi^{\prime
}\right\vert $ and $\left[  \sigma_{1}>\pi_{1}\right]  =\left[  \sigma_{1}%
>\pi_{1}^{\prime}\right]  $).

Next, we claim that%
\begin{equation}
\left\{  \operatorname*{st}\left(  \tau\right)  \ \mid\ \tau\in S_{\prec
}\left(  \sigma_{1}:\pi,\sigma_{\sim1}\right)  \right\}  =\left\{
\operatorname*{st}\left(  \tau\right)  \ \mid\ \tau\in S_{\prec}\left(
\sigma_{1}:\pi^{\prime},\sigma_{\sim1}\right)  \right\}  .
\label{pf.thm.head-comp.LRcomp.c1.pf.5}%
\end{equation}


[\textit{Proof of (\ref{pf.thm.head-comp.LRcomp.c1.pf.5}):} The permutations
$\sigma_{1}:\pi$ and $\sigma_{1}:\pi^{\prime}$ are clearly nonempty. Hence, if
$\sigma_{\sim1}$ is the empty permutation $\left(  {}\right)  $, then
$S_{\prec}\left(  \sigma_{1}:\pi,\sigma_{\sim1}\right)  =\left\{  \sigma
_{1}:\pi\right\}  $ and $S_{\prec}\left(  \sigma_{1}:\pi^{\prime},\sigma
_{\sim1}\right)  =\left\{  \sigma_{1}:\pi^{\prime}\right\}  $. Thus, if
$\sigma_{\sim1}$ is the empty permutation $\left(  {}\right)  $, then
(\ref{pf.thm.head-comp.LRcomp.c1.pf.5}) follows from%
\begin{align*}
&  \left\{  \operatorname*{st}\left(  \tau\right)  \ \mid\ \tau\in
\underbrace{S_{\prec}\left(  \sigma_{1}:\pi,\sigma_{\sim1}\right)
}_{=\left\{  \sigma_{1}:\pi\right\}  }\right\}  _{\operatorname*{multi}}\\
&  =\left\{  \operatorname*{st}\left(  \tau\right)  \ \mid\ \tau\in\left\{
\sigma_{1}:\pi\right\}  \right\}  _{\operatorname*{multi}}=\left\{
\underbrace{\operatorname*{st}\left(  \sigma_{1}:\pi\right)  }%
_{=\operatorname*{st}\left(  \sigma_{1}:\pi^{\prime}\right)  }\right\}
_{\operatorname*{multi}}\\
&  =\left\{  \operatorname*{st}\left(  \sigma_{1}:\pi^{\prime}\right)
\right\}  _{\operatorname*{multi}}=\left\{  \operatorname*{st}\left(
\tau\right)  \ \mid\ \tau\in\underbrace{\left\{  \sigma_{1}:\pi^{\prime
}\right\}  }_{=S_{\prec}\left(  \sigma_{1}:\pi^{\prime},\sigma_{\sim1}\right)
}\right\}  _{\operatorname*{multi}}\\
&  =\left\{  \operatorname*{st}\left(  \tau\right)  \ \mid\ \tau\in S_{\prec
}\left(  \sigma_{1}:\pi^{\prime},\sigma_{\sim1}\right)  \right\}
_{\operatorname*{multi}}.
\end{align*}
Thus, for the rest of our proof of (\ref{pf.thm.head-comp.LRcomp.c1.pf.5}), we
WLOG assume that $\sigma_{\sim1}$ is not the empty permutation $\left(
{}\right)  $. Thus, $\sigma_{\sim1}$ is nonempty.

But recall that $\left\vert \sigma_{\sim1}\right\vert =N-1$. Hence, the
induction hypothesis allows us to apply Claim 1 to $\sigma_{1}:\pi$,
$\sigma_{1}:\pi^{\prime}$ and $\sigma_{\sim1}$ instead of $\pi$, $\pi^{\prime
}$ and $\sigma$ (because we know that the permutations $\sigma_{\sim1}$ and
$\sigma_{1}:\pi$ are disjoint; that the permutations $\sigma_{\sim1}$ and
$\sigma_{1}:\pi^{\prime}$ are disjoint; that $\operatorname*{st}\left(
\sigma_{1}:\pi\right)  =\operatorname*{st}\left(  \sigma_{1}:\pi^{\prime
}\right)  $ and $\left\vert \sigma_{1}:\pi\right\vert =\left\vert \sigma
_{1}:\pi^{\prime}\right\vert $; and that $\left[  \underbrace{\left(
\sigma_{1}:\pi\right)  _{1}}_{=\left(  \sigma_{1}:\pi^{\prime}\right)  _{1}%
}>\left(  \sigma_{\sim1}\right)  _{1}\right]  =\left[  \left(  \sigma_{1}%
:\pi^{\prime}\right)  _{1}>\left(  \sigma_{\sim1}\right)  _{1}\right]  $). We
therefore obtain%
\[
\left\{  \operatorname*{st}\left(  \tau\right)  \ \mid\ \tau\in S_{\prec
}\left(  \sigma_{1}:\pi,\sigma_{\sim1}\right)  \right\}
_{\operatorname*{multi}}=\left\{  \operatorname*{st}\left(  \tau\right)
\ \mid\ \tau\in S_{\prec}\left(  \sigma_{1}:\pi^{\prime},\sigma_{\sim
1}\right)  \right\}  _{\operatorname*{multi}}%
\]
and%
\[
\left\{  \operatorname*{st}\left(  \tau\right)  \ \mid\ \tau\in S_{\succ
}\left(  \sigma_{1}:\pi,\sigma_{\sim1}\right)  \right\}
_{\operatorname*{multi}}=\left\{  \operatorname*{st}\left(  \tau\right)
\ \mid\ \tau\in S_{\succ}\left(  \sigma_{1}:\pi^{\prime},\sigma_{\sim
1}\right)  \right\}  _{\operatorname*{multi}}.
\]
The first of these two equalities is precisely
(\ref{pf.thm.head-comp.LRcomp.c1.pf.5}). Thus,
(\ref{pf.thm.head-comp.LRcomp.c1.pf.5}) is proven.]

Now,
\begin{align}
&  \left\{  \operatorname*{st}\left(  \tau\right)  \ \mid\ \tau\in
\underbrace{S_{\succ}\left(  \pi,\sigma\right)  }_{\substack{=S_{\prec}\left(
\sigma_{1}:\pi,\sigma_{\sim1}\right)  \\\text{(by
(\ref{pf.thm.head-comp.LRcomp.c1.pf.1}))}}}\right\}  _{\operatorname*{multi}%
}\nonumber\\
&  =\left\{  \operatorname*{st}\left(  \tau\right)  \ \mid\ \tau\in S_{\prec
}\left(  \sigma_{1}:\pi,\sigma_{\sim1}\right)  \right\}
_{\operatorname*{multi}}\nonumber\\
&  =\left\{  \operatorname*{st}\left(  \tau\right)  \ \mid\ \tau
\in\underbrace{S_{\prec}\left(  \sigma_{1}:\pi^{\prime},\sigma_{\sim1}\right)
}_{\substack{=S_{\succ}\left(  \pi^{\prime},\sigma\right)  \\\text{(by
(\ref{pf.thm.head-comp.LRcomp.c1.pf.2}))}}}\right\}  _{\operatorname*{multi}%
}\ \ \ \ \ \ \ \ \ \ \left(  \text{by (\ref{pf.thm.head-comp.LRcomp.c1.pf.5}%
)}\right) \nonumber\\
&  =\left\{  \operatorname*{st}\left(  \tau\right)  \ \mid\ \tau\in S_{\succ
}\left(  \pi^{\prime},\sigma\right)  \right\}  _{\operatorname*{multi}}.
\label{pf.thm.head-comp.LRcomp.c1.pf.6}%
\end{align}
This proves (\ref{pf.thm.head-comp.LRcomp.c1.eq4}). It remains to prove
(\ref{pf.thm.head-comp.LRcomp.c1.eq3}).

Lemma \ref{lem.LR.difference} \textbf{(a)} yields%
\begin{align}
&  \left\{  \operatorname*{st}\left(  \tau\right)  \ \mid\ \tau\in S_{\prec
}\left(  \pi,\sigma\right)  \right\}  _{\operatorname*{multi}}\nonumber\\
&  =\left\{  \operatorname*{st}\left(  \tau\right)  \ \mid\ \tau\in S\left(
\pi,\sigma\right)  \right\}  _{\operatorname*{multi}}-\left\{
\operatorname*{st}\left(  \tau\right)  \ \mid\ \tau\in S_{\succ}\left(
\pi,\sigma\right)  \right\}  _{\operatorname*{multi}}.
\label{pf.thm.head-comp.LRcomp.c1.pf.7}%
\end{align}
Lemma \ref{lem.LR.difference} \textbf{(a)} (applied to $\pi^{\prime}$ instead
of $\pi$) yields%
\begin{align}
&  \left\{  \operatorname*{st}\left(  \tau\right)  \ \mid\ \tau\in S_{\prec
}\left(  \pi^{\prime},\sigma\right)  \right\}  _{\operatorname*{multi}%
}\nonumber\\
&  =\left\{  \operatorname*{st}\left(  \tau\right)  \ \mid\ \tau\in S\left(
\pi^{\prime},\sigma\right)  \right\}  _{\operatorname*{multi}}-\left\{
\operatorname*{st}\left(  \tau\right)  \ \mid\ \tau\in S_{\succ}\left(
\pi^{\prime},\sigma\right)  \right\}  _{\operatorname*{multi}}.
\label{pf.thm.head-comp.LRcomp.c1.pf.8}%
\end{align}


But recall that the statistic $\operatorname*{st}$ is shuffle-compatible. In
other words, for any two disjoint permutations $\alpha$ and $\beta$, the
multiset%
\[
\left\{  \operatorname*{st}\left(  \tau\right)  \ \mid\ \tau\in S\left(
\alpha,\beta\right)  \right\}  _{\operatorname*{multi}}%
\]
depends only on $\operatorname*{st}\left(  \alpha\right)  $,
$\operatorname*{st}\left(  \beta\right)  $, $\left\vert \alpha\right\vert $
and $\left\vert \beta\right\vert $ (by the definition of
shuffle-compatibility). In other words, if $\alpha$ and $\beta$ are two
disjoint permutation, and if $\alpha^{\prime}$ and $\beta^{\prime}$ are two
disjoint permutations, and if we have%
\[
\operatorname*{st}\left(  \alpha\right)  =\operatorname*{st}\left(
\alpha^{\prime}\right)  ,\ \ \ \ \ \ \ \ \ \ \operatorname*{st}\left(
\beta\right)  =\operatorname*{st}\left(  \beta^{\prime}\right)
,\ \ \ \ \ \ \ \ \ \ \left\vert \alpha\right\vert =\left\vert \alpha^{\prime
}\right\vert \ \ \ \ \ \ \ \ \ \ \text{and}\ \ \ \ \ \ \ \ \ \ \left\vert
\beta\right\vert =\left\vert \beta^{\prime}\right\vert ,
\]
then%
\[
\left\{  \operatorname*{st}\left(  \tau\right)  \ \mid\ \tau\in S\left(
\alpha,\beta\right)  \right\}  _{\operatorname*{multi}}=\left\{
\operatorname*{st}\left(  \tau\right)  \ \mid\ \tau\in S\left(  \alpha
^{\prime},\beta^{\prime}\right)  \right\}  _{\operatorname*{multi}}.
\]
Applying this to $\alpha=\pi$, $\beta=\sigma$, $\alpha^{\prime}=\pi^{\prime}$
and $\beta^{\prime}=\sigma$, we obtain
\begin{equation}
\left\{  \operatorname*{st}\left(  \tau\right)  \ \mid\ \tau\in S\left(
\pi,\sigma\right)  \right\}  _{\operatorname*{multi}}=\left\{
\operatorname*{st}\left(  \tau\right)  \ \mid\ \tau\in S\left(  \pi^{\prime
},\sigma\right)  \right\}  _{\operatorname*{multi}}
\label{pf.thm.head-comp.LRcomp.c1.pf.9}%
\end{equation}
(since $\operatorname*{st}\left(  \pi\right)  =\operatorname*{st}\left(
\pi^{\prime}\right)  $, $\operatorname*{st}\left(  \sigma\right)
=\operatorname*{st}\left(  \sigma\right)  $, $\left\vert \pi\right\vert
=\left\vert \pi^{\prime}\right\vert $ and $\left\vert \sigma\right\vert
=\left\vert \sigma\right\vert $). Now, (\ref{pf.thm.head-comp.LRcomp.c1.pf.7})
becomes%
\begin{align*}
&  \left\{  \operatorname*{st}\left(  \tau\right)  \ \mid\ \tau\in S_{\prec
}\left(  \pi,\sigma\right)  \right\}  _{\operatorname*{multi}}\\
&  =\underbrace{\left\{  \operatorname*{st}\left(  \tau\right)  \ \mid
\ \tau\in S\left(  \pi,\sigma\right)  \right\}  _{\operatorname*{multi}}%
}_{\substack{=\left\{  \operatorname*{st}\left(  \tau\right)  \ \mid\ \tau\in
S\left(  \pi^{\prime},\sigma\right)  \right\}  _{\operatorname*{multi}%
}\\\text{(by (\ref{pf.thm.head-comp.LRcomp.c1.pf.9}))}}}-\underbrace{\left\{
\operatorname*{st}\left(  \tau\right)  \ \mid\ \tau\in S_{\succ}\left(
\pi,\sigma\right)  \right\}  _{\operatorname*{multi}}}_{\substack{=\left\{
\operatorname*{st}\left(  \tau\right)  \ \mid\ \tau\in S_{\succ}\left(
\pi^{\prime},\sigma\right)  \right\}  _{\operatorname*{multi}}\\\text{(by
(\ref{pf.thm.head-comp.LRcomp.c1.pf.6}))}}}\\
&  =\left\{  \operatorname*{st}\left(  \tau\right)  \ \mid\ \tau\in S\left(
\pi^{\prime},\sigma\right)  \right\}  _{\operatorname*{multi}}-\left\{
\operatorname*{st}\left(  \tau\right)  \ \mid\ \tau\in S_{\succ}\left(
\pi^{\prime},\sigma\right)  \right\}  _{\operatorname*{multi}}\\
&  =\left\{  \operatorname*{st}\left(  \tau\right)  \ \mid\ \tau\in S_{\prec
}\left(  \pi^{\prime},\sigma\right)  \right\}  _{\operatorname*{multi}%
}\ \ \ \ \ \ \ \ \ \ \left(  \text{by (\ref{pf.thm.head-comp.LRcomp.c1.pf.8}%
)}\right)  .
\end{align*}
Thus, (\ref{pf.thm.head-comp.LRcomp.c1.eq3}) is proven. Hence, we have proven
both (\ref{pf.thm.head-comp.LRcomp.c1.eq3}) and
(\ref{pf.thm.head-comp.LRcomp.c1.eq4}). This shows that Claim 1 holds for our
$\pi$, $\pi^{\prime}$ and $\sigma$. This completes the induction step. Thus,
Claim 1 is proven by induction.]

We shall next derive a \textquotedblleft mirror version\textquotedblright\ of
Claim 1:

\begin{statement}
\textit{Claim 2:} Let $\pi$, $\sigma$ and $\sigma^{\prime}$ be three nonempty
permutations. Assume that $\pi$ and $\sigma$ are disjoint. Assume that $\pi$
and $\sigma^{\prime}$ are disjoint. Assume furthermore that%
\[
\operatorname*{st}\left(  \sigma\right)  =\operatorname*{st}\left(
\sigma^{\prime}\right)  ,\ \ \ \ \ \ \ \ \ \ \left\vert \sigma\right\vert
=\left\vert \sigma^{\prime}\right\vert \ \ \ \ \ \ \ \ \ \ \text{and}%
\ \ \ \ \ \ \ \ \ \ \left[  \pi_{1}>\sigma_{1}\right]  =\left[  \pi_{1}%
>\sigma_{1}^{\prime}\right]  .
\]
Then,
\begin{align*}
&  \left\{  \operatorname*{st}\left(  \tau\right)  \ \mid\ \tau\in S_{\prec
}\left(  \pi,\sigma\right)  \right\}  _{\operatorname*{multi}}\\
&  =\left\{  \operatorname*{st}\left(  \tau\right)  \ \mid\ \tau\in S_{\prec
}\left(  \pi,\sigma^{\prime}\right)  \right\}  _{\operatorname*{multi}}%
\end{align*}
and%
\begin{align*}
&  \left\{  \operatorname*{st}\left(  \tau\right)  \ \mid\ \tau\in S_{\succ
}\left(  \pi,\sigma\right)  \right\}  _{\operatorname*{multi}}\\
&  =\left\{  \operatorname*{st}\left(  \tau\right)  \ \mid\ \tau\in S_{\succ
}\left(  \pi,\sigma^{\prime}\right)  \right\}  _{\operatorname*{multi}}.
\end{align*}

\end{statement}

[\textit{Proof of Claim 2:} We have $\sigma_{1}\neq\pi_{1}$ (since $\pi$ and
$\sigma$ are disjoint). Thus, the statement $\left(  \sigma_{1}>\pi
_{1}\right)  $ is equivalent to $\left(  \text{not }\pi_{1}>\sigma_{1}\right)
$. Hence, $\left[  \sigma_{1}>\pi_{1}\right]  =\left[  \text{not }\pi
_{1}>\sigma_{1}\right]  =1-\left[  \pi_{1}>\sigma_{1}\right]  $. Similarly,
$\left[  \sigma_{1}^{\prime}>\pi_{1}\right]  =1-\left[  \pi_{1}>\sigma
_{1}^{\prime}\right]  $. Hence,%
\[
\left[  \sigma_{1}>\pi_{1}\right]  =1-\underbrace{\left[  \pi_{1}>\sigma
_{1}\right]  }_{=\left[  \pi_{1}>\sigma_{1}^{\prime}\right]  }=1-\left[
\pi_{1}>\sigma_{1}^{\prime}\right]  =\left[  \sigma_{1}^{\prime}>\pi
_{1}\right]  .
\]
Hence, Claim 1 (applied to $\sigma$, $\sigma^{\prime}$ and $\pi$ instead of
$\pi$, $\pi^{\prime}$ and $\sigma$) shows that%
\begin{align*}
&  \left\{  \operatorname*{st}\left(  \tau\right)  \ \mid\ \tau\in S_{\prec
}\left(  \sigma,\pi\right)  \right\}  _{\operatorname*{multi}}\\
&  =\left\{  \operatorname*{st}\left(  \tau\right)  \ \mid\ \tau\in S_{\prec
}\left(  \sigma^{\prime},\pi\right)  \right\}  _{\operatorname*{multi}}%
\end{align*}
and%
\begin{align*}
&  \left\{  \operatorname*{st}\left(  \tau\right)  \ \mid\ \tau\in S_{\succ
}\left(  \sigma,\pi\right)  \right\}  _{\operatorname*{multi}}\\
&  =\left\{  \operatorname*{st}\left(  \tau\right)  \ \mid\ \tau\in S_{\succ
}\left(  \sigma^{\prime},\pi\right)  \right\}  _{\operatorname*{multi}}.
\end{align*}
But Proposition \ref{prop.LR.rec} \textbf{(a)} yields $S_{\prec}\left(
\pi,\sigma\right)  =S_{\succ}\left(  \sigma,\pi\right)  $. Similarly,
$S_{\prec}\left(  \pi,\sigma^{\prime}\right)  =S_{\succ}\left(  \sigma
^{\prime},\pi\right)  $. Also, Proposition \ref{prop.LR.rec} \textbf{(a)}
(applied to $\sigma$ and $\pi$ instead of $\pi$ and $\sigma$) yields
$S_{\prec}\left(  \sigma,\pi\right)  =S_{\succ}\left(  \pi,\sigma\right)  $.
Similarly, $S_{\prec}\left(  \sigma^{\prime},\pi\right)  =S_{\succ}\left(
\pi,\sigma^{\prime}\right)  $. Using all these equalities, we find%
\begin{align*}
&  \left\{  \operatorname*{st}\left(  \tau\right)  \ \mid\ \tau\in
\underbrace{S_{\prec}\left(  \pi,\sigma\right)  }_{=S_{\succ}\left(
\sigma,\pi\right)  }\right\}  _{\operatorname*{multi}}\\
&  =\left\{  \operatorname*{st}\left(  \tau\right)  \ \mid\ \tau\in S_{\succ
}\left(  \sigma,\pi\right)  \right\}  _{\operatorname*{multi}}=\left\{
\operatorname*{st}\left(  \tau\right)  \ \mid\ \tau\in\underbrace{S_{\succ
}\left(  \sigma^{\prime},\pi\right)  }_{=S_{\prec}\left(  \pi,\sigma^{\prime
}\right)  }\right\}  _{\operatorname*{multi}}\\
&  =\left\{  \operatorname*{st}\left(  \tau\right)  \ \mid\ \tau\in S_{\prec
}\left(  \pi,\sigma^{\prime}\right)  \right\}  _{\operatorname*{multi}}%
\end{align*}
and%
\begin{align*}
&  \left\{  \operatorname*{st}\left(  \tau\right)  \ \mid\ \tau\in
\underbrace{S_{\succ}\left(  \pi,\sigma\right)  }_{=S_{\prec}\left(
\sigma,\pi\right)  }\right\}  _{\operatorname*{multi}}\\
&  =\left\{  \operatorname*{st}\left(  \tau\right)  \ \mid\ \tau\in S_{\prec
}\left(  \sigma,\pi\right)  \right\}  _{\operatorname*{multi}}=\left\{
\operatorname*{st}\left(  \tau\right)  \ \mid\ \tau\in\underbrace{S_{\prec
}\left(  \sigma^{\prime},\pi\right)  }_{=S_{\succ}\left(  \pi,\sigma^{\prime
}\right)  }\right\}  _{\operatorname*{multi}}\\
&  =\left\{  \operatorname*{st}\left(  \tau\right)  \ \mid\ \tau\in S_{\succ
}\left(  \pi,\sigma^{\prime}\right)  \right\}  _{\operatorname*{multi}}.
\end{align*}
Thus, Claim 2 is proven.]

Finally, we are ready to take on the LR-shuffle-compatibility of
$\operatorname*{st}$:

\begin{statement}
\textit{Claim 3:} Let $\pi$ and $\sigma$ be two disjoint nonempty
permutations. Let $\pi^{\prime}$ and $\sigma^{\prime}$ be two disjoint
nonempty permutations. Assume that%
\begin{align*}
\operatorname*{st}\left(  \pi\right)   &  =\operatorname*{st}\left(
\pi^{\prime}\right)  ,\ \ \ \ \ \ \ \ \ \ \operatorname*{st}\left(
\sigma\right)  =\operatorname*{st}\left(  \sigma^{\prime}\right)  ,\\
\left\vert \pi\right\vert  &  =\left\vert \pi^{\prime}\right\vert
,\ \ \ \ \ \ \ \ \ \ \left\vert \sigma\right\vert =\left\vert \sigma^{\prime
}\right\vert \ \ \ \ \ \ \ \ \ \ \text{and}\ \ \ \ \ \ \ \ \ \ \left[  \pi
_{1}>\sigma_{1}\right]  =\left[  \pi_{1}^{\prime}>\sigma_{1}^{\prime}\right]
.
\end{align*}
Then,
\[
\left\{  \operatorname*{st}\left(  \tau\right)  \ \mid\ \tau\in S_{\prec
}\left(  \pi,\sigma\right)  \right\}  _{\operatorname*{multi}}=\left\{
\operatorname*{st}\left(  \tau\right)  \ \mid\ \tau\in S_{\prec}\left(
\pi^{\prime},\sigma^{\prime}\right)  \right\}  _{\operatorname*{multi}}%
\]
and%
\[
\left\{  \operatorname*{st}\left(  \tau\right)  \ \mid\ \tau\in S_{\succ
}\left(  \pi,\sigma\right)  \right\}  _{\operatorname*{multi}}=\left\{
\operatorname*{st}\left(  \tau\right)  \ \mid\ \tau\in S_{\succ}\left(
\pi^{\prime},\sigma^{\prime}\right)  \right\}  _{\operatorname*{multi}}.
\]

\end{statement}

[\textit{Proof of Claim 3:} We have $\left[  \pi_{1}>\sigma_{1}\right]
=\left[  \pi_{1}^{\prime}>\sigma_{1}^{\prime}\right]  $. Since $\left[
\pi_{1}>\sigma_{1}\right]  =\left[  \pi_{1}^{\prime}>\sigma_{1}^{\prime
}\right]  $ is either $1$ or $0$, we must therefore be in one of the following
two cases:

\textit{Case 1:} We have $\left[  \pi_{1}>\sigma_{1}\right]  =\left[  \pi
_{1}^{\prime}>\sigma_{1}^{\prime}\right]  =1$.

\textit{Case 2:} We have $\left[  \pi_{1}>\sigma_{1}\right]  =\left[  \pi
_{1}^{\prime}>\sigma_{1}^{\prime}\right]  =0$.

Let us first consider Case 1. In this case, we have $\left[  \pi_{1}%
>\sigma_{1}\right]  =\left[  \pi_{1}^{\prime}>\sigma_{1}^{\prime}\right]  =1$.

There clearly exists a positive integer $N$ that is larger than all entries of
$\sigma$ and larger than all entries of $\sigma^{\prime}$. Consider such an
$N$. Let $n=\left\vert \pi\right\vert $; thus, $\pi=\left(  \pi_{1},\pi
_{2},\ldots,\pi_{n}\right)  $. Let $\gamma$ be the permutation $\left(
\pi_{1}+N,\pi_{2}+N,\ldots,\pi_{n}+N\right)  $. This permutation $\gamma$ is
order-equivalent to $\pi$, but is disjoint from $\sigma$ (since all its
entries are $>N$, while all the entries of $\sigma$ are $<N$) and disjoint
from $\sigma^{\prime}$ (for similar reasons). Also, $\gamma_{1}%
=\underbrace{\pi_{1}}_{>0}+N>N>\sigma_{1}$, so that $\left[  \gamma_{1}%
>\sigma_{1}\right]  =1$. Similarly, $\left[  \gamma_{1}>\sigma_{1}^{\prime
}\right]  =1$.

The permutation $\gamma$ is order-equivalent to $\pi$. Thus,
$\operatorname*{st}\left(  \gamma\right)  =\operatorname*{st}\left(
\pi\right)  $ (since $\operatorname*{st}$ is a permutation statistic) and
$\left\vert \gamma\right\vert =\left\vert \pi\right\vert $. The permutation
$\gamma$ is furthermore nonempty (since it is order-equivalent to the nonempty
permutation $\pi$). Also, $\operatorname*{st}\left(  \gamma\right)
=\operatorname*{st}\left(  \pi\right)  =\operatorname*{st}\left(  \pi^{\prime
}\right)  $ and $\left\vert \gamma\right\vert =\left\vert \pi\right\vert
=\left\vert \pi^{\prime}\right\vert $. Moreover, $\left[  \pi_{1}>\sigma
_{1}\right]  =1=\left[  \gamma_{1}>\sigma_{1}\right]  $ and $\left[
\gamma_{1}>\sigma_{1}\right]  =1=\left[  \gamma_{1}>\sigma_{1}^{\prime
}\right]  $ and $\left[  \gamma_{1}>\sigma_{1}\right]  =1=\left[  \pi
_{1}^{\prime}>\sigma_{1}^{\prime}\right]  $. Hence, Claim 1 (applied to
$\gamma$ instead of $\pi^{\prime}$) yields%
\[
\left\{  \operatorname*{st}\left(  \tau\right)  \ \mid\ \tau\in S_{\prec
}\left(  \pi,\sigma\right)  \right\}  _{\operatorname*{multi}}=\left\{
\operatorname*{st}\left(  \tau\right)  \ \mid\ \tau\in S_{\prec}\left(
\gamma,\sigma\right)  \right\}  _{\operatorname*{multi}}%
\]
and%
\begin{equation}
\left\{  \operatorname*{st}\left(  \tau\right)  \ \mid\ \tau\in S_{\succ
}\left(  \pi,\sigma\right)  \right\}  _{\operatorname*{multi}}=\left\{
\operatorname*{st}\left(  \tau\right)  \ \mid\ \tau\in S_{\succ}\left(
\gamma,\sigma\right)  \right\}  _{\operatorname*{multi}}.\nonumber
\end{equation}
Furthermore, Claim 2 (applied to $\gamma$ instead of $\pi$) yields%
\[
\left\{  \operatorname*{st}\left(  \tau\right)  \ \mid\ \tau\in S_{\prec
}\left(  \gamma,\sigma\right)  \right\}  _{\operatorname*{multi}}=\left\{
\operatorname*{st}\left(  \tau\right)  \ \mid\ \tau\in S_{\prec}\left(
\gamma,\sigma^{\prime}\right)  \right\}  _{\operatorname*{multi}}%
\]
and%
\[
\left\{  \operatorname*{st}\left(  \tau\right)  \ \mid\ \tau\in S_{\succ
}\left(  \gamma,\sigma\right)  \right\}  _{\operatorname*{multi}}=\left\{
\operatorname*{st}\left(  \tau\right)  \ \mid\ \tau\in S_{\succ}\left(
\gamma,\sigma^{\prime}\right)  \right\}  _{\operatorname*{multi}}.
\]
Finally, Claim 1 (applied to $\gamma$ and $\sigma^{\prime}$ instead of $\pi$
and $\sigma$) yields%
\[
\left\{  \operatorname*{st}\left(  \tau\right)  \ \mid\ \tau\in S_{\prec
}\left(  \gamma,\sigma^{\prime}\right)  \right\}  _{\operatorname*{multi}%
}=\left\{  \operatorname*{st}\left(  \tau\right)  \ \mid\ \tau\in S_{\prec
}\left(  \pi^{\prime},\sigma^{\prime}\right)  \right\}
_{\operatorname*{multi}}%
\]
and%
\begin{equation}
\left\{  \operatorname*{st}\left(  \tau\right)  \ \mid\ \tau\in S_{\succ
}\left(  \gamma,\sigma^{\prime}\right)  \right\}  _{\operatorname*{multi}%
}=\left\{  \operatorname*{st}\left(  \tau\right)  \ \mid\ \tau\in S_{\succ
}\left(  \pi^{\prime},\sigma^{\prime}\right)  \right\}
_{\operatorname*{multi}}.\nonumber
\end{equation}
Combining the equalities we have found, we obtain%
\begin{align*}
\left\{  \operatorname*{st}\left(  \tau\right)  \ \mid\ \tau\in S_{\prec
}\left(  \pi,\sigma\right)  \right\}  _{\operatorname*{multi}}  &  =\left\{
\operatorname*{st}\left(  \tau\right)  \ \mid\ \tau\in S_{\prec}\left(
\gamma,\sigma\right)  \right\}  _{\operatorname*{multi}}\\
&  =\left\{  \operatorname*{st}\left(  \tau\right)  \ \mid\ \tau\in S_{\prec
}\left(  \gamma,\sigma^{\prime}\right)  \right\}  _{\operatorname*{multi}}\\
&  =\left\{  \operatorname*{st}\left(  \tau\right)  \ \mid\ \tau\in S_{\prec
}\left(  \pi^{\prime},\sigma^{\prime}\right)  \right\}
_{\operatorname*{multi}}.
\end{align*}
The same argument (but with the symbols \textquotedblleft$S_{\prec}%
$\textquotedblright\ and \textquotedblleft$S_{\succ}$\textquotedblright%
\ interchanged) yields
\[
\left\{  \operatorname*{st}\left(  \tau\right)  \ \mid\ \tau\in S_{\succ
}\left(  \pi,\sigma\right)  \right\}  _{\operatorname*{multi}}=\left\{
\operatorname*{st}\left(  \tau\right)  \ \mid\ \tau\in S_{\succ}\left(
\pi^{\prime},\sigma^{\prime}\right)  \right\}  _{\operatorname*{multi}}.
\]
Thus, Claim 3 is proven in Case 1.

Let us now consider Case 2. In this case, we have $\left[  \pi_{1}>\sigma
_{1}\right]  =\left[  \pi_{1}^{\prime}>\sigma_{1}^{\prime}\right]  =0$.

There clearly exists a positive integer $N$ that is larger than all entries of
$\pi$ and larger than all entries of $\pi^{\prime}$. Consider such an $N$. Set
$m=\left\vert \sigma\right\vert $. Thus, $\sigma=\left(  \sigma_{1},\sigma
_{2},\ldots,\sigma_{m}\right)  $ (since $\left\vert \sigma\right\vert =m$).
Let $\delta$ be the permutation $\left(  \sigma_{1}+N,\sigma_{2}%
+N,\ldots,\sigma_{m}+N\right)  $. This permutation $\delta$ is
order-equivalent to $\sigma$, but is disjoint from $\pi$ (since all its
entries are $>N$, while all the entries of $\pi$ are $<N$) and disjoint from
$\pi^{\prime}$ (for similar reasons). Also, $\delta_{1}=\underbrace{\sigma
_{1}}_{>0}+N>N>\pi_{1}$, so that we don't have $\pi_{1}>\delta_{1}$. Thus,
$\left[  \pi_{1}>\delta_{1}\right]  =0$. Similarly, $\left[  \pi_{1}^{\prime
}>\delta_{1}\right]  =0$.

The permutation $\delta$ is order-equivalent to $\sigma$. Thus,
$\operatorname*{st}\left(  \delta\right)  =\operatorname*{st}\left(
\sigma\right)  $ (since $\operatorname*{st}$ is a permutation statistic) and
$\left\vert \delta\right\vert =\left\vert \sigma\right\vert $. The permutation
$\delta$ is furthermore nonempty (since it is order-equivalent to the nonempty
permutation $\sigma$). Also, $\operatorname*{st}\left(  \delta\right)
=\operatorname*{st}\left(  \sigma\right)  =\operatorname*{st}\left(
\sigma^{\prime}\right)  $ and $\left\vert \delta\right\vert =\left\vert
\sigma\right\vert =\left\vert \sigma^{\prime}\right\vert $. Moreover, $\left[
\pi_{1}>\sigma_{1}\right]  =0=\left[  \pi_{1}>\delta_{1}\right]  $ and
$\left[  \pi_{1}>\delta_{1}\right]  =0=\left[  \pi_{1}^{\prime}>\delta
_{1}\right]  $ and $\left[  \pi_{1}^{\prime}>\delta_{1}\right]  =0=\left[
\pi_{1}^{\prime}>\sigma_{1}^{\prime}\right]  $. Hence, Claim 2 (applied to
$\delta$ instead of $\sigma^{\prime}$) yields%
\[
\left\{  \operatorname*{st}\left(  \tau\right)  \ \mid\ \tau\in S_{\prec
}\left(  \pi,\sigma\right)  \right\}  _{\operatorname*{multi}}=\left\{
\operatorname*{st}\left(  \tau\right)  \ \mid\ \tau\in S_{\prec}\left(
\pi,\delta\right)  \right\}  _{\operatorname*{multi}}%
\]
and%
\begin{equation}
\left\{  \operatorname*{st}\left(  \tau\right)  \ \mid\ \tau\in S_{\succ
}\left(  \pi,\sigma\right)  \right\}  _{\operatorname*{multi}}=\left\{
\operatorname*{st}\left(  \tau\right)  \ \mid\ \tau\in S_{\succ}\left(
\pi,\delta\right)  \right\}  _{\operatorname*{multi}}.\nonumber
\end{equation}
Furthermore, Claim 1 (applied to $\delta$ instead of $\sigma$) yields%
\[
\left\{  \operatorname*{st}\left(  \tau\right)  \ \mid\ \tau\in S_{\prec
}\left(  \pi,\delta\right)  \right\}  _{\operatorname*{multi}}=\left\{
\operatorname*{st}\left(  \tau\right)  \ \mid\ \tau\in S_{\prec}\left(
\pi^{\prime},\delta\right)  \right\}  _{\operatorname*{multi}}%
\]
and%
\[
\left\{  \operatorname*{st}\left(  \tau\right)  \ \mid\ \tau\in S_{\succ
}\left(  \pi,\delta\right)  \right\}  _{\operatorname*{multi}}=\left\{
\operatorname*{st}\left(  \tau\right)  \ \mid\ \tau\in S_{\succ}\left(
\pi^{\prime},\delta\right)  \right\}  _{\operatorname*{multi}}.
\]
Finally, Claim 2 (applied to $\pi^{\prime}$ and $\delta$ instead of $\pi$ and
$\sigma$) yields%
\[
\left\{  \operatorname*{st}\left(  \tau\right)  \ \mid\ \tau\in S_{\prec
}\left(  \pi^{\prime},\delta\right)  \right\}  _{\operatorname*{multi}%
}=\left\{  \operatorname*{st}\left(  \tau\right)  \ \mid\ \tau\in S_{\prec
}\left(  \pi^{\prime},\sigma^{\prime}\right)  \right\}
_{\operatorname*{multi}}%
\]
and%
\begin{equation}
\left\{  \operatorname*{st}\left(  \tau\right)  \ \mid\ \tau\in S_{\succ
}\left(  \pi^{\prime},\delta\right)  \right\}  _{\operatorname*{multi}%
}=\left\{  \operatorname*{st}\left(  \tau\right)  \ \mid\ \tau\in S_{\succ
}\left(  \pi^{\prime},\sigma^{\prime}\right)  \right\}
_{\operatorname*{multi}}.\nonumber
\end{equation}
Combining the equalities we have found, we obtain%
\begin{align*}
\left\{  \operatorname*{st}\left(  \tau\right)  \ \mid\ \tau\in S_{\prec
}\left(  \pi,\sigma\right)  \right\}  _{\operatorname*{multi}}  &  =\left\{
\operatorname*{st}\left(  \tau\right)  \ \mid\ \tau\in S_{\prec}\left(
\pi,\delta\right)  \right\}  _{\operatorname*{multi}}\\
&  =\left\{  \operatorname*{st}\left(  \tau\right)  \ \mid\ \tau\in S_{\prec
}\left(  \pi^{\prime},\delta\right)  \right\}  _{\operatorname*{multi}}\\
&  =\left\{  \operatorname*{st}\left(  \tau\right)  \ \mid\ \tau\in S_{\prec
}\left(  \pi^{\prime},\sigma^{\prime}\right)  \right\}
_{\operatorname*{multi}}.
\end{align*}
The same argument (but with the symbols \textquotedblleft$S_{\prec}%
$\textquotedblright\ and \textquotedblleft$S_{\succ}$\textquotedblright%
\ interchanged) yields
\[
\left\{  \operatorname*{st}\left(  \tau\right)  \ \mid\ \tau\in S_{\succ
}\left(  \pi,\sigma\right)  \right\}  _{\operatorname*{multi}}=\left\{
\operatorname*{st}\left(  \tau\right)  \ \mid\ \tau\in S_{\succ}\left(
\pi^{\prime},\sigma^{\prime}\right)  \right\}  _{\operatorname*{multi}}.
\]
Thus, Claim 3 is proven in Case 2.

We have now proven Claim 3 in each of the two Cases 1 and 2. Hence, Claim 3
always holds.]

Claim 3 says that for any two disjoint nonempty permutations $\pi$ and
$\sigma$, the multisets%
\[
\left\{  \operatorname*{st}\left(  \tau\right)  \ \mid\ \tau\in S_{\prec
}\left(  \pi,\sigma\right)  \right\}  _{\operatorname*{multi}}%
\ \ \ \ \ \ \ \ \ \ \text{and}\ \ \ \ \ \ \ \ \ \ \left\{  \operatorname*{st}%
\left(  \tau\right)  \ \mid\ \tau\in S_{\succ}\left(  \pi,\sigma\right)
\right\}  _{\operatorname*{multi}}%
\]
depend only on $\operatorname*{st}\left(  \pi\right)  $, $\operatorname*{st}%
\left(  \sigma\right)  $, $\left\vert \pi\right\vert $, $\left\vert
\sigma\right\vert $ and $\left[  \pi_{1}>\sigma_{1}\right]  $. In other words,
the statistic $\operatorname*{st}$ is LR-shuffle-compatible (by the definition
of \textquotedblleft LR-shuffle-compatible\textquotedblright). This proves
Theorem \ref{thm.head-comp.LRcomp}.
\end{proof}

Combining Theorem \ref{thm.head-comp.LRcomp} with Proposition
\ref{prop.head-comp.Pks}, we obtain the following:

\begin{theorem}
\label{thm.LRcomp.Pks}\textbf{(a)} The permutation statistic
$\operatorname*{Des}$ is LR-shuffle-compatible.

\textbf{(b)} The permutation statistic $\operatorname*{Lpk}$ is LR-shuffle-compatible.

\textbf{(c)} The permutation statistic $\operatorname*{Epk}$ is LR-shuffle-compatible.
\end{theorem}

\begin{proof}
[Proof of Theorem \ref{thm.LRcomp.Pks}.]\textbf{(a)} The permutation statistic
$\operatorname*{Des}$ is shuffle-compatible (by \cite[\S 2.4]{part1}) and
head-graft-compatible (by Proposition \ref{prop.head-comp.Pks} \textbf{(a)}).
Thus, Theorem \ref{thm.head-comp.LRcomp} (applied to $\operatorname*{st}%
=\operatorname*{Des}$) shows that the permutation statistic
$\operatorname*{Des}$ is LR-shuffle-compatible. This proves Theorem
\ref{thm.LRcomp.Pks} \textbf{(a)}.

\textbf{(b)} The permutation statistic $\operatorname*{Lpk}$ is
shuffle-compatible (by \cite[Theorem 4.9 \textbf{(a)}]{part1}) and
head-graft-compatible (by Proposition \ref{prop.head-comp.Pks} \textbf{(b)}).
Thus, Theorem \ref{thm.head-comp.LRcomp} (applied to $\operatorname*{st}%
=\operatorname*{Lpk}$) shows that the permutation statistic
$\operatorname*{Lpk}$ is LR-shuffle-compatible. This proves Theorem
\ref{thm.LRcomp.Pks} \textbf{(b)}.

\textbf{(c)} The permutation statistic $\operatorname*{Epk}$ is
shuffle-compatible (by Theorem \ref{thm.Epk.sh-co-a}) and
head-graft-compatible (by Proposition \ref{prop.head-comp.Pks} \textbf{(c)}).
Thus, Theorem \ref{thm.head-comp.LRcomp} (applied to $\operatorname*{st}%
=\operatorname*{Epk}$) shows that the permutation statistic
$\operatorname*{Epk}$ is LR-shuffle-compatible. This proves Theorem
\ref{thm.LRcomp.Pks} \textbf{(c)}.
\end{proof}

\subsection{\label{subsect.LR.others}Some other statistics}

The question of LR-shuffle-compatibility can be asked about any statistic. We
have so far answered it for $\operatorname*{Des}$, $\operatorname*{Pk}$,
$\operatorname*{Lpk}$, $\operatorname*{Rpk}$ and $\operatorname*{Epk}$. In
this section, we shall analyze it for some further statistics.

\subsubsection{The descent number $\operatorname*{des}$}

The permutation statistic $\operatorname*{des}$ (called the \textit{descent
number}) is defined as follows: For each permutation $\pi$, we set
$\operatorname*{des}\pi=\left\vert \operatorname*{Des}\pi\right\vert $ (that
is, $\operatorname*{des}\pi$ is the number of all descents of $\pi$). It was
proven in \cite[Theorem 4.6 \textbf{(a)}]{part1} that this statistic
$\operatorname*{des}$ is shuffle-compatible. We now claim the following:

\begin{proposition}
\label{prop.LRcomp.des}The permutation statistic $\operatorname*{des}$ is
head-graft-compatible and LR-shuffle-compatible.
\end{proposition}

\begin{proof}
[Proof of Proposition \ref{prop.LRcomp.des}.]From
(\ref{pf.prop.head-comp.Pks.a.1}), we easily obtain the following: If $\pi$ is
a nonempty permutation, and if $a$ is a letter that does not appear in $\pi$,
then%
\[
\operatorname*{des}\left(  a:\pi\right)  =\operatorname*{des}\pi+\left[
a>\pi_{1}\right]  .
\]
Thus, $\operatorname*{des}\left(  a:\pi\right)  $ depends only on
$\operatorname*{des}\pi$, $\left\vert \pi\right\vert $ and $\left[  a>\pi
_{1}\right]  $. In other words, $\operatorname*{des}$ is head-graft-compatible
(by the definition of \textquotedblleft
head-graft-compatible\textquotedblright). Hence, Theorem
\ref{thm.head-comp.LRcomp} (applied to $\operatorname*{st}=\operatorname*{des}%
$) shows that the permutation statistic $\operatorname*{des}$ is
LR-shuffle-compatible. This proves Proposition \ref{prop.LRcomp.des}.
\end{proof}

\subsubsection{The major index $\operatorname*{maj}$}

The permutation statistic $\operatorname*{maj}$ (called the \textit{major
index}) is defined as follows: For each permutation $\pi$, we set
$\operatorname*{maj}\pi=\sum_{i\in\operatorname*{Des}\pi}i$ (that is,
$\operatorname*{maj}\pi$ is the sum of all descents of $\pi$). It was proven
in \cite[Theorem 3.1 \textbf{(a)}]{part1} that this statistic
$\operatorname*{maj}$ is shuffle-compatible.

\begin{vershort}
However, $\operatorname*{maj}$ is neither head-graft-compatible nor
LR-shuffle-compatible. For example, if we take $\pi=\left(  5,4,2,3\right)  $,
$a=1$, $\pi^{\prime}=\left(  3,4,5,2\right)  $ and $a^{\prime}=1$, then we do
have%
\[
\operatorname*{maj}\left(  \pi\right)  =\operatorname*{maj}\left(  \pi
^{\prime}\right)  ,\ \ \ \ \ \ \ \ \ \ \left\vert \pi\right\vert =\left\vert
\pi^{\prime}\right\vert \ \ \ \ \ \ \ \ \ \ \text{and}%
\ \ \ \ \ \ \ \ \ \ \left[  a>\pi_{1}\right]  =\left[  a^{\prime}>\pi
_{1}^{\prime}\right]
\]
but we don't have $\operatorname*{maj}\left(  a:\pi\right)
=\operatorname*{maj}\left(  a^{\prime}:\pi^{\prime}\right)  $. Thus,
$\operatorname*{maj}$ is not head-graft-compatible. Using Corollary
\ref{cor.LRcomp.back} below, this entails that
$\operatorname*{maj}$ is not LR-shuffle-compatible.
\end{vershort}

\begin{verlong}
However, $\operatorname*{maj}$ is neither head-graft-compatible nor
LR-shuffle-compatible. For example, if we take $\pi=\left(  5,4,2,3\right)  $,
$a=1$, $\pi^{\prime}=\left(  3,4,5,2\right)  $ and $a^{\prime}=1$, then we do
have%
\[
\operatorname*{maj}\left(  \pi\right)  =\operatorname*{maj}\left(  \pi
^{\prime}\right)  ,\ \ \ \ \ \ \ \ \ \ \left\vert \pi\right\vert =\left\vert
\pi^{\prime}\right\vert \ \ \ \ \ \ \ \ \ \ \text{and}%
\ \ \ \ \ \ \ \ \ \ \left[  a>\pi_{1}\right]  =\left[  a^{\prime}>\pi
_{1}^{\prime}\right]
\]
but we don't have $\operatorname*{maj}\left(  a:\pi\right)
=\operatorname*{maj}\left(  a^{\prime}:\pi^{\prime}\right)  $. Thus,
$\operatorname*{maj}$ is not head-graft-compatible. Using Proposition
\ref{prop.LRcomp.head} below, this entails that $\operatorname*{maj}$ is not LR-shuffle-compatible.
\end{verlong}

\subsubsection{The joint statistic $\left(  \operatorname*{des}%
,\operatorname*{maj}\right)  $}

The next permutation statistic we shall study is the so-called joint statistic
$\left(  \operatorname*{des},\operatorname*{maj}\right)  $. This statistic is
defined as the permutation statistic that sends each permutation $\pi$ to the
ordered pair $\left(  \operatorname*{des}\pi,\operatorname*{maj}\pi\right)  $.
(Calling it $\left(  \operatorname*{des},\operatorname*{maj}\right)  $ is thus
a slight abuse of notation.) It was proven in \cite[Theorem 4.5 \textbf{(a)}%
]{part1} that this statistic $\left(  \operatorname*{des},\operatorname*{maj}%
\right)  $ is shuffle-compatible. We now claim the following:

\begin{proposition}
\label{prop.LRcomp.desmaj}The permutation statistic $\left(
\operatorname*{des},\operatorname*{maj}\right)  $ is head-graft-compatible and LR-shuffle-compatible.
\end{proposition}

\begin{proof}
[Proof of Proposition \ref{prop.LRcomp.desmaj}.]From
(\ref{pf.prop.head-comp.Pks.a.1}), we easily obtain the following: If $\pi$ is
a nonempty permutation, and if $a$ is a letter that does not appear in $\pi$,
then%
\begin{align*}
\operatorname*{des}\left(  a:\pi\right)   &  =\operatorname*{des}\pi+\left[
a>\pi_{1}\right]  \ \ \ \ \ \ \ \ \ \ \text{and}\\
\operatorname*{maj}\left(  a:\pi\right)   &  =\operatorname*{des}%
\pi+\operatorname*{maj}\pi+\left[  a>\pi_{1}\right]  .
\end{align*}
Thus, $\left(  \operatorname*{des},\operatorname*{maj}\right)  \left(
a:\pi\right)  $ depends only on $\left(  \operatorname*{des}%
,\operatorname*{maj}\right)  \left(  \pi\right)  $, $\left\vert \pi\right\vert
$ and $\left[  a>\pi_{1}\right]  $. In other words, $\left(
\operatorname*{des},\operatorname*{maj}\right)  $ is head-graft-compatible (by
the definition of \textquotedblleft head-graft-compatible\textquotedblright).
Hence, Theorem \ref{thm.head-comp.LRcomp} (applied to $\operatorname*{st}%
=\left(  \operatorname*{des},\operatorname*{maj}\right)  $) shows that the
permutation statistic $\left(  \operatorname*{des},\operatorname*{maj}\right)
$ is LR-shuffle-compatible. This proves Proposition \ref{prop.LRcomp.desmaj}.
\end{proof}

\subsubsection{The comajor index $\operatorname*{comaj}$}

The permutation statistic $\operatorname*{comaj}$ (called the \textit{comajor
index}) is defined as follows: For each permutation $\pi$, we set
$\operatorname*{comaj}\pi=\sum_{k\in\operatorname*{Des}\pi}\left(  n-k\right)
$, where $n=\left\vert \pi\right\vert $. It was proven in \cite[\S 3.2]{part1}
that this statistic $\operatorname*{comaj}$ is shuffle-compatible. We now
claim the following:

\begin{proposition}
\label{prop.LRcomp.comaj}The permutation statistic $\operatorname*{comaj}$ is
head-graft-compatible and LR-shuffle-compatible.
\end{proposition}

\begin{proof}
[Proof of Proposition \ref{prop.LRcomp.des}.]From
(\ref{pf.prop.head-comp.Pks.a.1}), we easily obtain the following: If $\pi$ is
a nonempty permutation, and if $a$ is a letter that does not appear in $\pi$,
then%
\[
\operatorname*{comaj}\left(  a:\pi\right)  =\operatorname*{comaj}\pi+\left[
a>\pi_{1}\right]  \cdot\left\vert \pi\right\vert .
\]
Thus, $\operatorname*{comaj}\left(  a:\pi\right)  $ depends only on
$\operatorname*{comaj}\pi$, $\left\vert \pi\right\vert $ and $\left[
a>\pi_{1}\right]  $. In other words, $\operatorname*{comaj}$ is
head-graft-compatible (by the definition of \textquotedblleft
head-graft-compatible\textquotedblright). Hence, Theorem
\ref{thm.head-comp.LRcomp} (applied to $\operatorname*{st}%
=\operatorname*{comaj}$) shows that the permutation statistic
$\operatorname*{comaj}$ is LR-shuffle-compatible. This proves Proposition
\ref{prop.LRcomp.comaj}.
\end{proof}

\subsection{Left- and right-shuffle-compatibility}

In this section, we shall study two notions closely related to LR-shuffle-compatibility:

\begin{definition}
\label{def.LR.left-right}Let $\operatorname*{st}$ be a permutation statistic.

\textbf{(a)} We say that $\operatorname*{st}$ is
\textit{left-shuffle-compatible} if for any two disjoint nonempty permutations
$\pi$ and $\sigma$ having the property that $\pi_{1}>\sigma_{1}$, the multiset
$\left\{  \operatorname*{st}\left(  \tau\right)  \ \mid\ \tau\in S_{\prec
}\left(  \pi,\sigma\right)  \right\}  _{\operatorname*{multi}}$ depends only
on $\operatorname*{st}\left(  \pi\right)  $, $\operatorname*{st}\left(
\sigma\right)  $, $\left\vert \pi\right\vert $ and $\left\vert \sigma
\right\vert $.

\textbf{(b)} We say that $\operatorname*{st}$ is
\textit{right-shuffle-compatible} if for any two disjoint nonempty
permutations $\pi$ and $\sigma$ having the property that $\pi_{1}>\sigma_{1}$,
the multiset $\left\{  \operatorname*{st}\left(  \tau\right)  \ \mid\ \tau\in
S_{\succ}\left(  \pi,\sigma\right)  \right\}  _{\operatorname*{multi}}$
depends only on $\operatorname*{st}\left(  \pi\right)  $, $\operatorname*{st}%
\left(  \sigma\right)  $, $\left\vert \pi\right\vert $ and $\left\vert
\sigma\right\vert $.
\end{definition}

For a shuffle-compatible permutation statistic, these two notions are
equivalent to the notions of LR-shuffle-compatibility and
head-graft-compatibility, as the following proposition reveals:

\begin{proposition}
\label{prop.LRcomp.equivs}Let $\operatorname*{st}$ be a shuffle-compatible
permutation statistic. Then, the following assertions are equivalent:

\begin{itemize}
\item \textit{Assertion }$\mathcal{A}_{1}$\textit{:} The statistic
$\operatorname*{st}$ is LR-shuffle-compatible.

\item \textit{Assertion }$\mathcal{A}_{2}$\textit{:} The statistic
$\operatorname*{st}$ is left-shuffle-compatible.

\item \textit{Assertion }$\mathcal{A}_{3}$\textit{:} The statistic
$\operatorname*{st}$ is right-shuffle-compatible.

\item \textit{Assertion }$\mathcal{A}_{4}$\textit{:} The statistic
$\operatorname*{st}$ is head-graft-compatible.
\end{itemize}
\end{proposition}

\begin{vershort}
\begin{proof}
[Proof of Proposition \ref{prop.LRcomp.equivs}.]Omitted; see \cite{verlong}.
\end{proof}
\end{vershort}

\begin{verlong}
\begin{proof}
[Proof of Proposition \ref{prop.LRcomp.equivs} (sketched).]We shall prove the
implications $\mathcal{A}_{1}\Longrightarrow\mathcal{A}_{2}$, $\mathcal{A}%
_{2}\Longrightarrow\mathcal{A}_{3}$, $\mathcal{A}_{3}\Longrightarrow
\mathcal{A}_{2}$, $\mathcal{A}_{3}\Longrightarrow\mathcal{A}_{4}$ and
$\mathcal{A}_{4}\Longrightarrow\mathcal{A}_{1}$.

The implication $\mathcal{A}_{4}\Longrightarrow\mathcal{A}_{1}$ follows from
Theorem \ref{thm.head-comp.LRcomp}.

\begin{noncompile}
Let us first recall that $\operatorname*{st}$ is shuffle-compatible. In other
words, the following holds:

\begin{statement}
\textit{Claim 0:} Let $\pi$ and $\sigma$ be two disjoint permutations. Let
$\pi^{\prime}$ and $\sigma^{\prime}$ be two disjoint permutations. Assume that%
\begin{align*}
\operatorname*{st}\left(  \pi\right)   &  =\operatorname*{st}\left(
\pi^{\prime}\right)  ,\ \ \ \ \ \ \ \ \ \ \operatorname*{st}\left(
\sigma\right)  =\operatorname*{st}\left(  \sigma^{\prime}\right)  ,\\
\left\vert \pi\right\vert  &  =\left\vert \pi^{\prime}\right\vert
\ \ \ \ \ \ \ \ \ \ \text{and}\ \ \ \ \ \ \ \ \ \ \left\vert \sigma\right\vert
=\left\vert \sigma^{\prime}\right\vert .
\end{align*}
Then,
\[
\left\{  \operatorname*{st}\left(  \tau\right)  \ \mid\ \tau\in S\left(
\pi,\sigma\right)  \right\}  _{\operatorname*{multi}}=\left\{
\operatorname*{st}\left(  \tau\right)  \ \mid\ \tau\in S\left(  \pi^{\prime
},\sigma^{\prime}\right)  \right\}  _{\operatorname*{multi}}.
\]

\end{statement}
\end{noncompile}

\textit{Proof of the implication }$\mathcal{A}_{1}\Longrightarrow
\mathcal{A}_{2}$\textit{:} Assume that Assertion $\mathcal{A}_{1}$ holds. In
other words, the statistic $\operatorname*{st}$ is LR-shuffle-compatible. We
shall show that Assertion $\mathcal{A}_{2}$ holds.

The statistic $\operatorname*{st}$ is LR-shuffle-compatible. In other words,
for any two disjoint nonempty permutations $\pi$ and $\sigma$, the multisets%
\[
\left\{  \operatorname*{st}\left(  \tau\right)  \ \mid\ \tau\in S_{\prec
}\left(  \pi,\sigma\right)  \right\}  _{\operatorname*{multi}}%
\ \ \ \ \ \ \ \ \ \ \text{and}\ \ \ \ \ \ \ \ \ \ \left\{  \operatorname*{st}%
\left(  \tau\right)  \ \mid\ \tau\in S_{\succ}\left(  \pi,\sigma\right)
\right\}  _{\operatorname*{multi}}%
\]
depend only on $\operatorname*{st}\left(  \pi\right)  $, $\operatorname*{st}%
\left(  \sigma\right)  $, $\left\vert \pi\right\vert $, $\left\vert
\sigma\right\vert $ and $\left[  \pi_{1}>\sigma_{1}\right]  $. Hence, for any
two disjoint nonempty permutations $\pi$ and $\sigma$, the multiset $\left\{
\operatorname*{st}\left(  \tau\right)  \ \mid\ \tau\in S_{\prec}\left(
\pi,\sigma\right)  \right\}  _{\operatorname*{multi}}$ depends only on
$\operatorname*{st}\left(  \pi\right)  $, $\operatorname*{st}\left(
\sigma\right)  $, $\left\vert \pi\right\vert $, $\left\vert \sigma\right\vert
$ and $\left[  \pi_{1}>\sigma_{1}\right]  $. Hence, for any two disjoint
nonempty permutations $\pi$ and $\sigma$ having the property that $\pi
_{1}>\sigma_{1}$, the multiset \newline$\left\{  \operatorname*{st}\left(
\tau\right)  \ \mid\ \tau\in S_{\prec}\left(  \pi,\sigma\right)  \right\}
_{\operatorname*{multi}}$ depends only on $\operatorname*{st}\left(
\pi\right)  $, $\operatorname*{st}\left(  \sigma\right)  $, $\left\vert
\pi\right\vert $ and $\left\vert \sigma\right\vert $ (indeed, it no longer
depends on $\left[  \pi_{1}>\sigma_{1}\right]  $, because our condition
$\pi_{1}>\sigma_{1}$ ensures that $\left[  \pi_{1}>\sigma_{1}\right]  =1$). In
other words, the statistic $\operatorname*{st}$ is left-shuffle-compatible (by
the definition of \textquotedblleft left-shuffle-compatible\textquotedblright%
). In other words, Assertion $\mathcal{A}_{2}$ holds. This proves the
implication $\mathcal{A}_{1}\Longrightarrow\mathcal{A}_{2}$.

\textit{Proof of the implication }$\mathcal{A}_{2}\Longrightarrow
\mathcal{A}_{3}$\textit{:} Assume that Assertion $\mathcal{A}_{2}$ holds. In
other words, the statistic $\operatorname*{st}$ is left-shuffle-compatible. We
shall show that Assertion $\mathcal{A}_{3}$ holds.

If $\pi$ and $\sigma$ are two disjoint nonempty permutations, then%
\begin{align*}
&  \left\{  \operatorname*{st}\left(  \tau\right)  \ \mid\ \tau\in S_{\succ
}\left(  \pi,\sigma\right)  \right\}  _{\operatorname*{multi}}\\
&  =\left\{  \operatorname*{st}\left(  \tau\right)  \ \mid\ \tau\in S\left(
\pi,\sigma\right)  \right\}  _{\operatorname*{multi}}-\left\{
\operatorname*{st}\left(  \tau\right)  \ \mid\ \tau\in S_{\prec}\left(
\pi,\sigma\right)  \right\}  _{\operatorname*{multi}}%
\end{align*}
(by Lemma \ref{lem.LR.difference} \textbf{(b)}). Hence, if $\pi$ and $\sigma$
are two disjoint nonempty permutations, then the multiset $\left\{
\operatorname*{st}\left(  \tau\right)  \ \mid\ \tau\in S_{\succ}\left(
\pi,\sigma\right)  \right\}  _{\operatorname*{multi}}$ is uniquely determined
by $\left\{  \operatorname*{st}\left(  \tau\right)  \ \mid\ \tau\in S\left(
\pi,\sigma\right)  \right\}  _{\operatorname*{multi}}$ and $\left\{
\operatorname*{st}\left(  \tau\right)  \ \mid\ \tau\in S_{\prec}\left(
\pi,\sigma\right)  \right\}  _{\operatorname*{multi}}$.

Thus, altogether, we know that for any two disjoint nonempty permutations
$\pi$ and $\sigma$ having the property that $\pi_{1}>\sigma_{1}$, the
following holds:

\begin{itemize}
\item The multiset $\left\{  \operatorname*{st}\left(  \tau\right)
\ \mid\ \tau\in S_{\prec}\left(  \pi,\sigma\right)  \right\}
_{\operatorname*{multi}}$ depends only on $\operatorname*{st}\left(
\pi\right)  $, $\operatorname*{st}\left(  \sigma\right)  $, $\left\vert
\pi\right\vert $ and $\left\vert \sigma\right\vert $ (since the statistic
$\operatorname*{st}$ is left-shuffle-compatible);

\item The multiset $\left\{  \operatorname*{st}\left(  \tau\right)
\ \mid\ \tau\in S\left(  \pi,\sigma\right)  \right\}  _{\operatorname*{multi}%
}$ depends only on $\operatorname*{st}\left(  \pi\right)  $,
$\operatorname*{st}\left(  \sigma\right)  $, $\left\vert \pi\right\vert $ and
$\left\vert \sigma\right\vert $ (since the statistic $\operatorname*{st}$ is shuffle-compatible);

\item The multiset $\left\{  \operatorname*{st}\left(  \tau\right)
\ \mid\ \tau\in S_{\succ}\left(  \pi,\sigma\right)  \right\}
_{\operatorname*{multi}}$ is uniquely determined by \newline$\left\{
\operatorname*{st}\left(  \tau\right)  \ \mid\ \tau\in S\left(  \pi
,\sigma\right)  \right\}  _{\operatorname*{multi}}$ and $\left\{
\operatorname*{st}\left(  \tau\right)  \ \mid\ \tau\in S_{\prec}\left(
\pi,\sigma\right)  \right\}  _{\operatorname*{multi}}$.
\end{itemize}

Hence, the multiset $\left\{  \operatorname*{st}\left(  \tau\right)
\ \mid\ \tau\in S_{\succ}\left(  \pi,\sigma\right)  \right\}
_{\operatorname*{multi}}$ also depends only on $\operatorname*{st}\left(
\pi\right)  $, $\operatorname*{st}\left(  \sigma\right)  $, $\left\vert
\pi\right\vert $ and $\left\vert \sigma\right\vert $. In other words, the
statistic $\operatorname*{st}$ is right-shuffle-compatible (by the definition
of \textquotedblleft right-shuffle-compatible\textquotedblright). In other
words, Assertion $\mathcal{A}_{3}$ holds. This proves the implication
$\mathcal{A}_{2}\Longrightarrow\mathcal{A}_{3}$.

\textit{Proof of the implication }$\mathcal{A}_{3}\Longrightarrow
\mathcal{A}_{2}$\textit{:} The proof of the implication $\mathcal{A}%
_{3}\Longrightarrow\mathcal{A}_{2}$ is entirely analogous to the above proof
of the implication $\mathcal{A}_{2}\Longrightarrow\mathcal{A}_{3}$ (except
that we need to interchange \textquotedblleft left\textquotedblright\ and
\textquotedblleft right\textquotedblright\ and likewise interchange
\textquotedblleft$S_{\prec}$\textquotedblright\ and \textquotedblleft%
$S_{\succ}$\textquotedblright).

\textit{Proof of the implication }$\mathcal{A}_{3}\Longrightarrow
\mathcal{A}_{4}$\textit{:} Assume that Assertion $\mathcal{A}_{3}$ holds. We
shall show that Assertion $\mathcal{A}_{4}$ holds.

We have already proven the implication $\mathcal{A}_{3}\Longrightarrow
\mathcal{A}_{2}$. Thus, Assertion $\mathcal{A}_{2}$ holds (since
$\mathcal{A}_{3}$ holds). In other words, the statistic $\operatorname*{st}$
is left-shuffle-compatible. In other words, the following claim holds:

\begin{statement}
\textit{Claim 1:} Let $\pi$ and $\sigma$ be two disjoint nonempty permutations
having the property that $\pi_{1}>\sigma_{1}$. Let $\pi^{\prime}$ and
$\sigma^{\prime}$ be two disjoint nonempty permutations having the property
that $\pi_{1}^{\prime}>\sigma_{1}^{\prime}$. Assume that%
\begin{align*}
\operatorname*{st}\left(  \pi\right)   &  =\operatorname*{st}\left(
\pi^{\prime}\right)  ,\ \ \ \ \ \ \ \ \ \ \operatorname*{st}\left(
\sigma\right)  =\operatorname*{st}\left(  \sigma^{\prime}\right)  ,\\
\left\vert \pi\right\vert  &  =\left\vert \pi^{\prime}\right\vert
\ \ \ \ \ \ \ \ \ \ \text{and}\ \ \ \ \ \ \ \ \ \ \left\vert \sigma\right\vert
=\left\vert \sigma^{\prime}\right\vert .
\end{align*}
Then,
\[
\left\{  \operatorname*{st}\left(  \tau\right)  \ \mid\ \tau\in S_{\prec
}\left(  \pi,\sigma\right)  \right\}  _{\operatorname*{multi}}=\left\{
\operatorname*{st}\left(  \tau\right)  \ \mid\ \tau\in S_{\prec}\left(
\pi^{\prime},\sigma^{\prime}\right)  \right\}  _{\operatorname*{multi}}.
\]

\end{statement}

Also, Assertion $\mathcal{A}_{3}$ holds. In other words, the statistic
$\operatorname*{st}$ is right-shuffle-compatible. In other words, the
following claim holds:

\begin{statement}
\textit{Claim 2:} Let $\pi$ and $\sigma$ be two disjoint nonempty permutations
having the property that $\pi_{1}>\sigma_{1}$. Let $\pi^{\prime}$ and
$\sigma^{\prime}$ be two disjoint nonempty permutations having the property
that $\pi_{1}^{\prime}>\sigma_{1}^{\prime}$. Assume that%
\begin{align*}
\operatorname*{st}\left(  \pi\right)   &  =\operatorname*{st}\left(
\pi^{\prime}\right)  ,\ \ \ \ \ \ \ \ \ \ \operatorname*{st}\left(
\sigma\right)  =\operatorname*{st}\left(  \sigma^{\prime}\right)  ,\\
\left\vert \pi\right\vert  &  =\left\vert \pi^{\prime}\right\vert
\ \ \ \ \ \ \ \ \ \ \text{and}\ \ \ \ \ \ \ \ \ \ \left\vert \sigma\right\vert
=\left\vert \sigma^{\prime}\right\vert .
\end{align*}
Then,
\[
\left\{  \operatorname*{st}\left(  \tau\right)  \ \mid\ \tau\in S_{\succ
}\left(  \pi,\sigma\right)  \right\}  _{\operatorname*{multi}}=\left\{
\operatorname*{st}\left(  \tau\right)  \ \mid\ \tau\in S_{\succ}\left(
\pi^{\prime},\sigma^{\prime}\right)  \right\}  _{\operatorname*{multi}}.
\]

\end{statement}

We are now going to prove the following claim:

\begin{statement}
\textit{Claim 3:} Let $\pi$ be a nonempty permutation, and let $a$ be a letter
that does not appear in $\pi$. Let $\pi^{\prime}$ be a nonempty permutation,
and let $a^{\prime}$ be a letter that does not appear in $\pi^{\prime}$.
Assume that%
\[
\operatorname*{st}\left(  \pi\right)  =\operatorname*{st}\left(  \pi^{\prime
}\right)  ,\ \ \ \ \ \ \ \ \ \ \left\vert \pi\right\vert =\left\vert
\pi^{\prime}\right\vert \ \ \ \ \ \ \ \ \ \ \text{and}%
\ \ \ \ \ \ \ \ \ \ \left[  a>\pi_{1}\right]  =\left[  a^{\prime}>\pi
_{1}^{\prime}\right]  .
\]
Then, $\operatorname*{st}\left(  a:\pi\right)  =\operatorname*{st}\left(
a^{\prime}:\pi^{\prime}\right)  $.
\end{statement}

[\textit{Proof of Claim 3:} The $1$-permutations $\left(  a\right)  $ and
$\left(  a^{\prime}\right)  $ are order-equivalent. Thus, $\operatorname*{st}%
\left(  \left(  a\right)  \right)  =\operatorname*{st}\left(  \left(
a^{\prime}\right)  \right)  $ (since $\operatorname*{st}$ is a permutation
statistic). Also, $\left\vert \left(  a\right)  \right\vert =1=\left\vert
\left(  a^{\prime}\right)  \right\vert $.

We are in one of the following two cases:

\textit{Case 1:} We have $a>\pi_{1}$.

\textit{Case 2:} We have $a\leq\pi_{1}$.

Let us first consider Case 1. In this case, we have $a>\pi_{1}$. Thus,
$\left[  a>\pi_{1}\right]  =1$. Comparing this with $\left[  a>\pi_{1}\right]
=\left[  a^{\prime}>\pi_{1}^{\prime}\right]  $, we obtain $\left[  a^{\prime
}>\pi_{1}^{\prime}\right]  =1$, so that $a^{\prime}>\pi_{1}^{\prime}$.

Consider the $1$-permutations $\left(  a\right)  $ and $\left(  a^{\prime
}\right)  $. Then, the first entry of $\left(  a\right)  $ is $\left(
a\right)  _{1}=a>\pi_{1}$. Also, $\left(  a^{\prime}\right)  _{1}=a^{\prime
}>\pi_{1}^{\prime}$. Also, the permutations $\left(  a\right)  $ and $\pi$ are
disjoint (since $a$ does not appear in $\pi$). Similarly, the permutations
$\left(  a^{\prime}\right)  $ and $\pi^{\prime}$ are disjoint. Furthermore,
$\operatorname*{st}\left(  \left(  a\right)  \right)  =\operatorname*{st}%
\left(  \left(  a^{\prime}\right)  \right)  $, $\operatorname*{st}\left(
\pi\right)  =\operatorname*{st}\left(  \pi^{\prime}\right)  $, $\left\vert
\left(  a\right)  \right\vert =\left\vert \left(  a^{\prime}\right)
\right\vert $ and $\left\vert \pi\right\vert =\left\vert \pi^{\prime
}\right\vert $. Hence, Claim 1 (applied to $\left(  a\right)  $, $\pi$,
$\left(  a^{\prime}\right)  $ and $\pi^{\prime}$ instead of $\pi$, $\sigma$,
$\pi^{\prime}$ and $\sigma^{\prime}$) yields%
\begin{equation}
\left\{  \operatorname*{st}\left(  \tau\right)  \ \mid\ \tau\in S_{\prec
}\left(  \left(  a\right)  ,\pi\right)  \right\}  _{\operatorname*{multi}%
}=\left\{  \operatorname*{st}\left(  \tau\right)  \ \mid\ \tau\in S_{\prec
}\left(  \left(  a^{\prime}\right)  ,\pi^{\prime}\right)  \right\}
_{\operatorname*{multi}}. \label{pf.prop.LRcomp.equivs.3to4.4}%
\end{equation}
But $S_{\prec}\left(  \left(  a\right)  ,\pi\right)  =\left\{  a:\pi\right\}
$. Hence,%
\[
\left\{  \operatorname*{st}\left(  \tau\right)  \ \mid\ \tau\in S_{\prec
}\left(  \left(  a\right)  ,\pi\right)  \right\}  _{\operatorname*{multi}%
}=\left\{  \operatorname*{st}\left(  \tau\right)  \ \mid\ \tau\in\left\{
a:\pi\right\}  \right\}  _{\operatorname*{multi}}=\left\{  \operatorname*{st}%
\left(  a:\pi\right)  \right\}  _{\operatorname*{multi}}.
\]
Similarly,%
\[
\left\{  \operatorname*{st}\left(  \tau\right)  \ \mid\ \tau\in S_{\prec
}\left(  \left(  a^{\prime}\right)  ,\pi^{\prime}\right)  \right\}
_{\operatorname*{multi}}=\left\{  \operatorname*{st}\left(  a^{\prime}%
:\pi^{\prime}\right)  \right\}  _{\operatorname*{multi}}.
\]
In light of the last two equalities, the equality
(\ref{pf.prop.LRcomp.equivs.3to4.4}) rewrites as $\left\{  \operatorname*{st}%
\left(  a:\pi\right)  \right\}  _{\operatorname*{multi}}=\left\{
\operatorname*{st}\left(  a^{\prime}:\pi^{\prime}\right)  \right\}
_{\operatorname*{multi}}$. In other words, $\operatorname*{st}\left(
a:\pi\right)  =\operatorname*{st}\left(  a^{\prime}:\pi^{\prime}\right)  $.
Thus, we have proven $\operatorname*{st}\left(  a:\pi\right)
=\operatorname*{st}\left(  a^{\prime}:\pi^{\prime}\right)  $ in Case 1.

Let us now consider Case 2. In this case, we have $a\leq\pi_{1}$. But $a$ does
not appear in $\pi$; therefore, $a\neq\pi_{1}$. Combined with $a\leq\pi_{1}$,
this yields $a<\pi_{1}$. In other words, $\pi_{1}>a$. Also, from $a\leq\pi
_{1}$, we obtain $\left[  a>\pi_{1}\right]  =0$. Comparing this with $\left[
a>\pi_{1}\right]  =\left[  a^{\prime}>\pi_{1}^{\prime}\right]  $, we obtain
$\left[  a^{\prime}>\pi_{1}^{\prime}\right]  =0$, so that $a^{\prime}\leq
\pi_{1}^{\prime}$. But $a^{\prime}$ does not appear in $\pi^{\prime}$; thus,
$a^{\prime}\neq\pi_{1}^{\prime}$. Combined with $a^{\prime}\leq\pi_{1}%
^{\prime}$, this yields $a^{\prime}<\pi_{1}^{\prime}$. In other words,
$\pi_{1}^{\prime}>a^{\prime}$.

Consider the $1$-permutations $\left(  a\right)  $ and $\left(  a^{\prime
}\right)  $. Then, the first entry of $\left(  a\right)  $ is $\left(
a\right)  _{1}=a$; similarly, $\left(  a^{\prime}\right)  _{1}=a^{\prime}$.
Now, $\pi_{1}>a=\left(  a\right)  _{1}$ and $\pi_{1}^{\prime}>a^{\prime
}=\left(  a^{\prime}\right)  _{1}$. Also, the permutations $\pi$ and $\left(
a\right)  $ are disjoint (since $a$ does not appear in $\pi$). Similarly, the
permutations $\pi^{\prime}$ and $\left(  a^{\prime}\right)  $ are disjoint.
Furthermore, $\operatorname*{st}\left(  \left(  a\right)  \right)
=\operatorname*{st}\left(  \left(  a^{\prime}\right)  \right)  $,
$\operatorname*{st}\left(  \pi\right)  =\operatorname*{st}\left(  \pi^{\prime
}\right)  $, $\left\vert \left(  a\right)  \right\vert =\left\vert \left(
a^{\prime}\right)  \right\vert $ and $\left\vert \pi\right\vert =\left\vert
\pi^{\prime}\right\vert $. Hence, Claim 2 (applied to $\pi$, $\left(
a\right)  $, $\pi^{\prime}$ and $\left(  a^{\prime}\right)  $ instead of $\pi
$, $\sigma$, $\pi^{\prime}$ and $\sigma^{\prime}$) yields%
\begin{equation}
\left\{  \operatorname*{st}\left(  \tau\right)  \ \mid\ \tau\in S_{\succ
}\left(  \pi,\left(  a\right)  \right)  \right\}  _{\operatorname*{multi}%
}=\left\{  \operatorname*{st}\left(  \tau\right)  \ \mid\ \tau\in S_{\succ
}\left(  \pi^{\prime},\left(  a^{\prime}\right)  \right)  \right\}
_{\operatorname*{multi}}. \label{pf.prop.LRcomp.equivs.3to4.5}%
\end{equation}
But $S_{\succ}\left(  \pi,\left(  a\right)  \right)  =\left\{  a:\pi\right\}
$. Hence,%
\[
\left\{  \operatorname*{st}\left(  \tau\right)  \ \mid\ \tau\in S_{\succ
}\left(  \pi,\left(  a\right)  \right)  \right\}  _{\operatorname*{multi}%
}=\left\{  \operatorname*{st}\left(  \tau\right)  \ \mid\ \tau\in\left\{
a:\pi\right\}  \right\}  _{\operatorname*{multi}}=\left\{  \operatorname*{st}%
\left(  a:\pi\right)  \right\}  _{\operatorname*{multi}}.
\]
Similarly,%
\[
\left\{  \operatorname*{st}\left(  \tau\right)  \ \mid\ \tau\in S_{\succ
}\left(  \pi^{\prime},\left(  a^{\prime}\right)  \right)  \right\}
_{\operatorname*{multi}}=\left\{  \operatorname*{st}\left(  a^{\prime}%
:\pi^{\prime}\right)  \right\}  _{\operatorname*{multi}}.
\]
In light of the last two equalities, the equality
(\ref{pf.prop.LRcomp.equivs.3to4.5}) rewrites as $\left\{  \operatorname*{st}%
\left(  a:\pi\right)  \right\}  _{\operatorname*{multi}}=\left\{
\operatorname*{st}\left(  a^{\prime}:\pi^{\prime}\right)  \right\}
_{\operatorname*{multi}}$. In other words, $\operatorname*{st}\left(
a:\pi\right)  =\operatorname*{st}\left(  a^{\prime}:\pi^{\prime}\right)  $.
Thus, we have proven $\operatorname*{st}\left(  a:\pi\right)
=\operatorname*{st}\left(  a^{\prime}:\pi^{\prime}\right)  $ in Case 2.

We have now proven $\operatorname*{st}\left(  a:\pi\right)
=\operatorname*{st}\left(  a^{\prime}:\pi^{\prime}\right)  $ in both Cases 1
and 2. Therefore, $\operatorname*{st}\left(  a:\pi\right)  =\operatorname*{st}%
\left(  a^{\prime}:\pi^{\prime}\right)  $ always holds. This proves Claim 3.]

Claim 3 shows that the statistic $\operatorname*{st}$ is
head-graft-compatible. In other words, Assertion $\mathcal{A}_{4}$ holds. This
proves the implication $\mathcal{A}_{3}\Longrightarrow\mathcal{A}_{4}$.

We have now proven the implications $\mathcal{A}_{1}\Longrightarrow
\mathcal{A}_{2}$, $\mathcal{A}_{2}\Longrightarrow\mathcal{A}_{3}$,
$\mathcal{A}_{3}\Longrightarrow\mathcal{A}_{4}$ and $\mathcal{A}%
_{4}\Longrightarrow\mathcal{A}_{1}$. Thus, all four statements $\mathcal{A}%
_{1}$, $\mathcal{A}_{2}$, $\mathcal{A}_{3}$ and $\mathcal{A}_{4}$ are
equivalent. This proves Proposition \ref{prop.LRcomp.equivs}.
\end{proof}
\end{verlong}

Note that on their own, the properties of left-shuffle-compatibility and
right-shuffle-compatibility are not equivalent. For example, the permutation
statistic that sends each nonempty permutation $\pi$ to the truth value
$\left[  \pi_{1}>\pi_{i}\text{ for all }i\right]  $ (and the $0$-permutation
$\left(  {}\right)  $ to $0$) is left-shuffle-compatible (because in the
definition of left-shuffle-compatibility, all the $\operatorname*{st}\left(
\tau\right)  $ will be $0$), but not right-shuffle-compatible.

\subsection{Properties of compatible statistics}

\begin{vershort}
Let us state some more facts on compatibility properties. We refer to
\cite{verlong} for their proofs. We begin with a converse to Theorem
\ref{thm.head-comp.LRcomp}:
\end{vershort}

\begin{verlong}
The following converse to Theorem \ref{thm.head-comp.LRcomp} holds:

\begin{proposition}
\label{prop.LRcomp.head}Let $\operatorname*{st}$ be a permutation statistic
that is LR-shuffle-compatible. Then, $\operatorname*{st}$ is
head-graft-compatible and shuffle-compatible.
\end{proposition}

Before we can prove this, we need three lemmas:

\begin{lemma}
\label{lem.LRcomp.head-l1}Let $\operatorname*{st}$ be a head-graft-compatible
permutation statistic. If $\alpha$ and $\beta$ are two permutations satisfying
$\operatorname*{Des}\alpha=\operatorname*{Des}\beta$ and $\left\vert
\alpha\right\vert =\left\vert \beta\right\vert $, then, $\operatorname*{st}%
\left(  \alpha\right)  =\operatorname*{st}\left(  \beta\right)  $.
\end{lemma}

\begin{proof}
[Proof of Lemma \ref{lem.LRcomp.head-l1} (sketched).]We must prove the
following claim:

\begin{statement}
\textit{Claim 1:} Let $\alpha$ and $\beta$ be two permutations satisfying
$\operatorname*{Des}\alpha=\operatorname*{Des}\beta$ and $\left\vert
\alpha\right\vert =\left\vert \beta\right\vert $. Then, $\operatorname*{st}%
\left(  \alpha\right)  =\operatorname*{st}\left(  \beta\right)  $.
\end{statement}

[\textit{Proof of Claim 1:} We shall prove Claim 1 by induction on $\left\vert
\alpha\right\vert $:

\textit{Induction base:} If $\left\vert \alpha\right\vert =0$, then Claim 1 is
true (because if $\left\vert \alpha\right\vert =0$, then both $\alpha$ and
$\beta$ equal the empty $0$-permutation $\left(  {}\right)  $, and therefore
satisfy $\alpha=\beta$ and thus $\operatorname*{st}\left(  \alpha\right)
=\operatorname*{st}\left(  \beta\right)  $). This completes the induction base.

\textit{Induction step:} Let $N$ be a positive integer. Assume (as the
induction hypothesis) that Claim 1 holds for $\left\vert \alpha\right\vert
=N-1$. We must now prove that Claim 1 holds for $\left\vert \alpha\right\vert
=N$.

Let $\alpha$ and $\beta$ be as in Claim 1, and assume that $\left\vert
\alpha\right\vert =N$. We must prove that $\operatorname*{st}\left(
\alpha\right)  =\operatorname*{st}\left(  \beta\right)  $.

If $N=1$, then this holds\footnote{\textit{Proof.} Assume that $N=1$. Thus,
$\alpha$ and $\beta$ are $1$-permutations (since $\left\vert \alpha\right\vert
=N=1$ and $\left\vert \beta\right\vert =\left\vert \alpha\right\vert =N=1$)
and thus are order-equivalent. Hence, $\operatorname*{st}\left(
\alpha\right)  =\operatorname*{st}\left(  \beta\right)  $ (since
$\operatorname*{st}$ is a permutation statistic).}. Hence, for the rest of
this proof, we WLOG assume that $N\neq1$. Hence, $N\geq2$ (since $N$ is a
positive integer).

The permutation $\alpha$ is nonempty (since $\left\vert \alpha\right\vert
=N>0$). Thus, $\alpha_{\sim1}$ and $\alpha_{1}$ are well-defined. Similarly,
$\beta_{\sim1}$ and $\beta_{1}$ are well-defined (since $\left\vert
\beta\right\vert =\left\vert \alpha\right\vert =N>0$). Clearly, the letter
$\alpha_{1}$ does not appear in $\alpha_{\sim1}$ (since the letters of the
permutation $\alpha$ are distinct). Similarly, the letter $\beta_{1}$ does not
appear in $\beta_{\sim1}$.

We have $\left\vert \alpha_{\sim1}\right\vert =\underbrace{\left\vert
\alpha\right\vert }_{=N}-1=N-1$ and similarly $\left\vert \beta_{\sim
1}\right\vert =N-1$. Thus, $\left\vert \alpha_{\sim1}\right\vert
=N-1=\left\vert \beta_{\sim1}\right\vert $. Also, $\left\vert \alpha_{\sim
1}\right\vert =N-1\geq1$ (since $N\geq2$); thus, the permutation $\alpha
_{\sim1}$ is nonempty. Similarly, $\beta_{\sim1}$ is nonempty.

It is easy to see that
\begin{align*}
\operatorname*{Des}\left(  \alpha_{\sim1}\right)   &  =\left\{  i-1\ \mid
\ i\in\left(  \operatorname*{Des}\alpha\right)  \setminus\left\{  1\right\}
\right\}  \ \ \ \ \ \ \ \ \ \ \text{and}\\
\operatorname*{Des}\left(  \beta_{\sim1}\right)   &  =\left\{  i-1\ \mid
\ i\in\left(  \operatorname*{Des}\beta\right)  \setminus\left\{  1\right\}
\right\}  .
\end{align*}
The right hand sides of these two equalities are equal (since
$\operatorname*{Des}\alpha=\operatorname*{Des}\beta$). Thus, their left hand
sides are equal as well. In other words, $\operatorname*{Des}\left(
\alpha_{\sim1}\right)  =\operatorname*{Des}\left(  \beta_{\sim1}\right)  $.

Moreover, $\left\vert \alpha_{\sim1}\right\vert =\underbrace{\left\vert
\alpha\right\vert }_{=N}-1=N-1$. Hence, the induction hypothesis reveals that
we can apply Claim 1 to $\alpha_{\sim1}$ and $\beta_{\sim1}$ instead of
$\alpha$ and $\beta$. We thus obtain $\operatorname*{st}\left(  \alpha_{\sim
1}\right)  =\operatorname*{st}\left(  \beta_{\sim1}\right)  $.

Furthermore, $\left(  \alpha_{\sim1}\right)  _{1}=\alpha_{2}$, so that
\[
\left[  \alpha_{1}>\left(  \alpha_{\sim1}\right)  _{1}\right]  =\left[
\alpha_{1}>\alpha_{2}\right]  =\left[  1\in\operatorname*{Des}\alpha\right]
.
\]
Similarly,%
\[
\left[  \beta_{1}>\left(  \beta_{\sim1}\right)  _{1}\right]  =\left[
1\in\operatorname*{Des}\beta\right]  .
\]
The right hand sides of these two equalities are equal (since
$\operatorname*{Des}\alpha=\operatorname*{Des}\beta$). Thus, their left hand
sides are equal as well. In other words, $\left[  \alpha_{1}>\left(
\alpha_{\sim1}\right)  _{1}\right]  =\left[  \beta_{1}>\left(  \beta_{\sim
1}\right)  _{1}\right]  $.

But recall that $\operatorname*{st}$ is head-graft-compatible. In other words,
every nonempty permutation $\pi$, every letter $a$ that does not appear in
$\pi$, every nonempty permutation $\pi^{\prime}$ and every letter $a^{\prime}$
that does not appear in $\pi^{\prime}$ satisfying%
\[
\operatorname*{st}\left(  \pi\right)  =\operatorname*{st}\left(  \pi^{\prime
}\right)  ,\ \ \ \ \ \ \ \ \ \ \left\vert \pi\right\vert =\left\vert
\pi^{\prime}\right\vert \ \ \ \ \ \ \ \ \ \ \text{and}%
\ \ \ \ \ \ \ \ \ \ \left[  a>\pi_{1}\right]  =\left[  a^{\prime}>\pi
_{1}^{\prime}\right]
\]
satisfy $\operatorname*{st}\left(  a:\pi\right)  =\operatorname*{st}\left(
a^{\prime}:\pi^{\prime}\right)  $. Applying this to $\pi=\alpha_{\sim1}$,
$a=\alpha_{1}$, $\pi^{\prime}=\beta_{\sim1}$ and $a^{\prime}=\beta_{1}$, we
obtain $\operatorname*{st}\left(  \alpha_{1}:\alpha_{\sim1}\right)
=\operatorname*{st}\left(  \beta_{1}:\beta_{\sim1}\right)  $. In view of
$\alpha_{1}:\alpha_{\sim1}=\alpha$ and $\beta_{1}:\beta_{\sim1}=\beta$, this
rewrites as $\operatorname*{st}\left(  \alpha\right)  =\operatorname*{st}%
\left(  \beta\right)  $.

Thus, we have proven that $\operatorname*{st}\left(  \alpha\right)
=\operatorname*{st}\left(  \beta\right)  $. Hence, Claim 1 holds for
$\left\vert \alpha\right\vert =N$. This completes the induction step. Thus,
Claim 1 is proven.]

Lemma \ref{lem.LRcomp.head-l1} follows immediately from Claim 1.
\end{proof}

\begin{lemma}
\label{lem.LRcomp.head-l2}Let $\operatorname*{st}$ be a head-graft-compatible
permutation statistic. Let $X$ be the codomain of $\operatorname*{st}$. Let
$n\in\mathbb{N}$. Then, there exists a map
\[
F_{n}:\left\{  \text{subsets of }\left[  n-1\right]  \right\}  \rightarrow X
\]
such that every $n$-permutation $\tau$ satisfies $\operatorname*{st}\left(
\tau\right)  =F_{n}\left(  \operatorname*{Des}\tau\right)  $.
\end{lemma}

\begin{proof}
[Proof of Lemma \ref{lem.LRcomp.head-l2}.]We define the map $F_{n}$ as follows:

Let $Z$ be any subset of $\left[  n-1\right]  $. Then, it is well-known that
there exists some $n$-permutation $\tau$ satisfying $Z=\operatorname*{Des}%
\tau$. Pick any such $\tau$. Then, $\operatorname*{st}\left(  \tau\right)  $
does not depend on the choice of $\tau$ (because if $\alpha$ and $\beta$ are
two different $n$-permutations $\tau$ satisfying $Z=\operatorname*{Des}\tau$,
then Lemma \ref{lem.LRcomp.head-l1} yields $\operatorname*{st}\left(
\alpha\right)  =\operatorname*{st}\left(  \beta\right)  $). Hence, we can set
$F_{n}\left(  Z\right)  $ to be $\operatorname*{st}\left(  \tau\right)  $.

Thus, we have defined $F_{n}\left(  Z\right)  $ for each subset $Z$ of
$\left[  n-1\right]  $. This completes the definition of $F_{n}$. This
definition shows that every $n$-permutation $\tau$ satisfies
$\operatorname*{st}\left(  \tau\right)  =F_{n}\left(  \operatorname*{Des}%
\tau\right)  $. Thus, Lemma \ref{lem.LRcomp.head-l2} is proven.
\end{proof}

\begin{lemma}
\label{lem.LRcomp.head-l3}Let $\pi$ and $\sigma$ be two disjoint permutations.
Let $\pi^{\prime}$ and $\sigma^{\prime}$ be two disjoint permutations. Assume
that%
\begin{align*}
\operatorname*{Des}\pi &  =\operatorname*{Des}\pi^{\prime}%
,\ \ \ \ \ \ \ \ \ \ \operatorname*{Des}\sigma=\operatorname*{Des}%
\sigma^{\prime},\\
\left\vert \pi\right\vert  &  =\left\vert \pi^{\prime}\right\vert
\ \ \ \ \ \ \ \ \ \ \text{and}\ \ \ \ \ \ \ \ \ \ \left\vert \sigma\right\vert
=\left\vert \sigma^{\prime}\right\vert .
\end{align*}
Then,
\[
\left\{  \operatorname*{Des}\tau\ \mid\ \tau\in S\left(  \pi,\sigma\right)
\right\}  _{\operatorname*{multi}}=\left\{  \operatorname*{Des}\tau
\ \mid\ \tau\in S\left(  \pi^{\prime},\sigma^{\prime}\right)  \right\}
_{\operatorname*{multi}}.
\]

\end{lemma}

\begin{proof}
[Proof of Lemma \ref{lem.LRcomp.head-l3}.]Lemma \ref{lem.LRcomp.head-l3} is
simply the statement that the permutation statistic $\operatorname*{Des}$ is
shuffle-compatible. But this has been proven in \cite[\S 2.4]{part1}.
\end{proof}

\begin{proof}
[Proof of Proposition \ref{prop.LRcomp.head} (sketched).]We know that
$\operatorname*{st}$ is LR-shuffle-compatible. In other words, the following holds:

\begin{statement}
\textit{Claim 1:} Let $\pi$ and $\sigma$ be two disjoint nonempty
permutations. Let $\pi^{\prime}$ and $\sigma^{\prime}$ be two disjoint
nonempty permutations. Assume that%
\begin{align*}
\operatorname*{st}\left(  \pi\right)   &  =\operatorname*{st}\left(
\pi^{\prime}\right)  ,\ \ \ \ \ \ \ \ \ \ \operatorname*{st}\left(
\sigma\right)  =\operatorname*{st}\left(  \sigma^{\prime}\right)  ,\\
\left\vert \pi\right\vert  &  =\left\vert \pi^{\prime}\right\vert
,\ \ \ \ \ \ \ \ \ \ \left\vert \sigma\right\vert =\left\vert \sigma^{\prime
}\right\vert \ \ \ \ \ \ \ \ \ \ \text{and}\ \ \ \ \ \ \ \ \ \ \left[  \pi
_{1}>\sigma_{1}\right]  =\left[  \pi_{1}^{\prime}>\sigma_{1}^{\prime}\right]
.
\end{align*}
Then,
\[
\left\{  \operatorname*{st}\left(  \tau\right)  \ \mid\ \tau\in S_{\prec
}\left(  \pi,\sigma\right)  \right\}  _{\operatorname*{multi}}=\left\{
\operatorname*{st}\left(  \tau\right)  \ \mid\ \tau\in S_{\prec}\left(
\pi^{\prime},\sigma^{\prime}\right)  \right\}  _{\operatorname*{multi}}%
\]
and%
\[
\left\{  \operatorname*{st}\left(  \tau\right)  \ \mid\ \tau\in S_{\succ
}\left(  \pi,\sigma\right)  \right\}  _{\operatorname*{multi}}=\left\{
\operatorname*{st}\left(  \tau\right)  \ \mid\ \tau\in S_{\succ}\left(
\pi^{\prime},\sigma^{\prime}\right)  \right\}  _{\operatorname*{multi}}.
\]

\end{statement}

Next, we want to show that $\operatorname*{st}$ is head-graft-compatible. In
other words, we want to show that the following holds:

\begin{statement}
\textit{Claim 2:} Let $\pi$ be a nonempty permutation, and let $a$ be a letter
that does not appear in $\pi$. Let $\pi^{\prime}$ be a nonempty permutation,
and let $a^{\prime}$ be a letter that does not appear in $\pi^{\prime}$.
Assume that%
\[
\operatorname*{st}\left(  \pi\right)  =\operatorname*{st}\left(  \pi^{\prime
}\right)  ,\ \ \ \ \ \ \ \ \ \ \left\vert \pi\right\vert =\left\vert
\pi^{\prime}\right\vert \ \ \ \ \ \ \ \ \ \ \text{and}%
\ \ \ \ \ \ \ \ \ \ \left[  a>\pi_{1}\right]  =\left[  a^{\prime}>\pi
_{1}^{\prime}\right]  .
\]
Then, $\operatorname*{st}\left(  a:\pi\right)  =\operatorname*{st}\left(
a^{\prime}:\pi^{\prime}\right)  $.
\end{statement}

[\textit{Proof of Claim 2:} The $1$-permutations $\left(  a\right)  $ and
$\left(  a^{\prime}\right)  $ are order-equivalent. Thus, $\operatorname*{st}%
\left(  \left(  a\right)  \right)  =\operatorname*{st}\left(  \left(
a^{\prime}\right)  \right)  $ (since $\operatorname*{st}$ is a permutation statistic).

The $1$-permutations $\left(  a\right)  $ and $\left(  a^{\prime}\right)  $
satisfy $\left(  a\right)  _{1}=a$ and $\left(  a^{\prime}\right)
_{1}=a^{\prime}$. Thus, the inequality $\left[  a>\pi_{1}\right]  =\left[
a^{\prime}>\pi_{1}^{\prime}\right]  $ (which is true by assumption) rewrites
as $\left[  \left(  a\right)  _{1}>\pi_{1}\right]  =\left[  \left(  a^{\prime
}\right)  _{1}>\pi_{1}^{\prime}\right]  $. Also, $\operatorname*{st}\left(
\left(  a\right)  \right)  =\operatorname*{st}\left(  \left(  a^{\prime
}\right)  \right)  $ and $\left\vert \left(  a\right)  \right\vert
=1=\left\vert \left(  a^{\prime}\right)  \right\vert $. Also, the
$1$-permutation $\left(  a\right)  $ is disjoint from $\pi$ (since $a$ does
not appear in $\pi$). Similarly, the $1$-permutation $\left(  a^{\prime
}\right)  $ is disjoint from $\pi^{\prime}$. Thus, Claim 1 (applied to
$\sigma=\left(  a\right)  $ and $\sigma^{\prime}=\left(  a^{\prime}\right)  $)
yields
\[
\left\{  \operatorname*{st}\left(  \tau\right)  \ \mid\ \tau\in S_{\prec
}\left(  \pi,\left(  a\right)  \right)  \right\}  _{\operatorname*{multi}%
}=\left\{  \operatorname*{st}\left(  \tau\right)  \ \mid\ \tau\in S_{\prec
}\left(  \pi^{\prime},\left(  a^{\prime}\right)  \right)  \right\}
_{\operatorname*{multi}}%
\]
and%
\[
\left\{  \operatorname*{st}\left(  \tau\right)  \ \mid\ \tau\in S_{\succ
}\left(  \pi,\left(  a\right)  \right)  \right\}  _{\operatorname*{multi}%
}=\left\{  \operatorname*{st}\left(  \tau\right)  \ \mid\ \tau\in S_{\succ
}\left(  \pi^{\prime},\left(  a^{\prime}\right)  \right)  \right\}
_{\operatorname*{multi}}.
\]
But it is easily seen that $S_{\succ}\left(  \pi,\left(  a\right)  \right)
=\left\{  a:\pi\right\}  $. Hence,%
\[
\left\{  \operatorname*{st}\left(  \tau\right)  \ \mid\ \tau\in S_{\succ
}\left(  \pi,\left(  a\right)  \right)  \right\}  _{\operatorname*{multi}%
}=\left\{  \operatorname*{st}\left(  \tau\right)  \ \mid\ \tau\in\left\{
a:\pi\right\}  \right\}  _{\operatorname*{multi}}=\left\{  \operatorname*{st}%
\left(  a:\pi\right)  \right\}  _{\operatorname*{multi}}.
\]
Similarly,%
\[
\left\{  \operatorname*{st}\left(  \tau\right)  \ \mid\ \tau\in S_{\succ
}\left(  \pi^{\prime},\left(  a^{\prime}\right)  \right)  \right\}
_{\operatorname*{multi}}=\left\{  \operatorname*{st}\left(  a^{\prime}%
:\pi^{\prime}\right)  \right\}  _{\operatorname*{multi}}.
\]
Thus,%
\begin{align*}
\left\{  \operatorname*{st}\left(  a:\pi\right)  \right\}
_{\operatorname*{multi}}  &  =\left\{  \operatorname*{st}\left(  \tau\right)
\ \mid\ \tau\in S_{\succ}\left(  \pi,\left(  a\right)  \right)  \right\}
_{\operatorname*{multi}}\\
&  =\left\{  \operatorname*{st}\left(  \tau\right)  \ \mid\ \tau\in S_{\succ
}\left(  \pi^{\prime},\left(  a^{\prime}\right)  \right)  \right\}
_{\operatorname*{multi}}=\left\{  \operatorname*{st}\left(  a^{\prime}%
:\pi^{\prime}\right)  \right\}  _{\operatorname*{multi}},
\end{align*}
so that $\operatorname*{st}\left(  a:\pi\right)  =\operatorname*{st}\left(
a^{\prime}:\pi^{\prime}\right)  $. This proves Claim 2.]

Now, Claim 2 shows that $\operatorname*{st}$ is head-graft-compatible. It
remains to show that $\operatorname*{st}$ is shuffle-compatible. First, we
show an auxiliary statement:

\begin{statement}
\textit{Claim 3:} Let $\pi$, $\pi^{\prime}$ and $\sigma$ be three
permutations. Assume that $\pi$ and $\pi^{\prime}$ are order-equivalent.
Assume that $\pi$ and $\sigma$ are disjoint. Assume that $\pi^{\prime}$ and
$\sigma$ are disjoint. Then,%
\[
\left\{  \operatorname*{st}\left(  \tau\right)  \ \mid\ \tau\in S\left(
\pi,\sigma\right)  \right\}  _{\operatorname*{multi}}=\left\{
\operatorname*{st}\left(  \tau\right)  \ \mid\ \tau\in S\left(  \pi^{\prime
},\sigma\right)  \right\}  _{\operatorname*{multi}}.
\]

\end{statement}

[\textit{Proof of Claim 3:} We have $\left\vert \pi\right\vert =\left\vert
\pi^{\prime}\right\vert $ (since $\pi$ and $\pi^{\prime}$ are
order-equivalent). Define $n\in\mathbb{N}$ by $n=\left\vert \pi\right\vert
=\left\vert \pi^{\prime}\right\vert $. Let $X$ be the codomain of
$\operatorname*{st}$. Lemma \ref{lem.LRcomp.head-l2} shows that there exists a
map
\[
F_{n}:\left\{  \text{subsets of }\left[  n-1\right]  \right\}  \rightarrow X
\]
such that every $n$-permutation $\tau$ satisfies
\begin{equation}
\operatorname*{st}\left(  \tau\right)  =F_{n}\left(  \operatorname*{Des}%
\tau\right)  . \label{pf.lem.LRcomp.head-l2.c3.pf.1}%
\end{equation}
Consider this map $F_{n}$.

We know that $\operatorname*{Des}$ is a permutation statistic. Thus,
$\operatorname*{Des}\pi=\operatorname*{Des}\left(  \pi^{\prime}\right)  $
(since $\pi$ and $\pi^{\prime}$ are order-equivalent). Also, $\left\vert
\pi\right\vert =\left\vert \pi^{\prime}\right\vert $, $\operatorname*{Des}%
\sigma=\operatorname*{Des}\sigma$ and $\left\vert \sigma\right\vert
=\left\vert \sigma\right\vert $. Hence, Lemma \ref{lem.LRcomp.head-l3}
(applied to $\sigma^{\prime}=\sigma$) yields
\begin{equation}
\left\{  \operatorname*{Des}\tau\ \mid\ \tau\in S\left(  \pi,\sigma\right)
\right\}  _{\operatorname*{multi}}=\left\{  \operatorname*{Des}\tau
\ \mid\ \tau\in S\left(  \pi^{\prime},\sigma\right)  \right\}
_{\operatorname*{multi}}. \label{pf.lem.LRcomp.head-l2.c3.pf.2}%
\end{equation}
Now,
\begin{align*}
&  \left\{  \underbrace{\operatorname*{st}\left(  \tau\right)  }%
_{\substack{=F_{n}\left(  \operatorname*{Des}\tau\right)  \\\text{(by
(\ref{pf.lem.LRcomp.head-l2.c3.pf.1}))}}}\ \mid\ \tau\in S\left(  \pi
,\sigma\right)  \right\}  _{\operatorname*{multi}}\\
&  =\left\{  F_{n}\left(  \operatorname*{Des}\tau\right)  \ \mid\ \tau\in
S\left(  \pi,\sigma\right)  \right\}  _{\operatorname*{multi}}=F_{n}\left(
\left\{  \operatorname*{Des}\tau\ \mid\ \tau\in S\left(  \pi,\sigma\right)
\right\}  _{\operatorname*{multi}}\right) \\
&  =F_{n}\left(  \left\{  \operatorname*{Des}\tau\ \mid\ \tau\in S\left(
\pi^{\prime},\sigma\right)  \right\}  _{\operatorname*{multi}}\right)
\ \ \ \ \ \ \ \ \ \ \left(  \text{by (\ref{pf.lem.LRcomp.head-l2.c3.pf.2}%
)}\right) \\
&  =\left\{  \underbrace{F_{n}\left(  \operatorname*{Des}\tau\right)
}_{\substack{=\operatorname*{st}\left(  \tau\right)  \\\text{(by
(\ref{pf.lem.LRcomp.head-l2.c3.pf.1}))}}}\ \mid\ \tau\in S\left(  \pi^{\prime
},\sigma\right)  \right\}  _{\operatorname*{multi}}=\left\{
\operatorname*{st}\left(  \tau\right)  \ \mid\ \tau\in S\left(  \pi^{\prime
},\sigma\right)  \right\}  _{\operatorname*{multi}}.
\end{align*}
This proves Claim 3.]

Let us finally prove that $\operatorname*{st}$ is shuffle-compatible. In other
words, let us prove the following claim:

\begin{statement}
\textit{Claim 4:} Let $\pi$ and $\sigma$ be two disjoint permutations. Let
$\pi^{\prime}$ and $\sigma^{\prime}$ be two disjoint permutations. Assume that%
\begin{align*}
\operatorname*{st}\left(  \pi\right)   &  =\operatorname*{st}\left(
\pi^{\prime}\right)  ,\ \ \ \ \ \ \ \ \ \ \operatorname*{st}\left(
\sigma\right)  =\operatorname*{st}\left(  \sigma^{\prime}\right)  ,\\
\left\vert \pi\right\vert  &  =\left\vert \pi^{\prime}\right\vert
\ \ \ \ \ \ \ \ \ \ \text{and}\ \ \ \ \ \ \ \ \ \ \left\vert \sigma\right\vert
=\left\vert \sigma^{\prime}\right\vert .
\end{align*}
Then,
\[
\left\{  \operatorname*{st}\left(  \tau\right)  \ \mid\ \tau\in S\left(
\pi,\sigma\right)  \right\}  _{\operatorname*{multi}}=\left\{
\operatorname*{st}\left(  \tau\right)  \ \mid\ \tau\in S\left(  \pi^{\prime
},\sigma^{\prime}\right)  \right\}  _{\operatorname*{multi}}.
\]

\end{statement}

[\textit{Proof of Claim 4:} Recall that $S\left(  \pi,\sigma\right)  =S\left(
\sigma,\pi\right)  $ and $S\left(  \pi^{\prime},\sigma^{\prime}\right)
=S\left(  \sigma^{\prime},\pi^{\prime}\right)  $. Hence, Claim 4 does not
change if we swap $\pi$ with $\sigma$ while simultaneously swapping
$\pi^{\prime}$ with $\sigma^{\prime}$. Thus, we WLOG assume that $\pi_{1}%
\geq\sigma_{1}$ (since otherwise, we can ensure this by performing these
swaps). But $\pi_{1}\neq\sigma_{1}$ (since $\pi$ and $\sigma$ are disjoint).
Combining this with $\pi_{1}\geq\sigma_{1}$, we obtain $\pi_{1}>\sigma_{1}$.

Define $n\in\mathbb{N}$ by $n=\left\vert \pi\right\vert =\left\vert
\pi^{\prime}\right\vert $. Define $m\in\mathbb{N}$ by $m=\left\vert
\sigma\right\vert =\left\vert \sigma^{\prime}\right\vert $.

If $n=0$, then both $\pi$ and $\pi^{\prime}$ equal the $0$-permutation
$\left(  {}\right)  $ (since $n=\left\vert \pi\right\vert =\left\vert
\pi^{\prime}\right\vert $). Hence, if $n=0$, then Claim 4 is true (because in
this case, we have%
\begin{align*}
\left\{  \operatorname*{st}\left(  \tau\right)  \ \mid\ \tau\in S\left(
\underbrace{\pi}_{=\left(  {}\right)  },\sigma\right)  \right\}
_{\operatorname*{multi}}  &  =\left\{  \operatorname*{st}\left(  \tau\right)
\ \mid\ \tau\in\underbrace{S\left(  \left(  {}\right)  ,\sigma\right)
}_{=\left\{  \sigma\right\}  }\right\}  _{\operatorname*{multi}}\\
&  =\left\{  \operatorname*{st}\left(  \tau\right)  \ \mid\ \tau\in\left\{
\sigma\right\}  \right\}  _{\operatorname*{multi}}=\left\{  \operatorname*{st}%
\left(  \sigma\right)  \right\}  _{\operatorname*{multi}}%
\end{align*}
and similarly $\left\{  \operatorname*{st}\left(  \tau\right)  \ \mid\ \tau\in
S\left(  \pi^{\prime},\sigma^{\prime}\right)  \right\}
_{\operatorname*{multi}}=\left\{  \operatorname*{st}\left(  \sigma^{\prime
}\right)  \right\}  _{\operatorname*{multi}}$, and therefore the assertion of
Claim 4 reduces to $\left\{  \operatorname*{st}\left(  \sigma\right)
\right\}  _{\operatorname*{multi}}=\left\{  \operatorname*{st}\left(
\sigma^{\prime}\right)  \right\}  _{\operatorname*{multi}}$, which follows
immediately from the assumption $\operatorname*{st}\left(  \sigma\right)
=\operatorname*{st}\left(  \sigma^{\prime}\right)  $). Thus, for the rest of
this proof, we WLOG assume that $n\neq0$. For similar reasons, we WLOG assume
that $m\neq0$.

The permutations $\pi$ and $\pi^{\prime}$ are nonempty (since $\left\vert
\pi\right\vert =\left\vert \pi^{\prime}\right\vert =n\neq0$). Similarly, the
permutations $\sigma$ and $\sigma^{\prime}$ are nonempty.

There clearly exists a positive integer $N$ that is larger than all entries of
$\sigma$ and larger than all entries of $\sigma^{\prime}$. Consider such an
$N$. From $n=\left\vert \pi^{\prime}\right\vert $, we obtain $\pi^{\prime
}=\left(  \pi_{1}^{\prime},\pi_{2}^{\prime},\ldots,\pi_{n}^{\prime}\right)  $.
Let $\gamma$ be the permutation $\left(  \pi_{1}^{\prime}+N,\pi_{2}^{\prime
}+N,\ldots,\pi_{n}^{\prime}+N\right)  $. This permutation $\gamma$ is
order-equivalent to $\pi^{\prime}$, but is disjoint from $\sigma$ (since all
its entries are $>N$, while all the entries of $\sigma$ are $<N$) and disjoint
from $\sigma^{\prime}$ (for similar reasons). Also, $\gamma_{1}%
=\underbrace{\pi_{1}^{\prime}}_{>0}+N>N>\sigma_{1}$, so that $\left[
\gamma_{1}>\sigma_{1}\right]  =1$. Similarly, $\left[  \gamma_{1}>\sigma
_{1}^{\prime}\right]  =1$.

The permutation $\gamma$ is order-equivalent to $\pi^{\prime}$. Thus,
$\operatorname*{st}\left(  \gamma\right)  =\operatorname*{st}\left(
\pi^{\prime}\right)  $ (since $\operatorname*{st}$ is a permutation statistic)
and $\left\vert \gamma\right\vert =\left\vert \pi^{\prime}\right\vert $. The
permutation $\gamma$ is furthermore nonempty (since it is order-equivalent to
the nonempty permutation $\pi^{\prime}$). Also, $\operatorname*{st}\left(
\gamma\right)  =\operatorname*{st}\left(  \pi^{\prime}\right)
=\operatorname*{st}\left(  \pi\right)  $ and $\left\vert \gamma\right\vert
=\left\vert \pi^{\prime}\right\vert =\left\vert \pi\right\vert $. Moreover,
from $\pi_{1}>\sigma_{1}$, we obtain $\left[  \pi_{1}>\sigma_{1}\right]
=1=\left[  \gamma_{1}>\sigma_{1}^{\prime}\right]  $. Hence, Claim 1 (applied
to $\gamma$ instead of $\pi^{\prime}$) yields
\begin{equation}
\left\{  \operatorname*{st}\left(  \tau\right)  \ \mid\ \tau\in S_{\prec
}\left(  \pi,\sigma\right)  \right\}  _{\operatorname*{multi}}=\left\{
\operatorname*{st}\left(  \tau\right)  \ \mid\ \tau\in S_{\prec}\left(
\gamma,\sigma^{\prime}\right)  \right\}  _{\operatorname*{multi}}
\label{pf.lem.LRcomp.head-l2.c4.pf.5}%
\end{equation}
and%
\begin{equation}
\left\{  \operatorname*{st}\left(  \tau\right)  \ \mid\ \tau\in S_{\succ
}\left(  \pi,\sigma\right)  \right\}  _{\operatorname*{multi}}=\left\{
\operatorname*{st}\left(  \tau\right)  \ \mid\ \tau\in S_{\succ}\left(
\gamma,\sigma^{\prime}\right)  \right\}  _{\operatorname*{multi}}.
\label{pf.lem.LRcomp.head-l2.c4.pf.6}%
\end{equation}


Now, if $A$ and $B$ are two finite multisets, then $A+B$ shall denote the
multiset union of $A$ and $B$; this is the multiset $C$ such that each object
$g$ satisfies $\left\vert C\right\vert _{g}=\left\vert A\right\vert
_{g}+\left\vert B\right\vert _{g}$. Then, it is easy to see that%
\begin{align*}
&  \left\{  \operatorname*{st}\left(  \tau\right)  \ \mid\ \tau\in S\left(
\pi,\sigma\right)  \right\}  _{\operatorname*{multi}}\\
&  =\left\{  \operatorname*{st}\left(  \tau\right)  \ \mid\ \tau\in S_{\prec
}\left(  \pi,\sigma\right)  \right\}  _{\operatorname*{multi}}+\left\{
\operatorname*{st}\left(  \tau\right)  \ \mid\ \tau\in S_{\succ}\left(
\pi,\sigma\right)  \right\}  _{\operatorname*{multi}}%
\end{align*}
and
\begin{align*}
&  \left\{  \operatorname*{st}\left(  \tau\right)  \ \mid\ \tau\in S\left(
\gamma,\sigma^{\prime}\right)  \right\}  _{\operatorname*{multi}}\\
&  =\left\{  \operatorname*{st}\left(  \tau\right)  \ \mid\ \tau\in S_{\prec
}\left(  \gamma,\sigma^{\prime}\right)  \right\}  _{\operatorname*{multi}%
}+\left\{  \operatorname*{st}\left(  \tau\right)  \ \mid\ \tau\in S_{\succ
}\left(  \gamma,\sigma^{\prime}\right)  \right\}  _{\operatorname*{multi}}.
\end{align*}
Hence, by adding together the equalities (\ref{pf.lem.LRcomp.head-l2.c4.pf.5})
and (\ref{pf.lem.LRcomp.head-l2.c4.pf.6}) (using the operation $+$ that we
have just defined), we obtain the equality%
\[
\left\{  \operatorname*{st}\left(  \tau\right)  \ \mid\ \tau\in S\left(
\pi,\sigma\right)  \right\}  _{\operatorname*{multi}}=\left\{
\operatorname*{st}\left(  \tau\right)  \ \mid\ \tau\in S\left(  \gamma
,\sigma^{\prime}\right)  \right\}  _{\operatorname*{multi}}.
\]


On the other hand, Claim 3 (applied to $\gamma$ and $\sigma^{\prime}$ instead
of $\pi$ and $\sigma$) yields%
\[
\left\{  \operatorname*{st}\left(  \tau\right)  \ \mid\ \tau\in S\left(
\gamma,\sigma^{\prime}\right)  \right\}  _{\operatorname*{multi}}=\left\{
\operatorname*{st}\left(  \tau\right)  \ \mid\ \tau\in S\left(  \pi^{\prime
},\sigma^{\prime}\right)  \right\}  _{\operatorname*{multi}}.
\]
Hence,%
\begin{align*}
\left\{  \operatorname*{st}\left(  \tau\right)  \ \mid\ \tau\in S\left(
\pi,\sigma\right)  \right\}  _{\operatorname*{multi}}  &  =\left\{
\operatorname*{st}\left(  \tau\right)  \ \mid\ \tau\in S\left(  \gamma
,\sigma^{\prime}\right)  \right\}  _{\operatorname*{multi}}\\
&  =\left\{  \operatorname*{st}\left(  \tau\right)  \ \mid\ \tau\in S\left(
\pi^{\prime},\sigma^{\prime}\right)  \right\}  _{\operatorname*{multi}}.
\end{align*}
This proves Claim 4.]

Claim 4 shows that $\operatorname*{st}$ is shuffle-compatible. This completes
the proof of Proposition \ref{prop.LRcomp.head}.
\end{proof}
\end{verlong}

\begin{corollary}
\label{cor.LRcomp.back}Let $\operatorname*{st}$ be a LR-shuffle-compatible
permutation statistic. Then, $\operatorname*{st}$ is shuffle-compatible,
left-shuffle-compatible, right-shuffle-compatible and head-graft-compatible.
\end{corollary}

\begin{verlong}
\begin{proof}
[Proof of Corollary \ref{cor.LRcomp.back}.]Proposition \ref{prop.LRcomp.head}
yields that $\operatorname*{st}$ is head-graft-compatible and
shuffle-compatible. Thus, Proposition \ref{prop.LRcomp.equivs} yields that
$\operatorname*{st}$ is left-shuffle-compatible and right-shuffle-compatible
as well. This proves Corollary \ref{cor.LRcomp.back}.
\end{proof}
\end{verlong}

\begin{corollary}
\label{cor.LRcomp.two}Let $\operatorname*{st}$ be a permutation statistic that
is left-shuffle-compatible and right-shuffle-compatible. Then,
$\operatorname*{st}$ is LR-shuffle-compatible.
\end{corollary}

\begin{verlong}
\begin{proof}
[Proof of Corollary \ref{cor.LRcomp.two}.]We have assumed that
$\operatorname*{st}$ is left-shuffle-compatible. In other words, the following
claim holds:In other words, the following claim holds:

\begin{statement}
\textit{Claim 1:} Let $\pi$ and $\sigma$ be two disjoint nonempty permutations
having the property that $\pi_{1}>\sigma_{1}$. Let $\pi^{\prime}$ and
$\sigma^{\prime}$ be two disjoint nonempty permutations having the property
that $\pi_{1}^{\prime}>\sigma_{1}^{\prime}$. Assume that%
\begin{align*}
\operatorname*{st}\left(  \pi\right)   &  =\operatorname*{st}\left(
\pi^{\prime}\right)  ,\ \ \ \ \ \ \ \ \ \ \operatorname*{st}\left(
\sigma\right)  =\operatorname*{st}\left(  \sigma^{\prime}\right)  ,\\
\left\vert \pi\right\vert  &  =\left\vert \pi^{\prime}\right\vert
\ \ \ \ \ \ \ \ \ \ \text{and}\ \ \ \ \ \ \ \ \ \ \left\vert \sigma\right\vert
=\left\vert \sigma^{\prime}\right\vert .
\end{align*}
Then,
\[
\left\{  \operatorname*{st}\left(  \tau\right)  \ \mid\ \tau\in S_{\prec
}\left(  \pi,\sigma\right)  \right\}  _{\operatorname*{multi}}=\left\{
\operatorname*{st}\left(  \tau\right)  \ \mid\ \tau\in S_{\prec}\left(
\pi^{\prime},\sigma^{\prime}\right)  \right\}  _{\operatorname*{multi}}.
\]

\end{statement}

Also, the statistic $\operatorname*{st}$ is right-shuffle-compatible. In other
words, the following claim holds:

\begin{statement}
\textit{Claim 2:} Let $\pi$ and $\sigma$ be two disjoint nonempty permutations
having the property that $\pi_{1}>\sigma_{1}$. Let $\pi^{\prime}$ and
$\sigma^{\prime}$ be two disjoint nonempty permutations having the property
that $\pi_{1}^{\prime}>\sigma_{1}^{\prime}$. Assume that%
\begin{align*}
\operatorname*{st}\left(  \pi\right)   &  =\operatorname*{st}\left(
\pi^{\prime}\right)  ,\ \ \ \ \ \ \ \ \ \ \operatorname*{st}\left(
\sigma\right)  =\operatorname*{st}\left(  \sigma^{\prime}\right)  ,\\
\left\vert \pi\right\vert  &  =\left\vert \pi^{\prime}\right\vert
\ \ \ \ \ \ \ \ \ \ \text{and}\ \ \ \ \ \ \ \ \ \ \left\vert \sigma\right\vert
=\left\vert \sigma^{\prime}\right\vert .
\end{align*}
Then,
\[
\left\{  \operatorname*{st}\left(  \tau\right)  \ \mid\ \tau\in S_{\succ
}\left(  \pi,\sigma\right)  \right\}  _{\operatorname*{multi}}=\left\{
\operatorname*{st}\left(  \tau\right)  \ \mid\ \tau\in S_{\succ}\left(
\pi^{\prime},\sigma^{\prime}\right)  \right\}  _{\operatorname*{multi}}.
\]

\end{statement}

On the other hand, we want to prove that $\operatorname*{st}$ is
LR-shuffle-compatible. In other words, we want to prove the following claim:

\begin{statement}
\textit{Claim 3:} Let $\pi$ and $\sigma$ be two disjoint nonempty
permutations. Let $\pi^{\prime}$ and $\sigma^{\prime}$ be two disjoint
nonempty permutations. Assume that%
\begin{align*}
\operatorname*{st}\left(  \pi\right)   &  =\operatorname*{st}\left(
\pi^{\prime}\right)  ,\ \ \ \ \ \ \ \ \ \ \operatorname*{st}\left(
\sigma\right)  =\operatorname*{st}\left(  \sigma^{\prime}\right)  ,\\
\left\vert \pi\right\vert  &  =\left\vert \pi^{\prime}\right\vert
,\ \ \ \ \ \ \ \ \ \ \left\vert \sigma\right\vert =\left\vert \sigma^{\prime
}\right\vert \ \ \ \ \ \ \ \ \ \ \text{and}\ \ \ \ \ \ \ \ \ \ \left[  \pi
_{1}>\sigma_{1}\right]  =\left[  \pi_{1}^{\prime}>\sigma_{1}^{\prime}\right]
.
\end{align*}
Then,
\[
\left\{  \operatorname*{st}\left(  \tau\right)  \ \mid\ \tau\in S_{\prec
}\left(  \pi,\sigma\right)  \right\}  _{\operatorname*{multi}}=\left\{
\operatorname*{st}\left(  \tau\right)  \ \mid\ \tau\in S_{\prec}\left(
\pi^{\prime},\sigma^{\prime}\right)  \right\}  _{\operatorname*{multi}}%
\]
and%
\[
\left\{  \operatorname*{st}\left(  \tau\right)  \ \mid\ \tau\in S_{\succ
}\left(  \pi,\sigma\right)  \right\}  _{\operatorname*{multi}}=\left\{
\operatorname*{st}\left(  \tau\right)  \ \mid\ \tau\in S_{\succ}\left(
\pi^{\prime},\sigma^{\prime}\right)  \right\}  _{\operatorname*{multi}}.
\]

\end{statement}

[\textit{Proof of Claim 3:} We are in one of the following two cases:

\textit{Case 1:} We have $\pi_{1}>\sigma_{1}$.

\textit{Case 2:} We have $\pi_{1}\leq\sigma_{1}$.

Let us first consider Case 1. In this case, we have $\pi_{1}>\sigma_{1}$.
Hence, $\left[  \pi_{1}>\sigma_{1}\right]  =1$. Comparing this with $\left[
\pi_{1}>\sigma_{1}\right]  =\left[  \pi_{1}^{\prime}>\sigma_{1}^{\prime
}\right]  $, we find $\left[  \pi_{1}^{\prime}>\sigma_{1}^{\prime}\right]
=1$. Hence, $\pi_{1}^{\prime}>\sigma_{1}^{\prime}$. Recall also that $\pi
_{1}>\sigma_{1}$. Hence, Claim 1 yields%
\[
\left\{  \operatorname*{st}\left(  \tau\right)  \ \mid\ \tau\in S_{\prec
}\left(  \pi,\sigma\right)  \right\}  _{\operatorname*{multi}}=\left\{
\operatorname*{st}\left(  \tau\right)  \ \mid\ \tau\in S_{\prec}\left(
\pi^{\prime},\sigma^{\prime}\right)  \right\}  _{\operatorname*{multi}}.
\]
But Claim 2 yields%
\[
\left\{  \operatorname*{st}\left(  \tau\right)  \ \mid\ \tau\in S_{\succ
}\left(  \pi,\sigma\right)  \right\}  _{\operatorname*{multi}}=\left\{
\operatorname*{st}\left(  \tau\right)  \ \mid\ \tau\in S_{\succ}\left(
\pi^{\prime},\sigma^{\prime}\right)  \right\}  _{\operatorname*{multi}}.
\]
Thus, Claim 3 is proven in Case 1.

Let us next consider Case 2. In this case, we have $\pi_{1}\leq\sigma_{1}$.

Applying Proposition \ref{prop.LR.rec} \textbf{(a)} to various pairs of
disjoint permutations, we obtain $S_{\prec}\left(  \pi,\sigma\right)
=S_{\succ}\left(  \sigma,\pi\right)  $ and $S_{\prec}\left(  \pi^{\prime
},\sigma^{\prime}\right)  =S_{\succ}\left(  \sigma^{\prime},\pi^{\prime
}\right)  $ and $S_{\prec}\left(  \sigma,\pi\right)  =S_{\succ}\left(
\pi,\sigma\right)  $ and $S_{\prec}\left(  \sigma^{\prime},\pi^{\prime
}\right)  =S_{\succ}\left(  \pi^{\prime},\sigma^{\prime}\right)  $.

But $\pi_{1}\neq\sigma_{1}$ (since $\pi$ and $\sigma$ are disjoint). Combined
with $\pi_{1}\leq\sigma_{1}$, this yields $\pi_{1}<\sigma_{1}$; thus,
$\sigma_{1}>\pi_{1}$. Also, $\left[  \pi_{1}>\sigma_{1}\right]  =0$ (since
$\pi_{1}\leq\sigma_{1}$). Comparing this with $\left[  \pi_{1}>\sigma
_{1}\right]  =\left[  \pi_{1}^{\prime}>\sigma_{1}^{\prime}\right]  $, we find
$\left[  \pi_{1}^{\prime}>\sigma_{1}^{\prime}\right]  =0$. Hence, $\pi
_{1}^{\prime}\leq\sigma_{1}^{\prime}$. But $\pi_{1}^{\prime}\neq\sigma
_{1}^{\prime}$ (since $\pi^{\prime}$ and $\sigma^{\prime}$ are disjoint).
Combined with $\pi_{1}^{\prime}\leq\sigma_{1}^{\prime}$, this yields $\pi
_{1}^{\prime}<\sigma_{1}^{\prime}$; thus, $\sigma_{1}^{\prime}>\pi_{1}%
^{\prime}$. Hence, Claim 2 (applied to $\sigma$, $\pi$, $\sigma^{\prime}$ and
$\pi^{\prime}$ instead of $\pi$, $\sigma$, $\pi^{\prime}$ and $\sigma^{\prime
}$) yields%
\[
\left\{  \operatorname*{st}\left(  \tau\right)  \ \mid\ \tau\in S_{\succ
}\left(  \sigma,\pi\right)  \right\}  _{\operatorname*{multi}}=\left\{
\operatorname*{st}\left(  \tau\right)  \ \mid\ \tau\in S_{\succ}\left(
\sigma^{\prime},\pi^{\prime}\right)  \right\}  _{\operatorname*{multi}}.
\]
In light of $S_{\succ}\left(  \sigma,\pi\right)  =S_{\prec}\left(  \pi
,\sigma\right)  $ and $S_{\succ}\left(  \sigma^{\prime},\pi^{\prime}\right)
=S_{\prec}\left(  \pi^{\prime},\sigma^{\prime}\right)  $, this rewrites as
\[
\left\{  \operatorname*{st}\left(  \tau\right)  \ \mid\ \tau\in S_{\prec
}\left(  \pi,\sigma\right)  \right\}  _{\operatorname*{multi}}=\left\{
\operatorname*{st}\left(  \tau\right)  \ \mid\ \tau\in S_{\prec}\left(
\pi^{\prime},\sigma^{\prime}\right)  \right\}  _{\operatorname*{multi}}.
\]


Also, Claim 1 (applied to $\sigma$, $\pi$, $\sigma^{\prime}$ and $\pi^{\prime
}$ instead of $\pi$, $\sigma$, $\pi^{\prime}$ and $\sigma^{\prime}$) yields%
\[
\left\{  \operatorname*{st}\left(  \tau\right)  \ \mid\ \tau\in S_{\prec
}\left(  \sigma,\pi\right)  \right\}  _{\operatorname*{multi}}=\left\{
\operatorname*{st}\left(  \tau\right)  \ \mid\ \tau\in S_{\prec}\left(
\sigma^{\prime},\pi^{\prime}\right)  \right\}  _{\operatorname*{multi}}.
\]
In light of $S_{\prec}\left(  \sigma,\pi\right)  =S_{\succ}\left(  \pi
,\sigma\right)  $ and $S_{\prec}\left(  \sigma^{\prime},\pi^{\prime}\right)
=S_{\succ}\left(  \pi^{\prime},\sigma^{\prime}\right)  $, this rewrites as%
\[
\left\{  \operatorname*{st}\left(  \tau\right)  \ \mid\ \tau\in S_{\succ
}\left(  \pi,\sigma\right)  \right\}  _{\operatorname*{multi}}=\left\{
\operatorname*{st}\left(  \tau\right)  \ \mid\ \tau\in S_{\succ}\left(
\pi^{\prime},\sigma^{\prime}\right)  \right\}  _{\operatorname*{multi}}.
\]
Thus, Claim 3 is proven in Case 2.

We have now proven Claim 3 in both Cases 1 and 2. Thus, Claim 3 is always proven.]

Claim 3 shows that $\operatorname*{st}$ is LR-shuffle-compatible. This proves
Corollary \ref{cor.LRcomp.two}.
\end{proof}
\end{verlong}

\section{\label{sect.Descent}Descent statistics and quasisymmetric functions}

In this section, we shall recall the concepts of descent statistics and their
shuffle algebras (introduced in \cite{part1}), and apply them to
$\operatorname*{Epk}$.

\subsection{Compositions}

\begin{definition}
A \textit{composition} is a finite list of positive integers. If $I=\left(
i_{1},i_{2},\ldots,i_{n}\right)  $ is a composition, then the nonnegative
integer $i_{1}+i_{2}+\cdots+i_{n}$ is called the \textit{size} of $I$ and is
denoted by $\left\vert I\right\vert $; we furthermore say that $I$ is a
\textit{composition of }$\left\vert I\right\vert $.
\end{definition}

\begin{definition}
\label{def.comps-to-sets}Let $n\in\mathbb{N}$. For each composition $I=\left(
i_{1},i_{2},\ldots,i_{k}\right)  $ of $n$, we define a subset
$\operatorname*{Des}I$ of $\left[  n-1\right]  $ by%
\begin{align*}
\operatorname*{Des}I  &  =\left\{  i_{1},i_{1}+i_{2},i_{1}+i_{2}+i_{3}%
,\ldots,i_{1}+i_{2}+\cdots+i_{k-1}\right\} \\
&  =\left\{  i_{1}+i_{2}+\cdots+i_{s}\ \mid\ s\in\left[  k-1\right]  \right\}
.
\end{align*}


On the other hand, for each subset $A=\left\{  a_{1}<a_{2}<\cdots
<a_{k}\right\}  $ of $\left[  n-1\right]  $, we define a composition
$\operatorname*{Comp}A$ of $n$ by%
\[
\operatorname*{Comp}A=\left(  a_{1},a_{2}-a_{1},a_{3}-a_{2},\ldots
,a_{k}-a_{k-1},n-a_{k}\right)  .
\]
(The definition of $\operatorname*{Comp}A$ should be understood to give
$\operatorname*{Comp}A=\left(  n\right)  $ if $A=\varnothing$. Note that
$\operatorname*{Comp}A$ depends not only on the set $A$ itself, but also on
$n$. We hope that $n$ will always be clear from the context when we use this notation.)

We thus have defined a map $\operatorname*{Des}$ (from the set of all
compositions of $n$ to the set of all subsets of $\left[  n-1\right]  $) and a
map $\operatorname*{Comp}$ (in the opposite direction). These two maps are
mutually inverse bijections.
\end{definition}

\begin{definition}
\label{def.CompDes}Let $n\in\mathbb{N}$. Let $\pi=\left(  \pi_{1},\pi
_{2},\ldots,\pi_{n}\right)  $ be an $n$-permutation. The \textit{descent
composition} of $\pi$ is defined to be the composition $\operatorname*{Comp}%
\left(  \operatorname*{Des}\pi\right)  $ of $n$. This composition is denoted
by $\operatorname*{Comp}\pi$.
\end{definition}

For example, the $6$-permutation $\pi=\left(  4,1,3,9,6,8\right)  $ has
$\operatorname*{Comp}\pi=\left(  1,3,2\right)  $. For another example, the
$6$-permutation $\pi=\left(  1,4,3,2,9,8\right)  $ has $\operatorname*{Comp}%
\pi=\left(  2,1,2,1\right)  $.

Definition \ref{def.CompDes} defines the permutation statistic
$\operatorname*{Comp}$, whose codomain is the set of all compositions.

\subsection{Descent statistics}

\begin{definition}
Let $\operatorname*{st}$ be a permutation statistic. We say that
$\operatorname*{st}$ is a \textit{descent statistic} if and only if
$\operatorname*{st}\pi$ (for $\pi$ a permutation) depends only on the descent
composition $\operatorname*{Comp}\pi$ of $\pi$. In other words,
$\operatorname*{st}$ is a descent statistic if and only if every two
permutations $\pi$ and $\sigma$ satisfying $\operatorname*{Comp}%
\pi=\operatorname*{Comp}\sigma$ satisfy $\operatorname*{st}\pi
=\operatorname*{st}\sigma$.
\end{definition}

Equivalently, a permutation statistic $\operatorname*{st}$ is a descent
statistic if and only if every two permutations $\pi$ and $\sigma$ satisfying
$\left\vert \pi\right\vert =\left\vert \sigma\right\vert $ and
$\operatorname*{Des}\pi=\operatorname*{Des}\sigma$ satisfy $\operatorname*{st}%
\pi=\operatorname*{st}\sigma$. (This is indeed equivalent, because for two
permutations $\pi$ and $\sigma$, the condition $\left(  \left\vert
\pi\right\vert =\left\vert \sigma\right\vert \text{ and }\operatorname*{Des}%
\pi=\operatorname*{Des}\sigma\right)  $ is equivalent to $\left(
\operatorname*{Comp}\pi=\operatorname*{Comp}\sigma\right)  $.)

For example, the permutation statistic $\operatorname*{Des}$ is a descent
statistic, because each permutation $\pi$ satisfies $\operatorname*{Des}%
\pi=\operatorname*{Des}\left(  \operatorname*{Comp}\pi\right)  $. Also,
$\operatorname*{Pk}$ is a descent statistic, since each permutation $\pi$
satisfies%
\[
\operatorname*{Pk}\pi=\left(  \operatorname*{Des}\pi\right)  \setminus\left(
\left\{  1\right\}  \cup\left(  \operatorname*{Des}\pi+1\right)  \right)  ,
\]
where $\operatorname*{Des}\pi+1$ denotes the set $\left\{  i+1\ \mid
\ i\in\operatorname*{Des}\pi\right\}  $ (and, as we have just said,
$\operatorname*{Des}\pi$ can be recovered from $\operatorname*{Comp}\pi$).
Furthermore, $\operatorname*{Epk}$ is a descent statistic, since each
$n$-permutation $\pi$ (for a positive integer $n$) satisfies%
\[
\operatorname*{Epk}\pi=\left(  \operatorname*{Des}\pi\cup\left\{  n\right\}
\right)  \setminus\left(  \operatorname*{Des}\pi+1\right)
\]
(and both $\operatorname*{Des}\pi$ and $n$ can be recovered from
$\operatorname*{Comp}\pi$). The permutation statistics $\operatorname*{Lpk}$
and $\operatorname*{Rpk}$ (and, of course, $\operatorname*{Comp}$) are descent
statistics as well, as one can easily check.

\begin{vershort}
The following fact (proven in \cite{verlong}) should be contrasted with
\cite[Conjecture 6.11]{part1}:
\end{vershort}

\begin{verlong}
Notice the following:
\end{verlong}

\begin{proposition}
\label{prop.des-stat.hgc}Every head-graft-compatible permutation statistic is
a descent statistic.
\end{proposition}

\begin{verlong}
\begin{proof}
[Proof of Proposition \ref{prop.des-stat.hgc}.]Let $\operatorname*{st}$ be a
head-graft-compatible permutation statistic. Then, we must show that
$\operatorname*{st}$ is a descent statistic. But this follows directly from
Lemma \ref{lem.LRcomp.head-l1}.
\end{proof}

Combining Proposition \ref{prop.des-stat.hgc} with Corollary
\ref{cor.LRcomp.back}, we conclude that every LR-shuffle-compatible
permutation statistic is a descent statistic. This should be contrasted with
\cite[Conjecture 6.11]{part1}.
\end{verlong}

\begin{definition}
\label{def.des-stat.stI}Let $\operatorname*{st}$ be a descent statistic. Then,
we can regard $\operatorname*{st}$ as a map from the set of all compositions
(rather than from the set of all permutations). Namely, for any composition
$I$, we define $\operatorname*{st}I$ (an element of the codomain of
$\operatorname*{st}$) by setting%
\[
\operatorname*{st}I=\operatorname*{st}\pi\ \ \ \ \ \ \ \ \ \ \text{for any
permutation }\pi\text{ satisfying }\operatorname*{Comp}\pi=I.
\]
This is well-defined (because for every composition $I$, there exists at least
one permutation $\pi$ satisfying $\operatorname*{Comp}\pi=I$, and all such
permutations $\pi$ have the same value of $\operatorname*{st}\pi$). In the
following, we shall regard every descent statistic $\operatorname*{st}$
simultaneously as a map from the set of all permutations and as a map from the
set of all compositions.
\end{definition}

Note that this definition leads to a new interpretation of
$\operatorname*{Des}I$ for a composition $I$: It is now defined as
$\operatorname*{Des}\pi$ for any permutation $\pi$ satisfying
$\operatorname*{Comp}\pi=I$. This could clash with the old meaning of
$\operatorname*{Des}I$ introduced in Definition \ref{def.comps-to-sets}.
Fortunately, these two meanings of $\operatorname*{Des}I$ are exactly the
same, so there is no conflict of notation.

However, Definition \ref{def.des-stat.stI} causes an ambiguity for expressions
like \textquotedblleft$\operatorname*{Des}\left(  i_{1},i_{2},\ldots
,i_{n}\right)  $\textquotedblright: Here, the \textquotedblleft$\left(
i_{1},i_{2},\ldots,i_{n}\right)  $\textquotedblright\ might be understood
either as a permutation, or as a composition, and the resulting descent sets
$\operatorname*{Des}\left(  i_{1},i_{2},\ldots,i_{n}\right)  $ are not the
same. A similar ambiguity occurs for any descent statistic $\operatorname*{st}%
$ instead of $\operatorname*{Des}$. We hope that this ambiguity will not arise
in this paper due to our explicit typecasting of permutations and
compositions; but the reader should be warned that it can arise if one takes
the notation too literally.

\begin{definition}
\label{def.des-stat.eq-comp}Let $\operatorname*{st}$ be a descent statistic.

\textbf{(a)} Two compositions $J$ and $K$ are said to be $\operatorname*{st}%
$\textit{-equivalent} if and only if they have the same size and satisfy
$\operatorname*{st}J=\operatorname*{st}K$. Equivalently, two compositions $J$
and $K$ are $\operatorname*{st}$-equivalent if and only if there exist two
$\operatorname*{st}$-equivalent permutations $\pi$ and $\sigma$ satisfying
$J=\operatorname*{Comp}\pi$ and $K=\operatorname*{Comp}\sigma$.

\textbf{(b)} The relation \textquotedblleft$\operatorname*{st}$%
-equivalent\textquotedblright\ is an equivalence relation on compositions; its
equivalence classes are called $\operatorname*{st}$\textit{-equivalence
classes of compositions}.
\end{definition}

\subsection{Quasisymmetric functions}

We now recall the definition of quasisymmetric functions; see \cite[Chapter
5]{HopfComb} (and various other modern textbooks) for more details about this:

\begin{definition}
Consider the ring of power series $\mathbb{Q}\left[  \left[  x_{1},x_{2}%
,x_{3},\ldots\right]  \right]  $ in infinitely many commuting indeterminates
over $\mathbb{Q}$. A power series $f\in\mathbb{Q}\left[  \left[  x_{1}%
,x_{2},x_{3},\ldots\right]  \right]  $ is said to be \textit{quasisymmetric}
if it has the following property: For any positive integers $a_{1}%
,a_{2},\ldots,a_{k}$ and any two strictly increasing sequences $\left(
i_{1}<i_{2}<\cdots<i_{k}\right)  $ and $\left(  j_{1}<j_{2}<\cdots
<j_{k}\right)  $ of positive integers, the coefficient of $x_{i_{1}}^{a_{1}%
}x_{i_{2}}^{a_{2}}\cdots x_{i_{k}}^{a_{k}}$ in $f$ equals the coefficient of
$x_{j_{1}}^{a_{1}}x_{j_{2}}^{a_{2}}\cdots x_{j_{k}}^{a_{k}}$ in $f$. A
\textit{quasisymmetric function} is a quasisymmetric power series
$f\in\mathbb{Q}\left[  \left[  x_{1},x_{2},x_{3},\ldots\right]  \right]  $
that has bounded degree (i.e., there exists an $N\in\mathbb{N}$ such that each
monomial appearing in $f$ has degree $\leq N$). The quasisymmetric functions
form a $\mathbb{Q}$-subalgebra of $\mathbb{Q}\left[  \left[  x_{1},x_{2}%
,x_{3},\ldots\right]  \right]  $; this $\mathbb{Q}$-subalgebra is denoted by
$\operatorname*{QSym}$ and called the \textit{ring of quasisymmetric
functions} over $\mathbb{Q}$.
\end{definition}

The $\mathbb{Q}$-algebra $\operatorname*{QSym}$ has much interesting structure
(e.g., it is a Hopf algebra), some of which we will introduce later when we
need it. One simple yet crucial feature of $\operatorname*{QSym}$ that we will
immediately use is the \textit{fundamental basis} of $\operatorname*{QSym}$:

\begin{definition}
For any composition $\alpha$, we define the \textit{fundamental quasisymmetric
function} $F_{\alpha}$ to be the power series%
\[
\sum_{\substack{i_{1}\leq i_{2}\leq\cdots\leq i_{n};\\i_{j}<i_{j+1}\text{ for
each }j\in\operatorname*{Des}\alpha}}x_{i_{1}}x_{i_{2}}\cdots x_{i_{n}}%
\in\operatorname*{QSym},
\]
where $n=\left\vert \alpha\right\vert $ is the size of $\alpha$. The family
$\left(  F_{\alpha}\right)  _{\alpha\text{ is a composition}}$ is a basis of
the $\mathbb{Q}$-vector space $\operatorname*{QSym}$; it is known as the
\textit{fundamental basis} of $\operatorname*{QSym}$.
\end{definition}

We notice that the fundamental quasisymmetric function $F_{\alpha}$ is denoted
by $L_{\alpha}$ in \cite[\S 5.2]{HopfComb}.

The multiplication of fundamental quasisymmetric functions is intimately
related to shuffles of permutations:

\begin{verlong}
\begin{theorem}
\label{thm.4.1}Let $\pi$ and $\sigma$ be two disjoint permutations. Let
$J=\operatorname*{Comp}\pi$ and $K=\operatorname*{Comp}\sigma$. For any
composition $L$, let $c_{J,K}^{L}$ be the number of permutations with descent
composition $L$ among the shuffles of $\pi$ and $\sigma$. Then,%
\[
F_{J}F_{K}=\sum_{L}c_{J,K}^{L}F_{L}%
\]
(where the sum is over all compositions $L$).
\end{theorem}

Theorem \ref{thm.4.1} is \cite[Theorem 4.1]{part1}; it can also be written in
the following form:
\end{verlong}

\begin{proposition}
\label{prop.4.1.rewr}Let $\pi$ and $\sigma$ be two disjoint permutations.
Then,%
\[
F_{\operatorname*{Comp}\pi}F_{\operatorname*{Comp}\sigma}=\sum_{\chi\in
S\left(  \pi,\sigma\right)  }F_{\operatorname*{Comp}\chi}.
\]

\end{proposition}

\begin{vershort}
Proposition \ref{prop.4.1.rewr} is a restatement of \cite[Theorem 4.1]{part1},
and is proven in \cite[(5.2.6)]{HopfComb} (which makes the additional
requirement that the letters of $\pi$ are $1,2,\ldots,\left\vert
\pi\right\vert $ and the letters of $\sigma$ are $\left\vert \pi\right\vert
+1,\left\vert \pi\right\vert +2,\ldots,\left\vert \pi\right\vert +\left\vert
\sigma\right\vert $; but this requirement is not used in the proof and thus
can be dropped).
\end{vershort}

\begin{verlong}
For a proof of Proposition \ref{prop.4.1.rewr} (and therefore also of the
equivalent Theorem \ref{thm.4.1}), we refer to \cite[(5.2.6)]{HopfComb} (which
makes the additional requirement that the letters of $\pi$ are $1,2,\ldots
,\left\vert \pi\right\vert $ and the letters of $\sigma$ are $\left\vert
\pi\right\vert +1,\left\vert \pi\right\vert +2,\ldots,\left\vert
\pi\right\vert +\left\vert \sigma\right\vert $; but this requirement is not
used in the proof and thus can be dropped).
\end{verlong}

\subsection{Shuffle algebras}

Any shuffle-compatible permutation statistic $\operatorname*{st}$ gives rise
to a \textit{shuffle algebra} $\mathcal{A}_{\operatorname*{st}}$, defined as follows:

\begin{definition}
\label{def.Ast}Let $\operatorname*{st}$ be a shuffle-compatible permutation
statistic. For each permutation $\pi$, let $\left[  \pi\right]
_{\operatorname*{st}}$ denote the $\operatorname*{st}$-equivalence class of
$\pi$.

Let $\mathcal{A}_{\operatorname*{st}}$ be the free $\mathbb{Q}$-vector space
whose basis is the set of all $\operatorname*{st}$-equivalence classes of
permutations. We define a multiplication on $\mathcal{A}_{\operatorname*{st}}$
by setting%
\[
\left[  \pi\right]  _{\operatorname*{st}}\left[  \sigma\right]
_{\operatorname*{st}}=\sum_{\tau\in S\left(  \pi,\sigma\right)  }\left[
\tau\right]  _{\operatorname*{st}}%
\]
for any two disjoint permutations $\pi$ and $\sigma$. It is easy to see that
this multiplication is well-defined and associative, and turns $\mathcal{A}%
_{\operatorname*{st}}$ into a $\mathbb{Q}$-algebra whose unity is the
$\operatorname*{st}$-equivalence class of the empty permutation. (In the
particular case when $\operatorname*{st}$ is a descent statistic, this shall
be proven again in Proposition \ref{prop.Ast.alg} \textbf{(a)} below.) This
$\mathbb{Q}$-algebra is denoted by $\mathcal{A}_{\operatorname*{st}}$, and is
called the \textit{shuffle algebra} of $\operatorname*{st}$. It is a graded
$\mathbb{Q}$-algebra; its $n$-th graded component (for each $n\in\mathbb{N}$)
is spanned by the $\operatorname*{st}$-equivalence classes of all $n$-permutations.
\end{definition}

This definition originates in \cite[\S 3.1]{part1}.

\begin{proposition}
\label{prop.Ast.alg}Let $\operatorname*{st}$ be a shuffle-compatible descent statistic.

\textbf{(a)} The multiplication on $\mathcal{A}_{\operatorname*{st}}$ defined
in Definition \ref{def.Ast} is well-defined and associative, and turns
$\mathcal{A}_{\operatorname*{st}}$ into a $\mathbb{Q}$-algebra whose unity is
the $\operatorname*{st}$-equivalence class of the empty permutation.

\textbf{(b)} There is a surjective $\mathbb{Q}$-algebra homomorphism
$p_{\operatorname*{st}}:\operatorname*{QSym}\rightarrow\mathcal{A}%
_{\operatorname*{st}}$ that satisfies
\[
p_{\operatorname*{st}}\left(  F_{\operatorname*{Comp}\pi}\right)  =\left[
\pi\right]  _{\operatorname*{st}}\ \ \ \ \ \ \ \ \ \ \text{for every
permutation }\pi.
\]

\end{proposition}

A central property of the shuffle algebra $\mathcal{A}_{\operatorname*{st}}$
of a shuffle-compatible descent statistic is its relation to
$\operatorname*{QSym}$. This relation is given by \cite[Theorem 4.3]{part1},
which we restate as follows:

\begin{theorem}
\label{thm.4.3}Let $\operatorname*{st}$ be a descent statistic.

\textbf{(a)} The descent statistic $\operatorname*{st}$ is shuffle-compatible
if and only if there exist a $\mathbb{Q}$-algebra $A$ with basis $\left(
u_{\alpha}\right)  $ (indexed by $\operatorname*{st}$-equivalence classes
$\alpha$ of compositions) and a $\mathbb{Q}$-algebra homomorphism
$\phi_{\operatorname*{st}}:\operatorname*{QSym}\rightarrow A$ with the
property that whenever $\alpha$ is an $\operatorname*{st}$-equivalence class
of compositions, we have%
\[
\phi_{\operatorname*{st}}\left(  F_{L}\right)  =u_{\alpha}%
\ \ \ \ \ \ \ \ \ \ \text{for each }L\in\alpha.
\]


\textbf{(b)} In this case, the $\mathbb{Q}$-linear map%
\[
\mathcal{A}_{\operatorname*{st}}\rightarrow A,\ \ \ \ \ \ \ \ \ \ \left[
\pi\right]  _{\operatorname*{st}}\mapsto u_{\alpha},
\]
where $\alpha$ is the $\operatorname*{st}$-equivalence class of the
composition $\operatorname*{Comp}\pi$, is a $\mathbb{Q}$-algebra isomorphism
$\mathcal{A}_{\operatorname*{st}}\rightarrow A$.
\end{theorem}

\begin{vershort}
Proofs of Proposition \ref{prop.Ast.alg} and Theorem \ref{thm.4.3}
(independent of \cite{part1}) can be found in \cite{verlong}.
\end{vershort}

\begin{verlong}
For the sake of completeness, we shall give proofs of Proposition
\ref{prop.Ast.alg} and Theorem \ref{thm.4.3} (independent of \cite{part1}) in
Subsection \ref{subsect.K.pfAsh}.
\end{verlong}

\begin{noncompile}
We shall perhaps also use some other notations that were introduced in
\cite{part1}. The reader should thus consult \cite{part1} for unexplained notations.
\end{noncompile}

\subsection{The shuffle algebra of $\operatorname*{Epk}$}

Theorem \ref{thm.Epk.sh-co-a} yields that the permutation statistic
$\operatorname*{Epk}$ is shuffle-compatible. Hence, the shuffle algebra
$\mathcal{A}_{\operatorname*{Epk}}$ is well-defined. We have little to say
about it:

\begin{theorem}
\label{thm.Epk.AEpk}\textbf{(a)} The shuffle algebra $\mathcal{A}%
_{\operatorname*{Epk}}$ is a graded quotient algebra of $\operatorname*{QSym}$.

\textbf{(b)} Let $n$ be a positive integer. The $n$-th graded component of
$\mathcal{A}_{\operatorname*{Epk}}$ has dimension $f_{n+2}-1$, where $\left(
f_{0},f_{1},f_{2},\ldots\right)  $ is the Fibonacci sequence (defined by
$f_{0}=0$ and $f_{1}=1$ and the recursive relation $f_{m}=f_{m-1}+f_{m-2}$ for
all $m\geq2$).
\end{theorem}

\begin{vershort}
\begin{proof}
[Proof of Theorem \ref{thm.Epk.AEpk}.]See \cite{verlong}.
\end{proof}
\end{vershort}

\begin{verlong}
\begin{proof}
[Proof of Theorem \ref{thm.Epk.AEpk} (sketched).]\textbf{(a)} Proposition
\ref{prop.Ast.alg} \textbf{(b)} (applied to $\operatorname*{st}%
=\operatorname*{Epk}$) yields a surjective $\mathbb{Q}$-algebra homomorphism
$p_{\operatorname*{Epk}}:\operatorname*{QSym}\rightarrow\mathcal{A}%
_{\operatorname*{Epk}}$. It is easy to see that this $p_{\operatorname*{Epk}}$
is furthermore graded (i.e., degree-preserving). Thus, $\mathcal{A}%
_{\operatorname*{Epk}}$ is isomorphic to a quotient of $\operatorname*{QSym}$
as a graded algebra. This proves Theorem \ref{thm.Epk.AEpk} \textbf{(a)}.

\textbf{(b)} The $n$-th graded component of $\mathcal{A}_{\operatorname*{Epk}%
}$ has a basis indexed by $\operatorname*{Epk}$-equivalence classes of
compositions of $n$. These latter classes are in bijection with
$\operatorname*{Epk}$-equivalence classes of $n$-permutations. In turn, the
latter classes are in bijection with the images of $n$-permutations under the
map $\operatorname*{Epk}$. Finally, the latter images are the lacunar nonempty
subsets of $\left[  n\right]  $ (according to Proposition \ref{prop.when-Epk}%
). The number of the latter subsets is $f_{n+2}-1$, because a known fact (see,
e.g., \cite[Exercise 1.35 \textbf{a.}]{Stanley-EC1}) says that the number of
lacunar subsets of $\left[  n\right]  $ is $f_{n+2}$. Combining all of the
preceding sentences, we conclude that the dimension of the $n$-th graded
component of $\mathcal{A}_{\operatorname*{Epk}}$ is $f_{n+2}-1$. This proves
Theorem \ref{thm.Epk.AEpk} \textbf{(b)}.
\end{proof}
\end{verlong}

We can give a somewhat tautological description of $\mathcal{A}%
_{\operatorname*{Epk}}$ using the notations of Section \ref{sect.Zenri}:

\begin{definition}
Let $\Pi_{\mathcal{Z}}$ be the $\mathbb{Q}$-vector subspace of
$\operatorname*{Pow}\mathcal{N}$ spanned by the family $\left(  K_{n,\Lambda
}^{\mathcal{Z}}\right)  _{n>0;\ \Lambda\subseteq\left[  n\right]  \text{ is
lacunar and nonempty}}\cup\left(  K_{0,\varnothing}^{\mathcal{Z}}\right)  $.
Then, $\Pi_{\mathcal{Z}}$ is also the $\mathbb{Q}$-vector subspace of
$\operatorname*{Pow}\mathcal{N}$ spanned by the family $\left(
K_{n,\operatorname*{Epk}\pi}^{\mathcal{Z}}\right)  _{n\in\mathbb{N}%
;\ \pi\text{ is an }n\text{-permutation}}$ (by Proposition \ref{prop.when-Epk}%
). In other words, $\Pi_{\mathcal{Z}}$ is also the $\mathbb{Q}$-vector
subspace of $\operatorname*{Pow}\mathcal{N}$ spanned by the family $\left(
\Gamma_{\mathcal{Z}}\left(  \pi\right)  \right)  _{n\in\mathbb{N};\ \pi\text{
is an }n\text{-permutation}}$ (because of (\ref{eq.def.KnL.2})). Hence,
Corollary \ref{cor.prod2} shows that $\Pi_{\mathcal{Z}}$ is closed under
multiplication. Since furthermore $\Gamma_{\mathcal{Z}}\left(
\left( \right) \right)  =1$ (for the empty $0$-permutation
$\left( \right)$), we can thus conclude that $\Pi
_{\mathcal{Z}}$ is a $\mathbb{Q}$-subalgebra of $\operatorname*{Pow}%
\mathcal{N}$.
\end{definition}

\begin{theorem}
\label{thm.Epk.sh-co}The $\mathbb{Q}$-linear map given by%
\[
\left[  \pi\right]  _{\operatorname*{Epk}}\mapsto K_{n,\operatorname*{Epk}%
}^{\mathcal{Z}}%
\]
is a $\mathbb{Q}$-algebra isomorphism from $\mathcal{A}_{\operatorname*{Epk}}$
to $\Pi_{\mathcal{Z}}$.
\end{theorem}

\begin{vershort}
\begin{proof}
[Proof of Theorem \ref{thm.Epk.sh-co}.]See \cite{verlong}.
\end{proof}
\end{vershort}

\begin{verlong}
In the process of proving Theorem \ref{thm.Epk.sh-co}, we will also prove
Theorem \ref{thm.Epk.sh-co-a} again.

\begin{proof}
[Proof of Theorem \ref{thm.Epk.sh-co} (sketched).]For each positive integer
$n$ and each $n$-permutation $\pi$, we have%
\[
\operatorname*{Epk}\pi=\left(  \operatorname*{Des}\pi\cup\left\{  n\right\}
\right)  \setminus\left(  \operatorname*{Des}\pi+1\right)
\]
(by Proposition \ref{prop.Epk.through-Des}). Thus, $\operatorname*{Epk}\pi$ is
uniquely determined by $\operatorname*{Des}\pi$ and $n$. (Of course, this
holds for $n=0$ as well, because in this case only one $\pi$ exists.) Hence,
$\operatorname*{Epk}$ is a descent statistic. Thus, for every composition $L$,
a set $\operatorname*{Epk}L$ is defined (according to Definition
\ref{def.des-stat.stI}); explicitly, $\operatorname*{Epk}L=\operatorname*{Epk}%
\pi$ whenever $\pi$ is a permutation satisfying $\operatorname*{Comp}\pi=L$.

\begin{noncompile}
We can spell out this definition: For each positive integer $n$ and each
composition $L$ of $n$, the subset $\operatorname*{Epk}L$ of $\left[
n\right]  $ is given by%
\[
\operatorname*{Epk}L=\left(  \operatorname*{Des}L\cup\left\{  n\right\}
\right)  \setminus\left(  \operatorname*{Des}L+1\right)  .
\]

\end{noncompile}

Recall that $\left(  F_{L}\right)  _{L\text{ is a composition}}$ is a basis of
the $\mathbb{Q}$-vector space $\operatorname*{QSym}$. Let $\phi
_{\operatorname*{Epk}}:\operatorname*{QSym}\rightarrow\Pi_{\mathcal{Z}}$ be
the $\mathbb{Q}$-linear map that sends each $F_{L}$ (for each $n\in\mathbb{N}$
and each composition $L$ of $n$) to $K_{n,\operatorname*{Epk}L}^{\mathcal{Z}%
}\in\Pi_{\mathcal{Z}}$. This $\mathbb{Q}$-linear map $\phi
_{\operatorname*{Epk}}$ respects multiplication\footnote{\textit{Proof.} Let
$n\in\mathbb{N}$ and $m\in\mathbb{N}$. Let $J$ be a composition of $n$, and
let $K$ be a composition of $m$. Fix an $n$-permutation $\pi$ satisfying
$\operatorname*{Comp}\pi=J$, and fix an $m$-permutation $\sigma$ satisfying
$\operatorname*{Comp}\sigma=K$. Now,%
\begin{align*}
&  \underbrace{\phi_{\operatorname*{Epk}}\left(  F_{J}\right)  }%
_{\substack{=K_{n,\operatorname*{Epk}J}^{\mathcal{Z}}=K_{n,\operatorname*{Epk}%
\pi}^{\mathcal{Z}}\\\text{(since }\operatorname*{Epk}J=\operatorname*{Epk}%
\pi\text{)}}}\cdot\underbrace{\phi_{\operatorname*{Epk}}\left(  F_{K}\right)
}_{\substack{=K_{m,\operatorname*{Epk}K}^{\mathcal{Z}}%
=K_{m,\operatorname*{Epk}\sigma}^{\mathcal{Z}}\\\text{(since }%
\operatorname*{Epk}K=\operatorname*{Epk}\sigma\text{)}}}\\
&  =K_{n,\operatorname*{Epk}\pi}^{\mathcal{Z}}\cdot K_{m,\operatorname*{Epk}%
\sigma}^{\mathcal{Z}}=\sum_{\tau\in S\left(  \pi,\sigma\right)  }%
K_{n+m,\operatorname*{Epk}\tau}^{\mathcal{Z}}\ \ \ \ \ \ \ \ \ \ \left(
\text{by Corollary \ref{cor.KnEpk-prodrule}}\right)  .
\end{align*}
Comparing this with%
\begin{align*}
\phi_{\operatorname*{Epk}}\left(  \underbrace{F_{J}F_{K}}_{\substack{=\sum
_{\tau\in S\left(  \pi,\sigma\right)  }F_{\operatorname*{Comp}\tau}\\\text{(by
Proposition~\ref{prop.4.1.rewr})}}}\right)   &  =\phi_{\operatorname*{Epk}%
}\left(  \sum_{\tau\in S\left(  \pi,\sigma\right)  }F_{\operatorname*{Comp}%
\tau}\right)  =\sum_{\tau\in S\left(  \pi,\sigma\right)  }\underbrace{\phi
_{\operatorname*{Epk}}\left(  F_{\operatorname*{Comp}\tau}\right)
}_{\substack{=K_{n+m,\operatorname*{Epk}\left(  \operatorname*{Comp}%
\tau\right)  }^{\mathcal{Z}}=K_{n+m,\operatorname*{Epk}\tau}^{\mathcal{Z}%
}\\\text{(since }\operatorname*{Epk}\left(  \operatorname*{Comp}\tau\right)
=\operatorname*{Epk}\tau\text{)}}}\\
&  =\sum_{\tau\in S\left(  \pi,\sigma\right)  }K_{n+m,\operatorname*{Epk}\tau
}^{\mathcal{Z}},
\end{align*}
we obtain $\phi_{\operatorname*{Epk}}\left(  F_{J}\right)  \cdot
\phi_{\operatorname*{Epk}}\left(  F_{K}\right)  =\phi_{\operatorname*{Epk}%
}\left(  F_{J}F_{K}\right)  $. Since the map $\phi_{\operatorname*{Epk}}$ is
$\mathbb{Q}$-linear, this yields that $\phi_{\operatorname*{Epk}}$ respects
multiplication (since $\left(  F_{L}\right)  _{L\text{ is a composition}}$ is
a basis of the $\mathbb{Q}$-vector space $\operatorname*{QSym}$).} and sends
$1\in\operatorname*{QSym}$ to $1\in\Pi_{\mathcal{Z}}$\ \ \ \ \footnote{This is
easy, since $1=F_{\left(  {}\right)  }$.}. Thus, $\phi_{\operatorname*{Epk}}$
is a $\mathbb{Q}$-algebra homomorphism.

The family $\left(  K_{n,\Lambda}^{\mathcal{Z}}\right)  _{n>0;\ \Lambda
\subseteq\left[  n\right]  \text{ is lacunar and nonempty}}\cup\left(
K_{0,\varnothing}^{\mathcal{Z}}\right)  $ spans $\Pi_{\mathcal{Z}}$ (by the
definition of $\Pi_{\mathcal{Z}}$) and is $\mathbb{Q}$-linearly independent
(by Corollary \ref{cor.KnL.lindep-all}). Thus, it is a basis of $\Pi
_{\mathcal{Z}}$.

For each positive integer $n$, there is a canonical bijection between the
$\operatorname*{Epk}$-equivalence classes of $n$-permutations and the nonempty
lacunar subsets $\Lambda$ of $\left[  n\right]  $ (indeed, the bijection sends
any equivalence class $\left[  \pi\right]  _{\operatorname*{Epk}}$ to
$\operatorname*{Epk}\pi$)\ \ \ \ \footnote{Proposition \ref{prop.when-Epk}
shows that this is indeed a bijection.}. Hence, Theorem~\ref{thm.4.3}
\textbf{(a)} (applied to $A=\Pi_{\mathcal{Z}}$, $\operatorname*{st}%
=\operatorname*{Epk}$ and $u_{\alpha}=K_{n,\Lambda}^{\mathcal{Z}}$, where
$\alpha$ is an $\operatorname*{Epk}$-equivalence class of $n$-permutations and
where $\Lambda$ is the corresponding nonempty\footnote{or empty, if $n=0$}
lacunar subset) shows that the descent statistic $\operatorname*{Epk}$ is
shuffle-compatible. This proves Theorem \ref{thm.Epk.sh-co-a} again.
Theorem~\ref{thm.4.3} \textbf{(b)} then yields that the $\mathbb{Q}$-linear
map%
\[
\mathcal{A}_{\operatorname*{Epk}}\rightarrow\Pi_{\mathcal{Z}}%
,\ \ \ \ \ \ \ \ \ \ \left[  \pi\right]  _{\operatorname*{Epk}}\mapsto
K_{n,\operatorname*{Epk}\pi}^{\mathcal{Z}}%
\]
is a $\mathbb{Q}$-algebra isomorphism from $\mathcal{A}_{\operatorname*{Epk}}$
to $\Pi_{\mathcal{Z}}$. Hence, Theorem \ref{thm.Epk.sh-co} is proven.
\end{proof}
\end{verlong}

\section{\label{sect.kernel}The kernel of the map $\operatorname*{QSym}%
\rightarrow\mathcal{A}_{\operatorname*{Epk}}$}

\subsection{The kernel of a descent statistic}

Now, we shall focus on a feature of shuffle-compatible descent statistics that
seems to have been overlooked so far: their kernels.

\begin{vershort}
All proofs in this section are omitted; they can be found in \cite{verlong}.
\end{vershort}

\begin{definition}
Let $\operatorname*{st}$ be a descent statistic. Then, $\mathcal{K}%
_{\operatorname*{st}}$ shall mean the $\mathbb{Q}$-vector subspace of
$\operatorname*{QSym}$ spanned by all elements of the form $F_{J}-F_{K}$,
where $J$ and $K$ are two $\operatorname*{st}$-equivalent compositions. (See
Definition \ref{def.des-stat.eq-comp} \textbf{(a)} for the definition of
\textquotedblleft$\operatorname*{st}$-equivalent
compositions\textquotedblright.) We shall refer to $\mathcal{K}%
_{\operatorname*{st}}$ as the \textit{kernel} of $\operatorname*{st}$.
\end{definition}

The following basic linear-algebraic lemma will be useful:

\begin{lemma}
\label{lem.K.fist}Let $\operatorname*{st}$ be a descent statistic. Let $A$ be
a $\mathbb{Q}$-vector space with basis $\left(  u_{\alpha}\right)  $ indexed
by $\operatorname*{st}$-equivalence classes $\alpha$ of compositions. Let
$\phi_{\operatorname*{st}}:\operatorname*{QSym}\rightarrow A$ be a
$\mathbb{Q}$-linear map with the property that whenever $\alpha$ is an
$\operatorname*{st}$-equivalence class of compositions, we have%
\begin{equation}
\phi_{\operatorname*{st}}\left(  F_{L}\right)  =u_{\alpha}%
\ \ \ \ \ \ \ \ \ \ \text{for each }L\in\alpha. \label{pf.prop.K.ideal.dir1.1}%
\end{equation}


Then, $\operatorname*{Ker}\left(  \phi_{\operatorname*{st}}\right)
=\mathcal{K}_{\operatorname*{st}}$.
\end{lemma}

\begin{verlong}
\begin{proof}
[Proof of Lemma \ref{lem.K.fist}.]Let us first show that $\operatorname*{Ker}%
\left(  \phi_{\operatorname*{st}}\right)  \subseteq\mathcal{K}%
_{\operatorname*{st}}$.

Indeed, let $x\in\operatorname*{Ker}\left(  \phi_{\operatorname*{st}}\right)
$ be arbitrary. Write $x\in\operatorname*{QSym}$ in the form $x=\sum_{L}%
x_{L}F_{L}$, where the sum ranges over all compositions $L$, and where the
$x_{L}$ are elements of $\mathbb{Q}$ (all but finitely many of which are
zero). (This can be done, since $\left(  F_{L}\right)  _{L\text{ is a
composition}}$ is a basis of the $\mathbb{Q}$-vector space
$\operatorname*{QSym}$.) Now, $x\in\operatorname*{Ker}\left(  \phi
_{\operatorname*{st}}\right)  $, so that $\phi_{\operatorname*{st}}\left(
x\right)  =0$. Thus,%
\begin{align*}
0  &  =\phi_{\operatorname*{st}}\left(  x\right)  =\sum_{L}x_{L}%
\phi_{\operatorname*{st}}\left(  F_{L}\right)  \ \ \ \ \ \ \ \ \ \ \left(
\text{since }x=\sum_{L}x_{L}F_{L}\right) \\
&  =\sum_{\alpha}\sum_{L\in\alpha}x_{L}\underbrace{\phi_{\operatorname*{st}%
}\left(  F_{L}\right)  }_{\substack{=u_{\alpha}\\\text{(by
(\ref{pf.prop.K.ideal.dir1.1}))}}}\ \ \ \ \ \ \ \ \ \ \left(
\begin{array}
[c]{c}%
\text{where the first sum is over}\\
\text{all }\operatorname*{st}\text{-equivalence classes }\alpha\text{ of
compositions}%
\end{array}
\right) \\
&  =\sum_{\alpha}\sum_{L\in\alpha}x_{L}u_{\alpha}=\sum_{\alpha}\left(
\sum_{L\in\alpha}x_{L}\right)  u_{\alpha}.
\end{align*}
Since the family $\left(  u_{\alpha}\right)  $ is linearly independent
(because it is a basis of $A$), we thus conclude that
\begin{equation}
\sum_{L\in\alpha}x_{L}=0 \label{pf.prop.K.ideal.dir1.2}%
\end{equation}
for each $\operatorname*{st}$-equivalence class $\alpha$ of compositions.

Now, for each $\operatorname*{st}$-equivalence class $\alpha$ of compositions,
we fix an element $L_{\alpha}$ of $\alpha$. Then, for each $\operatorname*{st}%
$-equivalence class $\alpha$ of compositions and each composition $L\in\alpha
$, we have
\begin{equation}
F_{L}-F_{L_{\alpha}}\in\mathcal{K}_{\operatorname*{st}}
\label{pf.prop.K.ideal.dir1.3}%
\end{equation}
(since the compositions $L$ and $L_{\alpha}$ are $\operatorname*{st}%
$-equivalent\footnote{since both compositions $L$ and $L_{\alpha}$ lie in the
same $\operatorname*{st}$-equivalence class $\alpha$}).

Now,%
\begin{align*}
x  &  =\sum_{L}x_{L}F_{L}\\
&  =\sum_{\alpha}\sum_{L\in\alpha}x_{L}\underbrace{F_{L}}_{=\left(
F_{L}-F_{L_{\alpha}}\right)  +F_{L_{\alpha}}}\ \ \ \ \ \ \ \ \ \ \left(
\begin{array}
[c]{c}%
\text{where the first sum is over}\\
\text{all }\operatorname*{st}\text{-equivalence classes }\alpha\text{ of
compositions}%
\end{array}
\right) \\
&  =\sum_{\alpha}\sum_{L\in\alpha}x_{L}\left(  \left(  F_{L}-F_{L_{\alpha}%
}\right)  +F_{L_{\alpha}}\right) \\
&  =\sum_{\alpha}\sum_{L\in\alpha}x_{L}\underbrace{\left(  F_{L}-F_{L_{\alpha
}}\right)  }_{\substack{\in\mathcal{K}_{\operatorname*{st}}\\\text{(by
(\ref{pf.prop.K.ideal.dir1.3}))}}}+\sum_{\alpha}\underbrace{\sum_{L\in\alpha
}x_{L}}_{\substack{=0\\\text{(by (\ref{pf.prop.K.ideal.dir1.2}))}%
}}F_{L_{\alpha}}\\
&  \in\underbrace{\sum_{\alpha}\sum_{L\in\alpha}x_{L}\mathcal{K}%
_{\operatorname*{st}}}_{\subseteq\mathcal{K}_{\operatorname*{st}}%
}+\underbrace{\sum_{\alpha}0F_{L_{\alpha}}}_{=0}\subseteq\mathcal{K}%
_{\operatorname*{st}}.
\end{align*}


Now, forget that we fixed $x$. We thus have proven that $x\in\mathcal{K}%
_{\operatorname*{st}}$ for each $x\in\operatorname*{Ker}\left(  \phi
_{\operatorname*{st}}\right)  $. In other words, $\operatorname*{Ker}\left(
\phi_{\operatorname*{st}}\right)  \subseteq\mathcal{K}_{\operatorname*{st}}$.

Conversely, it is easy to see that $\mathcal{K}_{\operatorname*{st}}%
\subseteq\operatorname*{Ker}\left(  \phi_{\operatorname*{st}}\right)
$\ \ \ \ \footnote{\textit{Proof.} Recall that $\mathcal{K}%
_{\operatorname*{st}}$ is the $\mathbb{Q}$-vector subspace of
$\operatorname*{QSym}$ spanned by all elements of the form $F_{J}-F_{K}$,
where $J$ and $K$ are two $\operatorname*{st}$-equivalent compositions. Hence,
in order to prove that $\mathcal{K}_{\operatorname*{st}}\subseteq
\operatorname*{Ker}\left(  \phi_{\operatorname*{st}}\right)  $, it suffices to
show that $F_{J}-F_{K}\in\operatorname*{Ker}\left(  \phi_{\operatorname*{st}%
}\right)  $, whenever $J$ and $K$ are two $\operatorname*{st}$-equivalent
compositions. So let $J$ and $K$ be two $\operatorname*{st}$-equivalent
compositions. We must show that $F_{J}-F_{K}\in\operatorname*{Ker}\left(
\phi_{\operatorname*{st}}\right)  $.
\par
The two compositions $J$ and $K$ are $\operatorname*{st}$-equivalent. Hence,
they lie in one and the same $\operatorname*{st}$-equivalence class. Let
$\alpha$ be this $\operatorname*{st}$-equivalence class. Then, $J\in\alpha$
and therefore $\phi_{\operatorname*{st}}\left(  F_{J}\right)  =u_{\alpha}$ (by
(\ref{pf.prop.K.ideal.dir1.1}), applied to $L=J$). Similarly, $\phi
_{\operatorname*{st}}\left(  F_{K}\right)  =u_{\alpha}$. Now, $\phi
_{\operatorname*{st}}\left(  F_{J}-F_{K}\right)  =\underbrace{\phi
_{\operatorname*{st}}\left(  F_{J}\right)  }_{=u_{\alpha}}-\underbrace{\phi
_{\operatorname*{st}}\left(  F_{K}\right)  }_{=u_{\alpha}}=u_{\alpha
}-u_{\alpha}=0$. In other words, $F_{J}-F_{K}\in\operatorname*{Ker}\left(
\phi_{\operatorname*{st}}\right)  $. This completes our proof.}. Combining
this with $\operatorname*{Ker}\left(  \phi_{\operatorname*{st}}\right)
\subseteq\mathcal{K}_{\operatorname*{st}}$, we obtain $\mathcal{K}%
_{\operatorname*{st}}=\operatorname*{Ker}\left(  \phi_{\operatorname*{st}%
}\right)  $. This proves Lemma \ref{lem.K.fist}.
\end{proof}
\end{verlong}

Theorem~\ref{thm.4.3} easily yields the following fact:

\begin{proposition}
\label{prop.K.ideal}Let $\operatorname*{st}$ be a descent statistic. Then,
$\operatorname*{st}$ is shuffle-compatible if and only if $\mathcal{K}%
_{\operatorname*{st}}$ is an ideal of $\operatorname*{QSym}$. Furthermore, in
this case, $\mathcal{A}_{\operatorname*{st}}\cong\operatorname*{QSym}%
/\mathcal{K}_{\operatorname*{st}}$.
\end{proposition}

\begin{verlong}
\begin{proof}
[Proof of Proposition \ref{prop.K.ideal}.]$\Longrightarrow:$ Assume that
$\operatorname*{st}$ is shuffle-compatible. Thus, Theorem~\ref{thm.4.3}
\textbf{(a)} shows that there exist a $\mathbb{Q}$-algebra $A$ with basis
$\left(  u_{\alpha}\right)  $ indexed by $\operatorname*{st}$-equivalence
classes $\alpha$ of compositions, and a $\mathbb{Q}$-algebra homomorphism
$\phi_{\operatorname*{st}}:\operatorname*{QSym}\rightarrow A$ with the
property that whenever $\alpha$ is an $\operatorname*{st}$-equivalence class
of compositions, we have%
\begin{equation}
\phi_{\operatorname*{st}}\left(  F_{L}\right)  =u_{\alpha}%
\ \ \ \ \ \ \ \ \ \ \text{for each }L\in\alpha.
\label{pf.prop.K.ideal.dir1.13}%
\end{equation}
Consider this $A$ and this $\phi_{\operatorname*{st}}$.

Lemma \ref{lem.K.fist} yields that $\mathcal{K}_{\operatorname*{st}%
}=\operatorname*{Ker}\left(  \phi_{\operatorname*{st}}\right)  $. But the map
$\phi_{\operatorname*{st}}$ is a $\mathbb{Q}$-algebra homomorphism. Thus, its
kernel $\operatorname*{Ker}\left(  \phi_{\operatorname*{st}}\right)  $ is an
ideal of $\operatorname*{QSym}$. In other words, $\mathcal{K}%
_{\operatorname*{st}}$ is an ideal of $\operatorname*{QSym}$ (since
$\mathcal{K}_{\operatorname*{st}}=\operatorname*{Ker}\left(  \phi
_{\operatorname*{st}}\right)  $).

It remains to show that $\mathcal{A}_{\operatorname*{st}}\cong%
\operatorname*{QSym}/\mathcal{K}_{\operatorname*{st}}$. This is easy: Each
element of the basis $\left(  u_{\alpha}\right)  $ of the $\mathbb{Q}$-vector
space $A$ is contained in the image of $\phi_{\operatorname*{st}}$ (because of
(\ref{pf.prop.K.ideal.dir1.13})). Therefore, the homomorphism $\phi
_{\operatorname*{st}}$ is surjective. Thus, $\phi_{\operatorname*{st}}\left(
\operatorname*{QSym}\right)  =A$. Hence, $A=\phi_{\operatorname*{st}}\left(
\operatorname*{QSym}\right)  \cong\operatorname*{QSym}/\operatorname*{Ker}%
\left(  \phi_{\operatorname*{st}}\right)  $ (by the homomorphism theorem). But
Theorem \ref{thm.4.3} \textbf{(b)} shows that $\mathcal{A}_{\operatorname*{st}%
}\cong A$. Thus, $\mathcal{A}_{\operatorname*{st}}\cong A\cong%
\operatorname*{QSym}/\underbrace{\operatorname*{Ker}\left(  \phi
_{\operatorname*{st}}\right)  }_{=\mathcal{K}_{\operatorname*{st}}%
}=\operatorname*{QSym}/\mathcal{K}_{\operatorname*{st}}$. This finishes the
proof of the $\Longrightarrow$ direction of Proposition \ref{prop.K.ideal}.

$\Longleftarrow:$ Assume that $\mathcal{K}_{\operatorname*{st}}$ is an ideal
of $\operatorname*{QSym}$. We must prove that $\operatorname*{st}$ is shuffle-compatible.

We shall not use this direction of Proposition \ref{prop.K.ideal}, so let us
merely sketch the proof. Let $A$ be the $\mathbb{Q}$-algebra
$\operatorname*{QSym}/\mathcal{K}_{\operatorname*{st}}$. Let $\phi
_{\operatorname*{st}}$ be the canonical projection $\operatorname*{QSym}%
\rightarrow A$; this is clearly a $\mathbb{Q}$-algebra homomorphism.

For each $\operatorname*{st}$-equivalence class $\alpha$ of compositions, we
define an element $u_{\alpha}$ of $A$ by requiring that%
\[
u_{\alpha}=\phi_{\operatorname*{st}}\left(  F_{L}\right)
\ \ \ \ \ \ \ \ \ \ \text{whenever }L\in\alpha.
\]
This is easily seen to be well-defined, because the image $\phi
_{\operatorname*{st}}\left(  F_{L}\right)  $ depends only on $\alpha$ but not
on $L$ (indeed, if $J$ and $K$ are two elements of $\alpha$, then $J$ and $K$
are $\operatorname*{st}$-equivalent, whence $F_{J}-F_{K}\in\mathcal{K}%
_{\operatorname*{st}}$, whence $F_{J}\equiv F_{K}\operatorname{mod}%
\mathcal{K}_{\operatorname*{st}}$ and therefore $\phi_{\operatorname*{st}%
}\left(  F_{J}\right)  =\phi_{\operatorname*{st}}\left(  F_{K}\right)  $).

It is not hard to see that the family $\left(  u_{\alpha}\right)  $ (where
$\alpha$ ranges over all $\operatorname*{st}$-equivalence classes of
compositions) is a basis of the $\mathbb{Q}$-algebra $A$. Hence, Theorem
\ref{thm.4.3} \textbf{(a)} yields that $\operatorname*{st}$ is
shuffle-compatible. This proves the $\Longleftarrow$ direction of Proposition
\ref{prop.K.ideal}.
\end{proof}
\end{verlong}

\begin{corollary}
\label{cor.Epk.ideal}The kernel $\mathcal{K}_{\operatorname*{Epk}}$ of the
descent statistic $\operatorname*{Epk}$ is an ideal of $\operatorname*{QSym}$.
\end{corollary}

\begin{verlong}
\begin{proof}
[Proof of Corollary \ref{cor.Epk.ideal}.]This follows from Proposition
\ref{prop.K.ideal} (applied to $\operatorname*{st}=\operatorname*{Epk}$),
because of Theorem \ref{thm.Epk.sh-co-a}.
\end{proof}
\end{verlong}

We can study the kernel of any descent statistic; in particular, the case of
shuffle-compatible descent statistics appears interesting. Since
$\operatorname*{QSym}$ is isomorphic to a polynomial ring (as an algebra), it
has many ideals, which are rather hopeless to classify or tame; but the ones
obtained as kernels of shuffle-compatible descent statistics might be worth discussing.

\subsection{The F-generating set of $\mathcal{K}_{\operatorname*{Epk}}$}

Let us now focus on $\mathcal{K}_{\operatorname*{Epk}}$, the kernel of
$\operatorname*{Epk}$.

\begin{proposition}
\label{prop.K.Epk.F}If $J=\left(  j_{1},j_{2},\ldots,j_{m}\right)  $ and $K$
are two compositions, then we shall write $J\rightarrow K$ if there exists an
$\ell\in\left\{  2,3,\ldots,m\right\}  $ such that $j_{\ell}>2$ and $K=\left(
j_{1},j_{2},\ldots,j_{\ell-1},1,j_{\ell}-1,j_{\ell+1},j_{\ell+2},\ldots
,j_{m}\right)  $. (In other words, we write $J\rightarrow K$ if $K$ can be
obtained from $J$ by \textquotedblleft splitting\textquotedblright\ some entry
$j_{\ell}>2$ into two consecutive entries\footnotemark\ $1$ and $j_{\ell}-1$,
provided that this entry was not the first entry -- i.e., we had $\ell>1$ --
and that this entry was greater than $2$.)

The ideal $\mathcal{K}_{\operatorname*{Epk}}$ of $\operatorname*{QSym}$ is
spanned (as a $\mathbb{Q}$-vector space) by all differences of the form
$F_{J}-F_{K}$, where $J$ and $K$ are two compositions satisfying $J\rightarrow
K$.
\end{proposition}

\footnotetext{The word \textquotedblleft consecutive\textquotedblright\ here
means \textquotedblleft in consecutive positions of $J$\textquotedblright, not
\textquotedblleft consecutive integers\textquotedblright. So two consecutive
entries of $J$ are two entries of the form $j_{p}$ and $j_{p+1}$ for some
$p\in\left\{  1,2,\ldots,m-1\right\}  $.}

\begin{example}
We have $\left(  2,1,4,4\right)  \rightarrow\left(  2,1,1,3,4\right)  $, since
the composition $\left(  2,1,1,3,4\right)  $ is obtained from $\left(
2,1,4,4\right)  $ by splitting the third entry (which is $4>2$) into two
consecutive entries $1$ and $3$.

Similarly, $\left(  2,1,4,4\right)  \rightarrow\left(  2,1,4,1,3\right)  $.

But we do not have $\left(  3,1\right)  \rightarrow\left(  1,2,1\right)  $,
because splitting the first entry of the composition is not allowed in the
definition of the relation $\rightarrow$. Also, we do not have $\left(
1,2,1\right)  \rightarrow\left(  1,1,1,1\right)  $, because the entry we are
splitting must be $>2$.

Two compositions $J$ and $K$ satisfying $J\rightarrow K$ must necessarily
satisfy $\left\vert J\right\vert =\left\vert K\right\vert $.

Here are all relations $\rightarrow$ between compositions of size $4$:%
\[
\left(  1,3\right)  \rightarrow\left(  1,1,2\right)  .
\]


Here are all relations $\rightarrow$ between compositions of size $5$:%
\begin{align*}
\left(  1,4\right)   &  \rightarrow\left(  1,1,3\right)  ,\\
\left(  1,3,1\right)   &  \rightarrow\left(  1,1,2,1\right)  ,\\
\left(  1,1,3\right)   &  \rightarrow\left(  1,1,1,2\right)  ,\\
\left(  2,3\right)   &  \rightarrow\left(  2,1,2\right)  .
\end{align*}
There are no relations $\rightarrow$ between compositions of size $\leq3$.
\end{example}

\begin{verlong}
\begin{proof}
[Proof of Proposition \ref{prop.K.Epk.F}.]We begin by proving some simple claims.

\begin{statement}
\textit{Claim 1:} Let $n\in\mathbb{N}$. Let $J$ and $K$ be two compositions of
size $n$. Then, $J\rightarrow K$ if and only if there exists some $k\in\left[
n-1\right]  $ such that
\begin{align*}
\operatorname*{Des}K  &  =\operatorname*{Des}J\cup\left\{  k\right\}
,\ \ \ \ \ \ \ \ \ \ k\notin\operatorname*{Des}J,\\
k-1  &  \in\operatorname*{Des}J\ \ \ \ \ \ \ \ \ \ \text{and}%
\ \ \ \ \ \ \ \ \ \ k+1\notin\operatorname*{Des}J\cup\left\{  n\right\}  .
\end{align*}

\end{statement}

[\textit{Proof of Claim 1:} This is straightforward to check:
\textquotedblleft Splitting\textquotedblright\ an entry of a composition $C$
into two consecutive entries (summing up to the original entry) is always
tantamount to adding a new element to $\operatorname*{Des}C$. The rest is
translating conditions.]

If $n$ is a positive integer, and $L$ is any composition of $n$, then
\begin{equation}
\operatorname*{Epk}L=\left(  \operatorname*{Des}L\cup\left\{  n\right\}
\right)  \setminus\left(  \operatorname*{Des}L+1\right)  .
\label{pf.prop.K.Epk.F.1}%
\end{equation}
(This is a consequence of Proposition \ref{prop.Epk.through-Des}, applied to
any $n$-permutation $\pi$ satisfying $L=\operatorname*{Comp}\pi$.)

\begin{noncompile}
Also, $\operatorname*{Epk}L=\operatorname*{Lpk}L\cup\operatorname*{Rpk}L$ for
any integer $n\geq2$ and any composition $L$ of $n$.
\end{noncompile}

\begin{statement}
\textit{Claim 2:} Let $J$ and $K$ be two compositions satisfying $J\rightarrow
K$. Then, $\operatorname*{Epk}J=\operatorname*{Epk}K$.
\end{statement}

[\textit{Proof of Claim 2:} Easy consequence of Claim 1 and
(\ref{pf.prop.K.Epk.F.1}).]

For any two integers $a$ and $b$, we set $\left[  a,b\right]  =\left\{
a,a+1,\ldots,b\right\}  $. (This is an empty set if $a>b$.)

It is easy to see that every composition $J$ of size $n>0$ satisfies%
\begin{equation}
\left[  \max\left(  \operatorname*{Epk}J\right)  ,n-1\right]  \subseteq
\operatorname*{Des}J \label{pf.prop.K.Epk.F.2}%
\end{equation}
\footnote{\textit{Proof of (\ref{pf.prop.K.Epk.F.2}):} Let $J$ be a
composition of size $n>0$. We shall show that $\left[  \max\left(
\operatorname*{Epk}J\right)  ,n-1\right]  \subseteq\operatorname*{Des}J$.
\par
Indeed, assume the contrary. Thus, $\left[  \max\left(  \operatorname*{Epk}%
J\right)  ,n-1\right]  \not \subseteq \operatorname*{Des}J$. Hence, there
exists some $q\in\left[  \max\left(  \operatorname*{Epk}J\right)  ,n-1\right]
$ satisfying $q\notin\operatorname*{Des}J$. Let $r$ be the \textbf{largest}
such $q$.
\par
Thus, $r\in\left[  \max\left(  \operatorname*{Epk}J\right)  ,n-1\right]  $ but
$r\notin\operatorname*{Des}J$. From $r\in\left[  \max\left(
\operatorname*{Epk}J\right)  ,n-1\right]  \subseteq\left[  n-1\right]  $, we
obtain $r+1\in\left[  n\right]  $. Also, from $r\notin\operatorname*{Des}J$,
we obtain $r+1\notin\operatorname*{Des}J+1$.
\par
From $r\in\left[  \max\left(  \operatorname*{Epk}J\right)  ,n-1\right]  $, we
obtain $r\geq\max\left(  \operatorname*{Epk}J\right)  $, so that
$r+1>r\geq\max\left(  \operatorname*{Epk}J\right)  $ and therefore
$r+1\notin\operatorname*{Epk}J$ (since a number that is higher than
$\max\left(  \operatorname*{Epk}J\right)  $ cannot belong to
$\operatorname*{Epk}J$).
\par
From (\ref{pf.prop.K.Epk.F.1}), we obtain $\operatorname*{Epk}J=\left(
\operatorname*{Des}J\cup\left\{  n\right\}  \right)  \setminus\left(
\operatorname*{Des}J+1\right)  $.
\par
If we had $r+1\in\operatorname*{Des}J\cup\left\{  n\right\}  $, then we would
have $r+1\in\left(  \operatorname*{Des}J\cup\left\{  n\right\}  \right)
\setminus\left(  \operatorname*{Des}J+1\right)  $ (since $r+1\notin%
\operatorname*{Des}J+1$). This would contradict $r+1\notin\operatorname*{Epk}%
J=\left(  \operatorname*{Des}J\cup\left\{  n\right\}  \right)  \setminus
\left(  \operatorname*{Des}J+1\right)  $. Thus, we cannot have $r+1\in
\operatorname*{Des}J\cup\left\{  n\right\}  $. Therefore, $r+1\notin%
\operatorname*{Des}J\cup\left\{  n\right\}  $.
\par
Hence, $r+1\neq n$ (since $r+1\notin\operatorname*{Des}J\cup\left\{
n\right\}  $ but $n\in\left\{  n\right\}  \subseteq\operatorname*{Des}%
J\cup\left\{  n\right\}  $). Combined with $r+1\in\left[  n\right]  $, this
yields $r+1\in\left[  n\right]  \setminus\left\{  n\right\}  =\left[
n-1\right]  $. Combined with $r+1>\max\left(  \operatorname*{Epk}J\right)  $,
this yields $r+1\in\left[  \max\left(  \operatorname*{Epk}J\right)
,n-1\right]  $. Also, $r+1\notin\operatorname*{Des}J$ (since $r+1\notin%
\operatorname*{Des}J\cup\left\{  n\right\}  $).
\par
Thus, $r+1$ is a $q\in\left[  \max\left(  \operatorname*{Epk}J\right)
,n-1\right]  $ satisfying $q\notin\operatorname*{Des}J$. This contradicts the
fact that $r$ is the \textbf{largest} such $q$ (since $r+1$ is clearly larger
than $r$). This contradiction proves that our assumption was wrong; thus,
(\ref{pf.prop.K.Epk.F.2}) is proven.}.

For each $n\in\mathbb{N}$ and each subset $S$ of $\left[  n-1\right]  $, we
define a subset $S^{\circ}$ of $\left[  n-1\right]  $ as follows:%
\[
S_{n}^{\circ}=\left\{  s\in S\ \mid\ s-1\notin S\text{ or }\left[
s,n-1\right]  \subseteq S\right\}  .
\]


Also, for each $n\in\mathbb{N}$ and each nonempty subset $T$ of $\left[
n\right]  $, we define a subset $\rho_{n}\left(  T\right)  $ of $\left[
n-1\right]  $ as follows:%
\[
\rho_{n}\left(  T\right)  =%
\begin{cases}
T\setminus\left\{  n\right\}  , & \text{if }n\in T;\\
T\cup\left[  \max T,n-1\right]  , & \text{if }n\notin T
\end{cases}
.
\]


\begin{statement}
\textit{Claim 3:} Let $n\in\mathbb{N}$. Let $J$ be a composition of size $n$.
Then, $\left(  \operatorname*{Des}J\right)  _{n}^{\circ}=\rho_{n}\left(
\operatorname*{Epk}J\right)  $.
\end{statement}

[\textit{Proof of Claim 3:} Let $g\in\left(  \operatorname*{Des}J\right)
_{n}^{\circ}$. We shall show that $g\in\rho_{n}\left(  \operatorname*{Epk}%
J\right)  $.

We have $g\in\left(  \operatorname*{Des}J\right)  _{n}^{\circ}\subseteq
\operatorname*{Des}J$ (since $S_{n}^{\circ}\subseteq S$ for each subset $S$ of
$\left[  n-1\right]  $) and therefore $\operatorname*{Des}J\neq\varnothing$.
Hence, $J$ is not the empty composition. In other words, $n>0$.

From (\ref{pf.prop.K.Epk.F.1}), we obtain $\operatorname*{Epk}J=\left(
\operatorname*{Des}J\cup\left\{  n\right\}  \right)  \setminus\left(
\operatorname*{Des}J+1\right)  $. Thus, the set $\operatorname*{Epk}J$ is
disjoint from $\operatorname*{Des}J+1$. Furthermore, the set
$\operatorname*{Epk}J$ is nonempty\footnote{Indeed, it contains at least the
smallest element of the set $\operatorname*{Des}J\cup\left\{  n\right\}  $
(since $\operatorname*{Epk}J=\left(  \operatorname*{Des}J\cup\left\{
n\right\}  \right)  \setminus\left(  \operatorname*{Des}J+1\right)  $).}.

We have $g\in\left(  \operatorname*{Des}J\right)  _{n}^{\circ}$. Thus, $g$ is
an element of $\operatorname*{Des}J$ satisfying $g-1\notin\operatorname*{Des}%
J$ or $\left[  g,n-1\right]  \subseteq\operatorname*{Des}J$ (by the definition
of $\left(  \operatorname*{Des}J\right)  _{n}^{\circ}$). We are thus in one of
the following two cases:

\textit{Case 1:} We have $g-1\notin\operatorname*{Des}J$.

\textit{Case 2:} We have $\left[  g,n-1\right]  \subseteq\operatorname*{Des}J$.

Let us first consider Case 1. In this case, we have $g-1\notin%
\operatorname*{Des}J$. In other words, $g\notin\operatorname*{Des}J+1$.
Combined with $g\in\operatorname*{Des}J\subseteq\operatorname*{Des}%
J\cup\left\{  n\right\}  $, this yields $g\in\left(  \operatorname*{Des}%
J\cup\left\{  n\right\}  \right)  \setminus\left(  \operatorname*{Des}%
J+1\right)  =\operatorname*{Epk}J$. Moreover, $g\neq n$ (since $g\in
\operatorname*{Des}J\subseteq\left[  n-1\right]  $) and thus $g\in\left(
\operatorname*{Epk}J\right)  \setminus\left\{  n\right\}  $ (since
$g\in\operatorname*{Epk}J$). But each nonempty subset $T$ of $\left[
n\right]  $ satisfies $T\setminus\left\{  n\right\}  \subseteq\rho_{n}\left(
T\right)  $ (by the definition of $\rho_{n}\left(  T\right)  $). Applying this
to $T=\operatorname*{Epk}J$, we obtain $\left(  \operatorname*{Epk}J\right)
\setminus\left\{  n\right\}  \subseteq\rho_{n}\left(  \operatorname*{Epk}%
J\right)  $. Hence, $g\in\left(  \operatorname*{Epk}J\right)  \setminus
\left\{  n\right\}  \subseteq\rho_{n}\left(  \operatorname*{Epk}J\right)  $.
Thus, $g\in\rho_{n}\left(  \operatorname*{Epk}J\right)  $ is proven in Case 1.

Let us now consider Case 2. In this case, we have $\left[  g,n-1\right]
\subseteq\operatorname*{Des}J$. Hence, each of the elements $g,g+1,\ldots,n-1$
belongs to $\operatorname*{Des}J$. In other words, each of the elements
$g+1,g+2,\ldots,n$ belongs to $\operatorname*{Des}J+1$. Hence, none of the
elements $g+1,g+2,\ldots,n$ belongs to $\operatorname*{Epk}J$ (since the set
$\operatorname*{Epk}J$ is disjoint from $\operatorname*{Des}J+1$). Thus,
$\max\left(  \operatorname*{Epk}J\right)  \leq g$. Therefore, $g\in\left[
\max\left(  \operatorname*{Epk}J\right)  ,n-1\right]  $ (since $g\in
\operatorname*{Des}J\subseteq\left[  n-1\right]  $).

Also, $n\notin\operatorname*{Epk}J$\ \ \ \ \footnote{\textit{Proof.} Assume
the contrary. Thus, $n\in\operatorname*{Epk}J$. But none of the elements
$g+1,g+2,\ldots,n$ belongs to $\operatorname*{Epk}J$. Hence, $n$ is not among
the elements $g+1,g+2,\ldots,n$. Therefore, $g\geq n$, so that $g=n$. This
contradicts $g\in\operatorname*{Des}J\subseteq\left[  n-1\right]  $. This
contradiction shows that our assumption was wrong, qed.}. Hence, the
definition of $\rho_{n}\left(  \operatorname*{Epk}J\right)  $ yields $\rho
_{n}\left(  \operatorname*{Epk}J\right)  =\operatorname*{Epk}J\cup\left[
\max\left(  \operatorname*{Epk}J\right)  ,n-1\right]  $. Now,%
\[
g\in\left[  \max\left(  \operatorname*{Epk}J\right)  ,n-1\right]
\subseteq\operatorname*{Epk}J\cup\left[  \max\left(  \operatorname*{Epk}%
J\right)  ,n-1\right]  =\rho_{n}\left(  \operatorname*{Epk}J\right)  .
\]
Hence, $g\in\rho_{n}\left(  \operatorname*{Epk}J\right)  $ is proven in Case 2.

Thus, $g\in\rho_{n}\left(  \operatorname*{Epk}J\right)  $ is proven in both
Cases 1 and 2. This shows that $g\in\rho_{n}\left(  \operatorname*{Epk}%
J\right)  $ always holds.

Forget that we fixed $g$. We thus have proven that $g\in\rho_{n}\left(
\operatorname*{Epk}J\right)  $ for each $g\in\left(  \operatorname*{Des}%
J\right)  _{n}^{\circ}$. In other words, $\left(  \operatorname*{Des}J\right)
_{n}^{\circ}\subseteq\rho_{n}\left(  \operatorname*{Epk}J\right)  $.

Now, let $h\in\rho_{n}\left(  \operatorname*{Epk}J\right)  $ be arbitrary. We
shall prove that $h\in\left(  \operatorname*{Des}J\right)  _{n}^{\circ}$.

We are in one of the following two cases:

\textit{Case 1:} We have $n\in\operatorname*{Epk}J$.

\textit{Case 2:} We have $n\notin\operatorname*{Epk}J$.

Let us first consider Case 1. In this case, we have $n\in\operatorname*{Epk}%
J$, and thus $\rho_{n}\left(  \operatorname*{Epk}J\right)
=\operatorname*{Epk}J\setminus\left\{  n\right\}  $ (by the definition of
$\rho_{n}\left(  \operatorname*{Epk}J\right)  $). Hence,
\[
h\in\rho_{n}\left(  \operatorname*{Epk}J\right)  =\operatorname*{Epk}%
J\setminus\left\{  n\right\}  \subseteq\operatorname*{Epk}J=\left(
\operatorname*{Des}J\cup\left\{  n\right\}  \right)  \setminus\left(
\operatorname*{Des}J+1\right)  .
\]
In other words, $h\in\operatorname*{Des}J\cup\left\{  n\right\}  $ and
$h\notin\operatorname*{Des}J+1$. Since $h\in\operatorname*{Des}J\cup\left\{
n\right\}  $ and $h\neq n$ (because $h\in\rho_{n}\left(  \operatorname*{Epk}%
J\right)  \subseteq\left[  n-1\right]  $), we obtain $h\in\left(
\operatorname*{Des}J\cup\left\{  n\right\}  \right)  \setminus\left\{
n\right\}  \subseteq\operatorname*{Des}J$. From $h\notin\operatorname*{Des}%
J+1$, we obtain $h-1\notin\operatorname*{Des}J$. Thus, $h$ is an element of
$\operatorname*{Des}J$ satisfying $h-1\notin\operatorname*{Des}J$ or $\left[
h,n-1\right]  \subseteq\operatorname*{Des}J$ (in fact, $h-1\notin%
\operatorname*{Des}J$ holds). Thus, $h\in\left(  \operatorname*{Des}J\right)
_{n}^{\circ}$ (by the definition of $\left(  \operatorname*{Des}J\right)
_{n}^{\circ}$). Thus, $h\in\left(  \operatorname*{Des}J\right)  _{n}^{\circ}$
is proven in Case 1.

Let us now consider Case 2. In this case, we have $n\notin\operatorname*{Epk}%
J$. Hence, the definition of $\rho_{n}\left(  \operatorname*{Epk}J\right)  $
yields $\rho_{n}\left(  \operatorname*{Epk}J\right)  =\left(
\operatorname*{Epk}J\right)  \cup\left[  \max\left(  \operatorname*{Epk}%
J\right)  ,n-1\right]  $. Thus, $h\in\rho_{n}\left(  \operatorname*{Epk}%
J\right)  =\left(  \operatorname*{Epk}J\right)  \cup\left[  \max\left(
\operatorname*{Epk}J\right)  ,n-1\right]  $.

If $h\in\operatorname*{Epk}J$, then we can prove $h\in\left(
\operatorname*{Des}J\right)  _{n}^{\circ}$ just as in Case 1. Hence, let us
WLOG assume that we don't have $h\in\operatorname*{Epk}J$. Thus,
$h\notin\operatorname*{Epk}J$. Combined with $h\in\left(  \operatorname*{Epk}%
J\right)  \cup\left[  \max\left(  \operatorname*{Epk}J\right)  ,n-1\right]  $,
this yields%
\begin{align*}
h  &  \in\left(  \left(  \operatorname*{Epk}J\right)  \cup\left[  \max\left(
\operatorname*{Epk}J\right)  ,n-1\right]  \right)  \setminus\left(
\operatorname*{Epk}J\right)  =\left[  \max\left(  \operatorname*{Epk}J\right)
,n-1\right]  \setminus\left(  \operatorname*{Epk}J\right) \\
&  \subseteq\left[  \max\left(  \operatorname*{Epk}J\right)  ,n-1\right]
\subseteq\operatorname*{Des}J\ \ \ \ \ \ \ \ \ \ \left(  \text{by
(\ref{pf.prop.K.Epk.F.2})}\right)  .
\end{align*}
Moreover, from $h\in\left[  \max\left(  \operatorname*{Epk}J\right)
,n-1\right]  $, we obtain $h\geq\max\left(  \operatorname*{Epk}J\right)  $, so
that%
\[
\left[  h,n-1\right]  \subseteq\left[  \max\left(  \operatorname*{Epk}%
J\right)  ,n-1\right]  \subseteq\operatorname*{Des}%
J\ \ \ \ \ \ \ \ \ \ \left(  \text{by (\ref{pf.prop.K.Epk.F.2})}\right)  .
\]
Hence, $h$ is an element of $\operatorname*{Des}J$ satisfying $h-1\notin%
\operatorname*{Des}J$ or $\left[  h,n-1\right]  \subseteq\operatorname*{Des}J$
(namely, $\left[  h,n-1\right]  \subseteq\operatorname*{Des}J$). In other
words, $h\in\left(  \operatorname*{Des}J\right)  _{n}^{\circ}$ (by the
definition of $\left(  \operatorname*{Des}J\right)  _{n}^{\circ}$). Thus,
$h\in\left(  \operatorname*{Des}J\right)  _{n}^{\circ}$ is proven in Case 2.

We have now proven $h\in\left(  \operatorname*{Des}J\right)  _{n}^{\circ}$ in
both Cases 1 and 2. Hence, $h\in\left(  \operatorname*{Des}J\right)
_{n}^{\circ}$ always holds.

Forget that we fixed $h$. We thus have shown that $h\in\left(
\operatorname*{Des}J\right)  _{n}^{\circ}$ for each $h\in\rho_{n}\left(
\operatorname*{Epk}J\right)  $. In other words, $\rho_{n}\left(
\operatorname*{Epk}J\right)  \subseteq\left(  \operatorname*{Des}J\right)
_{n}^{\circ}$. Combining this with $\left(  \operatorname*{Des}J\right)
_{n}^{\circ}\subseteq\rho_{n}\left(  \operatorname*{Epk}J\right)  $, we obtain
$\left(  \operatorname*{Des}J\right)  _{n}^{\circ}=\rho_{n}\left(
\operatorname*{Epk}J\right)  $. This proves Claim 3.]

\begin{statement}
\textit{Claim 4:} Let $n\in\mathbb{N}$. Let $J$ and $K$ be two compositions of
size $n$ satisfying $\operatorname*{Epk}J=\operatorname*{Epk}K$. Then,
$\left(  \operatorname*{Des}J\right)  _{n}^{\circ}=\left(  \operatorname*{Des}%
K\right)  _{n}^{\circ}$.
\end{statement}

[\textit{Proof of Claim 4:} Claim 3 yields $\left(  \operatorname*{Des}%
J\right)  _{n}^{\circ}=\rho_{n}\left(  \operatorname*{Epk}J\right)  $ and
similarly $\left(  \operatorname*{Des}K\right)  _{n}^{\circ}=\rho_{n}\left(
\operatorname*{Epk}K\right)  $. Hence,%
\[
\left(  \operatorname*{Des}J\right)  _{n}^{\circ}=\rho_{n}\left(
\underbrace{\operatorname*{Epk}J}_{=\operatorname*{Epk}K}\right)  =\rho
_{n}\left(  \operatorname*{Epk}K\right)  =\left(  \operatorname*{Des}K\right)
_{n}^{\circ}.
\]
This proves Claim 4.]

We let $\overset{\ast}{\rightarrow}$ be the transitive-and-reflexive closure
of the relation $\rightarrow$. Thus, two compositions $J$ and $K$ satisfy
$J\overset{\ast}{\rightarrow}K$ if and only if there exists a sequence
$\left(  L_{0},L_{1},\ldots,L_{\ell}\right)  $ of compositions satisfying
$L_{0}=J$ and $L_{\ell}=K$ and $L_{0}\rightarrow L_{1}\rightarrow
\cdots\rightarrow L_{\ell}$.

\begin{statement}
\textit{Claim 5:} Let $n\in\mathbb{N}$. Let $K$ be a composition of size $n$.
Then, $\operatorname*{Comp}\left(  \left(  \operatorname*{Des}K\right)
_{n}^{\circ}\right)  \overset{\ast}{\rightarrow}K$.
\end{statement}

[\textit{Proof of Claim 5:} We shall prove Claim 5 by strong induction on
$\left\vert \left(  \operatorname*{Des}K\right)  \setminus\left(
\operatorname*{Des}K\right)  _{n}^{\circ}\right\vert $. Thus, we fix an
$n\in\mathbb{N}$ and a composition $K$ of size $n$, and we assume (as the
induction hypothesis) that each composition $J$ of size $n$ satisfying
$\left\vert \left(  \operatorname*{Des}J\right)  \setminus\left(
\operatorname*{Des}J\right)  _{n}^{\circ}\right\vert <\left\vert \left(
\operatorname*{Des}K\right)  \setminus\left(  \operatorname*{Des}K\right)
_{n}^{\circ}\right\vert $ satisfies $\operatorname*{Comp}\left(  \left(
\operatorname*{Des}J\right)  _{n}^{\circ}\right)  \overset{\ast}{\rightarrow
}J$. Our goal is to prove that $\operatorname*{Comp}\left(  \left(
\operatorname*{Des}K\right)  _{n}^{\circ}\right)  \overset{\ast}{\rightarrow
}K$.

Let $A=\operatorname*{Des}K$. Thus, $K=\operatorname*{Comp}A$ and
$A\subseteq\left[  n-1\right]  $.

Applying (\ref{pf.prop.K.Epk.F.1}) to $L=K$, we obtain $\operatorname*{Epk}%
K=\left(  \operatorname*{Des}K\cup\left\{  n\right\}  \right)  \setminus
\left(  \operatorname*{Des}K+1\right)  =\left(  A\cup\left\{  n\right\}
\right)  \setminus\left(  A+1\right)  $ (since $\operatorname*{Des}K=A$).

Also, $A_{n}^{\circ}\subseteq A$ (since $S_{n}^{\circ}\subseteq S$ for any
subset $S$ of $\left[  n-1\right]  $). If $A_{n}^{\circ}=A$, then we are done
(because if $A_{n}^{\circ}=A$, then $\operatorname*{Comp}\left(  \left(
\underbrace{\operatorname*{Des}K}_{=A}\right)  _{n}^{\circ}\right)
=\operatorname*{Comp}\left(  \underbrace{A_{n}^{\circ}}_{=A}\right)
=\operatorname*{Comp}A=K$, and therefore the reflexivity of $\overset{\ast
}{\rightarrow}$ shows that $\operatorname*{Comp}\left(  \left(
\operatorname*{Des}K\right)  _{n}^{\circ}\right)  \overset{\ast}{\rightarrow
}K$). Hence, we WLOG assume that $A_{n}^{\circ}\neq A$. Thus, $A_{n}^{\circ}$
is a proper subset of $A$ (since $A_{n}^{\circ}\subseteq A$). Therefore, there
exists a $q\in A$ satisfying $q\notin A_{n}^{\circ}$. Let $k$ be the
\textbf{largest} such $q$. Thus, $k\in A$ and $k\notin A_{n}^{\circ}$. Hence,
$k\in A\setminus A_{n}^{\circ}$. Also, $k\in A\subseteq\left[  n-1\right]  $.

Let $J=\operatorname*{Comp}\left(  A\setminus\left\{  k\right\}  \right)  $.
Thus, $\operatorname*{Des}J=A\setminus\left\{  k\right\}  $, so that
$A=\operatorname*{Des}J\cup\left\{  k\right\}  $ (since $k\in A$). Hence,
$\operatorname*{Des}K=A=\operatorname*{Des}J\cup\left\{  k\right\}  $. Also,
$k\notin A\setminus\left\{  k\right\}  =\operatorname*{Des}J$.

Furthermore, $k-1\in\operatorname*{Des}J$\ \ \ \ \footnote{\textit{Proof.}
Assume the contrary. Thus, $k-1\notin\operatorname*{Des}J$. Hence,
$k-1\notin\operatorname*{Des}J\cup\left\{  k\right\}  $ as well (since
$k-1\neq k$). In other words, $k-1\notin A$ (since $\operatorname*{Des}%
J\cup\left\{  k\right\}  =A$). Therefore, the element $k$ of $A$ satisfies
$k-1\notin A$ or $\left[  k,n-1\right]  \subseteq A$. Thus, the definition of
$A_{n}^{\circ}$ yields $k\in A_{n}^{\circ}$. This contradicts $k\notin
A_{n}^{\circ}$. This contradiction shows that our assumption was false; qed.}
and $k+1\notin\operatorname*{Des}J\cup\left\{  n\right\}  $%
\ \ \ \ \footnote{\textit{Proof.} Assume the contrary. Thus, $k+1\in
\operatorname*{Des}J\cup\left\{  n\right\}  $. In other words, we have
$k+1\in\operatorname*{Des}J$ or $k+1=n$. In other words, we are in one of the
following two cases:
\par
\textit{Case 1:} We have $k+1\in\operatorname*{Des}J$.
\par
\textit{Case 2:} We have $k+1=n$.
\par
Let us first consider Case 1. In this case, we have $k+1\in\operatorname*{Des}%
J$. Hence, $k+1\in\operatorname*{Des}J\subseteq\operatorname*{Des}%
J\cup\left\{  k\right\}  =A$. If we had $k+1\notin A_{n}^{\circ}$, then $k+1$
would be a $q\in A$ satisfying $q\notin A_{n}^{\circ}$; this would contradict
the fact that $k$ is the \textbf{largest} such $q$ (since $k+1$ is larger than
$k$). Hence, we cannot have $k+1\notin A_{n}^{\circ}$. Thus, we must have
$k+1\in A_{n}^{\circ}$. In other words, $k+1$ is an element of $A$ satisfying
$\left(  k+1\right)  -1\notin A$ or $\left[  k+1,n-1\right]  \subseteq A$ (by
the definition of $A_{n}^{\circ}$). Since $\left(  k+1\right)  -1\notin A$ is
impossible (because $\left(  k+1\right)  -1=k\in A$), we thus have $\left[
k+1,n-1\right]  \subseteq A$. Now, $\left[  k,n-1\right]
=\underbrace{\left\{  k\right\}  }_{\substack{\subseteq A\\\text{(since }k\in
A\text{)}}}\cup\underbrace{\left[  k+1,n-1\right]  }_{\subseteq A}\subseteq
A\cup A=A$. Thus, the element $k$ of $A$ satisfies $k-1\notin A$ or $\left[
k,n-1\right]  \subseteq A$. In other words, $k\in A_{n}^{\circ}$ (by the
definition of $A_{n}^{\circ}$). This contradicts $k\notin A_{n}^{\circ}$.
Thus, we have found a contradiction in Case 1.
\par
Let us now consider Case 2. In this case, we have $k+1=n$. Hence, $k=n-1$, so
that $\left[  k,n-1\right]  =\left\{  k\right\}  \subseteq A$ (since $k\in
A$). Thus, the element $k$ of $A$ satisfies $k-1\notin A$ or $\left[
k,n-1\right]  \subseteq A$. In other words, $k\in A_{n}^{\circ}$ (by the
definition of $A_{n}^{\circ}$). This contradicts $k\notin A_{n}^{\circ}$.
Thus, we have found a contradiction in Case 2.
\par
We have therefore found a contradiction in each of the two Cases 1 and 2.
Thus, we always get a contradiction, so our assumption must have been wrong.
Qed.}. Hence, we have found a $k\in\left[  n-1\right]  $ satisfying%
\begin{align*}
\operatorname*{Des}K  &  =\operatorname*{Des}J\cup\left\{  k\right\}
,\ \ \ \ \ \ \ \ \ \ k\notin\operatorname*{Des}J,\\
k-1  &  \in\operatorname*{Des}J\ \ \ \ \ \ \ \ \ \ \text{and}%
\ \ \ \ \ \ \ \ \ \ k+1\notin\operatorname*{Des}J\cup\left\{  n\right\}  .
\end{align*}
Therefore, Claim 1 yields $J\rightarrow K$. Thus, Claim 2 yields
$\operatorname*{Epk}J=\operatorname*{Epk}K$. Claim 4 therefore yields $\left(
\operatorname*{Des}J\right)  _{n}^{\circ}=\left(  \operatorname*{Des}K\right)
_{n}^{\circ}=A_{n}^{\circ}$ (since $\operatorname*{Des}K=A$). Thus,%
\begin{align*}
\left\vert \underbrace{\left(  \operatorname*{Des}J\right)  }_{=A\setminus
\left\{  k\right\}  }\setminus\underbrace{\left(  \operatorname*{Des}J\right)
_{n}^{\circ}}_{=A_{n}^{\circ}}\right\vert  &  =\left\vert \underbrace{\left(
A\setminus\left\{  k\right\}  \right)  \setminus A_{n}^{\circ}}_{=\left(
A\setminus A_{n}^{\circ}\right)  \setminus\left\{  k\right\}  }\right\vert
=\left\vert \left(  A\setminus A_{n}^{\circ}\right)  \setminus\left\{
k\right\}  \right\vert \\
&  =\left\vert A\setminus A_{n}^{\circ}\right\vert
-1\ \ \ \ \ \ \ \ \ \ \left(  \text{since }k\in A\setminus A_{n}^{\circ
}\right) \\
&  <\left\vert A\setminus A_{n}^{\circ}\right\vert =\left\vert \left(
\operatorname*{Des}K\right)  \setminus\left(  \operatorname*{Des}K\right)
_{n}^{\circ}\right\vert
\end{align*}
(since $A=\operatorname*{Des}K$). Thus, the induction hypothesis shows that
$\operatorname*{Comp}\left(  \left(  \operatorname*{Des}J\right)  _{n}^{\circ
}\right)  \overset{\ast}{\rightarrow}J$. Combining this with $J\rightarrow K$,
we obtain $\operatorname*{Comp}\left(  \left(  \operatorname*{Des}J\right)
_{n}^{\circ}\right)  \overset{\ast}{\rightarrow}K$ (since $\overset{\ast
}{\rightarrow}$ is the transitive-and-reflexive closure of the relation
$\rightarrow$). In light of $\left(  \operatorname*{Des}J\right)  _{n}^{\circ
}=\left(  \operatorname*{Des}K\right)  _{n}^{\circ}$, this rewrites as
$\operatorname*{Comp}\left(  \left(  \operatorname*{Des}K\right)  _{n}^{\circ
}\right)  \overset{\ast}{\rightarrow}K$. Thus, Claim 5 is proven by induction.]

Now, let $\mathcal{K}^{\prime}$ be the $\mathbb{Q}$-vector subspace of
$\operatorname*{QSym}$ spanned by all differences of the form $F_{J}-F_{K}$,
where $J$ and $K$ are two compositions satisfying $J\rightarrow K$.

\begin{statement}
\textit{Claim 6:} Let $J$ and $K$ be two compositions such that
$J\overset{\ast}{\rightarrow}K$. Then, $F_{J}-F_{K}\in\mathcal{K}^{\prime}$.
\end{statement}

[\textit{Proof of Claim 6:} We have $J\overset{\ast}{\rightarrow}K$. By the
definition of the relation $\overset{\ast}{\rightarrow}$, this means that
there exists a sequence $\left(  L_{0},L_{1},\ldots,L_{\ell}\right)  $ of
compositions satisfying $L_{0}=J$ and $L_{\ell}=K$ and $L_{0}\rightarrow
L_{1}\rightarrow\cdots\rightarrow L_{\ell}$. Consider this sequence. For each
$i\in\left\{  0,1,\ldots,\ell-1\right\}  $, we have $L_{i}\rightarrow L_{i+1}$
and thus $F_{L_{i}}-F_{L_{i+1}}\in\mathcal{K}^{\prime}$ (by the definition of
$\mathcal{K}^{\prime}$). Therefore, $\sum_{i=0}^{\ell-1}\left(  F_{L_{i}%
}-F_{L_{i+1}}\right)  \in\mathcal{K}^{\prime}$. In light of%
\begin{align*}
\sum_{i=0}^{\ell-1}\left(  F_{L_{i}}-F_{L_{i+1}}\right)   &  =F_{L_{0}%
}-F_{L_{\ell}}\ \ \ \ \ \ \ \ \ \ \left(  \text{by the telescope
principle}\right) \\
&  =F_{J}-F_{K}\ \ \ \ \ \ \ \ \ \ \left(  \text{since }L_{0}=J\text{ and
}L_{\ell}=K\right)  ,
\end{align*}
this rewrites as $F_{J}-F_{K}\in\mathcal{K}^{\prime}$. This proves Claim 6.]

\begin{statement}
\textit{Claim 7:} We have $\mathcal{K}_{\operatorname*{Epk}}\subseteq
\mathcal{K}^{\prime}$.
\end{statement}

[\textit{Proof of Claim 7:} Recall that $\mathcal{K}_{\operatorname*{Epk}}$ is
the $\mathbb{Q}$-vector subspace of $\operatorname*{QSym}$ spanned by all
elements of the form $F_{J}-F_{K}$, where $J$ and $K$ are two
$\operatorname*{Epk}$-equivalent compositions. Thus, it suffices to show that
if $J$ and $K$ are two $\operatorname*{Epk}$-equivalent compositions, then
$F_{J}-F_{K}\in\mathcal{K}^{\prime}$.

So let $J$ and $K$ be two $\operatorname*{Epk}$-equivalent compositions. We
must prove that $F_{J}-F_{K}\in\mathcal{K}^{\prime}$.

The compositions $J$ and $K$ are $\operatorname*{Epk}$-equivalent; in other
words, they have the same size and satisfy $\operatorname*{Epk}%
J=\operatorname*{Epk}K$. Let $n=\left\vert J\right\vert =\left\vert
K\right\vert $. (This is well-defined, since the compositions $J$ and $K$ have
the same size.)

Claim 4 yields $\left(  \operatorname*{Des}J\right)  _{n}^{\circ}=\left(
\operatorname*{Des}K\right)  _{n}^{\circ}$. But Claim 5 yields
$\operatorname*{Comp}\left(  \left(  \operatorname*{Des}K\right)  _{n}^{\circ
}\right)  \overset{\ast}{\rightarrow}K$. Hence, Claim 6 (applied to
$\operatorname*{Comp}\left(  \left(  \operatorname*{Des}K\right)  _{n}^{\circ
}\right)  $ instead of $J$) shows that $F_{\operatorname*{Comp}\left(  \left(
\operatorname*{Des}K\right)  _{n}^{\circ}\right)  }-F_{K}\in\mathcal{K}%
^{\prime}$. The same argument (applied to $J$ instead of $K$) shows that
$F_{\operatorname*{Comp}\left(  \left(  \operatorname*{Des}J\right)
_{n}^{\circ}\right)  }-F_{J}\in\mathcal{K}^{\prime}$. Now,%
\begin{align*}
&  \left(  F_{\operatorname*{Comp}\left(  \left(  \operatorname*{Des}K\right)
_{n}^{\circ}\right)  }-F_{K}\right)  -\left(  F_{\operatorname*{Comp}\left(
\left(  \operatorname*{Des}J\right)  _{n}^{\circ}\right)  }-F_{J}\right) \\
&  =\left(  F_{\operatorname*{Comp}\left(  \left(  \operatorname*{Des}%
K\right)  _{n}^{\circ}\right)  }-\underbrace{F_{\operatorname*{Comp}\left(
\left(  \operatorname*{Des}J\right)  _{n}^{\circ}\right)  }}%
_{\substack{=F_{\operatorname*{Comp}\left(  \left(  \operatorname*{Des}%
K\right)  _{n}^{\circ}\right)  }\\\text{(since }\left(  \operatorname*{Des}%
J\right)  _{n}^{\circ}=\left(  \operatorname*{Des}K\right)  _{n}^{\circ
}\text{)}}}\right)  +F_{J}-F_{K}\\
&  =\underbrace{\left(  F_{\operatorname*{Comp}\left(  \left(
\operatorname*{Des}K\right)  _{n}^{\circ}\right)  }-F_{\operatorname*{Comp}%
\left(  \left(  \operatorname*{Des}K\right)  _{n}^{\circ}\right)  }\right)
}_{=0}+F_{J}-F_{K}=F_{J}-F_{K},
\end{align*}
so that%
\[
F_{J}-F_{K}=\underbrace{\left(  F_{\operatorname*{Comp}\left(  \left(
\operatorname*{Des}K\right)  _{n}^{\circ}\right)  }-F_{K}\right)  }%
_{\in\mathcal{K}^{\prime}}-\underbrace{\left(  F_{\operatorname*{Comp}\left(
\left(  \operatorname*{Des}J\right)  _{n}^{\circ}\right)  }-F_{J}\right)
}_{\in\mathcal{K}^{\prime}}\in\mathcal{K}^{\prime}-\mathcal{K}^{\prime
}\subseteq\mathcal{K}^{\prime}.
\]
This proves Claim 7.]

\begin{statement}
\textit{Claim 8:} We have $\mathcal{K}^{\prime}\subseteq\mathcal{K}%
_{\operatorname*{Epk}}$.
\end{statement}

[\textit{Proof of Claim 8:} Recall that $\mathcal{K}^{\prime}$ is the
$\mathbb{Q}$-vector subspace of $\operatorname*{QSym}$ spanned by all
differences of the form $F_{J}-F_{K}$, where $J$ and $K$ are two compositions
satisfying $J\rightarrow K$. Thus, it suffices to show that if $J$ and $K$ are
two compositions satisfying $J\rightarrow K$, then $F_{J}-F_{K}\in
\mathcal{K}_{\operatorname*{Epk}}$.

So let $J$ and $K$ be two compositions satisfying $J\rightarrow K$. We must
prove that $F_{J}-F_{K}\in\mathcal{K}_{\operatorname*{Epk}}$.

We have $J\rightarrow K$ and thus $J\overset{\ast}{\rightarrow}K$ (since
$\overset{\ast}{\rightarrow}$ is the transitive-and-reflexive closure of the
relation $\rightarrow$). Hence, Claim 6 shows that $F_{J}-F_{K}\in
\mathcal{K}^{\prime}$. This proves Claim 8.]

Combining Claim 7 and Claim 8, we obtain $\mathcal{K}_{\operatorname*{Epk}%
}=\mathcal{K}^{\prime}$. Recalling the definition of $\mathcal{K}^{\prime}$,
we can rewrite this as follows: $\mathcal{K}_{\operatorname*{Epk}}$ is the
$\mathbb{Q}$-vector subspace of $\operatorname*{QSym}$ spanned by all
differences of the form $F_{J}-F_{K}$, where $J$ and $K$ are two compositions
satisfying $J\rightarrow K$. This proves Proposition \ref{prop.K.Epk.F}.
\end{proof}
\end{verlong}

\subsection{The M-generating set of $\mathcal{K}_{\operatorname*{Epk}}$}

Another characterization of the ideal $\mathcal{K}_{\operatorname*{Epk}}$ of
$\operatorname*{QSym}$ can be obtained using the monomial basis of
$\operatorname*{QSym}$. Let us first recall how said basis is defined:

For any composition $\alpha=\left(  \alpha_{1},\alpha_{2},\ldots,\alpha_{\ell
}\right)  $, we let%
\[
M_{\alpha}=\sum_{i_{1}<i_{2}<\cdots<i_{\ell}}x_{i_{1}}^{\alpha_{1}}x_{i_{2}%
}^{\alpha_{2}}\cdots x_{i_{\ell}}^{\alpha_{\ell}}%
\]
(where the sum is over all strictly increasing $\ell$-tuples $\left(
i_{1},i_{2},\ldots,i_{\ell}\right)  $ of positive integers). This power series
$M_{\alpha}$ belongs to $\operatorname*{QSym}$. The family $\left(  M_{\alpha
}\right)  _{\alpha\text{ is a composition}}$ is a basis of the $\mathbb{Q}%
$-vector space $\operatorname*{QSym}$; it is called the \textit{monomial
basis} of $\operatorname*{QSym}$.

\begin{proposition}
\label{prop.K.Epk.M}If $J=\left(  j_{1},j_{2},\ldots,j_{m}\right)  $ and $K$
are two compositions, then we shall write $J\underset{M}{\rightarrow}K$ if
there exists an $\ell\in\left\{  2,3,\ldots,m\right\}  $ such that $j_{\ell
}>2$ and $K=\left(  j_{1},j_{2},\ldots,j_{\ell-1},2,j_{\ell}-2,j_{\ell
+1},j_{\ell+2},\ldots,j_{m}\right)  $. (In other words, we write
$J\underset{M}{\rightarrow}K$ if $K$ can be obtained from $J$ by
\textquotedblleft splitting\textquotedblright\ some entry $j_{\ell}>2$ into
two consecutive entries $2$ and $j_{\ell}-2$, provided that this entry was not
the first entry -- i.e., we had $\ell>1$ -- and that this entry was greater
than $2$.)

The ideal $\mathcal{K}_{\operatorname*{Epk}}$ of $\operatorname*{QSym}$ is
spanned (as a $\mathbb{Q}$-vector space) by all sums of the form $M_{J}+M_{K}%
$, where $J$ and $K$ are two compositions satisfying
$J\underset{M}{\rightarrow}K$.
\end{proposition}

\begin{example}
We have $\left(  2,1,4,4\right)  \underset{M}{\rightarrow}\left(
2,1,2,2,4\right)  $, since the composition $\left(  2,1,2,2,4\right)  $ is
obtained from $\left(  2,1,4,4\right)  $ by splitting the third entry (which
is $4>2$) into two consecutive entries $2$ and $2$.

Similarly, $\left(  2,1,4,4\right)  \underset{M}{\rightarrow}\left(
2,1,4,2,2\right)  $ and $\left(  2,1,5,4\right)  \underset{M}{\rightarrow
}\left(  2,1,2,3,4\right)  $.

But we do not have $\left(  3,1\right)  \underset{M}{\rightarrow}\left(
2,1,1\right)  $, because splitting the first entry of the composition is not
allowed in the definition of the relation $\underset{M}{\rightarrow}$.

Two compositions $J$ and $K$ satisfying $J\underset{M}{\rightarrow}K$ must
necessarily satisfy $\left\vert J\right\vert =\left\vert K\right\vert $.

Here are all relations $\underset{M}{\rightarrow}$ between compositions of
size $4$:%
\[
\left(  1,3\right)  \underset{M}{\rightarrow}\left(  1,2,1\right)  .
\]


Here are all relations $\underset{M}{\rightarrow}$ between compositions of
size $5$:%
\begin{align*}
\left(  1,4\right)   &  \underset{M}{\rightarrow}\left(  1,2,2\right)  ,\\
\left(  1,3,1\right)   &  \underset{M}{\rightarrow}\left(  1,2,1,1\right)  ,\\
\left(  1,1,3\right)   &  \underset{M}{\rightarrow}\left(  1,1,2,1\right)  ,\\
\left(  2,3\right)   &  \underset{M}{\rightarrow}\left(  2,2,1\right)  .
\end{align*}
There are no relations $\underset{M}{\rightarrow}$ between compositions of
size $\leq3$.
\end{example}

\begin{verlong}
Before we start proving Proposition \ref{prop.K.Epk.M}, let us recall a basic
formula (\cite[(5.2.2)]{HopfComb}) that connects the monomial quasisymmetric
functions with the fundamental quasisymmetric functions:

\begin{proposition}
\label{prop.M-through-F.1}Let $n\in\mathbb{N}$. Let $\alpha$ be any
composition of $n$. Then,%
\[
M_{\alpha}=\sum_{\substack{\beta\text{ is a composition}\\\text{of }n\text{
that refines }\alpha}}\left(  -1\right)  ^{\ell\left(  \beta\right)
-\ell\left(  \alpha\right)  }F_{\beta}.
\]
Here, if $\gamma$ is any composition, then $\ell\left(  \gamma\right)  $
denotes the \textit{length} of $\gamma$ (that is, the number of entries of
$\gamma$).
\end{proposition}

\begin{proof}
[Proof of Proposition \ref{prop.M-through-F.1}.]This precisely \cite[(5.2.2)]%
{HopfComb}.
%(If the numbering shifts: This is the formula in Proposition 5.2.8.)

\end{proof}

\begin{proposition}
\label{prop.M-through-F.2}Let $n$ be a positive integer. Let $C$ be a subset
of $\left[  n-1\right]  $.

\textbf{(a)} Then,%
\[
M_{\operatorname*{Comp}C}=\sum_{B\supseteq C}\left(  -1\right)  ^{\left\vert
B\setminus C\right\vert }F_{\operatorname*{Comp}B}.
\]
(The bound variable $B$ in this sum and any similar sums is supposed to be a
subset of $\left[  n-1\right]  $; thus, the above sum ranges over all subsets
$B$ of $\left[  n-1\right]  $ satisfying $B\supseteq C$.)

\textbf{(b)} Let $k\in\left[  n-1\right]  $ be such that $k\notin C$. Then,%
\[
M_{\operatorname*{Comp}C}+M_{\operatorname*{Comp}\left(  C\cup\left\{
k\right\}  \right)  }=\sum_{\substack{B\supseteq C;\\k\notin B}}\left(
-1\right)  ^{\left\vert B\setminus C\right\vert }F_{\operatorname*{Comp}B}.
\]


\textbf{(c)} Let $k\in\left[  n-1\right]  $ be such that $k\notin C$ and
$k-1\notin C\cup\left\{  0\right\}  $. Then,%
\[
M_{\operatorname*{Comp}C}+M_{\operatorname*{Comp}\left(  C\cup\left\{
k\right\}  \right)  }=\sum_{\substack{B\supseteq C;\\k\notin B;\\k-1\notin
B}}\left(  -1\right)  ^{\left\vert B\setminus C\right\vert }\left(
F_{\operatorname*{Comp}B}-F_{\operatorname*{Comp}\left(  B\cup\left\{
k-1\right\}  \right)  }\right)  .
\]

\end{proposition}

\begin{proof}
[Proof of Proposition \ref{prop.M-through-F.2}.]\textbf{(a)} Proposition
\ref{prop.M-through-F.2} \textbf{(a)} is the result of applying Proposition
\ref{prop.M-through-F.1} to $\alpha=\operatorname*{Comp}C$ and using the
standard dictionary between compositions of $n$ and subsets of $\left[
n-1\right]  $.

\textbf{(b)} Proposition \ref{prop.M-through-F.2} \textbf{(a)} (applied to
$C\cup\left\{  k\right\}  $ instead of $C$) yields%
\begin{align}
M_{\operatorname*{Comp}\left(  C\cup\left\{  k\right\}  \right)  }  &
=\underbrace{\sum_{B\supseteq C\cup\left\{  k\right\}  }}_{=\sum
_{\substack{B\supseteq C;\\k\in B}}}\underbrace{\left(  -1\right)
^{\left\vert B\setminus\left(  C\cup\left\{  k\right\}  \right)  \right\vert
}}_{\substack{=\left(  -1\right)  ^{\left\vert B\setminus C\right\vert
-1}\\\text{(since }\left\vert B\setminus\left(  C\cup\left\{  k\right\}
\right)  \right\vert =\left\vert B\setminus C\right\vert -1\\\text{(since
}k\notin C\text{))}}}F_{\operatorname*{Comp}B}=\sum_{\substack{B\supseteq
C;\\k\in B}}\underbrace{\left(  -1\right)  ^{\left\vert B\setminus
C\right\vert -1}}_{=-\left(  -1\right)  ^{\left\vert B\setminus C\right\vert
}}F_{\operatorname*{Comp}B}\nonumber\\
&  =-\sum_{\substack{B\supseteq C;\\k\in B}}\left(  -1\right)  ^{\left\vert
B\setminus C\right\vert }F_{\operatorname*{Comp}B}.
\label{pf.prop.M-through-F.2.b.1}%
\end{align}
But Proposition \ref{prop.M-through-F.2} \textbf{(a)} yields
\begin{align*}
M_{\operatorname*{Comp}C}  &  =\sum_{B\supseteq C}\left(  -1\right)
^{\left\vert B\setminus C\right\vert }F_{\operatorname*{Comp}B}\\
&  =\sum_{\substack{B\supseteq C;\\k\in B}}\left(  -1\right)  ^{\left\vert
B\setminus C\right\vert }F_{\operatorname*{Comp}B}+\sum_{\substack{B\supseteq
C;\\k\notin B}}\left(  -1\right)  ^{\left\vert B\setminus C\right\vert
}F_{\operatorname*{Comp}B}.
\end{align*}
Adding (\ref{pf.prop.M-through-F.2.b.1}) to this equality, we obtain%
\begin{align*}
&  M_{\operatorname*{Comp}C}+M_{\operatorname*{Comp}\left(  C\cup\left\{
k\right\}  \right)  }\\
&  =\sum_{\substack{B\supseteq C;\\k\in B}}\left(  -1\right)  ^{\left\vert
B\setminus C\right\vert }F_{\operatorname*{Comp}B}+\sum_{\substack{B\supseteq
C;\\k\notin B}}\left(  -1\right)  ^{\left\vert B\setminus C\right\vert
}F_{\operatorname*{Comp}B}+\left(  -\sum_{\substack{B\supseteq C;\\k\in
B}}\left(  -1\right)  ^{\left\vert B\setminus C\right\vert }%
F_{\operatorname*{Comp}B}\right) \\
&  =\sum_{\substack{B\supseteq C;\\k\notin B}}\left(  -1\right)  ^{\left\vert
B\setminus C\right\vert }F_{\operatorname*{Comp}B}.
\end{align*}
This proves Proposition \ref{prop.M-through-F.2} \textbf{(b)}.

\textbf{(c)} We have $k-1\notin C\cup\left\{  0\right\}  $. Thus, $k-1\notin
C$ and $k-1\neq0$. From $k-1\neq0$, we obtain $k-1\in\left[  n-1\right]  $.

The map%
\begin{align}
&  \left\{  B\subseteq\left[  n-1\right]  \ \mid\ B\supseteq C\text{ and
}k\notin B\text{ and }k-1\notin B\right\} \nonumber\\
&  \rightarrow\left\{  B\subseteq\left[  n-1\right]  \ \mid\ B\supseteq
C\text{ and }k\notin B\text{ and }k-1\in B\right\} \nonumber
\end{align}
sending each $B$ to $B\cup\left\{  k-1\right\}  $ is a bijection (this is easy
to check using the facts that $k-1\notin C$ and $k-1\in\left[  n-1\right]  $).
We shall denote this map by $\Phi$.

Proposition \ref{prop.M-through-F.2} \textbf{(b)} yields
\begin{align*}
&  M_{\operatorname*{Comp}C}+M_{\operatorname*{Comp}\left(  C\cup\left\{
k\right\}  \right)  }\\
&  =\sum_{\substack{B\supseteq C;\\k\notin B}}\left(  -1\right)  ^{\left\vert
B\setminus C\right\vert }F_{\operatorname*{Comp}B}\\
&  =\sum_{\substack{B\supseteq C;\\k\notin B;\\k-1\notin B}}\left(  -1\right)
^{\left\vert B\setminus C\right\vert }F_{\operatorname*{Comp}B}%
+\underbrace{\sum_{\substack{B\supseteq C;\\k\notin B;\\k-1\in B}}\left(
-1\right)  ^{\left\vert B\setminus C\right\vert }F_{\operatorname*{Comp}B}%
}_{\substack{=\sum_{\substack{B\supseteq C;\\k\notin B;\\k-1\notin B}}\left(
-1\right)  ^{\left\vert \left(  B\cup\left\{  k-1\right\}  \right)  \setminus
C\right\vert }F_{\operatorname*{Comp}\left(  B\cup\left\{  k-1\right\}
\right)  }\\\text{(here, we have substituted }B\cup\left\{  k-1\right\}
\text{ for }B\text{ in the sum}\\\text{(since the map }\Phi\text{ is a
bijection))}}}\\
&  =\sum_{\substack{B\supseteq C;\\k\notin B;\\k-1\notin B}}\left(  -1\right)
^{\left\vert B\setminus C\right\vert }F_{\operatorname*{Comp}B}+\sum
_{\substack{B\supseteq C;\\k\notin B;\\k-1\notin B}}\underbrace{\left(
-1\right)  ^{\left\vert \left(  B\cup\left\{  k-1\right\}  \right)  \setminus
C\right\vert }}_{\substack{=\left(  -1\right)  ^{\left\vert B\setminus
C\right\vert +1}\\\text{(since }\left\vert \left(  B\cup\left\{  k-1\right\}
\right)  \setminus C\right\vert =\left\vert B\setminus C\right\vert
+1\\\text{(since }k-1\notin B\text{ and }k-1\notin C\text{))}}%
}F_{\operatorname*{Comp}\left(  B\cup\left\{  k-1\right\}  \right)  }\\
&  =\sum_{\substack{B\supseteq C;\\k\notin B;\\k-1\notin B}}\left(  -1\right)
^{\left\vert B\setminus C\right\vert }F_{\operatorname*{Comp}B}+\sum
_{\substack{B\supseteq C;\\k\notin B;\\k-1\notin B}}\underbrace{\left(
-1\right)  ^{\left\vert B\setminus C\right\vert +1}}_{=-\left(  -1\right)
^{\left\vert B\setminus C\right\vert }}F_{\operatorname*{Comp}\left(
B\cup\left\{  k-1\right\}  \right)  }\\
&  =\sum_{\substack{B\supseteq C;\\k\notin B;\\k-1\notin B}}\left(  -1\right)
^{\left\vert B\setminus C\right\vert }F_{\operatorname*{Comp}B}-\sum
_{\substack{B\supseteq C;\\k\notin B;\\k-1\notin B}}\left(  -1\right)
^{\left\vert B\setminus C\right\vert }F_{\operatorname*{Comp}\left(
B\cup\left\{  k-1\right\}  \right)  }\\
&  =\sum_{\substack{B\supseteq C;\\k\notin B;\\k-1\notin B}}\left(  -1\right)
^{\left\vert B\setminus C\right\vert }\left(  F_{\operatorname*{Comp}%
B}-F_{\operatorname*{Comp}\left(  B\cup\left\{  k-1\right\}  \right)
}\right)  .
\end{align*}
This proves Proposition \ref{prop.M-through-F.2} \textbf{(c)}.
\end{proof}
\end{verlong}

\begin{verlong}
\begin{proof}
[Proof of Proposition \ref{prop.K.Epk.M}.]We shall use the notation
$\left\langle f_{i}\ \mid\ i\in I\right\rangle $ for the $\mathbb{Q}$-linear
span of a family $\left(  f_{i}\right)  _{i\in I}$ of elements of a
$\mathbb{Q}$-vector space.

Define a $\mathbb{Q}$-vector subspace $\mathcal{M}$ of $\operatorname*{QSym}$
by%
\[
\mathcal{M}=\left\langle M_{J}+M_{K}\ \mid\ J\text{ and }K\text{ are
compositions satisfying }J\underset{M}{\rightarrow}K\right\rangle .
\]
Then, our goal is to prove that $\mathcal{K}_{\operatorname*{Epk}}%
=\mathcal{M}$.

We have%
\begin{align*}
\mathcal{M}  &  =\left\langle M_{J}+M_{K}\ \mid\ J\text{ and }K\text{ are
compositions satisfying }J\underset{M}{\rightarrow}K\right\rangle \\
&  =\sum_{n\in\mathbb{N}}\left\langle M_{J}+M_{K}\ \mid\ J\text{ and }K\text{
are compositions of }n\text{ satisfying }J\underset{M}{\rightarrow
}K\right\rangle
\end{align*}
(because if $J$ and $K$ are two compositions satisfying
$J\underset{M}{\rightarrow}K$, then $J$ and $K$ have the same size).

Consider the binary relation $\rightarrow$ defined in Proposition
\ref{prop.K.Epk.F}. Then, Proposition \ref{prop.K.Epk.F} yields%
\begin{align*}
\mathcal{K}_{\operatorname*{Epk}}  &  =\left\langle F_{J}-F_{K}\ \mid\ J\text{
and }K\text{ are compositions satisfying }J\rightarrow K\right\rangle \\
&  =\sum_{n\in\mathbb{N}}\left\langle F_{J}-F_{K}\ \mid\ J\text{ and }K\text{
are compositions of }n\text{ satisfying }J\rightarrow K\right\rangle
\end{align*}
(because if $J$ and $K$ are two compositions satisfying $J\rightarrow K$, then
$J$ and $K$ have the same size).

Now, fix $n\in\mathbb{N}$. Let $\Omega$ be the set of all pairs $\left(
C,k\right)  $ in which $C$ is a subset of $\left[  n-1\right]  $ and $k$ is an
element of $\left[  n-1\right]  $ satisfying $k\notin C$, $k-1\in C$ and
$k+1\notin C\cup\left\{  n\right\}  $.

For every $\left(  C,k\right)  \in\Omega$, we define two elements
$\mathbf{m}_{C,k}$ and $\mathbf{f}_{C,k}$ of $\operatorname*{QSym}$ by%
\begin{align}
\mathbf{m}_{C,k}  &  =M_{\operatorname*{Comp}C}+M_{\operatorname*{Comp}\left(
C\cup\left\{  k+1\right\}  \right)  }\ \ \ \ \ \ \ \ \ \ \text{and}%
\label{pf.prop.K.Epk.M.mCk=}\\
\mathbf{f}_{C,k}  &  =F_{\operatorname*{Comp}C}-F_{\operatorname*{Comp}\left(
C\cup\left\{  k\right\}  \right)  }. \label{pf.prop.K.Epk.M.fCk=}%
\end{align}
\footnote{These two elements are well-defined, because both $C\cup\left\{
k\right\}  $ and $C\cup\left\{  k+1\right\}  $ are subsets of $\left[
n-1\right]  $ (since $k+1\notin C\cup\left\{  n\right\}  $ shows that $k+1\neq
n$).}

We have the following:

\begin{statement}
\textit{Claim 1:} We have%
\begin{align*}
&  \left\langle M_{J}+M_{K}\ \mid\ J\text{ and }K\text{ are compositions of
}n\text{ satisfying }J\underset{M}{\rightarrow}K\right\rangle \\
&  =\left\langle \mathbf{m}_{C,k}\ \mid\ \left(  C,k\right)  \in
\Omega\right\rangle .
\end{align*}

\end{statement}

[\textit{Proof of Claim 1:} It is easy to see that two subsets $C$ and $D$ of
$\left[  n-1\right]  $ satisfy $\operatorname*{Comp}C\underset{M}{\rightarrow
}\operatorname*{Comp}D$ if and only if there exists some $k\in\left[
n-1\right]  $ satisfying $D=C\cup\left\{  k+1\right\}  $, $k\notin C$, $k-1\in
C$ and $k+1\notin C\cup\left\{  n\right\}  $.\ \ \ \ \footnote{To prove this,
recall that \textquotedblleft splitting\textquotedblright\ an entry of a
composition $J$ into two consecutive entries (summing up to the original
entry) is always tantamount to adding a new element to $\operatorname*{Des}J$.
It suffices to show that the conditions under which an entry of a composition
$J$ can be split in the definition of the relation $\underset{M}{\rightarrow}$
are precisely the conditions $k\notin C$, $k-1\in C$ and $k+1\notin
C\cup\left\{  n\right\}  $ on $C=\operatorname*{Des}J$. This is
straightforward.}. Thus,%
\begin{align*}
&  \left\langle M_{\operatorname*{Comp}C}+M_{\operatorname*{Comp}D}%
\ \mid\ C\text{ and }D\text{ are subsets of }\left[  n-1\right]  \right. \\
&  \ \ \ \ \ \ \ \ \ \ \ \ \ \ \ \ \ \ \ \ \left.  \text{satisfying
}\operatorname*{Comp}C\underset{M}{\rightarrow}\operatorname*{Comp}%
D\right\rangle \\
&  =\left\langle M_{\operatorname*{Comp}C}+M_{\operatorname*{Comp}D}%
\ \mid\ C\text{ and }D\text{ are subsets of }\left[  n-1\right]  \right. \\
&  \ \ \ \ \ \ \ \ \ \ \ \ \ \ \ \ \ \ \ \ \left.  \text{such that there
exists some }k\in\left[  n-1\right]  \right. \\
&  \ \ \ \ \ \ \ \ \ \ \ \ \ \ \ \ \ \ \ \ \left.  \text{satisfying }%
D=C\cup\left\{  k+1\right\}  \text{, }k\notin C\text{, }k-1\in C\text{ and
}k+1\notin C\cup\left\{  n\right\}  \right\rangle \\
&  =\left\langle M_{\operatorname*{Comp}C}+M_{\operatorname*{Comp}\left(
C\cup\left\{  k+1\right\}  \right)  }\ \mid\ C\subseteq\left[  n-1\right]
\text{ and }k\in\left[  n-1\right]  \right. \\
&  \ \ \ \ \ \ \ \ \ \ \ \ \ \ \ \ \ \ \ \ \left.  \text{are such that
}k\notin C\text{, }k-1\in C\text{ and }k+1\notin C\cup\left\{  n\right\}
\right\rangle \\
&  =\left\langle \underbrace{M_{\operatorname*{Comp}C}+M_{\operatorname*{Comp}%
\left(  C\cup\left\{  k+1\right\}  \right)  }}_{\substack{=\mathbf{m}%
_{C,k}\\\text{(by (\ref{pf.prop.K.Epk.M.mCk=}))}}}\ \mid\ \left(  C,k\right)
\in\Omega\right\rangle \\
&  \ \ \ \ \ \ \ \ \ \ \left(  \text{by the definition of }\Omega\right) \\
&  =\left\langle \mathbf{m}_{C,k}\ \mid\ \left(  C,k\right)  \in
\Omega\right\rangle .
\end{align*}
Now, recall that $\operatorname*{Comp}$ is a bijection between the subsets of
$\left[  n-1\right]  $ and the compositions of $n$. Hence,%
\begin{align*}
&  \left\langle M_{J}+M_{K}\ \mid\ J\text{ and }K\text{ are compositions of
}n\text{ satisfying }J\underset{M}{\rightarrow}K\right\rangle \\
&  =\left\langle M_{\operatorname*{Comp}C}+M_{\operatorname*{Comp}D}%
\ \mid\ C\text{ and }D\text{ are subsets of }\left[  n-1\right]  \right. \\
&  \ \ \ \ \ \ \ \ \ \ \ \ \ \ \ \ \ \ \ \ \left.  \text{satisfying
}\operatorname*{Comp}C\underset{M}{\rightarrow}\operatorname*{Comp}%
D\right\rangle \\
&  =\left\langle \mathbf{m}_{C,k}\ \mid\ \left(  C,k\right)  \in
\Omega\right\rangle .
\end{align*}
This proves Claim 1.]

\begin{statement}
\textit{Claim 2:} We have%
\begin{align*}
&  \left\langle F_{J}-F_{K}\ \mid\ J\text{ and }K\text{ are compositions of
}n\text{ satisfying }J\rightarrow K\right\rangle \\
&  =\left\langle \mathbf{f}_{C,k}\ \mid\ \left(  C,k\right)  \in
\Omega\right\rangle .
\end{align*}

\end{statement}

[\textit{Proof of Claim 2:} It is easy to see that two subsets $C$ and $D$ of
$\left[  n-1\right]  $ satisfy $\operatorname*{Comp}C\rightarrow
\operatorname*{Comp}D$ if and only if there exists some $k\in\left[
n-1\right]  $ satisfying $D=C\cup\left\{  k\right\}  $, $k\notin C$, $k-1\in
C$ and $k+1\notin C\cup\left\{  n\right\}  $.\ \ \ \ \footnote{To prove this,
recall that \textquotedblleft splitting\textquotedblright\ an entry of a
composition $J$ into two consecutive entries (summing up to the original
entry) is always tantamount to adding a new element to $\operatorname*{Des}J$.
It suffices to show that the conditions under which an entry of a composition
$J$ can be split in the definition of the relation $\rightarrow$ are precisely
the conditions $k\notin C$, $k-1\in C$ and $k+1\notin C\cup\left\{  n\right\}
$ on $C=\operatorname*{Des}J$. This is straightforward.}. Thus,%
\begin{align*}
&  \left\langle F_{\operatorname*{Comp}C}-F_{\operatorname*{Comp}D}%
\ \mid\ C\text{ and }D\text{ are subsets of }\left[  n-1\right]  \right. \\
&  \ \ \ \ \ \ \ \ \ \ \ \ \ \ \ \ \ \ \ \ \left.  \text{satisfying
}\operatorname*{Comp}C\rightarrow\operatorname*{Comp}D\right\rangle \\
&  =\left\langle F_{\operatorname*{Comp}C}-F_{\operatorname*{Comp}D}%
\ \mid\ C\text{ and }D\text{ are subsets of }\left[  n-1\right]  \right. \\
&  \ \ \ \ \ \ \ \ \ \ \ \ \ \ \ \ \ \ \ \ \left.  \text{such that there
exists some }k\in\left[  n-1\right]  \right. \\
&  \ \ \ \ \ \ \ \ \ \ \ \ \ \ \ \ \ \ \ \ \left.  \text{satisfying }%
D=C\cup\left\{  k\right\}  \text{, }k\notin C\text{, }k-1\in C\text{ and
}k+1\notin C\cup\left\{  n\right\}  \right\rangle \\
&  =\left\langle F_{\operatorname*{Comp}C}-F_{\operatorname*{Comp}\left(
C\cup\left\{  k\right\}  \right)  }\ \mid\ C\subseteq\left[  n-1\right]
\text{ and }k\in\left[  n-1\right]  \right. \\
&  \ \ \ \ \ \ \ \ \ \ \ \ \ \ \ \ \ \ \ \ \left.  \text{are such that
}k\notin C\text{, }k-1\in C\text{ and }k+1\notin C\cup\left\{  n\right\}
\right\rangle \\
&  =\left\langle \underbrace{F_{\operatorname*{Comp}C}-F_{\operatorname*{Comp}%
\left(  C\cup\left\{  k\right\}  \right)  }}_{\substack{=\mathbf{f}%
_{C,k}\\\text{(by (\ref{pf.prop.K.Epk.M.fCk=}))}}}\ \mid\ \left(  C,k\right)
\in\Omega\right\rangle \\
&  \ \ \ \ \ \ \ \ \ \ \left(  \text{by the definition of }\Omega\right) \\
&  =\left\langle \mathbf{f}_{C,k}\ \mid\ \left(  C,k\right)  \in
\Omega\right\rangle .
\end{align*}
Now, recall that $\operatorname*{Comp}$ is a bijection between the subsets of
$\left[  n-1\right]  $ and the compositions of $n$. Hence,%
\begin{align*}
&  \left\langle F_{J}-F_{K}\ \mid\ J\text{ and }K\text{ are compositions of
}n\text{ satisfying }J\rightarrow K\right\rangle \\
&  =\left\langle F_{\operatorname*{Comp}C}-F_{\operatorname*{Comp}D}%
\ \mid\ C\text{ and }D\text{ are subsets of }\left[  n-1\right]  \right. \\
&  \ \ \ \ \ \ \ \ \ \ \ \ \ \ \ \ \ \ \ \ \left.  \text{satisfying
}\operatorname*{Comp}C\rightarrow\operatorname*{Comp}D\right\rangle \\
&  =\left\langle \mathbf{f}_{C,k}\ \mid\ \left(  C,k\right)  \in
\Omega\right\rangle .
\end{align*}
This proves Claim 2.]

We define a partial order on the set $\Omega$ by setting%
\[
\left(  B,k\right)  \geq\left(  C,\ell\right)  \ \ \ \ \ \ \ \ \ \ \text{if
and only if}\ \ \ \ \ \ \ \ \ \ \left(  k=\ell\text{ and }B\supseteq C\right)
.
\]
Thus, $\Omega$ is a finite poset.

\begin{statement}
\textit{Claim 3:} For every $\left(  C,\ell\right)  \in\Omega$, we have%
\[
\mathbf{m}_{C,\ell}=\sum_{\substack{\left(  B,k\right)  \in\Omega;\\\left(
B,k\right)  \geq\left(  C,\ell\right)  }}\left(  -1\right)  ^{\left\vert
B\setminus C\right\vert }\mathbf{f}_{B,k}.
\]

\end{statement}

[\textit{Proof of Claim 3:} Let $\left(  C,\ell\right)  \in\Omega$. Thus, $C$
is a subset of $\left[  n-1\right]  $ and $\ell$ is an element of $\left[
n-1\right]  $ satisfying $\ell\notin C$, $\ell-1\in C$ and $\ell+1\notin
C\cup\left\{  n\right\}  $. From $\ell+1\notin C\cup\left\{  n\right\}  $, we
obtain $\ell+1\notin C$ and $\ell+1\neq n$. From $\ell+1\neq n$, we obtain
$\ell+1\in\left[  n-1\right]  $. Also, $\left(  \ell+1\right)  -1=\ell\notin
C\cup\left\{  0\right\}  $ (since $\ell\notin C$ and $\ell\neq0$). Thus,
Proposition \ref{prop.M-through-F.2} \textbf{(c)} (applied to $k=\ell+1$)
yields\footnote{Here and in the following, the bound variable $B$ in a sum
always is understood to be a subset of $\left[  n-1\right]  $.}%
\[
M_{\operatorname*{Comp}C}+M_{\operatorname*{Comp}\left(  C\cup\left\{
\ell+1\right\}  \right)  }=\sum_{\substack{B\supseteq C;\\\ell+1\notin
B;\\\ell\notin B}}\left(  -1\right)  ^{\left\vert B\setminus C\right\vert
}\left(  F_{\operatorname*{Comp}B}-F_{\operatorname*{Comp}\left(
B\cup\left\{  \ell\right\}  \right)  }\right)  .
\]


But every $B\subseteq\left[  n-1\right]  $ satisfying $B\supseteq C$ must
satisfy $\ell-1\in B$ (since $\ell-1\in C\subseteq B$). Hence, we can
manipulate summation signs as follows:%
\begin{align}
\sum_{\substack{B\supseteq C;\\\ell+1\notin B;\\\ell\notin B}}  &
=\sum_{\substack{B\supseteq C;\\\ell+1\notin B;\\\ell\notin B;\\\ell-1\in
B}}=\sum_{\substack{B\supseteq C;\\\ell+1\notin B\cup\left\{  n\right\}
;\\\ell\notin B;\\\ell-1\in B}}\ \ \ \ \ \ \ \ \ \ \left(
\begin{array}
[c]{c}%
\text{since }\ell+1\notin B\text{ is equivalent to }\ell+1\notin B\cup\left\{
n\right\} \\
\text{(because }\ell+1\neq n\text{)}%
\end{array}
\right) \nonumber\\
&  =\sum_{\substack{B\supseteq C;\\\ell\notin B;\\\ell-1\in B;\\\ell+1\notin
B\cup\left\{  n\right\}  }}\nonumber\\
&  =\sum_{\substack{B\supseteq C;\\\left(  B,\ell\right)  \in\Omega
}}\ \ \ \ \ \ \ \ \ \ \left(
\begin{array}
[c]{c}%
\text{since the}\\
\text{condition }\left(  \ell\notin B\text{, }\ell-1\in B\text{ and }%
\ell+1\notin B\cup\left\{  n\right\}  \right) \\
\text{on a subset }B\text{ of }\left[  n-1\right]  \text{ is equivalent to
}\left(  B,\ell\right)  \in\Omega\\
\text{(by the definition of }\Omega\text{)}%
\end{array}
\right) \nonumber\\
&  =\sum_{\substack{\left(  B,\ell\right)  \in\Omega;\\B\supseteq C}%
}=\sum_{\substack{\left(  B,k\right)  \in\Omega;\\k=\ell;\\B\supseteq C}%
}=\sum_{\substack{\left(  B,k\right)  \in\Omega;\\\left(  B,k\right)
\geq\left(  C,\ell\right)  }} \label{pf.prop.K.Epk.M.sum-man}%
\end{align}
(since the condition $\left(  k=\ell\text{ and }B\supseteq C\right)  $ on a
$\left(  B,k\right)  \in\Omega$ is equivalent to $\left(  B,k\right)
\geq\left(  C,\ell\right)  $ (by the definition of the partial order on
$\Omega$)).

Now, the definition of $\mathbf{m}_{C,\ell}$ yields%
\begin{align*}
\mathbf{m}_{C,\ell}  &  =M_{\operatorname*{Comp}C}+M_{\operatorname*{Comp}%
\left(  C\cup\left\{  \ell+1\right\}  \right)  }\\
&  =\underbrace{\sum_{\substack{B\supseteq C;\\\ell+1\notin B;\\\ell\notin
B}}}_{\substack{=\sum_{\substack{\left(  B,k\right)  \in\Omega;\\\left(
B,k\right)  \geq\left(  C,\ell\right)  }}\\\text{(by
(\ref{pf.prop.K.Epk.M.sum-man}))}}}\left(  -1\right)  ^{\left\vert B\setminus
C\right\vert }\left(  F_{\operatorname*{Comp}B}-F_{\operatorname*{Comp}\left(
B\cup\left\{  \ell\right\}  \right)  }\right) \\
&  =\sum_{\substack{\left(  B,k\right)  \in\Omega;\\\left(  B,k\right)
\geq\left(  C,\ell\right)  }}\left(  -1\right)  ^{\left\vert B\setminus
C\right\vert }\left(  F_{\operatorname*{Comp}B}%
-\underbrace{F_{\operatorname*{Comp}\left(  B\cup\left\{  \ell\right\}
\right)  }}_{\substack{=F_{\operatorname*{Comp}\left(  B\cup\left\{
k\right\}  \right)  }\\\text{(since }\ell=k\\\text{(since }\left(  B,k\right)
\geq\left(  C,\ell\right)  \text{ and thus }k=\ell\text{))}}}\right) \\
&  =\sum_{\substack{\left(  B,k\right)  \in\Omega;\\\left(  B,k\right)
\geq\left(  C,\ell\right)  }}\left(  -1\right)  ^{\left\vert B\setminus
C\right\vert }\underbrace{\left(  F_{\operatorname*{Comp}B}%
-F_{\operatorname*{Comp}\left(  B\cup\left\{  k\right\}  \right)  }\right)
}_{\substack{=\mathbf{f}_{B,k}\\\text{(since (\ref{pf.prop.K.Epk.M.fCk=})
yields}\\\mathbf{f}_{B,k}=F_{\operatorname*{Comp}B}-F_{\operatorname*{Comp}%
\left(  B\cup\left\{  k\right\}  \right)  }\text{)}}}\\
&  =\sum_{\substack{\left(  B,k\right)  \in\Omega;\\\left(  B,k\right)
\geq\left(  C,\ell\right)  }}\left(  -1\right)  ^{\left\vert B\setminus
C\right\vert }\mathbf{f}_{B,k}.
\end{align*}
This proves Claim 3.]

Now, Claim 3 shows that the family $\left(  \mathbf{m}_{C,k}\right)  _{\left(
C,k\right)  \in\Omega}$ expands triangularly with respect to the family
$\left(  \mathbf{f}_{C,k}\right)  _{\left(  C,k\right)  \in\Omega}$ with
respect to the poset structure on $\Omega$. Moreover, the expansion is
\textbf{uni}triangular (because if $\left(  B,k\right)  =\left(
C,\ell\right)  $, then $B=C$ and thus $\left(  -1\right)  ^{\left\vert
B\setminus C\right\vert }=\left(  -1\right)  ^{\left\vert C\setminus
C\right\vert }=\left(  -1\right)  ^{0}=1$) and thus invertibly triangular
(this means that the diagonal entries are invertible). Therefore, by a
standard fact from linear algebra (see, e.g., \cite[Corollary 11.1.19
\textbf{(b)}]{HopfComb}), we conclude that the span of the family $\left(
\mathbf{m}_{C,k}\right)  _{\left(  C,k\right)  \in\Omega}$ equals the span of
the family $\left(  \mathbf{f}_{C,k}\right)  _{\left(  C,k\right)  \in\Omega}%
$. In other words,%
\[
\left\langle \mathbf{m}_{C,k}\ \mid\ \left(  C,k\right)  \in\Omega
\right\rangle =\left\langle \mathbf{f}_{C,k}\ \mid\ \left(  C,k\right)
\in\Omega\right\rangle .
\]
Now, Claim 1 yields%
\begin{align*}
&  \left\langle M_{J}+M_{K}\ \mid\ J\text{ and }K\text{ are compositions of
}n\text{ satisfying }J\underset{M}{\rightarrow}K\right\rangle \\
&  =\left\langle \mathbf{m}_{C,k}\ \mid\ \left(  C,k\right)  \in
\Omega\right\rangle =\left\langle \mathbf{f}_{C,k}\ \mid\ \left(  C,k\right)
\in\Omega\right\rangle \\
&  =\left\langle F_{J}-F_{K}\ \mid\ J\text{ and }K\text{ are compositions of
}n\text{ satisfying }J\rightarrow K\right\rangle
\end{align*}
(by Claim 2).

Now, forget that we fixed $n$. We thus have proven that
\begin{align*}
&  \left\langle M_{J}+M_{K}\ \mid\ J\text{ and }K\text{ are compositions of
}n\text{ satisfying }J\underset{M}{\rightarrow}K\right\rangle \\
&  =\left\langle F_{J}-F_{K}\ \mid\ J\text{ and }K\text{ are compositions of
}n\text{ satisfying }J\rightarrow K\right\rangle
\end{align*}
for each $n\in\mathbb{N}$. Thus,%
\begin{align*}
&  \sum_{n\in\mathbb{N}}\left\langle M_{J}+M_{K}\ \mid\ J\text{ and }K\text{
are compositions of }n\text{ satisfying }J\underset{M}{\rightarrow
}K\right\rangle \\
&  =\sum_{n\in\mathbb{N}}\left\langle F_{J}-F_{K}\ \mid\ J\text{ and }K\text{
are compositions of }n\text{ satisfying }J\rightarrow K\right\rangle .
\end{align*}
In light of%
\[
\mathcal{K}_{\operatorname*{Epk}}=\sum_{n\in\mathbb{N}}\left\langle
F_{J}-F_{K}\ \mid\ J\text{ and }K\text{ are compositions of }n\text{
satisfying }J\rightarrow K\right\rangle
\]
and%
\[
\mathcal{M}=\sum_{n\in\mathbb{N}}\left\langle M_{J}+M_{K}\ \mid\ J\text{ and
}K\text{ are compositions of }n\text{ satisfying }J\underset{M}{\rightarrow
}K\right\rangle ,
\]
this rewrites as $\mathcal{M}=\mathcal{K}_{\operatorname*{Epk}}$. In other
words, $\mathcal{K}_{\operatorname*{Epk}}=\mathcal{M}$. This proves
Proposition \ref{prop.K.Epk.M}.
\end{proof}
\end{verlong}

\begin{question}
It is worth analyzing the kernels of other known descent statistics
(shuffle-compatible of not). We have seen above that $\mathcal{K}%
_{\operatorname*{Epk}}$ is a \textquotedblleft binomial
subspace\textquotedblright\ of $\operatorname*{QSym}$ in the fundamental basis
(i.e., it can be spanned by elements of the form $\lambda F_{J}+\mu F_{K}$ for
$J,K\in\operatorname*{Comp}$) and a \textquotedblleft binomial
subspace\textquotedblright\ of $\operatorname*{QSym}$ in the monomial basis
(i.e., it can be spanned by elements of the form $\lambda M_{J}+\mu M_{K}$ for
$J,K\in\operatorname*{Comp}$). What other descent statistics have this property?
\end{question}

\begin{verlong}
\subsection{\label{subsect.K.pfAsh}Appendix: Proof of Proposition
\ref{prop.Ast.alg} and Theorem \ref{thm.4.3}}

Let us now give proofs of Proposition \ref{prop.Ast.alg} and Theorem
\ref{thm.4.3}, which we have promised above. We will mostly rely on Lemma
\ref{lem.K.fist} and on Proposition \ref{prop.4.1.rewr}.

For the rest of Subsection \ref{subsect.K.pfAsh}, we shall make the following conventions:

\begin{convention}
Let $\operatorname*{st}$ be a permutation statistic. For each permutation
$\pi$, let $\left[  \pi\right]  _{\operatorname*{st}}$ denote the
$\operatorname*{st}$-equivalence class of $\pi$. Let $\mathcal{A}%
_{\operatorname*{st}}$ be the free $\mathbb{Q}$-vector space whose basis is
the set of all $\operatorname*{st}$-equivalence classes of permutations. (This
is well-defined whether or not $\operatorname*{st}$ is shuffle-compatible.)
\end{convention}

\begin{proof}
[Proof of Proposition \ref{prop.Ast.alg}.]A \textit{magmatic algebra} shall
mean a $\mathbb{Q}$-vector space equipped with a binary operation which is
written as multiplication (i.e., we write $ab$ for the image of a pair
$\left(  a,b\right)  $ under this operation), but is not required to be
associative (or have a unity). An (actual, i.e., associative unital) algebra
is thus a magmatic algebra whose multiplication is associative and has a
unity. In particular, any actual algebra is a magmatic algebra. A
\textit{magmatic algebra homomorphism} is a $\mathbb{Q}$-linear map between
two magmatic algebras that preserves the multiplication.

We make $\mathcal{A}_{\operatorname*{st}}$ into a magmatic algebra by setting%
\begin{equation}
\left[  \pi\right]  _{\operatorname*{st}}\left[  \sigma\right]
_{\operatorname*{st}}=\sum_{\tau\in S\left(  \pi,\sigma\right)  }\left[
\tau\right]  _{\operatorname*{st}} \label{pf.prop.Ast.alg.algdef}%
\end{equation}
for any two disjoint permutations $\pi$ and $\sigma$. This is well-defined,
because the right-hand side of (\ref{pf.prop.Ast.alg.algdef}) depends only on
the $\operatorname*{st}$-equivalence classes $\left[  \pi\right]
_{\operatorname*{st}}$ and $\left[  \sigma\right]  _{\operatorname*{st}}$
rather than on the permutations $\pi$ and $\sigma$ themselves (this is because
$\operatorname*{st}$ is shuffle-compatible).

Define a $\mathbb{Q}$-linear map $p:\operatorname*{QSym}\rightarrow
\mathcal{A}_{\operatorname*{st}}$ by requiring that%
\begin{align*}
p\left(  F_{L}\right)   &  =\left[  \pi\right]  _{\operatorname*{st}%
}\ \ \ \ \ \ \ \ \ \ \text{for every composition }L\text{ and every}\\
&  \ \ \ \ \ \ \ \ \ \ \ \ \ \ \ \ \ \ \ \ \text{permutation }\pi\text{ with
}\operatorname*{Comp}\pi=L.
\end{align*}
This is well-defined, because for any given composition $L$, any two
permutations $\pi$ with $\operatorname*{Comp}\pi=L$ will have the same
$\operatorname*{st}$-equivalence class $\left[  \pi\right]
_{\operatorname*{st}}$ (since $\operatorname*{st}$ is a descent statistic).

Thus, each permutation $\pi$ satisfies
\begin{equation}
p\left(  F_{\operatorname*{Comp}\pi}\right)  =\left[  \pi\right]
_{\operatorname*{st}} \label{pf.prop.Ast.alg.pFCp}%
\end{equation}
and therefore $\left[  \pi\right]  _{\operatorname*{st}}=p\left(
F_{\operatorname*{Comp}\pi}\right)  \in p\left(  \operatorname*{QSym}\right)
$. Hence, $\mathcal{A}_{\operatorname*{st}}\subseteq p\left(
\operatorname*{QSym}\right)  $ (since the $\operatorname*{st}$-equivalence
classes $\left[  \pi\right]  _{\operatorname*{st}}$ form a basis of
$\mathcal{A}_{\operatorname*{st}}$). Consequently, the map $p$ is surjective.

Moreover, we have%
\begin{equation}
p\left(  ab\right)  =p\left(  a\right)  p\left(  b\right)
\ \ \ \ \ \ \ \ \ \ \text{for all }a,b\in\operatorname*{QSym}.
\label{pf.prop.Ast.alg.pab}%
\end{equation}


[\textit{Proof of (\ref{pf.prop.Ast.alg.pab}):} Let $a,b\in
\operatorname*{QSym}$. We must prove the equality (\ref{pf.prop.Ast.alg.pab}).
Since this equality is $\mathbb{Q}$-linear in each of $a$ and $b$, we WLOG
assume that $a$ and $b$ belong to the fundamental basis of
$\operatorname*{QSym}$. That is, $a=F_{J}$ and $b=F_{K}$ for two compositions
$J$ and $K$. Consider these $J$ and $K$. Fix any two disjoint permutations
$\pi$ and $\sigma$ such that $\operatorname*{Comp}\pi=J$ and
$\operatorname*{Comp}\sigma=K$. (Such $\pi$ and $\sigma$ are easy to find.)
The definition of $p$ thus yields $p\left(  F_{J}\right)  =\left[  \pi\right]
_{\operatorname*{st}}$ and $p\left(  F_{K}\right)  =\left[  \sigma\right]
_{\operatorname*{st}}$. Hence,%
\begin{align}
p\left(  \underbrace{a}_{=F_{J}}\right)  p\left(  \underbrace{b}_{=F_{K}%
}\right)   &  =\underbrace{p\left(  F_{J}\right)  }_{=\left[  \pi\right]
_{\operatorname*{st}}}\underbrace{p\left(  F_{K}\right)  }_{=\left[
\sigma\right]  _{\operatorname*{st}}}=\left[  \pi\right]  _{\operatorname*{st}%
}\left[  \sigma\right]  _{\operatorname*{st}}\nonumber\\
&  =\sum_{\tau\in S\left(  \pi,\sigma\right)  }\left[  \tau\right]
_{\operatorname*{st}}\ \ \ \ \ \ \ \ \ \ \left(  \text{by
(\ref{pf.prop.Ast.alg.algdef})}\right) \nonumber\\
&  =\sum_{\chi\in S\left(  \pi,\sigma\right)  }\left[  \chi\right]
_{\operatorname*{st}} \label{pf.prop.Ast.alg.pab.pf.1}%
\end{align}
(here, we have renamed the summation index $\tau$ as $\chi$). On the other
hand, $a=F_{J}=F_{\operatorname*{Comp}\pi}$ (since $J=\operatorname*{Comp}\pi
$) and $b=F_{\operatorname*{Comp}\sigma}$ (similarly); multiplying these
equalities, we get%
\[
ab=F_{\operatorname*{Comp}\pi}F_{\operatorname*{Comp}\sigma}=\sum_{\chi\in
S\left(  \pi,\sigma\right)  }F_{\operatorname*{Comp}\chi}%
\]
(by Proposition \ref{prop.4.1.rewr}). Applying the map $p$ to this equality,
we find%
\begin{align*}
p\left(  ab\right)   &  =p\left(  \sum_{\chi\in S\left(  \pi,\sigma\right)
}F_{\operatorname*{Comp}\chi}\right)  =\sum_{\chi\in S\left(  \pi
,\sigma\right)  }\underbrace{p\left(  F_{\operatorname*{Comp}\chi}\right)
}_{\substack{=\left[  \chi\right]  _{\operatorname*{st}}\\\text{(by
(\ref{pf.prop.Ast.alg.pFCp}))}}}=\sum_{\chi\in S\left(  \pi,\sigma\right)
}\left[  \chi\right]  _{\operatorname*{st}}\\
&  =p\left(  a\right)  p\left(  b\right)  \ \ \ \ \ \ \ \ \ \ \left(  \text{by
(\ref{pf.prop.Ast.alg.pab.pf.1})}\right)  .
\end{align*}
This proves (\ref{pf.prop.Ast.alg.pab}).]

The equality (\ref{pf.prop.Ast.alg.pab}) shows that $p$ is a magmatic algebra
homomorphism (since $p$ is $\mathbb{Q}$-linear). Thus, using the surjectivity
of $p$, we can easily see that the magmatic algebra $\mathcal{A}%
_{\operatorname*{st}}$ is associative\footnote{\textit{Proof.} Let
$u,v,w\in\mathcal{A}_{\operatorname*{st}}$. We must show that $\left(
uv\right)  w=u\left(  vw\right)  $.
\par
There exist $a,b,c\in\operatorname*{QSym}$ such that $u=p\left(  a\right)  $,
$v=p\left(  b\right)  $ and $w=p\left(  c\right)  $ (since $p$ is surjective).
Fix such $a,b,c$. Since $\operatorname*{QSym}$ is an actual (i.e., associative
unital) algebra, we have%
\begin{align*}
p\left(  abc\right)   &  =p\left(  \left(  ab\right)  c\right)
=\underbrace{p\left(  ab\right)  }_{\substack{=p\left(  a\right)  p\left(
b\right)  \\\text{(since }p\text{ is a magmatic}\\\text{algebra homomorphism)}%
}}p\left(  c\right)  \ \ \ \ \ \ \ \ \ \ \left(  \text{since }p\text{ is a
magmatic algebra homomorphism}\right) \\
&  =\left(  \underbrace{p\left(  a\right)  }_{=u}\underbrace{p\left(
b\right)  }_{=v}\right)  \underbrace{p\left(  c\right)  }_{=w}=\left(
uv\right)  w.
\end{align*}
A similar argument shows that $p\left(  abc\right)  =u\left(  vw\right)  $.
Thus, $\left(  uv\right)  w=p\left(  abc\right)  =u\left(  vw\right)  $,
qed.}. In other words, the multiplication on $\mathcal{A}_{\operatorname*{st}%
}$ defined in Definition \ref{def.Ast} is associative. Moreover, it is clear
that the $\operatorname*{st}$-equivalence class of the empty permutation
serves as a neutral element for this multiplication (because if $\varnothing$
denotes the empty permutation, then $S\left(  \varnothing,\sigma\right)
=S\left(  \sigma,\varnothing\right)  =\left\{  \sigma\right\}  $ for every
permutation $\sigma$). Thus, the multiplication on $\mathcal{A}%
_{\operatorname*{st}}$ defined in Definition \ref{def.Ast} is well-defined and
associative, and turns $\mathcal{A}_{\operatorname*{st}}$ into a $\mathbb{Q}%
$-algebra whose unity is the $\operatorname*{st}$-equivalence class of the
empty permutation. This proves Proposition \ref{prop.Ast.alg} \textbf{(a)}.

\textbf{(b)} The map $p:\operatorname*{QSym}\rightarrow\mathcal{A}%
_{\operatorname*{st}}$ is $\mathbb{Q}$-linear and respects multiplication (by
(\ref{pf.prop.Ast.alg.pab})). Moreover, it sends the unity of
$\operatorname*{QSym}$ to the unity of the algebra $\mathcal{A}%
_{\operatorname*{st}}$\ \ \ \ \footnote{\textit{Proof.} The unity of
$\operatorname*{QSym}$ is $1=F_{\left(  {}\right)  }$, where $\left(
{}\right)  $ denotes the empty composition. Now, let $\varnothing$ denote the
empty permutation. Then, the $\operatorname*{st}$-equivalence class $\left[
\varnothing\right]  _{\operatorname*{st}}$ is the unity of the algebra
$\mathcal{A}_{\operatorname*{st}}$. But the empty permutation $\varnothing$
has descent composition $\operatorname*{Comp}\varnothing=\left(  {}\right)  $.
Hence, the definition of $p$ yields $p\left(  F_{\left(  {}\right)  }\right)
=\left[  \varnothing\right]  _{\operatorname*{st}}$. In view of what we just
said, this equality says that $p$ sends the unity of $\operatorname*{QSym}$ to
the unity of the algebra $\mathcal{A}_{\operatorname*{st}}$.}. Thus, $p$ is a
$\mathbb{Q}$-algebra homomorphism. Moreover, recall that $p$ is surjective and
satisfies $p\left(  F_{\operatorname*{Comp}\pi}\right)  =\left[  \pi\right]
_{\operatorname*{st}}$ for every permutation $\pi$. Hence, there is a
surjective $\mathbb{Q}$-algebra homomorphism $p_{\operatorname*{st}%
}:\operatorname*{QSym}\rightarrow\mathcal{A}_{\operatorname*{st}}$ that
satisfies
\[
p_{\operatorname*{st}}\left(  F_{\operatorname*{Comp}\pi}\right)  =\left[
\pi\right]  _{\operatorname*{st}}\ \ \ \ \ \ \ \ \ \ \text{for every
permutation }\pi
\]
(namely, $p_{\operatorname*{st}}=p$). This proves Proposition
\ref{prop.Ast.alg} \textbf{(b)}.
\end{proof}

\begin{proof}
[Proof of Theorem \ref{thm.4.3}.]\textbf{(a)} $\Longrightarrow:$ Assume that
$\operatorname*{st}$ is shuffle-compatible. Proposition \ref{prop.Ast.alg}
\textbf{(b)} shows that there is a surjective $\mathbb{Q}$-algebra
homomorphism $p_{\operatorname*{st}}:\operatorname*{QSym}\rightarrow
\mathcal{A}_{\operatorname*{st}}$ that satisfies
\begin{equation}
p_{\operatorname*{st}}\left(  F_{\operatorname*{Comp}\pi}\right)  =\left[
\pi\right]  _{\operatorname*{st}}\ \ \ \ \ \ \ \ \ \ \text{for every
permutation }\pi. \label{pf.thm.4.3.a.fwd.pst}%
\end{equation}
Consider this $p_{\operatorname*{st}}$.

If $\alpha$ is an $\operatorname*{st}$-equivalence class of compositions, then
we let $u_{\alpha}$ denote the $\operatorname*{st}$-equivalence class $\left[
\pi\right]  _{\operatorname*{st}}$ of all permutations $\pi$ whose descent
composition $\operatorname*{Comp}\pi$ belongs to $\alpha$. (This is indeed a
well-defined $\operatorname*{st}$-equivalence class, because
$\operatorname*{st}$ is a descent statistic.) This establishes a bijection
between the $\operatorname*{st}$-equivalence classes of compositions and the
$\operatorname*{st}$-equivalence classes of permutations. Thus, the family
$\left(  u_{\alpha}\right)  $ (indexed by $\operatorname*{st}$-equivalence
classes $\alpha$ of compositions) is just a reindexing of the basis of
$\mathcal{A}_{\operatorname*{st}}$ consisting of the $\operatorname*{st}%
$-equivalence classes $\left[  \pi\right]  _{\operatorname*{st}}$ of
permutations. Consequently, this family is a basis of the $\mathbb{Q}$-vector
space $\mathcal{A}_{\operatorname*{st}}$. Moreover, $p_{\operatorname*{st}}$
is a $\mathbb{Q}$-algebra homomorphism $\operatorname*{QSym}\rightarrow
\mathcal{A}_{\operatorname*{st}}$ with the property that whenever $\alpha$ is
an $\operatorname*{st}$-equivalence class of compositions, we have%
\[
p_{\operatorname*{st}}\left(  F_{L}\right)  =u_{\alpha}%
\ \ \ \ \ \ \ \ \ \ \text{for each }L\in\alpha.
\]
(Indeed, this follows from applying (\ref{pf.thm.4.3.a.fwd.pst}) to any
permutation $\pi$ satisfying $\operatorname*{Comp}\pi=L$.)

Thus, there exist a $\mathbb{Q}$-algebra $A$ (namely, $\mathcal{A}%
=\mathcal{A}_{\operatorname*{st}}$) with basis $\left(  u_{\alpha}\right)  $
(indexed by $\operatorname*{st}$-equivalence classes $\alpha$ of compositions)
and a $\mathbb{Q}$-algebra homomorphism $\phi_{\operatorname*{st}%
}:\operatorname*{QSym}\rightarrow A$ (namely, $\phi_{\operatorname*{st}%
}=p_{\operatorname*{st}}$) with the property that whenever $\alpha$ is an
$\operatorname*{st}$-equivalence class of compositions, we have%
\[
\phi_{\operatorname*{st}}\left(  F_{L}\right)  =u_{\alpha}%
\ \ \ \ \ \ \ \ \ \ \text{for each }L\in\alpha.
\]
This proves the $\Longrightarrow$ direction of Theorem \ref{thm.4.3}
\textbf{(a)}.

$\Longleftarrow:$ Assume that there exist a $\mathbb{Q}$-algebra $A$ with
basis $\left(  u_{\alpha}\right)  $ (indexed by $\operatorname*{st}%
$-equivalence classes $\alpha$ of compositions) and a $\mathbb{Q}$-algebra
homomorphism $\phi_{\operatorname*{st}}:\operatorname*{QSym}\rightarrow A$
with the property that whenever $\alpha$ is an $\operatorname*{st}%
$-equivalence class of compositions, we have%
\[
\phi_{\operatorname*{st}}\left(  F_{L}\right)  =u_{\alpha}%
\ \ \ \ \ \ \ \ \ \ \text{for each }L\in\alpha.
\]
Consider this $A$, this $\left(  u_{\alpha}\right)  $ and this $\phi
_{\operatorname*{st}}$. Lemma \ref{lem.K.fist} shows that $\operatorname*{Ker}%
\left(  \phi_{\operatorname*{st}}\right)  =\mathcal{K}_{\operatorname*{st}}$.
But $\operatorname*{Ker}\left(  \phi_{\operatorname*{st}}\right)  $ is an
ideal of $\operatorname*{QSym}$ (since $\phi_{\operatorname*{st}}$ is a
$\mathbb{Q}$-algebra homomorphism). In other words, $\mathcal{K}%
_{\operatorname*{st}}$ is an ideal of $\operatorname*{QSym}$ (since
$\operatorname*{Ker}\left(  \phi_{\operatorname*{st}}\right)  =\mathcal{K}%
_{\operatorname*{st}}$).

Now, consider any two disjoint permutations $\pi$ and $\sigma$. Also, consider
two further disjoint permutations $\pi^{\prime}$ and $\sigma^{\prime}$
satisfying $\operatorname*{st}\left(  \pi\right)  =\operatorname*{st}\left(
\pi^{\prime}\right)  $, $\operatorname*{st}\left(  \sigma\right)
=\operatorname*{st}\left(  \sigma^{\prime}\right)  $, $\left\vert
\pi\right\vert =\left\vert \pi^{\prime}\right\vert $ and $\left\vert
\sigma\right\vert =\left\vert \sigma^{\prime}\right\vert $. We shall show that
$\left\{  \operatorname*{st}\left(  \tau\right)  \ \mid\ \tau\in S\left(
\pi,\sigma\right)  \right\}  _{\operatorname*{multi}} =\left\{
\operatorname*{st}\left(  \tau\right)  \ \mid\ \tau\in S\left(  \pi^{\prime
},\sigma^{\prime}\right)  \right\}  _{\operatorname*{multi}} $ as multisets.
This will show that the multiset \newline$\left\{  \operatorname*{st}\left(
\tau\right)  \ \mid\ \tau\in S\left(  \pi,\sigma\right)  \right\}
_{\operatorname*{multi}} $ depends only on $\operatorname*{st}\left(
\pi\right)  $, $\operatorname*{st}\left(  \sigma\right)  $, $\left\vert
\pi\right\vert $ and $\left\vert \sigma\right\vert $.

From $\operatorname*{st}\left(  \pi\right)  =\operatorname*{st}\left(
\pi^{\prime}\right)  $ and $\left\vert \pi\right\vert =\left\vert \pi^{\prime
}\right\vert $, we conclude that $\pi$ and $\pi^{\prime}$ are
$\operatorname*{st}$-equivalent. In other words, $\operatorname*{Comp}\pi$ and
$\operatorname*{Comp}\left(  \pi^{\prime}\right)  $ are $\operatorname*{st}%
$-equivalent. Hence, $F_{\operatorname*{Comp}\pi}-F_{\operatorname*{Comp}%
\left(  \pi^{\prime}\right)  }\in\mathcal{K}_{\operatorname*{st}}$ (by the
definition of $\mathcal{K}_{\operatorname*{st}}$), so that
$F_{\operatorname*{Comp}\pi}\equiv F_{\operatorname*{Comp}\left(  \pi^{\prime
}\right)  }\operatorname{mod}\mathcal{K}_{\operatorname*{st}}$. Similarly,
$F_{\operatorname*{Comp}\sigma}\equiv F_{\operatorname*{Comp}\left(
\sigma^{\prime}\right)  }\operatorname{mod}\mathcal{K}_{\operatorname*{st}}$.
These two congruences, combined, yield $F_{\operatorname*{Comp}\pi
}F_{\operatorname*{Comp}\sigma}\equiv F_{\operatorname*{Comp}\left(
\pi^{\prime}\right)  }F_{\operatorname*{Comp}\left(  \sigma^{\prime}\right)
}\operatorname{mod}\mathcal{K}_{\operatorname*{st}}$, because $\mathcal{K}%
_{\operatorname*{st}}$ is an ideal of $\operatorname*{QSym}$.

Let $X$ be the codomain of the map $\operatorname*{st}$. Let $\mathbb{Q}%
\left[  X\right]  $ be the free $\mathbb{Q}$-vector space with basis $\left(
\left[  x\right]  \right)  _{x\in X}$. Then, we can define a $\mathbb{Q}%
$-linear map $\mathbf{st}:\operatorname*{QSym}\rightarrow\mathbb{Q}\left[
X\right]  ,\ F_{J}\mapsto\left[  \operatorname*{st}J\right]  $. This map
$\mathbf{st}$ sends each of the generators of $\mathcal{K}_{\operatorname*{st}%
}$ to $0$ (by the definition of $\mathcal{K}_{\operatorname*{st}}$), and
therefore sends the whole $\mathcal{K}_{\operatorname*{st}}$ to $0$. In other
words, $\mathbf{st}\left(  \mathcal{K}_{\operatorname*{st}}\right)  =0$.

We have $F_{\operatorname*{Comp}\pi}F_{\operatorname*{Comp}\sigma}\equiv
F_{\operatorname*{Comp}\left(  \pi^{\prime}\right)  }F_{\operatorname*{Comp}%
\left(  \sigma^{\prime}\right)  }\operatorname{mod}\mathcal{K}%
_{\operatorname*{st}}$ and thus%
\begin{equation}
\mathbf{st}\left(  F_{\operatorname*{Comp}\pi}F_{\operatorname*{Comp}\sigma
}\right)  =\mathbf{st}\left(  F_{\operatorname*{Comp}\left(  \pi^{\prime
}\right)  }F_{\operatorname*{Comp}\left(  \sigma^{\prime}\right)  }\right)
\label{pf.thm.4.3.a.fwd.steq}%
\end{equation}
(since $\mathbf{st}\left(  \mathcal{K}_{\operatorname*{st}}\right)  =0$). But
Proposition \ref{prop.4.1.rewr} yields%
\[
F_{\operatorname*{Comp}\pi}F_{\operatorname*{Comp}\sigma}=\sum_{\chi\in
S\left(  \pi,\sigma\right)  }F_{\operatorname*{Comp}\chi}.
\]
Applying the map $\mathbf{st}$ to both sides of this equality, we find%
\begin{align*}
\mathbf{st}\left(  F_{\operatorname*{Comp}\pi}F_{\operatorname*{Comp}\sigma
}\right)   &  =\mathbf{st}\left(  \sum_{\chi\in S\left(  \pi,\sigma\right)
}F_{\operatorname*{Comp}\chi}\right) \\
&  =\sum_{\chi\in S\left(  \pi,\sigma\right)  }\underbrace{\mathbf{st}\left(
F_{\operatorname*{Comp}\chi}\right)  }_{=\left[  \operatorname*{st}\left(
\operatorname*{Comp}\chi\right)  \right]  =\left[  \operatorname*{st}%
\chi\right]  }=\sum_{\chi\in S\left(  \pi,\sigma\right)  }\left[
\operatorname*{st}\chi\right]  .
\end{align*}
Similarly,%
\[
\mathbf{st}\left(  F_{\operatorname*{Comp}\left(  \pi^{\prime}\right)
}F_{\operatorname*{Comp}\left(  \sigma^{\prime}\right)  }\right)  =\sum
_{\chi\in S\left(  \pi^{\prime},\sigma^{\prime}\right)  }\left[
\operatorname*{st}\chi\right]  .
\]
But the left-hand sides of the last two equalities are equal (because of
(\ref{pf.thm.4.3.a.fwd.steq})); therefore, the right-hand sides must be equal
as well. In other words,
\[
\sum_{\chi\in S\left(  \pi,\sigma\right)  }\left[  \operatorname*{st}%
\chi\right]  =\sum_{\chi\in S\left(  \pi^{\prime},\sigma^{\prime}\right)
}\left[  \operatorname*{st}\chi\right]  .
\]
This shows exactly that $\left\{  \operatorname*{st}\left(  \chi\right)
\ \mid\ \chi\in S\left(  \pi,\sigma\right)  \right\}  _{\operatorname*{multi}}
=\left\{  \operatorname*{st}\left(  \chi\right)  \ \mid\ \chi\in S\left(
\pi^{\prime},\sigma^{\prime}\right)  \right\}  _{\operatorname*{multi}} $. In
other words, $\left\{  \operatorname*{st}\left(  \tau\right)  \ \mid\ \tau\in
S\left(  \pi,\sigma\right)  \right\}  _{\operatorname*{multi}} =\left\{
\operatorname*{st}\left(  \tau\right)  \ \mid\ \tau\in S\left(  \pi^{\prime
},\sigma^{\prime}\right)  \right\}  _{\operatorname*{multi}} $. Thus, we have
proven that the multiset $\left\{  \operatorname*{st}\left(  \tau\right)
\ \mid\ \tau\in S\left(  \pi,\sigma\right)  \right\}  _{\operatorname*{multi}}
$ depends only on $\operatorname*{st}\left(  \pi\right)  $,
$\operatorname*{st}\left(  \sigma\right)  $, $\left\vert \pi\right\vert $ and
$\left\vert \sigma\right\vert $. Hence, the statistic $\operatorname*{st}$ is
shuffle-compatible. This proves the $\Longleftarrow$ direction of Theorem
\ref{thm.4.3} \textbf{(a)}.

\textbf{(b)} Proposition \ref{prop.Ast.alg} \textbf{(b)} shows that there is a
surjective $\mathbb{Q}$-algebra homomorphism $p_{\operatorname*{st}%
}:\operatorname*{QSym}\rightarrow\mathcal{A}_{\operatorname*{st}}$ that
satisfies
\begin{equation}
p_{\operatorname*{st}}\left(  F_{\operatorname*{Comp}\pi}\right)  =\left[
\pi\right]  _{\operatorname*{st}}\ \ \ \ \ \ \ \ \ \ \text{for every
permutation }\pi. \label{pf.thm.4.3.b.pst}%
\end{equation}
Consider this $p_{\operatorname*{st}}$.

Let $\gamma$ be the $\mathbb{Q}$-linear map%
\[
\mathcal{A}_{\operatorname*{st}}\rightarrow A,\ \ \ \ \ \ \ \ \ \ \left[
\pi\right]  _{\operatorname*{st}}\mapsto u_{\alpha},
\]
where $\alpha$ is the $\operatorname*{st}$-equivalence class of the
composition $\operatorname*{Comp}\pi$. This map $\gamma$ is clearly
well-defined (since the $\operatorname*{st}$-equivalence classes $\left[
\pi\right]  _{\operatorname*{st}}$ form a basis of $\mathcal{A}%
_{\operatorname*{st}}$, and since the $\operatorname*{st}$-equivalence class
of the composition $\operatorname*{Comp}\pi$ depends only on the
$\operatorname*{st}$-equivalence class $\left[  \pi\right]
_{\operatorname*{st}}$ and not on the permutation $\pi$ itself). Moreover,
$\gamma$ sends a basis of $\mathcal{A}_{\operatorname*{st}}$ (the basis formed
by the $\operatorname*{st}$-equivalence classes $\left[  \pi\right]
_{\operatorname*{st}}$ of permutations) to a basis of $A$ (namely, to the
basis $\left(  u_{\alpha}\right)  $) bijectively; thus, $\gamma$ is an
isomorphism of $\mathbb{Q}$-vector spaces.

The diagram%
\[%
%TCIMACRO{\TeXButton{comm triangle}{\xymatrix{
%\QSym\arsurj[r]^{p_{\operatorname{st}}} \ar[dr]_{\phi_{\operatorname{st}}}
%& \mathcal{A}_{\operatorname{st}} \arsurj[d]^\gamma\\
%& A
%}}}%
%BeginExpansion
\xymatrix{
\QSym\arsurj[r]^{p_{\operatorname{st}}} \ar[dr]_{\phi_{\operatorname{st}}}
& \mathcal{A}_{\operatorname{st}} \arsurj[d]^\gamma\\
& A
}%
%EndExpansion
\]
is commutative (as one can easily check by tracing an arbitrary basis element
$F_{L}$ of $\operatorname*{QSym}$ through the diagram). Since the maps
$p_{\operatorname*{st}}$ and $\phi_{\operatorname*{st}}$ in this diagram are
$\mathbb{Q}$-algebra homomorphisms, and since $p_{\operatorname*{st}}$ is
surjective, we thus conclude that $\gamma$ is also a $\mathbb{Q}$-algebra
homomorphism\footnote{\textit{Proof.} Let $a,b\in\mathcal{A}%
_{\operatorname*{st}}$. We shall show that $\gamma\left(  ab\right)
=\gamma\left(  a\right)  \gamma\left(  b\right)  $.
\par
There exist $a^{\prime},b^{\prime}\in\operatorname*{QSym}$ such that
$a=p_{\operatorname*{st}}\left(  a^{\prime}\right)  $ and
$b=p_{\operatorname*{st}}\left(  b^{\prime}\right)  $ (since
$p_{\operatorname*{st}}$ is surjective). Consider these $a^{\prime},b^{\prime
}$. Then, $\gamma\left(  \underbrace{a}_{=p_{\operatorname*{st}}\left(
a^{\prime}\right)  }\right)  =\gamma\left(  p_{\operatorname*{st}}\left(
a^{\prime}\right)  \right)  =\phi_{\operatorname*{st}}\left(  a^{\prime
}\right)  $ (since the diagram is commutative) and $\gamma\left(  b\right)
=\phi_{\operatorname*{st}}\left(  b^{\prime}\right)  $ (similarly). But from
$a=p_{\operatorname*{st}}\left(  a^{\prime}\right)  $ and
$b=p_{\operatorname*{st}}\left(  b^{\prime}\right)  $, we obtain
$ab=p_{\operatorname*{st}}\left(  a^{\prime}\right)  p_{\operatorname*{st}%
}\left(  b^{\prime}\right)  =p_{\operatorname*{st}}\left(  a^{\prime}%
b^{\prime}\right)  $ (since $p_{\operatorname*{st}}$ is a $\mathbb{Q}$-algebra
homomorphism), so that%
\begin{align*}
\gamma\left(  ab\right)   &  =\gamma\left(  p_{\operatorname*{st}}\left(
a^{\prime}b^{\prime}\right)  \right)  =\phi_{\operatorname*{st}}\left(
a^{\prime}b^{\prime}\right)  \ \ \ \ \ \ \ \ \ \ \left(  \text{since the
diagram is commutative}\right) \\
&  =\underbrace{\phi_{\operatorname*{st}}\left(  a^{\prime}\right)  }%
_{=\gamma\left(  a\right)  }\underbrace{\phi_{\operatorname*{st}}\left(
b^{\prime}\right)  }_{=\gamma\left(  b\right)  }\ \ \ \ \ \ \ \ \ \ \left(
\text{since }\phi_{\operatorname*{st}}\text{ is a }\mathbb{Q}\text{-algebra
homomorphism}\right) \\
&  =\gamma\left(  a\right)  \gamma\left(  b\right)  .
\end{align*}
\par
Now, forget that we fixed $a,b$. We thus have proven that $\gamma\left(
ab\right)  =\gamma\left(  a\right)  \gamma\left(  b\right)  $ for all
$a,b\in\mathcal{A}_{\operatorname*{st}}$. Similarly, $\gamma\left(  1\right)
=1$. Hence, $\gamma$ is a $\mathbb{Q}$-algebra homomorphism (since $\gamma$ is
$\mathbb{Q}$-linear).}. Since $\gamma$ is an isomorphism of $\mathbb{Q}%
$-vector spaces, we thus conclude that $\gamma$ is a $\mathbb{Q}$-algebra
isomorphism $\mathcal{A}_{\operatorname*{st}}\rightarrow A$. This proves
Theorem \ref{thm.4.3} \textbf{(b)}.
\end{proof}
\end{verlong}

\section{\label{sect.dendri}Dendriform structures}

\begin{vershort}
Next, we shall recall the \textit{dendriform operations }$\left.
\prec\right.  $ and $\left.  \succeq\right.  $ on $\operatorname*{QSym}$
studied in \cite{dimcr}, and we shall connect these operations back to
LR-shuffle-compatibility. Since we consider this somewhat tangential to the
present paper, we merely summarize the main results here; more can be found in
\cite{verlong}.

\subsection{Two operations on $\operatorname*{QSym}$}

We begin with some definitions. We will use some notations from \cite{dimcr},
but we set $\mathbf{k}=\mathbb{Q}$ because we are working over the ring
$\mathbb{Q}$ in this paper. Monomials always mean formal expressions of the
form $x_{1}^{a_{1}}x_{2}^{a_{2}}x_{3}^{a_{3}}\cdots$ with $a_{1}+a_{2}%
+a_{3}+\cdots<\infty$ (see \cite[Section 2]{dimcr} for details). If
$\mathfrak{m}$ is a monomial, then $\operatorname*{Supp}\mathfrak{m}$ will
denote the finite subset
\[
\left\{  i\in\left\{  1,2,3,\ldots\right\}  \ \mid\ \text{the exponent with
which }x_{i}\text{ occurs in }\mathfrak{m}\text{ is }>0\right\}
\]
of $\left\{  1,2,3,\ldots\right\}  $. Next, we define two binary operations
\begin{align*}
&  \left.  \prec\right.  \ \left(  \text{called \textquotedblleft dendriform
less-than\textquotedblright; but it's an operation, not a relation}\right)
,\\
&  \left.  \succeq\right.  \ \left(  \text{called \textquotedblleft dendriform
greater-or-equal\textquotedblright; but it's an operation, not a
relation}\right)  ,
\end{align*}
on the ring $\mathbf{k}\left[  \left[  x_{1},x_{2},x_{3},\ldots\right]
\right]  $ of power series by first defining how they act on monomials:%
\begin{align*}
\mathfrak{m}\left.  \prec\right.  \mathfrak{n}  &  =\left\{
\begin{array}
[c]{l}%
\mathfrak{m}\cdot\mathfrak{n},\ \ \ \ \ \ \ \ \ \ \text{if }\min\left(
\operatorname*{Supp}\mathfrak{m}\right)  <\min\left(  \operatorname*{Supp}%
\mathfrak{n}\right)  ;\\
0,\ \ \ \ \ \ \ \ \ \ \text{if }\min\left(  \operatorname*{Supp}%
\mathfrak{m}\right)  \geq\min\left(  \operatorname*{Supp}\mathfrak{n}\right)
\end{array}
\right.  ;\\
\mathfrak{m}\left.  \succeq\right.  \mathfrak{n}  &  =\left\{
\begin{array}
[c]{l}%
\mathfrak{m}\cdot\mathfrak{n},\ \ \ \ \ \ \ \ \ \ \text{if }\min\left(
\operatorname*{Supp}\mathfrak{m}\right)  \geq\min\left(  \operatorname*{Supp}%
\mathfrak{n}\right)  ;\\
0,\ \ \ \ \ \ \ \ \ \ \text{if }\min\left(  \operatorname*{Supp}%
\mathfrak{m}\right)  <\min\left(  \operatorname*{Supp}\mathfrak{n}\right)
\end{array}
\right.  ;
\end{align*}
and then requiring that they all be $\mathbf{k}$-bilinear and continuous (so
their action on pairs of arbitrary power series can be computed by
\textquotedblleft opening the parentheses\textquotedblright). These operations
$\left.  \prec\right.  $ and $\left.  \succeq\right.  $ restrict to the subset
$\operatorname*{QSym}$ of $\mathbf{k}\left[  \left[  x_{1},x_{2},x_{3}%
,\ldots\right]  \right]  $ (this is proven in \cite[detailed version, Section
3]{dimcr}). They furthermore satisfy numerous relations:

\begin{itemize}
\item The dendriform operations satisfy the four rules
\begin{align*}
a\left.  \prec\right.  b+a\left.  \succeq\right.  b  &  =ab;\\
\left(  a\left.  \prec\right.  b\right)  \left.  \prec\right.  c  &  =a\left.
\prec\right.  \left(  bc\right)  ;\ \ \ \ \ \ \ \ \ \ \left(  a\left.
\succeq\right.  b\right)  \left.  \prec\right.  c=a\left.  \succeq\right.
\left(  b\left.  \prec\right.  c\right)  ;\\
a\left.  \succeq\right.  \left(  b\left.  \succeq\right.  c\right)   &
=\left(  ab\right)  \left.  \succeq\right.  c
\end{align*}
for all $a,b,c\in\mathbf{k}\left[  \left[  x_{1},x_{2},x_{3},\ldots\right]
\right]  $.

\item For any $a\in\mathbf{k}\left[  \left[  x_{1},x_{2},x_{3},\ldots\right]
\right]  $, we have%
\begin{align*}
1\left.  \prec\right.  a  &  =0;\ \ \ \ \ \ \ \ \ \ a\left.  \prec\right.
1=a-\varepsilon\left(  a\right)  ;\\
1\left.  \succeq\right.  a  &  =a;\ \ \ \ \ \ \ \ \ \ a\left.  \succeq\right.
1=\varepsilon\left(  a\right)  ,
\end{align*}
where $\varepsilon\left(  a\right)  $ denotes the constant term of the power
series $a$.
\end{itemize}

The operations $\left.  \prec\right.  $ and $\left.  \succeq\right.  $ are
sometimes called \textquotedblleft restricted products\textquotedblright\ due
to their similarity with the (regular) multiplication of $\operatorname*{QSym}%
$. In particular, they satisfy the following analogue of Proposition
\ref{prop.4.1.rewr}:

\begin{corollary}
Let $\pi$ and $\sigma$ be two disjoint nonempty permutations. Assume that
$\pi_{1}>\sigma_{1}$. Then,%
\[
F_{\operatorname*{Comp}\pi}\left.  \prec\right.  F_{\operatorname*{Comp}%
\sigma}=\sum_{\chi\in S_{\prec}\left(  \pi,\sigma\right)  }%
F_{\operatorname*{Comp}\chi}%
\]
and%
\[
F_{\operatorname*{Comp}\pi}\left.  \succeq\right.  F_{\operatorname*{Comp}%
\sigma}=\sum_{\chi\in S_{\succ}\left(  \pi,\sigma\right)  }%
F_{\operatorname*{Comp}\chi}.
\]

\end{corollary}

\subsection{Left- and right-shuffle-compatibility and ideals}

This corollary lets us relate the notions introduced in Definition
\ref{def.LR.left-right} to the operations $\left.  \prec\right.  $ and
$\left.  \succeq\right.  $. To state the precise connection, we need the
following notation:

\begin{definition}
Let $A$ be a $\mathbf{k}$-module equipped with some binary operation $\ast$
(written infix).

\textbf{(a)} If $B$ and $C$ are two $\mathbf{k}$-submodules of $A$, then
$B\ast C$ shall mean the $\mathbf{k}$-submodule of $A$ spanned by all elements
of the form $b\ast c$ with $b\in B$ and $c\in C$.

\textbf{(b)} A $\mathbf{k}$-submodule $M$ of $A$ is said to be a\textit{
}$\ast$\textit{-ideal} if and only if it satisfies $A\ast M\subseteq M$ and
$M\ast A\subseteq M$.
\end{definition}

Now, let us define two further variants of LR-shuffle-compatibility (to be
compared with those introduced in Definition \ref{def.LR.left-right}):

\begin{definition}
Let $\operatorname*{st}$ be a permutation statistic.

\textbf{(a)} We say that $\operatorname*{st}$ is \textit{weakly
left-shuffle-compatible} if for any two disjoint nonempty permutations $\pi$
and $\sigma$ having the property that%
\[
\text{each entry of }\pi\text{ is greater than each entry of }\sigma,
\]
the multiset $\left\{  \operatorname*{st}\left(  \tau\right)  \ \mid\ \tau\in
S_{\prec}\left(  \pi,\sigma\right)  \right\}  _{\operatorname*{multi}}$
depends only on $\operatorname*{st}\left(  \pi\right)  $, $\operatorname*{st}%
\left(  \sigma\right)  $, $\left\vert \pi\right\vert $ and $\left\vert
\sigma\right\vert $.

\textbf{(b)} We say that $\operatorname*{st}$ is \textit{weakly
right-shuffle-compatible} if for any two disjoint nonempty permutations $\pi$
and $\sigma$ having the property that%
\[
\text{each entry of }\pi\text{ is greater than each entry of }\sigma,
\]
the multiset $\left\{  \operatorname*{st}\left(  \tau\right)  \ \mid\ \tau\in
S_{\succ}\left(  \pi,\sigma\right)  \right\}  _{\operatorname*{multi}}$
depends only on $\operatorname*{st}\left(  \pi\right)  $, $\operatorname*{st}%
\left(  \sigma\right)  $, $\left\vert \pi\right\vert $ and $\left\vert
\sigma\right\vert $.
\end{definition}

Then, the following analogues to the first part of Proposition
\ref{prop.K.ideal} hold:

\begin{theorem}
\label{thm.dendri.K.ideal}Let $\operatorname*{st}$ be a descent statistic.
Then, the following three statements are equivalent:

\begin{itemize}
\item \textit{Statement A:} The statistic $\operatorname*{st}$ is left-shuffle-compatible.

\item \textit{Statement B:} The statistic $\operatorname*{st}$ is weakly left-shuffle-compatible.

\item \textit{Statement C:} The set $\mathcal{K}_{\operatorname*{st}}$ is an
$\left.  \prec\right.  $-ideal of $\operatorname*{QSym}$.
\end{itemize}
\end{theorem}

\begin{theorem}
\label{thm.dendri.K.ideal-R}Let $\operatorname*{st}$ be a descent statistic.
Then, the following three statements are equivalent:

\begin{itemize}
\item \textit{Statement A:} The statistic $\operatorname*{st}$ is right-shuffle-compatible.

\item \textit{Statement B:} The statistic $\operatorname*{st}$ is weakly right-shuffle-compatible.

\item \textit{Statement C:} The set $\mathcal{K}_{\operatorname*{st}}$ is an
$\left.  \succeq\right.  $-ideal of $\operatorname*{QSym}$.
\end{itemize}
\end{theorem}

\begin{corollary}
\label{cor.dendri.K.ideal-LR}Let $\operatorname*{st}$ be a permutation
statistic that is LR-shuffle-compatible. Then, $\operatorname*{st}$ is a
shuffle-compatible descent statistic, and the set $\mathcal{K}%
_{\operatorname*{st}}$ is an ideal and a $\left.  \prec\right.  $-ideal and a
$\left.  \succeq\right.  $-ideal of $\operatorname*{QSym}$.
\end{corollary}

\begin{corollary}
\label{cor.dendri.K.ideal-LRi}Let $\operatorname*{st}$ be a descent statistic
such that $\mathcal{K}_{\operatorname*{st}}$ is a $\left.  \prec\right.
$-ideal and a $\left.  \succeq\right.  $-ideal of $\operatorname*{QSym}$.
Then, $\operatorname*{st}$ is LR-shuffle-compatible and shuffle-compatible.
\end{corollary}

Theorem \ref{thm.dendri.K.ideal} can (for example) be applied to
$\operatorname*{st}=\operatorname*{Epk}$, which we know to be
LR-shuffle-compatible (from Theorem \ref{thm.LRcomp.Pks} \textbf{(c)}); the
result is that $\mathcal{K}_{\operatorname*{Epk}}$ is an ideal and a $\left.
\prec\right.  $-ideal and a $\left.  \succeq\right.  $-ideal of
$\operatorname*{QSym}$. The same can be said about $\operatorname*{Des}$ and
$\operatorname*{Lpk}$ and some other statistics.

Combining Theorem \ref{thm.dendri.K.ideal} with Theorem
\ref{thm.dendri.K.ideal-R}, we can also see that any descent statistic that is
weakly left-shuffle-compatible and weakly right-shuffle-compatible must
automatically be shuffle-compatible (because any $\left.  \prec\right.
$-ideal of $\operatorname*{QSym}$ that is also a $\left.  \succeq\right.
$-ideal of $\operatorname*{QSym}$ is an ideal of $\operatorname*{QSym}$ as
well). Note that this is only true for descent statistics! As far as arbitrary
permutation statistics are concerned, this is false; for example, the number
of inversions is weakly left-shuffle-compatible and weakly
right-shuffle-compatible but not shuffle-compatible.

We can state an analogue of Theorem
\ref{thm.4.3} \textbf{(a)}. Let us first define the notion of dendriform algebras:

\begin{definition}
\textbf{(a)} A \textit{dendriform algebra} over a field $\mathbf{k}$ means a
$\mathbf{k}$-algebra $A$ equipped with two further $\mathbf{k}$-bilinear
binary operations $\left.  \prec\right.  $ and $\left.  \succeq\right.  $
(these are operations, not relations, despite the symbols) from $A\times A$ to
$A$ that satisfy the four rules%
\begin{align*}
a\left.  \prec\right.  b+a\left.  \succeq\right.  b  &  =ab;\\
\left(  a\left.  \prec\right.  b\right)  \left.  \prec\right.  c  &  =a\left.
\prec\right.  \left(  bc\right)  ;\\
\left(  a\left.  \succeq\right.  b\right)  \left.  \prec\right.  c  &
=a\left.  \succeq\right.  \left(  b\left.  \prec\right.  c\right)  ;\\
a\left.  \succeq\right.  \left(  b\left.  \succeq\right.  c\right)   &
=\left(  ab\right)  \left.  \succeq\right.  c
\end{align*}
for all $a,b,c\in A$. (Depending on the situation, it is useful to also impose
a few axioms that relate the unity $1$ of the $\mathbf{k}$-algebra $A$ with
the operations $\left.  \prec\right.  $ and $\left.  \succeq\right.  $. For
example, we could require $1\left.  \prec\right.  a=a$ for each $a\in A$. For
what we are going to do in the following, it does not matter whether we make
this requirement.)

\textbf{(b)} If $A$ and $B$ are two dendriform algebras over $\mathbf{k}$,
then a \textit{dendriform algebra homomorphism} from $A$ to $B$ means a
$\mathbf{k}$-algebra homomorphism $\phi:A\rightarrow B$ preserving the
operations $\left.  \prec\right.  $ and $\left.  \succeq\right.  $ (that is,
satisfying $\phi\left(  a\left.  \prec\right.  b\right)  =\phi\left(
a\right)  \left.  \prec\right.  \phi\left(  b\right)  $ and $\phi\left(
a\left.  \succeq\right.  b\right)  =\phi\left(  a\right)  \left.
\succeq\right.  \phi\left(  b\right)  $ for all $a,b\in A$). (Some authors
only require it to be a $\mathbf{k}$-linear map instead of being a
$\mathbf{k}$-algebra homomorphism; this boils down to the question whether
$\phi\left(  1\right)  $ must be $1$ or not. This does not make a difference
for us here.)
\end{definition}

Thus, $\operatorname*{QSym}$ (with its two operations $\left.  \prec\right.  $
and $\left.  \succeq\right.  $) becomes a dendriform algebra over $\mathbb{Q}$.

Notice that if $A$ and $B$ are two dendriform algebras over $\mathbf{k}$, then
the kernel of any dendriform algebra homomorphism $A\rightarrow B$ is an
$\left.  \prec\right.  $-ideal and a $\left.  \succeq\right.  $-ideal of $A$.
Conversely, if $A$ is a dendriform algebra over $\mathbf{k}$, and $I$ is
simultaneously an $\left.  \prec\right.  $-ideal and a $\left.  \succeq
\right.  $-ideal of $A$, then $A/I$ canonically becomes a dendriform algebra,
and the canonical projection $A\rightarrow A/I$ becomes a dendriform algebra homomorphism.

Therefore, Corollary \ref{cor.dendri.K.ideal-LR} (and the $\mathcal{A}%
_{\operatorname*{st}}\cong\operatorname*{QSym}/\mathcal{K}_{\operatorname*{st}%
}$ isomorphism from Proposition \ref{prop.K.ideal}) yields the following:

\begin{corollary}
If a descent statistic $\operatorname*{st}$ is LR-shuffle-compatible, then its
shuffle algebra $\mathcal{A}_{\operatorname*{st}}$ canonically becomes a
dendriform algebra.
\end{corollary}

We furthermore have the following analogue of Theorem~\ref{thm.4.3}, which
easily follows from Theorem \ref{thm.dendri.K.ideal} and Theorem
\ref{thm.dendri.K.ideal-R}:

\begin{theorem}
\label{thm.dendri.4.3}Let $\operatorname*{st}$ be a descent statistic.

\textbf{(a)} The descent statistic $\operatorname*{st}$ is
left-shuffle-compatible and right-shuffle-compatible if and only if there
exist a dendriform algebra $A$ with
basis $\left(  u_{\alpha}\right)  $ (indexed by $\operatorname*{st}%
$-equivalence classes $\alpha$ of compositions)
and a dendriform algebra homomorphism
$\phi_{\operatorname*{st}}:\operatorname*{QSym} \rightarrow A$
with the property that whenever $\alpha$ is an
$\operatorname{st}$-equivalence class of compositions, we have
\[
\phi_{\operatorname*{st}}\left(  F_{L}\right)  =u_{\alpha}%
\ \ \ \ \ \ \ \ \ \ \text{for each }L\in\alpha.
\]

\textbf{(b)} In this case, the $\mathbb{Q}$-linear map%
\[
\mathcal{A}_{\operatorname*{st}}\rightarrow A,\ \ \ \ \ \ \ \ \ \ \left[
\pi\right]  _{\operatorname*{st}}\mapsto u_{\alpha},
\]
where $\alpha$ is the $\operatorname*{st}$-equivalence class of the
composition $\operatorname*{Comp}\pi$, is an isomorphism of dendriform
algebras $\mathcal{A}_{\operatorname*{st}}\rightarrow A$.
\end{theorem}

\begin{question}
Can the $\mathbb{Q}$-algebra $\operatorname*{Pow}\mathcal{N}$ from Definition
\ref{def.GammaZ} be endowed with two binary operations $\left.  \prec\right.
$ and $\left.  \succeq\right.  $ that make it into a dendriform algebra? Can
we then find an analogue of Proposition \ref{prop.prod1} along the following lines?

Let $\left(  P,\gamma\right)  $, $\left(  Q,\delta\right)  $ and $\left(
P\sqcup Q,\varepsilon\right)  $ be as in Proposition \ref{prop.prod1}. Assume
that each of the posets $P$ and $Q$ has a (global) minimum element; denote these
elements by $\min P$ and $\min Q$, respectively.
Let $P \left. \prec \right. Q$ be the poset obtained by adding
the relation $\min P<\min Q$ to $P\sqcup Q$.
Let $P \left. \succ \right. Q$ be the poset obtained by adding
the relation $\min P>\min Q$ to $P\sqcup Q$.
Then, we hope to have%
\begin{align*}
\Gamma_{\mathcal{Z}}\left(  P,\gamma\right)  \left.  \prec\right.
\Gamma_{\mathcal{Z}}\left(  Q,\delta\right)   &  =\Gamma_{\mathcal{Z}}\left(
P\left.  \prec\right.  Q,\varepsilon\right)  \ \ \ \ \ \ \ \ \ \ \text{and}\\
\Gamma_{\mathcal{Z}}\left(  P,\gamma\right)  \left.  \succeq\right.
\Gamma_{\mathcal{Z}}\left(  Q,\delta\right)   &  =\Gamma_{\mathcal{Z}}\left(
P\left.  \succeq\right.  Q,\varepsilon\right)  .
\end{align*}
\end{question}
\end{vershort}

\begin{verlong}
Next, we shall study how the ideal $\mathcal{K}_{\operatorname*{Epk}}$
interacts with some additional structure on $\operatorname*{QSym}$, viz. the
\textit{dendriform operations }$\left.  \prec\right.  $ and $\left.
\succeq\right.  $ and the \textquotedblleft runic\textquotedblright%
\ operations $\bel$ and $\tvi$. These operations were introduced in
\cite{dimcr}. Our study shall lead us back to the notions of
left-shuffle-compatibility\ and right-shuffle-compatibility from Section
\ref{sect.LR}. We shall reprove that $\operatorname*{Epk}$ is
left-shuffle-compatible and right-shuffle-compatible; similar studies can
probably be made for other descent statistics.

\subsection{Four operations on $\operatorname*{QSym}$}

We begin with some definitions. We will use some notations from \cite{dimcr},
but we set $\mathbf{k}=\mathbb{Q}$ because we are working over the ring
$\mathbb{Q}$ in this paper. Monomials always mean formal expressions of the
form $x_{1}^{a_{1}}x_{2}^{a_{2}}x_{3}^{a_{3}}\cdots$ with $a_{1}+a_{2}%
+a_{3}+\cdots<\infty$ (see \cite[Section 2]{dimcr} for details). If
$\mathfrak{m}$ is a monomial, then $\operatorname*{Supp}\mathfrak{m}$ will
denote the finite subset
\[
\left\{  i\in\left\{  1,2,3,\ldots\right\}  \ \mid\ \text{the exponent with
which }x_{i}\text{ occurs in }\mathfrak{m}\text{ is }>0\right\}
\]
of $\left\{  1,2,3,\ldots\right\}  $. Next, we define four binary operations
\begin{align*}
&  \left.  \prec\right.  \ \left(  \text{called \textquotedblleft dendriform
less-than\textquotedblright; but it's an operation, not a relation}\right)
,\\
&  \left.  \succeq\right.  \ \left(  \text{called \textquotedblleft dendriform
greater-or-equal\textquotedblright; but it's an operation, not a
relation}\right)  ,\\
&  \bel\ \left(  \text{called \textquotedblleft belgthor\textquotedblright%
}\right)  ,\\
&  \tvi\ \left(  \text{called \textquotedblleft tvimadur\textquotedblright%
}\right)
\end{align*}
on the ring $\mathbf{k}\left[  \left[  x_{1},x_{2},x_{3},\ldots\right]
\right]  $ of power series by first defining how they act on monomials:%
\begin{align*}
\mathfrak{m}\left.  \prec\right.  \mathfrak{n}  &  =\left\{
\begin{array}
[c]{l}%
\mathfrak{m}\cdot\mathfrak{n},\ \ \ \ \ \ \ \ \ \ \text{if }\min\left(
\operatorname*{Supp}\mathfrak{m}\right)  <\min\left(  \operatorname*{Supp}%
\mathfrak{n}\right)  ;\\
0,\ \ \ \ \ \ \ \ \ \ \text{if }\min\left(  \operatorname*{Supp}%
\mathfrak{m}\right)  \geq\min\left(  \operatorname*{Supp}\mathfrak{n}\right)
\end{array}
\right.  ;\\
\mathfrak{m}\left.  \succeq\right.  \mathfrak{n}  &  =\left\{
\begin{array}
[c]{l}%
\mathfrak{m}\cdot\mathfrak{n},\ \ \ \ \ \ \ \ \ \ \text{if }\min\left(
\operatorname*{Supp}\mathfrak{m}\right)  \geq\min\left(  \operatorname*{Supp}%
\mathfrak{n}\right)  ;\\
0,\ \ \ \ \ \ \ \ \ \ \text{if }\min\left(  \operatorname*{Supp}%
\mathfrak{m}\right)  <\min\left(  \operatorname*{Supp}\mathfrak{n}\right)
\end{array}
\right.  ;\\
\mathfrak{m}\bel\mathfrak{n}  &  =\left\{
\begin{array}
[c]{l}%
\mathfrak{m}\cdot\mathfrak{n},\ \ \ \ \ \ \ \ \ \ \text{if }\max\left(
\operatorname*{Supp}\mathfrak{m}\right)  \leq\min\left(  \operatorname*{Supp}%
\mathfrak{n}\right)  ;\\
0,\ \ \ \ \ \ \ \ \ \ \text{if }\max\left(  \operatorname*{Supp}%
\mathfrak{m}\right)  >\min\left(  \operatorname*{Supp}\mathfrak{n}\right)
\end{array}
\right.  ;\\
\mathfrak{m}\tvi\mathfrak{n}  &  =\left\{
\begin{array}
[c]{l}%
\mathfrak{m}\cdot\mathfrak{n},\ \ \ \ \ \ \ \ \ \ \text{if }\max\left(
\operatorname*{Supp}\mathfrak{m}\right)  <\min\left(  \operatorname*{Supp}%
\mathfrak{n}\right)  ;\\
0,\ \ \ \ \ \ \ \ \ \ \text{if }\max\left(  \operatorname*{Supp}%
\mathfrak{m}\right)  \geq\min\left(  \operatorname*{Supp}\mathfrak{n}\right)
\end{array}
\right.  ;
\end{align*}
and then requiring that they all be $\mathbf{k}$-bilinear and continuous (so
their action on pairs of arbitrary power series can be computed by
\textquotedblleft opening the parentheses\textquotedblright). These operations
$\left.  \prec\right.  $, $\left.  \succeq\right.  $, $\bel$ and $\tvi$ all
restrict to the subset $\operatorname*{QSym}$ of $\mathbf{k}\left[  \left[
x_{1},x_{2},x_{3},\ldots\right]  \right]  $ (this is proven in \cite[detailed
version, Section 3]{dimcr}). They furthermore satisfy numerous
relations\footnote{These relations are all easy to prove (by linearity, it
suffices to verify them on monomials only, and this verification is
straightforward). A proof of the associativity of $\bel$ was given in
\cite[detailed version, Proposition 3.4]{dimcr}.}:

\begin{itemize}
\item The dendriform operations satisfy the four rules
\begin{align}
a\left.  \prec\right.  b+a\left.  \succeq\right.  b  &
=ab;\label{eq.dendriform.1}\\
\left(  a\left.  \prec\right.  b\right)  \left.  \prec\right.  c  &  =a\left.
\prec\right.  \left(  bc\right)  ;\nonumber\\
\left(  a\left.  \succeq\right.  b\right)  \left.  \prec\right.  c  &
=a\left.  \succeq\right.  \left(  b\left.  \prec\right.  c\right)
;\nonumber\\
a\left.  \succeq\right.  \left(  b\left.  \succeq\right.  c\right)   &
=\left(  ab\right)  \left.  \succeq\right.  c\nonumber
\end{align}
for all $a,b,c\in\mathbf{k}\left[  \left[  x_{1},x_{2},x_{3},\ldots\right]
\right]  $. (In other words, they turn $\mathbf{k}\left[  \left[  x_{1}%
,x_{2},x_{3},\ldots\right]  \right]  $ into what is called a dendriform algebra.)

\item For any $a\in\mathbf{k}\left[  \left[  x_{1},x_{2},x_{3},\ldots\right]
\right]  $, we have%
\begin{align}
1\left.  \prec\right.  a  &  =0;\label{eq.dendriform.1<a}\\
a\left.  \prec\right.  1  &  =a-\varepsilon\left(  a\right)
;\label{eq.dendriform.a<1}\\
1\left.  \succeq\right.  a  &  =a;\label{eq.dendriform.1>=a}\\
a\left.  \succeq\right.  1  &  =\varepsilon\left(  a\right)  ,
\label{eq.dendriform.a>=1}%
\end{align}
where $\varepsilon\left(  a\right)  $ denotes the constant term of the power
series $a$.

\item The binary operation $\bel$ is associative and unital (with $1$ serving
as the unity).

\item The binary operation $\tvi$ is associative and unital (with $1$ serving
as the unity).
\end{itemize}

Recall that we are using the notations $M_{\alpha}$ for the monomial
quasisymmetric functions and $F_{\alpha}$ for the fundamental quasisymmetric functions.

\begin{itemize}
\item For any two nonempty compositions $\alpha$ and $\beta$, we have
$M_{\alpha}\bel  M_{\beta}=M_{\left[  \alpha,\beta\right]  }+M_{\alpha
\odot\beta}$, where $\left[  \alpha,\beta\right]  $ and $\alpha\odot\beta$ are
two compositions defined by%
\begin{align*}
\left[  \left(  \alpha_{1},\alpha_{2},\ldots,\alpha_{\ell}\right)  ,\left(
\beta_{1},\beta_{2},\ldots,\beta_{m}\right)  \right]   &  =\left(  \alpha
_{1},\alpha_{2},\ldots,\alpha_{\ell},\beta_{1},\beta_{2},\ldots,\beta
_{m}\right)  ;\\
\left(  \alpha_{1},\alpha_{2},\ldots,\alpha_{\ell}\right)  \odot\left(
\beta_{1},\beta_{2},\ldots,\beta_{m}\right)   &  =\left(  \alpha_{1}%
,\alpha_{2},\ldots,\alpha_{\ell-1},\alpha_{\ell}+\beta_{1},\beta_{2},\beta
_{3},\ldots,\beta_{m}\right)  .
\end{align*}


\item For any two compositions $\alpha$ and $\beta$, we have $M_{\alpha
}\tvi  M_{\beta}=M_{\left[  \alpha,\beta\right]  }$.

\item For any two compositions $\alpha$ and $\beta$, we have $F_{\alpha
}\bel  F_{\beta}=F_{\alpha\odot\beta}$. (Here, $\alpha\odot\beta$ is defined
to be $\alpha$ if $\beta$ is the empty composition, and is defined to be
$\beta$ if $\alpha$ is the empty composition.)

\item For any two compositions $\alpha$ and $\beta$, we have $F_{\alpha
}\tvi  F_{\beta}=F_{\left[  \alpha,\beta\right]  }$.
\end{itemize}

Furthermore, we shall use two theorems from \cite[detailed version, Section
3]{dimcr}:

\begin{theorem}
\label{thm.beldend}Let $S$ denote the antipode of the Hopf algebra
$\operatorname*{QSym}$. Let us use Sweedler's notation $\sum_{\left(
b\right)  }b_{\left(  1\right)  }\otimes b_{\left(  2\right)  }$ for
$\Delta\left(  b\right)  $, where $b$ is any element of $\operatorname*{QSym}%
$. Then,%
\[
\sum_{\left(  b\right)  }\left(  S\left(  b_{\left(  1\right)  }\right)
\bel  a\right)  b_{\left(  2\right)  }=a\left.  \prec\right.  b
\]
for any $a\in\mathbf{k}\left[  \left[  x_{1},x_{2},x_{3},\ldots\right]
\right]  $ and $b\in\operatorname*{QSym}$.
\end{theorem}

\begin{theorem}
\label{thm.tvidend'}Let $S$ denote the antipode of the Hopf algebra
$\operatorname*{QSym}$. Let us use Sweedler's notation $\sum_{\left(
b\right)  }b_{\left(  1\right)  }\otimes b_{\left(  2\right)  }$ for
$\Delta\left(  b\right)  $, where $b$ is any element of $\operatorname*{QSym}%
$. Then,%
\[
\sum_{\left(  b\right)  }\left(  S\left(  b_{\left(  1\right)  }\right)
\tvi  a\right)  b_{\left(  2\right)  }=b\left.  \succeq\right.  a
\]
for any $a\in\mathbf{k}\left[  \left[  x_{1},x_{2},x_{3},\ldots\right]
\right]  $ and $b\in\operatorname*{QSym}$.
\end{theorem}

(Notice that Theorem \ref{thm.tvidend'} differs from \cite[detailed version,
Theorem 3.15]{dimcr} in that we are writing $b\left.  \succeq\right.  a$
instead of $a\left.  \preceq\right.  b$. But this is the same thing, since
$a\left.  \preceq\right.  b=b\left.  \succeq\right.  a$ for all $a,b\in
\mathbf{k}\left[  \left[  x_{1},x_{2},x_{3},\ldots\right]  \right]  $.)

\begin{noncompile}
The following fact is neat, but has so far been useless:

\begin{proposition}
\label{prop.tvibel.antipode}Let $S$ denote the antipode of the Hopf algebra
$\operatorname*{QSym}$. Let $a\in\operatorname*{QSym}$ and $b\in
\operatorname*{QSym}$. Then,%
\begin{align*}
S\left(  a\tvi b\right)   &  =S\left(  b\right)  \bel S\left(  a\right)
\ \ \ \ \ \ \ \ \ \ \text{and}\\
S\left(  a\bel b\right)   &  =S\left(  b\right)  \tvi S\left(  a\right)  .
\end{align*}

\end{proposition}
\end{noncompile}

\begin{noncompile}
Let us restate Theorem \ref{thm.beldend} and Theorem \ref{thm.tvidend'} using
a slight variation on the comultiplication map $\Delta$:

\begin{theorem}
\label{thm.beldend.Del'}Let $S$ denote the antipode of the Hopf algebra
$\operatorname*{QSym}$. Let us use the Sweedler-like notation $\sum_{\left(
b\right)  }b_{\left[  1\right]  }\otimes b_{\left[  2\right]  }$ for
$\Delta\left(  b\right)  -1\otimes b$, where $b$ is any element of
$\operatorname*{QSym}$. Then,%
\[
\sum_{\left(  b\right)  }\left(  S\left(  b_{\left[  1\right]  }\right)
\bel
a\right)  b_{\left[  2\right]  }=-a\mathfrak{\left.  \succeq\right.  }b
\]
for any $a\in\mathbf{k}\left[  \left[  x_{1},x_{2},x_{3},\ldots\right]
\right]  $ and $b\in\operatorname*{QSym}$.
\end{theorem}

\begin{theorem}
\label{thm.tvidend'.Del'}Let $S$ denote the antipode of the Hopf algebra
$\operatorname*{QSym}$. Let us use the Sweedler-like notation $\sum_{\left(
b\right)  }b_{\left[  1\right]  }\otimes b_{\left[  2\right]  }$ for
$\Delta\left(  b\right)  -1\otimes b$, where $b$ is any element of
$\operatorname*{QSym}$. Then,%
\[
\sum_{\left(  b\right)  }\left(  S\left(  b_{\left[  1\right]  }\right)
\tvi
a\right)  b_{\left[  2\right]  }=-b\left.  \prec\right.  a
\]
for any $a\in\mathbf{k}\left[  \left[  x_{1},x_{2},x_{3},\ldots\right]
\right]  $ and $b\in\operatorname*{QSym}$.
\end{theorem}

\begin{proof}
[Proof of Theorem \ref{thm.beldend.Del'}.]Let us use Sweedler's notation, too.
From $\sum_{\left(  b\right)  }b_{\left[  1\right]  }\otimes b_{\left[
2\right]  }=\Delta\left(  b\right)  -1\otimes b=\sum_{\left(  b\right)
}b_{\left(  1\right)  }\otimes b_{\left(  2\right)  }-1\otimes b$, we have%
\begin{align*}
\sum_{\left(  b\right)  }\left(  S\left(  b_{\left[  1\right]  }\right)
\bel a\right)  b_{\left[  2\right]  }  &  =\underbrace{\sum_{\left(  b\right)
}\left(  S\left(  b_{\left(  1\right)  }\right)  \bel a\right)  b_{\left(
2\right)  }}_{\substack{=a\left.  \prec\right.  b\\\text{(by Theorem
\ref{thm.beldend})}}}-\underbrace{\left(  1\bel a\right)  }_{=a}b\\
&  =a\left.  \prec\right.  b-ab=-a\left.  \succeq\right.  b
\end{align*}
(since $a\left.  \prec\right.  b+a\left.  \succeq\right.  b=ab$). This proves
Theorem \ref{thm.beldend.Del'}.
\end{proof}

\begin{proof}
[Proof of Theorem \ref{thm.tvidend'.Del'}.]Analogous.
\end{proof}
\end{noncompile}

\subsection{Ideals}

\begin{definition}
Let $A$ be a $\mathbf{k}$-module equipped with some binary operation $\ast$
(written infix).

\textbf{(a)} If $B$ and $C$ are two $\mathbf{k}$-submodules of $A$, then
$B\ast C$ shall mean the $\mathbf{k}$-submodule of $A$ spanned by all elements
of the form $b\ast c$ with $b\in B$ and $c\in C$.

\textbf{(b)} A $\mathbf{k}$-submodule $M$ of $A$ is said to be a \textit{left
}$\ast$\textit{-ideal} if and only if it satisfies $A\ast M\subseteq M$.

\textbf{(c)} A $\mathbf{k}$-submodule $M$ of $A$ is said to be a \textit{right
}$\ast$\textit{-ideal} if and only if it satisfies $M\ast A\subseteq M$.

\textbf{(d)} A $\mathbf{k}$-submodule $M$ of $A$ is said to be a\textit{
}$\ast$\textit{-ideal} if and only if it is both a left $\ast$-ideal and a
right $\ast$-ideal.
\end{definition}

\begin{theorem}
\label{thm.ideal-crit2}Let $M$ be an ideal of $\operatorname*{QSym}$. Let
$A=\operatorname*{QSym}$.

\textbf{(a)} If $A\bel  M\subseteq M$, then $M\left.  \prec\right.  A\subseteq
M$.

\textbf{(b)} If $A\tvi  M\subseteq M$, then $A\left.  \succeq\right.
M\subseteq M$.

\textbf{(c)} If $A\tvi  M\subseteq M$ and $A\bel  M\subseteq M$, then $M$ is a
$\left.  \prec\right.  $-ideal and a $\left.  \succeq\right.  $-ideal of
$\operatorname*{QSym}$.
\end{theorem}

\begin{proof}
[Proof of Theorem \ref{thm.ideal-crit2}.]\textbf{(a)} Assume that
$A\bel  M\subseteq M$. If $a\in M$ and $b\in A$, then%
\begin{align*}
a\left.  \prec\right.  b  &  =\sum_{\left(  b\right)  }\left(
\underbrace{S\left(  b_{\left(  1\right)  }\right)  }_{\in A}%
\bel \underbrace{a}_{\in M}\right)  \underbrace{b_{\left(  2\right)  }}_{\in
A}\ \ \ \ \ \ \ \ \ \ \left(  \text{by Theorem \ref{thm.beldend}}\right) \\
&  \in\underbrace{\left(  A\bel M\right)  }_{\subseteq M}A\subseteq
MA\subseteq M\ \ \ \ \ \ \ \ \ \ \left(  \text{since }M\text{ is an ideal of
}A\right)  .
\end{align*}
Thus, $M\left.  \prec\right.  A\subseteq M$. This proves Theorem
\ref{thm.ideal-crit2} \textbf{(a)}.

\textbf{(b)} Assume that $A\tvi  M\subseteq M$. If $a\in M$ and $b\in A$, then%
\begin{align*}
b\left.  \succeq\right.  a  &  =\sum_{\left(  b\right)  }\left(
\underbrace{S\left(  b_{\left(  1\right)  }\right)  }_{\in A}%
\tvi \underbrace{a}_{\in M}\right)  \underbrace{b_{\left(  2\right)  }}_{\in
A}\ \ \ \ \ \ \ \ \ \ \left(  \text{by Theorem \ref{thm.tvidend'}}\right) \\
&  \in\underbrace{\left(  A\tvi M\right)  }_{\subseteq M}A\subseteq
MA\subseteq M\ \ \ \ \ \ \ \ \ \ \left(  \text{since }M\text{ is an ideal of
}A\right)  .
\end{align*}
Thus, $A\left.  \succeq\right.  M\subseteq M$. This proves Theorem
\ref{thm.ideal-crit2} \textbf{(b)}.

\textbf{(c)} Assume that $A\tvi  M\subseteq M$ and $A\bel  M\subseteq M$.
Then, Theorem \ref{thm.ideal-crit2} \textbf{(b)} yields $A\left.
\succeq\right.  M\subseteq M$. Thus, $M$ is a left $\left.  \succeq\right.  $-ideal.

Now, any $b\in M$ and $a\in A$ satisfy%
\begin{align*}
a\left.  \prec\right.  b  &  =\underbrace{a}_{\in A}\underbrace{b}_{\in
M}-\underbrace{a}_{\in A}\left.  \succeq\right.  \underbrace{b}_{\in
M}\ \ \ \ \ \ \ \ \ \ \left(  \text{by (\ref{eq.dendriform.1})}\right) \\
&  \in\underbrace{AM}_{\substack{\subseteq M\\\text{(since }M\text{ is an
ideal of }A\text{)}}}-\underbrace{A\left.  \succeq\right.  M}_{\subseteq
M}\subseteq M-M\subseteq M.
\end{align*}
In other words, $A\left.  \prec\right.  M\subseteq M$. In other words, $M$ is
a left $\left.  \prec\right.  $-ideal.

But Theorem \ref{thm.ideal-crit2} \textbf{(a)} yields $M\left.  \prec\right.
A\subseteq M$. In other words, $M$ is a right $\left.  \prec\right.  $-ideal.

Any $a\in M$ and $b\in A$ satisfy%
\begin{align*}
a\left.  \succeq\right.  b  &  =\underbrace{a}_{\in M}\underbrace{b}_{\in
A}-\underbrace{a}_{\in M}\left.  \prec\right.  \underbrace{b}_{\in
A}\ \ \ \ \ \ \ \ \ \ \left(  \text{by (\ref{eq.dendriform.1})}\right) \\
&  \in\underbrace{MA}_{\substack{\subseteq M\\\text{(since }M\text{ is an
ideal of }A\text{)}}}-\underbrace{M\left.  \prec\right.  A}_{\subseteq
M}\subseteq M-M\subseteq M.
\end{align*}
In other words, $M\left.  \succeq\right.  A\subseteq M$. In other words, $M$
is a right $\left.  \succeq\right.  $-ideal.

Hence, $M$ is a $\left.  \prec\right.  $-ideal (since $M$ is a left $\left.
\prec\right.  $-ideal and a right $\left.  \prec\right.  $-ideal) and a
$\left.  \succeq\right.  $-ideal (since $M$ is a left $\left.  \succeq\right.
$-ideal and a right $\left.  \succeq\right.  $-ideal). This proves Theorem
\ref{thm.ideal-crit2} \textbf{(c)}.
\end{proof}

Another simple fact is the following:

\begin{proposition}
\label{prop.ideal-crit3}Let $M$ be simultaneously a $\left.  \prec\right.
$-ideal and a $\left.  \succeq\right.  $-ideal of $\operatorname*{QSym}$.
Then, $M$ is an ideal of $\operatorname*{QSym}$.
\end{proposition}

\begin{proof}
[Proof of Proposition \ref{prop.ideal-crit3}.]Any $a\in M$ and $b\in
\operatorname*{QSym}$ satisfy%
\begin{align*}
ab  &  =\underbrace{a}_{\in M}\left.  \prec\right.  \underbrace{b}%
_{\in\operatorname*{QSym}}+\underbrace{a}_{\in M}\left.  \succeq\right.
\underbrace{b}_{\in\operatorname*{QSym}}\ \ \ \ \ \ \ \ \ \ \left(  \text{by
(\ref{eq.dendriform.1})}\right) \\
&  \in\underbrace{M\left.  \prec\right.  \operatorname*{QSym}}%
_{\substack{\subseteq M\\\text{(since }M\text{ is a }\left.  \prec\right.
\text{-ideal of }\operatorname*{QSym}\text{)}}}+\underbrace{M\left.
\succeq\right.  \operatorname*{QSym}}_{\substack{\subseteq M\\\text{(since
}M\text{ is a }\left.  \succeq\right.  \text{-ideal of }\operatorname*{QSym}%
\text{)}}}\subseteq M+M\subseteq M.
\end{align*}
In other words, $M$ is an ideal of $\operatorname*{QSym}$. This proves
Proposition \ref{prop.ideal-crit3}.
\end{proof}

\begin{question}
Proposition \ref{prop.ideal-crit3} says that if a $\mathbb{Q}$-vector subspace
$M$ of $\operatorname*{QSym}$ is simultaneously a $\left.  \prec\right.
$-ideal and an $\left.  \succeq\right.  $-ideal, then it is also an ideal.
Similarly, if $M$ is an ideal and a $\left.  \prec\right.  $-ideal, then it is
a $\left.  \succeq\right.  $-ideal. Can we state any other such criteria?
\end{question}

\subsection{Application to $\mathcal{K}_{\operatorname*{Epk}}$}

We now claim the following:

\begin{theorem}
\label{thm.Epk.dend}The ideal $\mathcal{K}_{\operatorname*{Epk}}$ of
$\operatorname*{QSym}$ is a $\tvi  $-ideal, a $\bel  $-ideal, a $\left.
\prec\right.  $-ideal and a $\left.  \succeq\right.  $-ideal of
$\operatorname*{QSym}$.
\end{theorem}

\begin{proof}
[Proof of Theorem \ref{thm.Epk.dend}.]Let $A=\operatorname*{QSym}$. Corollary
\ref{cor.Epk.ideal} shows that $\mathcal{K}_{\operatorname*{Epk}}$ is an ideal
of $\operatorname*{QSym}$.

Let us recall the binary relation $\rightarrow$ on the set of compositions
defined in Proposition \ref{prop.K.Epk.F}.

\begin{statement}
\textit{Claim 1:} Let $J$ and $K$ be two compositions satisfying $J\rightarrow
K$. Let $G$ be a further composition. Then, $\left[  G,J\right]
\rightarrow\left[  G,K\right]  $.
\end{statement}

[\textit{Proof of Claim 1:} Write the composition $J$ in the form $J=\left(
j_{1},j_{2},\ldots,j_{m}\right)  $. Write the composition $G$ in the form
$G=\left(  g_{1},g_{2},\ldots,g_{p}\right)  $.

We have $J\rightarrow K$. In other words, there exists an $\ell\in\left\{
2,3,\ldots,m\right\}  $ such that $j_{\ell}>2$ and $K=\left(  j_{1}%
,j_{2},\ldots,j_{\ell-1},1,j_{\ell}-1,j_{\ell+1},j_{\ell+2},\ldots
,j_{m}\right)  $ (by the definition of the relation $\rightarrow$). Consider
this $\ell$. Clearly, $\ell>1$ (since $\ell\in\left\{  2,3,\ldots,m\right\}
$), so that $p+\ell>\underbrace{p}_{\geq0}+1\geq1$.

From $G=\left(  g_{1},g_{2},\ldots,g_{p}\right)  $ and $J=\left(  j_{1}%
,j_{2},\ldots,j_{m}\right)  $, we obtain%
\begin{equation}
\left[  G,J\right]  =\left(  g_{1},g_{2},\ldots,g_{p},j_{1},j_{2},\ldots
,j_{m}\right)  . \label{pf.thm.Epk.dend.c1.pf.GJ=}%
\end{equation}
From $G=\left(  g_{1},g_{2},\ldots,g_{p}\right)  $ and $K=\left(  j_{1}%
,j_{2},\ldots,j_{\ell-1},1,j_{\ell}-1,j_{\ell+1},j_{\ell+2},\ldots
,j_{m}\right)  $, we obtain%
\begin{equation}
\left[  G,K\right]  =\left(  g_{1},g_{2},\ldots,g_{p},j_{1},j_{2}%
,\ldots,j_{\ell-1},1,j_{\ell}-1,j_{\ell+1},j_{\ell+2},\ldots,j_{m}\right)  .
\label{pf.thm.Epk.dend.c1.pf.GK=}%
\end{equation}
From looking at (\ref{pf.thm.Epk.dend.c1.pf.GJ=}) and
(\ref{pf.thm.Epk.dend.c1.pf.GK=}), we conclude immediately that the
composition $\left[  G,K\right]  $ is obtained from $\left[  G,J\right]  $ by
\textquotedblleft splitting\textquotedblright\ the entry $j_{\ell}>2$ into two
consecutive entries $1$ and $j_{\ell}-1$, and that this entry $j_{\ell}$ was
not the first entry (indeed, this entry is the $\left(  p+\ell\right)  $-th
entry, but $p+\ell>1$). Hence, $\left[  G,J\right]  \rightarrow\left[
G,K\right]  $ (by the definition of the relation $\rightarrow$). This proves
Claim 1.]

\begin{statement}
\textit{Claim 2:} We have $A\tvi  \mathcal{K}_{\operatorname*{Epk}}%
\subseteq\mathcal{K}_{\operatorname*{Epk}}$.
\end{statement}

[\textit{Proof of Claim 2:} We must show that $a\tvi  m\in\mathcal{K}%
_{\operatorname*{Epk}}$ for every $a\in A$ and $m\in\mathcal{K}%
_{\operatorname*{Epk}}$. So let us fix $a\in A$ and $m\in\mathcal{K}%
_{\operatorname*{Epk}}$.

Proposition \ref{prop.K.Epk.F} shows that the $\mathbb{Q}$-vector space
$\mathcal{K}_{\operatorname*{Epk}}$ is spanned by all differences of the form
$F_{J}-F_{K}$, where $J$ and $K$ are two compositions satisfying $J\rightarrow
K$. Hence, we can WLOG assume that $m$ is such a difference (because the
relation $a\tvi  m\in\mathcal{K}_{\operatorname*{Epk}}$, which we must prove,
is $\mathbb{Q}$-linear in $m$). Assume this. Thus, $m=F_{J}-F_{K}$ for some
two compositions $J$ and $K$ satisfying $J\rightarrow K$. Consider these $J$
and $K$.

From $J\rightarrow K$, we easily conclude that the composition $J$ is
nonempty. Thus, $\left\vert J\right\vert \neq0$. But from $J\rightarrow K$, we
also obtain $\left\vert J\right\vert =\left\vert K\right\vert $. Hence,
$\left\vert K\right\vert =\left\vert J\right\vert \neq0$. Thus, the
composition $K$ is nonempty.

Recall that the family $\left(  F_{L}\right)  _{L\text{ is a composition}}$ is
a basis of the $\mathbb{Q}$-vector space $\operatorname*{QSym}=A$. Hence, we
can WLOG assume that $a$ belongs to this family (since the relation
$a\tvi  m\in\mathcal{K}_{\operatorname*{Epk}}$, which we must prove, is
$\mathbb{Q}$-linear in $a$). Assume this. Thus, $a=F_{G}$ for some composition
$G$. Consider this $G$.

If $G$ is the empty composition, then $a=F_{G}=1$, and therefore
$\underbrace{a}_{=1}\tvi  m=1\tvi  m=m\in\mathcal{K}_{\operatorname*{Epk}}$
holds. Thus, for the rest of this proof, we WLOG assume that $G$ is not the
empty composition. Thus, $G$ is nonempty.

Recall that for any two compositions $\alpha$ and $\beta$, we have $F_{\alpha
}\tvi  F_{\beta}=F_{\left[  \alpha,\beta\right]  }$. Applying this to
$\alpha=G$ and $\beta=J$, we obtain $F_{G}\tvi  F_{J}=F_{\left[  G,J\right]
}$. Similarly, $F_{G}\tvi  F_{K}=F_{\left[  G,K\right]  }$.

But Claim 1 yields $\left[  G,J\right]  \rightarrow\left[  G,K\right]  $.
Hence, the difference $F_{\left[  G,J\right]  }-F_{\left[  G,K\right]  }$ is
one of the differences which span the ideal $\mathcal{K}_{\operatorname*{Epk}%
}$ according to Proposition \ref{prop.K.Epk.F}. Thus, in particular, this
difference lies in $\mathcal{K}_{\operatorname*{Epk}}$. In other words,
$F_{\left[  G,J\right]  }-F_{\left[  G,K\right]  }\in\mathcal{K}%
_{\operatorname*{Epk}}$.

Now,%
\begin{align*}
\underbrace{a}_{=F_{G}}\tvi \underbrace{m}_{=F_{J}-F_{K}}  &  =F_{G}%
\tvi \left(  F_{J}-F_{K}\right)  =\underbrace{F_{G}\tvi F_{J}}_{=F_{\left[
G,J\right]  }}-\underbrace{F_{G}\tvi F_{K}}_{=F_{\left[  G,K\right]  }}\\
&  =F_{\left[  G,J\right]  }-F_{\left[  G,K\right]  }\in\mathcal{K}%
_{\operatorname*{Epk}}.
\end{align*}
This proves Claim 2.]

\begin{statement}
\textit{Claim 3:} Let $J$ and $K$ be two compositions satisfying $J\rightarrow
K$. Let $G$ be a further composition. Then, $\left[  J,G\right]
\rightarrow\left[  K,G\right]  $.
\end{statement}

[\textit{Proof of Claim 3:} This is proven in the same way as we proved Claim
1, with the only difference that $j_{\ell}$ is now the $\ell$-th entry of
$\left[  J,G\right]  $ and not the $\left(  p+\ell\right)  $-th entry (but
this is still sufficient, since $\ell>1$).]

\begin{statement}
\textit{Claim 4:} We have $\mathcal{K}_{\operatorname*{Epk}}\tvi  A\subseteq
\mathcal{K}_{\operatorname*{Epk}}$.
\end{statement}

[\textit{Proof of Claim 4:} This is proven in the same way as we proved Claim
2, with the only difference that now we need to use Claim 3 instead of Claim 1.]

Combining Claim 2 and Claim 4, we conclude that $\mathcal{K}%
_{\operatorname*{Epk}}$ is a $\tvi  $-ideal of $A=\operatorname*{QSym}$.

\begin{statement}
\textit{Claim 5:} Let $J$ and $K$ be two nonempty compositions satisfying
$J\rightarrow K$. Let $G$ be a further nonempty composition. Then, $G\odot
J\rightarrow G\odot K$.
\end{statement}

[\textit{Proof of Claim 5:} Write the composition $J$ in the form $J=\left(
j_{1},j_{2},\ldots,j_{m}\right)  $. Write the composition $G$ in the form
$G=\left(  g_{1},g_{2},\ldots,g_{p}\right)  $. Thus, $p>0$ (since the
composition $G$ is nonempty).

We have $J\rightarrow K$. In other words, there exists an $\ell\in\left\{
2,3,\ldots,m\right\}  $ such that $j_{\ell}>2$ and $K=\left(  j_{1}%
,j_{2},\ldots,j_{\ell-1},1,j_{\ell}-1,j_{\ell+1},j_{\ell+2},\ldots
,j_{m}\right)  $ (by the definition of the relation $\rightarrow$). Consider
this $\ell$. Clearly, $\ell\geq2$ (since $\ell\in\left\{  2,3,\ldots
,m\right\}  $), so that $\underbrace{p}_{>0}+\underbrace{\ell}_{\geq
2}-1>0+2-1=1$.

From $G=\left(  g_{1},g_{2},\ldots,g_{p}\right)  $ and $J=\left(  j_{1}%
,j_{2},\ldots,j_{m}\right)  $, we obtain%
\begin{equation}
G\odot J=\left(  g_{1},g_{2},\ldots,g_{p-1},g_{p}+j_{1},j_{2},j_{3}%
,\ldots,j_{m}\right)  . \label{pf.thm.Epk.dend.c5.pf.GJ=}%
\end{equation}
From $G=\left(  g_{1},g_{2},\ldots,g_{p}\right)  $ and $K=\left(  j_{1}%
,j_{2},\ldots,j_{\ell-1},1,j_{\ell}-1,j_{\ell+1},j_{\ell+2},\ldots
,j_{m}\right)  $, we obtain%
\begin{equation}
G\odot K=\left(  g_{1},g_{2},\ldots,g_{p-1},g_{p}+j_{1},j_{2},j_{3}%
,\ldots,j_{\ell-1},1,j_{\ell}-1,j_{\ell+1},j_{\ell+2},\ldots,j_{m}\right)
\label{pf.thm.Epk.dend.c5.pf.GK=}%
\end{equation}
(notice that the $g_{p}+j_{1}$ term is \textbf{not} a $g_{p}+1$ term, because
$\ell\geq2$).

From looking at (\ref{pf.thm.Epk.dend.c5.pf.GJ=}) and
(\ref{pf.thm.Epk.dend.c5.pf.GK=}), we conclude immediately that the
composition $G\odot K$ is obtained from $G\odot J$ by \textquotedblleft
splitting\textquotedblright\ the entry $j_{\ell}>2$ into two consecutive
entries $1$ and $j_{\ell}-1$, and that this entry $j_{\ell}$ was not the first
entry (indeed, this entry is the $\left(  p+\ell-1\right)  $-th entry, but
$p+\ell-1>1$). Hence, $G\odot J\rightarrow G\odot K$ (by the definition of the
relation $\rightarrow$). This proves Claim 5.]

\begin{statement}
\textit{Claim 6:} We have $A\bel\mathcal{K}_{\operatorname*{Epk}}%
\subseteq\mathcal{K}_{\operatorname*{Epk}}$.
\end{statement}

[\textit{Proof of Claim 6:} This is proven in the same way as we proved Claim
2, with the only difference that now we need to use Claim 5 instead of Claim 1
and that we need to use the formula $F_{\alpha}\bel  F_{\beta}=F_{\alpha
\odot\beta}$ instead of $F_{\alpha}\tvi  F_{\beta}=F_{\left[  \alpha
,\beta\right]  }$.]

\begin{statement}
\textit{Claim 7:} Let $J$ and $K$ be two nonempty compositions satisfying
$J\rightarrow K$. Let $G$ be a further nonempty composition. Then, $J\odot
G\rightarrow K\odot G$.
\end{statement}

[\textit{Proof of Claim 7:} Write the composition $J$ in the form $J=\left(
j_{1},j_{2},\ldots,j_{m}\right)  $. Write the composition $G$ in the form
$G=\left(  g_{1},g_{2},\ldots,g_{p}\right)  $. Thus, $p>0$ (since the
composition $G$ is nonempty).

We have $J\rightarrow K$. In other words, there exists an $\ell\in\left\{
2,3,\ldots,m\right\}  $ such that $j_{\ell}>2$ and $K=\left(  j_{1}%
,j_{2},\ldots,j_{\ell-1},1,j_{\ell}-1,j_{\ell+1},j_{\ell+2},\ldots
,j_{m}\right)  $ (by the definition of the relation $\rightarrow$). Consider
this $\ell$. Clearly, $\ell\geq2$ (since $\ell\in\left\{  2,3,\ldots
,m\right\}  $), so that $\ell>1$.

From $G=\left(  g_{1},g_{2},\ldots,g_{p}\right)  $ and $J=\left(  j_{1}%
,j_{2},\ldots,j_{m}\right)  $, we obtain%
\begin{equation}
J\odot G=\left(  j_{1},j_{2},\ldots,j_{m-1},j_{m}+g_{1},g_{2},g_{3}%
,\ldots,g_{p}\right)  . \label{pf.thm.Epk.dend.c7.pf.JG=}%
\end{equation}
Now, we distinguish between the following two cases:

\textit{Case 1:} We have $\ell=m$.

\textit{Case 2:} We have $\ell\neq m$.

Let us first consider Case 1. In this case, we have $\ell=m$. Thus,
$m=\ell\geq2>1$ and $j_{m}+g_{1}=\underbrace{j_{\ell}}_{>2}+\underbrace{g_{1}%
}_{\geq0}>2$.

From $G=\left(  g_{1},g_{2},\ldots,g_{p}\right)  $ and
\begin{align*}
K  &  =\left(  j_{1},j_{2},\ldots,j_{\ell-1},1,j_{\ell}-1,j_{\ell+1}%
,j_{\ell+2},\ldots,j_{m}\right) \\
&  =\left(  j_{1},j_{2},\ldots,j_{m-1},1,j_{m}-1\right)
\ \ \ \ \ \ \ \ \ \ \left(  \text{since }\ell=m\right)  ,
\end{align*}
we obtain%
\begin{align}
K\odot G  &  =\left(  j_{1},j_{2},\ldots,j_{m-1},1,\left(  j_{m}-1\right)
+g_{1},g_{2},g_{3},\ldots,g_{p}\right) \nonumber\\
&  =\left(  j_{1},j_{2},\ldots,j_{m-1},1,j_{m}+g_{1}-1,g_{2},g_{3}%
,\ldots,g_{p}\right)  . \label{pf.thm.Epk.dend.c7.pf.KG=c1}%
\end{align}


From looking at (\ref{pf.thm.Epk.dend.c7.pf.JG=}) and
(\ref{pf.thm.Epk.dend.c7.pf.KG=c1}), we conclude immediately that the
composition $K\odot G$ is obtained from $J\odot G$ by \textquotedblleft
splitting\textquotedblright\ the entry $j_{m}+g_{1}>2$ into two consecutive
entries $1$ and $j_{m}+g_{1}-1$, and that this entry $j_{m}+g_{1}$ was not the
first entry (indeed, this entry is the $m$-th entry, but $m>1$). Hence,
$J\odot G\rightarrow K\odot G$ (by the definition of the relation
$\rightarrow$). This proves Claim 7 in Case 1.

Let us next consider Case 2. In this case, we have $\ell\neq m$. Hence,
$\ell\in\left\{  2,3,\ldots,m-1\right\}  $ (since $\ell\in\left\{
2,3,\ldots,m\right\}  $).

From $G=\left(  g_{1},g_{2},\ldots,g_{p}\right)  $ and $K=\left(  j_{1}%
,j_{2},\ldots,j_{\ell-1},1,j_{\ell}-1,j_{\ell+1},j_{\ell+2},\ldots
,j_{m}\right)  $, we obtain%
\begin{align}
&  K\odot G\nonumber\\
&  =\left(  j_{1},j_{2},\ldots,j_{\ell-1},1,j_{\ell}-1,j_{\ell+1},j_{\ell
+2},\ldots,j_{m-1},j_{m}+g_{1},g_{2},g_{3},\ldots,g_{p}\right)
\label{pf.thm.Epk.dend.c7.pf.KG=c2}%
\end{align}
(notice that the $j_{m}+g_{1}$ term is \textbf{not} a $\left(  j_{\ell
}-1\right)  +g_{1}$ term, because $\ell\neq m$).

From looking at (\ref{pf.thm.Epk.dend.c7.pf.JG=}) and
(\ref{pf.thm.Epk.dend.c7.pf.KG=c2}), we conclude immediately that the
composition $K\odot G$ is obtained from $J\odot G$ by \textquotedblleft
splitting\textquotedblright\ the entry $j_{\ell}>2$ into two consecutive
entries $1$ and $j_{\ell}-1$, and that this entry $j_{\ell}$ was not the first
entry (indeed, this entry is the $\ell$-th entry, but $\ell>1$). Hence,
$J\odot G\rightarrow K\odot G$ (by the definition of the relation
$\rightarrow$). This proves Claim 7 in Case 2.

We have now proven Claim 7 in both Cases 1 and 2. Thus, Claim 7 always holds.]

\begin{statement}
\textit{Claim 8:} We have $\mathcal{K}_{\operatorname*{Epk}}\bel
A\subseteq\mathcal{K}_{\operatorname*{Epk}}$.
\end{statement}

[\textit{Proof of Claim 8:} This is proven in the same way as we proved Claim
6, with the only difference that now we need to use Claim 7 instead of Claim 5.]

Combining Claim 6 and Claim 8, we conclude that $\mathcal{K}%
_{\operatorname*{Epk}}$ is a $\bel  $-ideal of $A=\operatorname*{QSym}$.

Finally, Theorem \ref{thm.ideal-crit2} \textbf{(c)} (applied to $M=\mathcal{K}%
_{\operatorname*{Epk}}$) shows that $\mathcal{K}_{\operatorname*{Epk}}$ is a
$\left.  \prec\right.  $-ideal and a $\left.  \succeq\right.  $-ideal of
$\operatorname*{QSym}$.

Thus, altogether, we have proven that $\mathcal{K}_{\operatorname*{Epk}}$ is a
$\tvi$-ideal, a $\bel$-ideal, a $\left.  \prec\right.  $-ideal and a $\left.
\succeq\right.  $-ideal of $\operatorname*{QSym}$. This proves Theorem
\ref{thm.Epk.dend}.
\end{proof}

\begin{question}
What other descent statistics $\operatorname*{st}$ have the property that
$\mathcal{K}_{\operatorname*{st}}$ is an $\tvi$-ideal, $\bel$-ideal, $\left.
\prec\right.  $-ideal and/or $\left.  \succeq\right.  $-ideal? We will see
some answers in Subsection \ref{subsect.dendri.other-stats}, but a more
systematic study would be interesting.
\end{question}

\subsection{Dendriform shuffle-compatibility}

We have seen (in Proposition \ref{prop.K.ideal}) that the kernel
$\mathcal{K}_{\operatorname*{st}}$ of a descent statistic $\operatorname*{st}$
is an ideal of $\operatorname*{QSym}$ if and only if $\operatorname*{st}$ is
shuffle-compatible. It is natural to ask whether similar combinatorial
interpretations exist for when the kernel $\mathcal{K}_{\operatorname*{st}}$
of a descent statistic $\operatorname*{st}$ is a $\tvi$-ideal, a $\bel$-ideal,
a $\left.  \prec\right.  $-ideals or a $\left.  \succeq\right.  $-ideal. In
this section, we shall prove such interpretations.

Recall Definition \ref{def.LRshuf} and Definition \ref{def.LR.pi1}. The
following theorem is analogous to Theorem~\ref{thm.4.1}:

\begin{theorem}
\label{thm.dendri.4.1}Let $\pi$ be a nonempty permutation with descent
composition $J$. Let $\sigma$ be a nonempty permutation with descent
composition $K$. Assume that the permutations $\pi$ and $\sigma$ are disjoint,
and that $\pi_{1}>\sigma_{1}$. For any composition $L$, let $c_{J,K}^{L,\prec
}$ be the number of permutations with descent composition $L$ among the left
shuffles of $\pi$ and $\sigma$, and let $c_{J,K}^{L,\succ}$ be the number of
permutations with descent composition $L$ among the right shuffles of $\pi$
and $\sigma$. Then,%
\[
F_{J}\left.  \prec\right.  F_{K}=\sum_{L}c_{J,K}^{L,\prec}F_{L}%
\]
and%
\[
F_{J}\left.  \succeq\right.  F_{K}=\sum_{L}c_{J,K}^{L,\succ}F_{L}.
\]

\end{theorem}

Note the condition $\pi_{1}>\sigma_{1}$, which is not present in
Theorem~\ref{thm.4.1}, and which makes Theorem \ref{thm.dendri.4.1} somewhat
harder to apply.

Theorem \ref{thm.dendri.4.1} can be proven similarly to \cite[(5.2.6)]%
{HopfComb}, but in lieu of the application of \cite[Lemma 5.2.17]{HopfComb},
it requires the following fact:

\begin{proposition}
\label{prop.dendri.4.1.lem}We shall use the notations of \cite[Section
5.2]{HopfComb}. Let $P$ and $Q$ be two disjoint labelled posets, each of which
has a minimum element. Assume that $\min P>_{\mathbb{Z}}\min Q$. Consider the
disjoint union $P\sqcup Q$ of $P$ and $Q$.

\textbf{(a)} Add a further relation $\min P<\min Q$ to $P\sqcup Q$; denote the
resulting labelled poset by $P\left.  \prec\right.  Q$. Then, $F_{P}\left(
\mathbf{x}\right)  \left.  \prec\right.  F_{Q}\left(  \mathbf{x}\right)
=F_{P\left.  \prec\right.  Q}\left(  \mathbf{x}\right)  $.

\textbf{(b)} Add a further relation $\min P>\min Q$ to $P\sqcup Q$; denote the
resulting labelled poset by $P\left.  \succeq\right.  Q$. Then, $F_{P}\left(
\mathbf{x}\right)  \left.  \succeq\right.  F_{Q}\left(  \mathbf{x}\right)
=F_{P\left.  \succeq\right.  Q}\left(  \mathbf{x}\right)  $.
\end{proposition}

Note that the notion of a ``labelled poset'' is understood in the sense
of \cite[Definition 5.2.1]{HopfComb} here; thus, a labelled poset needs no
labelling, but instead its underlying set must be a finite set of integers.

Note also that a \textit{minimum element} of a poset $P$ means an element
$m \in P$ such that every $p \in P$ satisfies $m \leq p$.
This is not the same as a minimal element.

\begin{proof}[Proof of Proposition~\ref{prop.dendri.4.1.lem} (sketched).]
We imitate the proof of \cite[Lemma 5.2.17]{HopfComb}:

We will need two simple auxiliary claims:

\begin{statement}
\textit{Observation 1:} Let $R$ be a labelled poset. Let $u$ and $v$ be
two elements of $R$ such that $u >_{\ZZ} v$.
Assume that we don't have $u > v$ in $R$.
Let $f$ be an $R$-partition.
Let $R^{\prime}$ be the labelled poset obtained from $R$ by adding the
relation $u < v$ to $R$.
Then, $f$ is an $R^{\prime}$-partition if and only if
$f \left(u\right) < f\left(v\right)$.
\end{statement}

[\textit{Proof of Observation 1:} The ``only if'' direction is obvious.
The ``if'' direction is easily derived from the definition of an
$R^{\prime}$-partition, once you recall that any pair
$\left(x, y\right)$ of elements of $R^{\prime}$ that satisfies $x < y$
in $R^{\prime}$ must either already satisfy $x < y$ in $R$, or satisfy
$x \leq u$ and $v \leq y$ in $R$. The details are left to the reader.]

\begin{statement}
\textit{Observation 2:} Let $R$ be a labelled poset. Let $u$ and $v$ be
two elements of $R$ such that $u <_{\ZZ} v$.
Assume that we don't have $u > v$ in $R$.
Let $f$ be an $R$-partition.
Let $R^{\prime}$ be the labelled poset obtained from $R$ by adding the
relation $u < v$ to $R$.
Then, $f$ is an $R^{\prime}$-partition if and only if
$f \left(u\right) \leq f\left(v\right)$.
\end{statement}

[\textit{Proof of Observation 2:} Similar to Observation 1.]

\textbf{(a)} If $f : P \sqcup Q \to \left\{1,2,3,\ldots\right\}$ is a
$P \left. \prec \right. Q$-partition, then its restrictions $f\mid_P$
and $f\mid_Q$ are a $P$-partition and a $Q$-partition, respectively, and
have the property that
$\left(f\mid_P\right) \left( \min P \right) < \left(f\mid_Q\right) \left( \min Q \right)$
(indeed, this property must hold because $\min P < \min Q$ in
$P \left. \prec \right. Q$ but $\min P >_{\ZZ} \min Q$).
Conversely, any pair of a $P$-partition $g$ and a $Q$-partition $h$
having the property that $g \left(\min P\right) < h \left(\min Q\right)$
can be combined to form a
$P \left. \prec \right. Q$-partition\footnote{To see this, we need to
apply Observation 1 to $R = P \sqcup Q$, $u = \min P$, $v = \min Q$
and $R^{\prime} = P \left. \prec \right. Q$.}.

Using the notations of \cite[Definition 5.2.1]{HopfComb}, we have
\begin{equation}
\sum_{f\text{ is a }P\left.  \prec\right.  Q\text{-partition}}\mathbf{x}%
_{f}=\sum_{\substack{g\text{ is a }P\text{-partition;}\\h\text{ is a
}Q\text{-partition;}\\g\left(  \min P\right)  <h\left(  \min Q\right)
}}\mathbf{x}_{g}\mathbf{x}_{h}\label{pf.prop.dendri.4.1.lem.a.1}%
\end{equation}
(since the previous two sentences establish a bijection between the addends on
the left hand side of this equality and the addends on its right hand side).

But the definitions of $F_{P}\left(  \mathbf{x}\right)  $ and $F_{Q}\left(
\mathbf{x}\right)  $ yield
$F_{P}\left(  \mathbf{x}\right)
=  \sum_{\substack{g\text{ is a } P\text{-partition}}}\mathbf{x}_{g}$
and
$F_{Q}\left( \mathbf{x}\right)
= \sum_{\substack{h\text{ is a }Q\text{-partition}}}\mathbf{x}_{h}$.
Thus,
\begin{align*}
& F_{P}\left(  \mathbf{x}\right)  \left.  \prec\right.  F_{Q}\left(
\mathbf{x}\right)   \\
& =\left(  \sum_{\substack{g\text{ is a }%
P\text{-partition}}}\mathbf{x}_{g}\right)  \left.  \prec\right.  \left(
\sum_{\substack{h\text{ is a }Q\text{-partition}}}\mathbf{x}_{h}\right)  \\
& =\sum_{\substack{g\text{ is a }P\text{-partition;}\\h\text{ is a
}Q\text{-partition}}}\underbrace{\mathbf{x}_{g}\left.  \prec\right.
\mathbf{x}_{h}}_{\substack{=\left\{
\begin{array}
[c]{l}%
\mathbf{x}_{g}\mathbf{x}_{h},\ \ \ \ \ \ \ \ \ \ \text{if }\min\left(
\operatorname*{Supp}\left(  \mathbf{x}_{g}\right)  \right)  <\min\left(
\operatorname*{Supp}\left(  \mathbf{x}_{h}\right)  \right)  ;\\
0,\ \ \ \ \ \ \ \ \ \ \text{if }\min\left(  \operatorname*{Supp}\left(
\mathbf{x}_{g}\right)  \right)  \geq\min\left(  \operatorname*{Supp}\left(
\mathbf{x}_{h}\right)  \right)
\end{array}
\right.  \\\text{(by the definition of }\mathbf{x}_{g}\left.  \prec\right.
\mathbf{x}_{h}\text{)}}}\\
& =\sum_{\substack{g\text{ is a }P\text{-partition;}\\h\text{ is a
}Q\text{-partition}}}\underbrace{\left\{
\begin{array}
[c]{l}%
\mathbf{x}_{g}\mathbf{x}_{h},\ \ \ \ \ \ \ \ \ \ \text{if }\min\left(
\operatorname*{Supp}\left(  \mathbf{x}_{g}\right)  \right)  <\min\left(
\operatorname*{Supp}\left(  \mathbf{x}_{h}\right)  \right)  ;\\
0,\ \ \ \ \ \ \ \ \ \ \text{if }\min\left(  \operatorname*{Supp}\left(
\mathbf{x}_{g}\right)  \right)  \geq\min\left(  \operatorname*{Supp}\left(
\mathbf{x}_{h}\right)  \right)
\end{array}
\right.  }_{\substack{=\left\{
\begin{array}
[c]{l}%
\mathbf{x}_{g}\mathbf{x}_{h},\ \ \ \ \ \ \ \ \ \ \text{if }g\left(  \min
P\right)  <h\left(  \min Q\right)  ;\\
0,\ \ \ \ \ \ \ \ \ \ \text{if }g\left(  \min P\right)  \geq h\left(  \min
Q\right)
\end{array}
\right.  \\\text{(since }\min\left(  \operatorname*{Supp}\left(
\mathbf{x}_{g}\right)  \right)  =g\left(  \min P\right)  \text{ (because the
map }g\text{ is a }P\text{-partition, thus}\\\text{weakly increasing on
}P\text{, and therefore }g\left(  \min P\right)  =\min\left(  g\left(
P\right)  \right)  =\min\left(  \operatorname*{Supp}\left(  \mathbf{x}%
_{g}\right)  \right)  \text{)}\\\text{and similarly }\min\left(
\operatorname*{Supp}\left(  \mathbf{x}_{h}\right)  \right)  =h\left(  \min
Q\right)  \text{)}}}\\
& =\sum_{\substack{g\text{ is a }P\text{-partition;}\\h\text{ is a
}Q\text{-partition}}}\left\{
\begin{array}
[c]{l}%
\mathbf{x}_{g}\mathbf{x}_{h},\ \ \ \ \ \ \ \ \ \ \text{if }g\left(  \min
P\right)  <h\left(  \min Q\right)  ;\\
0,\ \ \ \ \ \ \ \ \ \ \text{if }g\left(  \min P\right)  \geq h\left(  \min
Q\right)
\end{array}
\right.  \\
& =\sum_{\substack{g\text{ is a }P\text{-partition;}\\h\text{ is a
}Q\text{-partition;}\\g\left(  \min P\right)  <h\left(  \min Q\right)
}}\mathbf{x}_{g}\mathbf{x}_{h}=\sum_{f\text{ is a }P\left.  \prec\right.
Q\text{-partition}}\mathbf{x}_{f}\ \ \ \ \ \ \ \ \ \ \left(  \text{by
(\ref{pf.prop.dendri.4.1.lem.a.1})}\right)  \\
& =F_{P\left.  \prec\right.  Q}\left(  \mathbf{x}\right)
\end{align*}
(by the definition of $F_{P\left.  \prec\right.  Q}\left(  \mathbf{x}\right)
$). This proves Proposition \ref{prop.dendri.4.1.lem} \textbf{(a)}.

\textbf{(b)} Proposition \ref{prop.dendri.4.1.lem} \textbf{(b)} is proven
similarly to Proposition \ref{prop.dendri.4.1.lem} \textbf{(a)}
(but this time we need to use Observation 2 instead of Observation 1).
\end{proof}

Also, the following simple fact is used:

\begin{lemma}
\label{lem.dendri.4.1.lem2}We shall use the notations of \cite[Section
5.2]{HopfComb}. Let $P$ and $Q$ be two disjoint posets, each of which has a
minimum element. Consider the disjoint union $P\sqcup Q$ of $P$ and $Q$ as the
set-theoretic union $P\cup Q$. Assume that $P$ and $Q$ are subsets of
$\mathbb{P}$ (the set of positive integers); thus, any linear extension of $P$
or of $Q$ or of $P\sqcup Q$ is a permutation (a word with letters in
$\mathbb{P}$).

\textbf{(a)} Add a further relation $\min P<\min Q$ to $P\sqcup Q$; denote the
resulting poset by $P\left.  \prec\right.  Q$. Then,
\[
\mathcal{L}\left(  P\left.  \prec\right.  Q\right)  =\bigsqcup_{\pi
\in\mathcal{L}\left(  P\right)  ;\ \sigma\in\mathcal{L}\left(  Q\right)
}S_{\prec}\left(  \pi,\sigma\right)  .
\]


\textbf{(b)} Add a further relation $\min P>\min Q$ to $P\sqcup Q$; denote the
resulting poset by $P\left.  \succeq\right.  Q$. Then,
\[
\mathcal{L}\left(  P\left.  \succeq\right.  Q\right)  =\bigsqcup_{\pi
\in\mathcal{L}\left(  P\right)  ;\ \sigma\in\mathcal{L}\left(  Q\right)
}S_{\succ}\left(  \pi,\sigma\right)  .
\]

\end{lemma}

\begin{proof}
[Proof of Lemma \ref{lem.dendri.4.1.lem2} (sketched).]Recall that we regard
linear extensions of a finite poset $R$ as lists of elements of $R$. Thus, if
$R$ is a finite poset, if $u$ and $v$ are two elements of $R$, and if $w$ is a
linear extension of $R$, then we have $u<v$ in $w$ if and only if $u$ appears
before $v$ in the list $w$.

We shall also use the following notation: If $w$ is a list of elements of some
set $U$, and if $V$ is a subset of $U$, then $w\mid_{V}$ means the result of
removing all entries from $w$ that don't belong to $V$. For example, $\left(
2,7,1,6,3,4\right)  \mid_{\left\{  1,2,4,5,6\right\}  }=\left(
2,1,6,4\right)  $.

Now, consider our two posets $P$ and $Q$. Recall that a linear extension of
$P\sqcup Q$ is simply a list $\left(  w_{1},w_{2},\ldots,w_{n}\right)  $ of
all elements of $P\sqcup Q$ such that no two integers $i<j$ satisfy $w_{i}\geq
w_{j}$ in $P\sqcup Q$. In other words, a linear extension of $P\sqcup Q$ is a
list $w$ of all elements of $P\sqcup Q$ such that if $x$ and $y$ are two
elements of $P\sqcup Q$ satisfying $x<y$ in $P\sqcup Q$, then $x$ appears
before\footnote{\textquotedblleft Before\textquotedblright\ doesn't imply
\textquotedblleft immediately before\textquotedblright. For example, $2$
appears before $4$ in the list $\left(  1,2,3,4\right)  $.} $y$ in the list
$w$. By the definition of $P\sqcup Q$, this rewrites as follows: A linear
extension of $P\sqcup Q$ is a list $w$ of all elements of $P\sqcup Q$ with the
following two properties:

\begin{itemize}
\item If $x$ and $y$ are two elements of $P$ satisfying $x<y$ in $P$, then $x$
appears before $y$ in the list $w$.

\item If $x$ and $y$ are two elements of $Q$ satisfying $x<y$ in $Q$, then $x$
appears before $y$ in the list $w$.
\end{itemize}

In other words, a linear extension $P\sqcup Q$ is a list $w$ of all elements
of $P\sqcup Q$ such that $w\mid_{P}$ is a linear extension of $P$ and
$w\mid_{Q}$ is a linear extension of $Q$.

This yields the following:

\begin{itemize}
\item If $w$ is a linear extension of $P\sqcup Q$, then $w\mid_{P}$ is a
linear extension of $P$, and $w\mid_{Q}$ is a linear extension of $Q$, and we
have $w\in S\left(  w\mid_{P},w\mid_{Q}\right)  $ (since both $w\mid_{P}$ and
$w\mid_{Q}$ are subsequences of $w$, and their sizes add up to the size of
$w$). Therefore, if $w$ is a linear extension of $P\sqcup Q$, then
$w\in\bigcup_{\pi\in\mathcal{L}\left(  P\right)  ;\ \sigma\in\mathcal{L}%
\left(  Q\right)  }S\left(  \pi,\sigma\right)  $ (because $w\in S\left(
\pi,\sigma\right)  $ for $\pi=w\mid_{P}\in\mathcal{L}\left(  P\right)  $ and
$\sigma=w\mid_{Q}\in\mathcal{L}\left(  Q\right)  $).

\item Conversely, any $w\in\bigcup_{\pi\in\mathcal{L}\left(  P\right)
;\ \sigma\in\mathcal{L}\left(  Q\right)  }S\left(  \pi,\sigma\right)  $ is a
linear extension of $P\sqcup Q$.
\end{itemize}

Combining these two facts, we conclude the following:

\begin{statement}
\textit{Observation 1:} The linear extensions of $P\sqcup Q$ are precisely the
elements of $\bigcup_{\pi\in\mathcal{L}\left(  P\right)  ;\ \sigma
\in\mathcal{L}\left(  Q\right)  }S\left(  \pi,\sigma\right)  $.
\end{statement}

Next, we notice the following:

\begin{statement}
\textit{Observation 2:} The union $\bigcup_{\pi\in\mathcal{L}\left(  P\right)
;\ \sigma\in\mathcal{L}\left(  Q\right)  }S\left(  \pi,\sigma\right)  $ is a
disjoint union (i.e., the sets $S\left(  \pi,\sigma\right)  $ for distinct
pairs $\left(  \pi,\sigma\right)  $ are disjoint).
\end{statement}

[\textit{Proof of Observation 2:} If we are given an element $w\in S\left(
\pi,\sigma\right)  $ for some $\pi\in\mathcal{L}\left(  P\right)  $ and
$\sigma\in\mathcal{L}\left(  Q\right)  $, then we can uniquely reconstruct
$\left(  \pi,\sigma\right)  $ from $w$ (namely, $\left(  \pi,\sigma\right)  $
is given by $\pi=w\mid_{P}$ and $\sigma=w\mid_{Q}$). Thus, the the sets
$S\left(  \pi,\sigma\right)  $ for distinct pairs $\left(  \pi,\sigma\right)
$ are disjoint. This proves Observation 2.]

We will furthermore need a simple auxiliary claim:

\begin{statement}
\textit{Observation 3:} Let $R$ be a finite poset. Let $u$ and $v$ be two
elements of $R$. Assume that we don't have $u>v$ in $R$. Let $w$ be a linear
extension of $R$. Let $R^{\prime}$ be the poset obtained from $R$ by adding
the relation $u<v$ to $R$. Then, $w$ is a linear extension of $R^{\prime}$ if
and only if we have $u<v$ in $w$.
\end{statement}

[\textit{Proof of Observation 3:} The \textquotedblleft only
if\textquotedblright\ direction is obvious. The \textquotedblleft
if\textquotedblright\ direction is easily derived from the definition of a
linear extension, once you recall that any pair $\left(  x,y\right)  $ of
elements of $R^{\prime}$ that satisfies $x<y$ in $R^{\prime}$ must either
already satisfy $x<y$ in $R$, or satisfy $x\leq u$ and $v\leq y$ in $R$. The
details are left to the reader.]

\textbf{(a)} Observation 2 shows that the union $\bigcup_{\pi\in
\mathcal{L}\left(  P\right)  ;\ \sigma\in\mathcal{L}\left(  Q\right)
}S\left(  \pi,\sigma\right)  $ is a disjoint union. Hence, the union
$\bigcup_{\pi\in\mathcal{L}\left(  P\right)  ;\ \sigma\in\mathcal{L}\left(
Q\right)  }S_{\prec}\left(  \pi,\sigma\right)  $ is a disjoint union as well
(since $S_{\prec}\left(  \pi,\sigma\right)  $ is a subset of $S\left(
\pi,\sigma\right)  $ for all $\pi$ and $\sigma$).

We don't have $\min P>\min Q$ in $P\sqcup Q$. Thus, applying Observation 3 to
$R=P\sqcup Q$, $u=\min P$, $v=\min Q$ and $R^{\prime}=P\left.  \prec\right.
Q$, we obtain the following:

\begin{statement}
\textit{Observation 4:} Let $w$ be a linear extension of $P\sqcup Q$. Then,
$w$ is a linear extension of $P\left.  \prec\right.  Q$ if and only if we have
$\min P<\min Q$ in $w$.
\end{statement}

Next, we claim that
\begin{equation}
\mathcal{L}\left(  P\left.  \prec\right.  Q\right)  \subseteq\bigcup_{\pi
\in\mathcal{L}\left(  P\right)  ;\ \sigma\in\mathcal{L}\left(  Q\right)
}S_{\prec}\left(  \pi,\sigma\right)  .\label{pf.lem.dendri.4.1.lem2.a.5}%
\end{equation}


[\textit{Proof:} Let $w\in\mathcal{L}\left(  P\left.  \prec\right.  Q\right)
$. Thus, $w$ is a linear extension of $P\left.  \prec\right.  Q$. Hence, $w$
is a linear extension of $P\sqcup Q$, and we have $\min P<\min Q$ in $w$ (by
Observation 4). Since $w$ is a linear extension of $P\sqcup Q$, we have
$w\in\bigcup_{\pi\in\mathcal{L}\left(  P\right)  ;\ \sigma\in\mathcal{L}%
\left(  Q\right)  }S\left(  \pi,\sigma\right)  $ (by Observation 1). In other
words, $w\in S\left(  \pi,\sigma\right)  $ for some $\pi\in\mathcal{L}\left(
P\right)  $ and $\sigma\in\mathcal{L}\left(  Q\right)  $. Consider these $\pi$
and $\sigma$.

The element $\min P$ is the minimum element of the poset $P$, and thus must be
the first letter of $\pi$ (since $\pi$ is a linear extension of $P$).
Similarly, $\min Q$ must be the first letter of $\sigma$. But $w\in S\left(
\pi,\sigma\right)  $. Hence, the first letter of $w$ is either the first
letter of $\pi$ or the first letter of $\sigma$.

We have $\min P<\min Q$ in $w$. In other words, $\min P$ appears before $\min
Q$ in the list $w$. Hence, $\min Q$ cannot be the first letter of $w$. In
other words, the first letter of $w$ cannot be $\min Q$. In other words, the
first letter of $w$ cannot be the first letter of $\sigma$ (since $\min Q$ is
the first letter of $\sigma$). Thus, the first letter of $w$ is the first
letter of $\pi$ (since the first letter of $w$ is either the first letter of
$\pi$ or the first letter of $\sigma$). In other words, $w$ is a left shuffle
of $\pi$ and $\sigma$ (since $w\in S\left(  \pi,\sigma\right)  $). In other
words, $w\in S_{\prec}\left(  \pi,\sigma\right)  $ (since $S_{\prec}\left(
\pi,\sigma\right)  $ is the set of all left shuffles of $\pi$ and $\sigma$).

Now, forget that we have defined $\pi$ and $\sigma$. We thus have shown that
$w\in S_{\prec}\left(  \pi,\sigma\right)  $ for some $\pi\in\mathcal{L}\left(
P\right)  $ and $\sigma\in\mathcal{L}\left(  Q\right)  $. Thus, $w\in
\bigcup_{\pi\in\mathcal{L}\left(  P\right)  ;\ \sigma\in\mathcal{L}\left(
Q\right)  }S_{\prec}\left(  \pi,\sigma\right)  $. Since we have proven this
for any $w\in\mathcal{L}\left(  P\left.  \prec\right.  Q\right)  $, we thus
have proven (\ref{pf.lem.dendri.4.1.lem2.a.5}).]

On the other hand, we claim that
\begin{equation}
\bigcup_{\pi\in\mathcal{L}\left(  P\right)  ;\ \sigma\in\mathcal{L}\left(
Q\right)  }S_{\prec}\left(  \pi,\sigma\right)  \subseteq\mathcal{L}\left(
P\left.  \prec\right.  Q\right)  .\label{pf.lem.dendri.4.1.lem2.a.6}%
\end{equation}


[\textit{Proof:} Let $w\in\bigcup_{\pi\in\mathcal{L}\left(  P\right)
;\ \sigma\in\mathcal{L}\left(  Q\right)  }S_{\prec}\left(  \pi,\sigma\right)
$.

We have%
\[
w\in\bigcup_{\pi\in\mathcal{L}\left(  P\right)  ;\ \sigma\in\mathcal{L}\left(
Q\right)  }\underbrace{S_{\prec}\left(  \pi,\sigma\right)  }_{\subseteq
S\left(  \pi,\sigma\right)  }\subseteq\bigcup_{\pi\in\mathcal{L}\left(
P\right)  ;\ \sigma\in\mathcal{L}\left(  Q\right)  }S\left(  \pi
,\sigma\right)  .
\]
Thus, $w$ is a linear extension of $P\sqcup Q$ (by Observation 1).

Also, $w\in\bigcup_{\pi\in\mathcal{L}\left(  P\right)  ;\ \sigma
\in\mathcal{L}\left(  Q\right)  }S_{\prec}\left(  \pi,\sigma\right)  $.
Thus, $w\in S_{\prec}\left(  \pi,\sigma\right)  $ for some $\pi\in
\mathcal{L}\left(  P\right)  $ and $\sigma\in\mathcal{L}\left(  Q\right)  $.
Consider these $\pi$ and $\sigma$. From $w\in S_{\prec}\left(  \pi
,\sigma\right)  $, we conclude that $w$ is a left shuffle of $\pi$ and
$\sigma$. In other words, $w$ is a shuffle of $\pi$ and $\sigma$ such that the
first letter of $w$ is the first letter of $\pi$.

The element $\min P$ is the minimum element of the poset $P$, and thus must be
the first letter of $\pi$ (since $\pi$ is a linear extension of $P$). In other
words, $\min P$ is the first letter of $w$ (since the first letter of $w$ is
the first letter of $\pi$). Hence, $\min P$ appears before $\min Q$ in the
list $w$ (since $\min P$ and $\min Q$ are two distinct letters of $w$). In
other words, we have $\min P<\min Q$ in $w$. Hence, Observation 4 yields that
$w$ is a linear extension of $P\left.  \prec\right.  Q$ (since $w$ is a linear
extension of $P\sqcup Q$). In other words, $w\in\mathcal{L}\left(  P\left.
\prec\right.  Q\right)  $. Since we have proven this for any $w\in\bigcup
_{\pi\in\mathcal{L}\left(  P\right)  ;\ \sigma\in\mathcal{L}\left(  Q\right)
}S_{\prec}\left(  \pi,\sigma\right)  $, we thus have proven
(\ref{pf.lem.dendri.4.1.lem2.a.6}).]

Combining (\ref{pf.lem.dendri.4.1.lem2.a.5}) and
(\ref{pf.lem.dendri.4.1.lem2.a.6}), we obtain
\[
\mathcal{L}\left(  P\left.  \prec\right.  Q\right)  =\bigcup_{\pi
\in\mathcal{L}\left(  P\right)  ;\ \sigma\in\mathcal{L}\left(  Q\right)
}S_{\prec}\left(  \pi,\sigma\right)  =\bigsqcup_{\pi\in\mathcal{L}\left(
P\right)  ;\ \sigma\in\mathcal{L}\left(  Q\right)  }S_{\prec}\left(
\pi,\sigma\right)
\]
(since the union $\bigcup_{\pi\in\mathcal{L}\left(  P\right)  ;\ \sigma
\in\mathcal{L}\left(  Q\right)  }S_{\prec}\left(  \pi,\sigma\right)  $ is a
disjoint union). This proves Lemma \ref{lem.dendri.4.1.lem2} \textbf{(a)}.

\textbf{(b)} Lemma \ref{lem.dendri.4.1.lem2} \textbf{(b)} is proven similarly
to Lemma \ref{lem.dendri.4.1.lem2} \textbf{(a)}.
\end{proof}

We can now prove Theorem \ref{thm.dendri.4.1}.
First, let us rewrite Theorem \ref{thm.dendri.4.1} as follows:\footnote{Recall
that for any permutation $\varphi$, we have let $\operatorname*{Comp}\varphi$
denote the descent composition of $\varphi$.}

\begin{corollary}
\label{cor.dendri.4.1}Let $\pi$ and $\sigma$ be two disjoint nonempty
permutations. Assume that $\pi_{1}>\sigma_{1}$. Then,%
\[
F_{\operatorname*{Comp}\pi}\left.  \prec\right.  F_{\operatorname*{Comp}%
\sigma}=\sum_{\chi\in S_{\prec}\left(  \pi,\sigma\right)  }%
F_{\operatorname*{Comp}\chi}%
\]
and%
\[
F_{\operatorname*{Comp}\pi}\left.  \succeq\right.  F_{\operatorname*{Comp}%
\sigma}=\sum_{\chi\in S_{\succ}\left(  \pi,\sigma\right)  }%
F_{\operatorname*{Comp}\chi}.
\]

\end{corollary}

\begin{proof}
[Proof of Corollary \ref{cor.dendri.4.1} (sketched).] Let $n=\left\vert
\pi\right\vert $ and $m=\left\vert \sigma\right\vert $. We shall use the
notations of \cite[Section 5.2]{HopfComb}; in particular,
\textquotedblleft labelled poset\textquotedblright\ will be defined as in
\cite[Definition 5.2.1]{HopfComb}.

If $R$ is a labelled poset, and if $w$ is a linear extension of $R$,
then $w$ can be regarded as a labelled poset itself, but also as a
permutation (since $w$ is a list of distinct elements of $R$, and thus
a word over the alphabet $\mathbb{P}$ with no two equal letters).
The first interpretation (as a labelled poset)
gives rise to a quasisymmetric function
$F_w\left(\xx\right)$ (defined as in \cite[Definition 5.2.1]{HopfComb}).
The second interpretation (as a permutation) leads to a composition
$\operatorname{Comp} w$. These two objects are connected by the equality
\begin{align}
F_w \left( \xx \right) = F_{\operatorname{Comp} w} .
\label{pf.cor.dendri.4.1.Fw}
\end{align}
(This follows from \cite[Proposition 5.2.10]{HopfComb}; but keep in mind that
\cite[Proposition 5.2.10]{HopfComb} denotes $F_{\operatorname*{Comp}\pi}$
by $L_{\alpha}$ in this context.)

Let $P$ be the labelled poset whose elements are $\pi_{1},\pi_{2},\ldots
,\pi_{n}$ and whose order is the total order given by $\pi_{1}<\pi_{2}%
<\cdots<\pi_{n}$. Thus, $P$ is totally ordered, and its minimum element is
$\min P=\pi_{1}$. Also, \cite[Proposition 5.2.10]{HopfComb} yields
$F_{P}\left(  \mathbf{x}\right)  =F_{\operatorname*{Comp}\pi}$. (Keep in mind
that \cite[Proposition 5.2.10]{HopfComb} denotes $F_{\operatorname*{Comp}\pi}$
by $L_{\alpha}$ in this context.)

Let $Q$ be the labelled poset whose elements are $\sigma_{1},\sigma_{2}%
,\ldots,\sigma_{m}$ and whose order is the total order given by $\sigma
_{1}<\sigma_{2}<\cdots<\sigma_{m}$. Thus, $Q$ is totally ordered, and its
minimum element is $\min Q=\sigma_{1}$. Also, \cite[Proposition 5.2.10]%
{HopfComb} yields $F_{Q}\left(  \mathbf{x}\right)  =F_{\operatorname*{Comp}%
\sigma}$.

The posets $P$ and $Q$ are disjoint (since the permutations $\pi$ and $\sigma$
are disjoint). Define three labelled posets $P\sqcup Q$, $P\left.
\prec\right.  Q$ and $P\left.  \succ\right.  Q$ as in Proposition
\ref{prop.dendri.4.1.lem}.

Lemma \ref{lem.dendri.4.1.lem2} \textbf{(a)} yields $\mathcal{L}\left(
P\left.  \prec\right.  Q\right)  =\bigsqcup_{\pi^{\prime}\in\mathcal{L}\left(
P\right)  ;\ \sigma^{\prime}\in\mathcal{L}\left(  Q\right)  }S_{\prec}\left(
\pi^{\prime},\sigma^{\prime}\right)  $ (where we are using the letters
$\pi^{\prime}$ and $\sigma^{\prime}$ for our subscripts, since the letters
$\pi$ and $\sigma$ are already taken). But $P$ is totally ordered; thus, there
exists only one linear extension $\pi^{\prime}\in\mathcal{L}\left(  P\right)
$, namely, $\pi^{\prime}=\pi$. In other words, $\mathcal{L}\left(  P\right)
=\left\{  \pi\right\}  $. Similarly, $\mathcal{L}\left(  Q\right)  =\left\{
\sigma\right\}  $. Hence,%
\begin{align}
\mathcal{L}\left(  P\left.  \prec\right.  Q\right)    & =\bigsqcup
_{\pi^{\prime}\in\mathcal{L}\left(  P\right)  ;\ \sigma^{\prime}\in
\mathcal{L}\left(  Q\right)  }S_{\prec}\left(  \pi^{\prime},\sigma^{\prime
}\right) \nonumber\\
&  =\bigsqcup_{\pi^{\prime}\in\left\{  \pi\right\}  ;\ \sigma^{\prime
}\in\left\{  \sigma\right\}  }S_{\prec}\left(  \pi^{\prime},\sigma^{\prime
}\right)  \ \ \ \ \ \ \ \ \ \ \left(
\begin{array}
[c]{c}%
\text{since }\mathcal{L}\left(  P\right)  =\left\{  \pi\right\}  \\
\text{and }\mathcal{L}\left(  Q\right)  =\left\{  \sigma\right\}
\end{array}
\right)  \nonumber\\
& =S_{\prec}\left(  \pi,\sigma\right)  .\label{pf.cor.dendri.4.1.L1}%
\end{align}


Recall that every labelled poset $R$ satisfies%
\begin{equation}
F_{R}\left(  \mathbf{x}\right)  =\sum_{w\in\mathcal{L}\left(  R\right)  }%
F_{w}\left(  \mathbf{x}\right)  \label{pf.cor.dendri.4.1.FR}%
\end{equation}
(by \cite[Theorem 5.2.11]{HopfComb}).

Now, $\pi_{1}>\sigma_{1}$ in $\mathbb{Z}$. In other words, $\pi_{1}%
>_{\mathbb{Z}}\sigma_{1}$. In other words, $\min P>_{\mathbb{Z}}\min Q$ (since
$\min P=\pi_{1}$ and $\min Q=\sigma_{1}$). Thus,
\begin{align*}
& \underbrace{F_{\operatorname*{Comp}\pi}}_{=F_{P}\left(  \mathbf{x}\right)
}\left.  \prec\right.  \underbrace{F_{\operatorname*{Comp}\sigma}}%
_{=F_{Q}\left(  \mathbf{x}\right)  }\\
& =F_{P}\left(  \mathbf{x}\right)  \left.  \prec\right.  F_{Q}\left(
\mathbf{x}\right)  =F_{P\left.  \prec\right.  Q}\left(  \mathbf{x}\right)
\ \ \ \ \ \ \ \ \ \ \left(  \text{by Proposition \ref{prop.dendri.4.1.lem}
\textbf{(a)}}\right)  \\
& =\sum_{w\in\mathcal{L}\left(  P\left.  \prec\right.  Q\right)  }F_{w}\left(
\mathbf{x}\right)  \ \ \ \ \ \ \ \ \ \ \left(  \text{by
(\ref{pf.cor.dendri.4.1.FR})}\right)  \\
& =\sum_{w\in S_{\prec}\left(  \pi,\sigma\right)  }\underbrace{F_{w}\left(
\mathbf{x}\right)  }_{\substack{=F_{\operatorname*{Comp}w}\\\text{(by
\eqref{pf.cor.dendri.4.1.Fw})}}}\ \ \ \ \ \ \ \ \ \ \left(  \text{by
(\ref{pf.cor.dendri.4.1.L1})}\right)  \\
& =\sum_{w\in S_{\prec}\left(  \pi,\sigma\right)  }F_{\operatorname*{Comp}%
w}=\sum_{\chi\in S_{\prec}\left(  \pi,\sigma\right)  }F_{\operatorname*{Comp}%
\chi}.
\end{align*}
A similar argument (using Proposition \ref{prop.dendri.4.1.lem} \textbf{(b)}
and Lemma \ref{lem.dendri.4.1.lem2} \textbf{(b)} instead of Proposition
\ref{prop.dendri.4.1.lem} \textbf{(a)} and Lemma \ref{lem.dendri.4.1.lem2}
\textbf{(a)}) shows that%
\[
F_{\operatorname*{Comp}\pi}\left.  \succeq\right.  F_{\operatorname*{Comp}%
\sigma}=\sum_{\chi\in S_{\succ}\left(  \pi,\sigma\right)  }%
F_{\operatorname*{Comp}\chi}.
\]
Thus, Corollary \ref{cor.dendri.4.1} is proven.
\end{proof}

Hence, Theorem \ref{thm.dendri.4.1} is proven as well (since Corollary
\ref{cor.dendri.4.1} was just a restatement of Theorem \ref{thm.dendri.4.1}).

Now, let us define two further variants of LR-shuffle-compatibility (to be
compared with those introduced in Definition \ref{def.LR.left-right}):

\begin{definition}
Let $\operatorname*{st}$ be a permutation statistic.

\textbf{(a)} We say that $\operatorname*{st}$ is \textit{weakly
left-shuffle-compatible} if for any two disjoint nonempty permutations $\pi$
and $\sigma$ having the property that%
\begin{equation}
\text{each entry of }\pi\text{ is greater than each entry of }\sigma,
\label{eq.def.dendri.dsc.weak-ass}%
\end{equation}
the multiset $\left\{  \operatorname*{st}\left(  \tau\right)  \ \mid\ \tau\in
S_{\prec}\left(  \pi,\sigma\right)  \right\}  _{\operatorname*{multi}}$
depends only on $\operatorname*{st}\left(  \pi\right)  $, $\operatorname*{st}%
\left(  \sigma\right)  $, $\left\vert \pi\right\vert $ and $\left\vert
\sigma\right\vert $.

\textbf{(b)} We say that $\operatorname*{st}$ is \textit{weakly
right-shuffle-compatible} if for any two disjoint nonempty permutations $\pi$
and $\sigma$ having the property that%
\[
\text{each entry of }\pi\text{ is greater than each entry of }\sigma,
\]
the multiset $\left\{  \operatorname*{st}\left(  \tau\right)  \ \mid\ \tau\in
S_{\succ}\left(  \pi,\sigma\right)  \right\}  _{\operatorname*{multi}}$
depends only on $\operatorname*{st}\left(  \pi\right)  $, $\operatorname*{st}%
\left(  \sigma\right)  $, $\left\vert \pi\right\vert $ and $\left\vert
\sigma\right\vert $.
\end{definition}

Then, the following analogues to the first part of Proposition
\ref{prop.K.ideal} hold:

\begin{theorem}
\label{thm.dendri.K.ideal}Let $\operatorname*{st}$ be a descent statistic.
Then, the following three statements are equivalent:

\begin{itemize}
\item \textit{Statement A:} The statistic $\operatorname*{st}$ is left-shuffle-compatible.

\item \textit{Statement B:} The statistic $\operatorname*{st}$ is weakly left-shuffle-compatible.

\item \textit{Statement C:} The set $\mathcal{K}_{\operatorname*{st}}$ is an
$\left.  \prec\right.  $-ideal of $\operatorname*{QSym}$.
\end{itemize}
\end{theorem}

\begin{theorem}
\label{thm.dendri.K.ideal-R}Let $\operatorname*{st}$ be a descent statistic.
Then, the following three statements are equivalent:

\begin{itemize}
\item \textit{Statement A:} The statistic $\operatorname*{st}$ is right-shuffle-compatible.

\item \textit{Statement B:} The statistic $\operatorname*{st}$ is weakly right-shuffle-compatible.

\item \textit{Statement C:} The set $\mathcal{K}_{\operatorname*{st}}$ is an
$\left.  \succeq\right.  $-ideal of $\operatorname*{QSym}$.
\end{itemize}
\end{theorem}

Let us prove Theorem \ref{thm.dendri.K.ideal} directly, without using shuffle algebras:

\begin{proof}
[Proof of Theorem \ref{thm.dendri.K.ideal} (sketched).]The implication
A$\Longrightarrow$B is obvious.

\textit{Proof of the implication B}$\Longrightarrow$\textit{C:} Assume that
Statement B holds. Thus, the statistic $\operatorname*{st}$ is weakly left-shuffle-compatible.

Let us show that the set $\mathcal{K}_{\operatorname*{st}}$ is a $\left.
\prec\right.  $-ideal of $\operatorname*{QSym}$. Indeed, it suffices to show
that every two $\operatorname*{st}$-equivalent compositions $J$ and $K$ and
every further composition $L$ satisfy%
\begin{equation}
\left(  F_{J}-F_{K}\right)  \left.  \prec\right.  F_{L}\in\mathcal{K}%
_{\operatorname*{st}}\ \ \ \ \ \ \ \ \ \ \text{and}\ \ \ \ \ \ \ \ \ \ F_{L}%
\left.  \prec\right.  \left(  F_{J}-F_{K}\right)  \in\mathcal{K}%
_{\operatorname*{st}} \label{pf.thm.dendri.K.ideal.two-incs}%
\end{equation}
(because of the definition of $\mathcal{K}_{\operatorname*{st}}$). So let $J$
and $K$ be two $\operatorname*{st}$-equivalent compositions, and let $L$ be a
further composition. If $J=K$, then (\ref{pf.thm.dendri.K.ideal.two-incs})
follows immediately from realizing that $F_{J}-F_{K}=0$; thus, we WLOG assume
that $J\neq K$. But $\left\vert J\right\vert =\left\vert K\right\vert $ (since
$J$ and $K$ are $\operatorname*{st}$-equivalent). Hence, $\left\vert
J\right\vert =\left\vert K\right\vert >0$ (since otherwise, we would have
$\left\vert J\right\vert =\left\vert K\right\vert =0$, which would imply that
both $J$ and $K$ would be the empty composition, contradicting $J\neq K$).
Thus, the power series $F_{J}$ and $F_{K}$ are homogeneous of degree
$\left\vert J\right\vert =\left\vert K\right\vert >0$; consequently,
$\varepsilon\left(  F_{J}\right)  =0$ and $\varepsilon\left(  F_{K}\right)
=0$. Hence, $\varepsilon\left(  F_{J}-F_{K}\right)  =\underbrace{\varepsilon
\left(  F_{J}\right)  }_{=0}-\underbrace{\varepsilon\left(  F_{K}\right)
}_{=0}=0$.

The compositions $J$ and $K$ are nonempty (since $\left\vert J\right\vert
=\left\vert K\right\vert >0$). If $L$ is empty, then
(\ref{pf.thm.dendri.K.ideal.two-incs}) holds for easy reasons (indeed, we have
$F_{L}=1$ in this case, and therefore (\ref{eq.dendriform.a<1}) yields
\[
\left(  F_{J}-F_{K}\right)  \left.  \prec\right.  F_{L}=\left(  F_{J}%
-F_{K}\right)  -\underbrace{\varepsilon\left(  F_{J}-F_{K}\right)
}_{\substack{=0}}=F_{J}-F_{K}\in\mathcal{K}_{\operatorname*{st}},
\]
and similarly (\ref{eq.dendriform.1<a}) leads to $F_{L}\left.  \prec\right.
\left(  F_{J}-F_{K}\right)  \in\mathcal{K}_{\operatorname*{st}}$). Hence, we
WLOG assume that $L$ is nonempty.

Pick three disjoint permutations $\varphi$, $\psi$ and $\sigma$ having descent
compositions $J$, $K$ and $L$, respectively, and having the property that%
\[
\text{each entry of }\varphi\text{ is greater than each entry of }\sigma
\]
and%
\[
\text{each entry of }\psi\text{ is greater than each entry of }\sigma\text{.}%
\]
(Such permutations $\varphi$, $\psi$ and $\sigma$ exist, since the set
$\mathbb{P}$ is infinite.)

The permutations $\varphi$ and $\psi$ are $\operatorname*{st}$-equivalent
(since their descent compositions $J$ and $K$ are $\operatorname*{st}%
$-equivalent). In other words, $\left\vert \varphi\right\vert =\left\vert
\psi\right\vert $ and $\operatorname*{st}\left(  \varphi\right)
=\operatorname*{st}\left(  \psi\right)  $.

The statistic $\operatorname*{st}$ is weakly left-shuffle-compatible. Thus,
the multiset\newline$\left\{  \operatorname*{st}\left(  \tau\right)
\ \mid\ \tau\in S_{\prec}\left(  \pi,\sigma\right)  \right\}
_{\operatorname*{multi}}$ (where $\pi$ is a nonempty permutation disjoint from
$\sigma$ and having the property that each entry of $\pi$ is greater than each
entry of $\sigma$) depends only on $\operatorname*{st}\left(  \pi\right)  $
and $\left\vert \pi\right\vert $ (by the definition of \textquotedblleft
weakly left-shuffle-compatible\textquotedblright)\footnote{Recall that
$\sigma$ is fixed here, which is why we don't have to say that it depends on
$\operatorname*{st}\left(  \sigma\right)  $ and $\left\vert \sigma\right\vert
$ as well.}. Therefore, the multisets $\left\{  \operatorname*{st}\left(
\tau\right)  \ \mid\ \tau\in S_{\prec}\left(  \varphi,\sigma\right)  \right\}
_{\operatorname*{multi}}$ and \newline$\left\{  \operatorname*{st}\left(
\tau\right)  \ \mid\ \tau\in S_{\prec}\left(  \psi,\sigma\right)  \right\}
_{\operatorname*{multi}}$ are equal (since $\left\vert \varphi\right\vert
=\left\vert \psi\right\vert $ and $\operatorname*{st}\left(  \varphi\right)
=\operatorname*{st}\left(  \psi\right)  $). Hence, there exists a bijection
$\alpha:S_{\prec}\left(  \varphi,\sigma\right)  \rightarrow S_{\prec}\left(
\psi,\sigma\right)  $ such that each $\chi\in S_{\prec}\left(  \varphi
,\sigma\right)  $ satisfies%
\begin{equation}
\operatorname*{st}\left(  \alpha\left(  \chi\right)  \right)
=\operatorname*{st}\left(  \chi\right)  . \label{pf.thm.dendri.K.ideal.alpha}%
\end{equation}
Consider this $\alpha$. Clearly, each $\chi\in S_{\prec}\left(  \varphi
,\sigma\right)  $ satisfies%
\[
\left(  \chi\text{ and }\alpha\left(  \chi\right)  \text{ are }%
\operatorname*{st}\text{-equivalent}\right)
\]
(because of (\ref{pf.thm.dendri.K.ideal.alpha}) and since $\left\vert
\chi\right\vert =\underbrace{\left\vert \varphi\right\vert }_{=\left\vert
\psi\right\vert }+\left\vert \sigma\right\vert =\left\vert \psi\right\vert
+\left\vert \sigma\right\vert =\left\vert \alpha\left(  \chi\right)
\right\vert $) and therefore%
\[
\left(  \operatorname*{Comp}\chi\text{ and }\operatorname*{Comp}\left(
\alpha\left(  \chi\right)  \right)  \text{ are }\operatorname*{st}%
\text{-equivalent}\right)
\]
(since $\operatorname*{st}$ is a descent statistic) and thus
$F_{\operatorname*{Comp}\chi}-F_{\operatorname*{Comp}\left(  \alpha\left(
\chi\right)  \right)  }\in\mathcal{K}_{\operatorname*{st}}$ (by the definition
of $\mathcal{K}_{\operatorname*{st}}$) and therefore%
\begin{equation}
F_{\operatorname*{Comp}\chi}\equiv F_{\operatorname*{Comp}\left(
\alpha\left(  \chi\right)  \right)  }\operatorname{mod}\mathcal{K}%
_{\operatorname*{st}}. \label{pf.thm.dendri.K.ideal.alphaeq}%
\end{equation}


The first claim of Corollary \ref{cor.dendri.4.1} yields%
\begin{align*}
F_{\operatorname*{Comp}\varphi}\left.  \prec\right.  F_{\operatorname*{Comp}%
\sigma}  &  =\sum_{\chi\in S_{\prec}\left(  \varphi,\sigma\right)
}F_{\operatorname*{Comp}\chi}\ \ \ \ \ \ \ \ \ \ \text{and}\\
F_{\operatorname*{Comp}\psi}\left.  \prec\right.  F_{\operatorname*{Comp}%
\sigma}  &  =\sum_{\chi\in S_{\prec}\left(  \psi,\sigma\right)  }%
F_{\operatorname*{Comp}\chi}.
\end{align*}
Hence,%
\begin{align*}
F_{\operatorname*{Comp}\varphi}\left.  \prec\right.  F_{\operatorname*{Comp}%
\sigma}  &  =\sum_{\chi\in S_{\prec}\left(  \varphi,\sigma\right)
}\underbrace{F_{\operatorname*{Comp}\chi}}_{\substack{\equiv
F_{\operatorname*{Comp}\left(  \alpha\left(  \chi\right)  \right)
}\operatorname{mod}\mathcal{K}_{\operatorname*{st}}\\\text{(by
(\ref{pf.thm.dendri.K.ideal.alphaeq}))}}}\equiv\sum_{\chi\in S_{\prec}\left(
\varphi,\sigma\right)  }F_{\operatorname*{Comp}\left(  \alpha\left(
\chi\right)  \right)  }\\
&  =\sum_{\chi\in S_{\prec}\left(  \psi,\sigma\right)  }%
F_{\operatorname*{Comp}\chi}\\
&  \ \ \ \ \ \ \ \ \ \ \left(
\begin{array}
[c]{c}%
\text{here, we have substituted }\chi\text{ for }\alpha\left(  \chi\right)
\text{ in the sum,}\\
\text{since the map }\alpha:S_{\prec}\left(  \varphi,\sigma\right)
\rightarrow S_{\prec}\left(  \psi,\sigma\right)  \text{ is a bijection}%
\end{array}
\right) \\
&  =F_{\operatorname*{Comp}\psi}\left.  \prec\right.  F_{\operatorname*{Comp}%
\sigma}\operatorname{mod}\mathcal{K}_{\operatorname*{st}}.
\end{align*}
Since $\operatorname*{Comp}\varphi=J$, $\operatorname*{Comp}\psi=K$ and
$\operatorname*{Comp}\sigma=L$ (by the definition of $\varphi$, $\psi$ and
$\sigma$), this rewrites as $F_{J}\left.  \prec\right.  F_{L}\equiv
F_{K}\left.  \prec\right.  F_{L}\operatorname{mod}\mathcal{K}%
_{\operatorname*{st}}$. In other words, $F_{J}\left.  \prec\right.
F_{L}-F_{K}\left.  \prec\right.  F_{L}\in\mathcal{K}_{\operatorname*{st}}$. In
other words, $\left(  F_{J}-F_{K}\right)  \left.  \prec\right.  F_{L}%
\in\mathcal{K}_{\operatorname*{st}}$. This proves the first claim of
(\ref{pf.thm.dendri.K.ideal.two-incs}). The second is proven similarly.
Altogether, we thus conclude that $\mathcal{K}_{\operatorname*{st}}$ is a
$\left.  \prec\right.  $-ideal of $\operatorname*{QSym}$. In other words,
Statement C holds. This proves the implication B$\Longrightarrow$C.

\textit{Proof of the implication C}$\Longrightarrow$\textit{A:} Assume that
Statement C holds. Thus, the set $\mathcal{K}_{\operatorname*{st}}$ is an
$\left.  \prec\right.  $-ideal of $\operatorname*{QSym}$.

Let $X$ be the codomain of the map $\operatorname*{st}$. Let $\mathbb{Q}%
\left[  X\right]  $ be the free $\mathbb{Q}$-vector space with basis $\left(
\left[  x\right]  \right)  _{x\in X}$. Then, we can define a $\mathbb{Q}%
$-linear map $\mathbf{st}:\operatorname*{QSym}\rightarrow\mathbb{Q}\left[
X\right]  ,\ F_{J}\mapsto\left[  \operatorname*{st}J\right]  $. This map
$\mathbf{st}$ sends each of the generators of $\mathcal{K}_{\operatorname*{st}%
}$ to $0$ (by the definition of $\mathcal{K}_{\operatorname*{st}}$), and
therefore sends the whole $\mathcal{K}_{\operatorname*{st}}$ to $0$. In other
words, $\mathbf{st}\left(  \mathcal{K}_{\operatorname*{st}}\right)  =0$.

Now, consider any two disjoint nonempty permutations $\pi$ and $\sigma$ having
the property that $\pi_{1}>\sigma_{1}$. Also, consider two further disjoint
nonempty permutations $\pi^{\prime}$ and $\sigma^{\prime}$ having the property
that $\pi_{1}^{\prime}>\sigma_{1}^{\prime}$ and satisfying $\operatorname*{st}%
\left(  \pi\right)  =\operatorname*{st}\left(  \pi^{\prime}\right)  $,
$\operatorname*{st}\left(  \sigma\right)  =\operatorname*{st}\left(
\sigma^{\prime}\right)  $, $\left\vert \pi\right\vert =\left\vert \pi^{\prime
}\right\vert $ and $\left\vert \sigma\right\vert =\left\vert \sigma^{\prime
}\right\vert $. We shall show that $\left\{  \operatorname*{st}\left(
\tau\right)  \ \mid\ \tau\in S_{\prec}\left(  \pi,\sigma\right)  \right\}
_{\operatorname*{multi}}=\left\{  \operatorname*{st}\left(  \tau\right)
\ \mid\ \tau\in S_{\prec}\left(  \pi^{\prime},\sigma^{\prime}\right)
\right\}  _{\operatorname*{multi}}$. This will show that the multiset
$\left\{  \operatorname*{st}\left(  \tau\right)  \ \mid\ \tau\in S_{\prec
}\left(  \pi,\sigma\right)  \right\}  _{\operatorname*{multi}}$ depends only
on $\operatorname*{st}\left(  \pi\right)  $, $\operatorname*{st}\left(
\sigma\right)  $, $\left\vert \pi\right\vert $ and $\left\vert \sigma
\right\vert $.

From $\operatorname*{st}\left(  \pi\right)  =\operatorname*{st}\left(
\pi^{\prime}\right)  $ and $\left\vert \pi\right\vert =\left\vert \pi^{\prime
}\right\vert $, we conclude that $\pi$ and $\pi^{\prime}$ are
$\operatorname*{st}$-equivalent. In other words, $\operatorname*{Comp}\pi$ and
$\operatorname*{Comp}\left(  \pi^{\prime}\right)  $ are $\operatorname*{st}%
$-equivalent. Hence, $F_{\operatorname*{Comp}\pi}-F_{\operatorname*{Comp}%
\left(  \pi^{\prime}\right)  }\in\mathcal{K}_{\operatorname*{st}}$ (by the
definition of $\mathcal{K}_{\operatorname*{st}}$), so that
$F_{\operatorname*{Comp}\pi}\equiv F_{\operatorname*{Comp}\left(  \pi^{\prime
}\right)  }\operatorname{mod}\mathcal{K}_{\operatorname*{st}}$. Similarly,
$F_{\operatorname*{Comp}\sigma}\equiv F_{\operatorname*{Comp}\left(
\sigma^{\prime}\right)  }\operatorname{mod}\mathcal{K}_{\operatorname*{st}}$.
These two congruences, combined, yield $F_{\operatorname*{Comp}\pi}\left.
\prec\right.  F_{\operatorname*{Comp}\sigma}\equiv F_{\operatorname*{Comp}%
\left(  \pi^{\prime}\right)  }\left.  \prec\right.  F_{\operatorname*{Comp}%
\left(  \sigma^{\prime}\right)  }\operatorname{mod}\mathcal{K}%
_{\operatorname*{st}}$. (Indeed, we can conclude $a\left.  \prec\right.
c\equiv b\left.  \prec\right.  d\operatorname{mod}\mathcal{K}%
_{\operatorname*{st}}$ whenever we have $a\equiv b\operatorname{mod}%
\mathcal{K}_{\operatorname*{st}}$ and $c\equiv d\operatorname{mod}%
\mathcal{K}_{\operatorname*{st}}$; this is because we know that $\mathcal{K}%
_{\operatorname*{st}}$ is a $\left.  \prec\right.  $-ideal of
$\operatorname*{QSym}$.)

From $F_{\operatorname*{Comp}\pi}\left.  \prec\right.  F_{\operatorname*{Comp}%
\sigma}\equiv F_{\operatorname*{Comp}\left(  \pi^{\prime}\right)  }\left.
\prec\right.  F_{\operatorname*{Comp}\left(  \sigma^{\prime}\right)
}\operatorname{mod}\mathcal{K}_{\operatorname*{st}}$, we obtain%
\begin{equation}
\mathbf{st}\left(  F_{\operatorname*{Comp}\pi}\left.  \prec\right.
F_{\operatorname*{Comp}\sigma}\right)  =\mathbf{st}\left(
F_{\operatorname*{Comp}\left(  \pi^{\prime}\right)  }\left.  \prec\right.
F_{\operatorname*{Comp}\left(  \sigma^{\prime}\right)  }\right)
\label{pf.thm.dendri.K.ideal.steq}%
\end{equation}
(since $\mathbf{st}\left(  \mathcal{K}_{\operatorname*{st}}\right)  =0$).

The first claim of Corollary \ref{cor.dendri.4.1} yields%
\[
F_{\operatorname*{Comp}\pi}\left.  \prec\right.  F_{\operatorname*{Comp}%
\sigma}=\sum_{\chi\in S_{\prec}\left(  \pi,\sigma\right)  }%
F_{\operatorname*{Comp}\chi}.
\]
Applying the map $\mathbf{st}$ to both sides of this equality, we find%
\begin{align*}
\mathbf{st}\left(  F_{\operatorname*{Comp}\pi}\left.  \prec\right.
F_{\operatorname*{Comp}\sigma}\right)   &  =\mathbf{st}\left(  \sum_{\chi\in
S_{\prec}\left(  \pi,\sigma\right)  }F_{\operatorname*{Comp}\chi}\right) \\
&  =\sum_{\chi\in S_{\prec}\left(  \pi,\sigma\right)  }\underbrace{\mathbf{st}%
\left(  F_{\operatorname*{Comp}\chi}\right)  }_{=\left[  \operatorname*{st}%
\left(  \operatorname*{Comp}\chi\right)  \right]  =\left[  \operatorname*{st}%
\chi\right]  }=\sum_{\chi\in S_{\prec}\left(  \pi,\sigma\right)  }\left[
\operatorname*{st}\chi\right]  .
\end{align*}
Similarly,%
\[
\mathbf{st}\left(  F_{\operatorname*{Comp}\left(  \pi^{\prime}\right)
}\left.  \prec\right.  F_{\operatorname*{Comp}\left(  \sigma^{\prime}\right)
}\right)  =\sum_{\chi\in S_{\prec}\left(  \pi^{\prime},\sigma^{\prime}\right)
}\left[  \operatorname*{st}\chi\right]  .
\]
But the left-hand sides of the last two equalities are equal (because of
(\ref{pf.thm.dendri.K.ideal.steq})); therefore, the right-hand sides must be
equal as well. In other words,
\[
\sum_{\chi\in S_{\prec}\left(  \pi,\sigma\right)  }\left[  \operatorname*{st}%
\chi\right]  =\sum_{\chi\in S_{\prec}\left(  \pi^{\prime},\sigma^{\prime
}\right)  }\left[  \operatorname*{st}\chi\right]  .
\]
This shows exactly that $\left\{  \operatorname*{st}\left(  \chi\right)
\ \mid\ \chi\in S_{\prec}\left(  \pi,\sigma\right)  \right\}
_{\operatorname*{multi}} =\left\{  \operatorname*{st}\left(  \chi\right)
\ \mid\ \chi\in S_{\prec}\left(  \pi^{\prime},\sigma^{\prime}\right)
\right\}  _{\operatorname*{multi}} $. In other words, $\left\{
\operatorname*{st}\left(  \tau\right)  \ \mid\ \tau\in S_{\prec}\left(
\pi,\sigma\right)  \right\}  _{\operatorname*{multi}} =\left\{
\operatorname*{st}\left(  \tau\right)  \ \mid\ \tau\in S_{\prec}\left(
\pi^{\prime},\sigma^{\prime}\right)  \right\}  _{\operatorname*{multi}} $.
Thus, we have proven that the multiset $\left\{  \operatorname*{st}\left(
\tau\right)  \ \mid\ \tau\in S_{\prec}\left(  \pi,\sigma\right)  \right\}
_{\operatorname*{multi}} $ depends only on $\operatorname*{st}\left(
\pi\right)  $, $\operatorname*{st}\left(  \sigma\right)  $, $\left\vert
\pi\right\vert $ and $\left\vert \sigma\right\vert $. Hence, the statistic
$\operatorname*{st}$ is left-shuffle-compatible. In other words, Statement A
holds. This proves the implication C$\Longrightarrow$A.

Now that we have proven all three implications A$\Longrightarrow$B,
B$\Longrightarrow$C and C$\Longrightarrow$A, the proof of Theorem
\ref{thm.dendri.K.ideal} is complete.
\end{proof}

\begin{proof}
[Proof of Theorem \ref{thm.dendri.K.ideal-R}.]The proof of Theorem
\ref{thm.dendri.K.ideal-R} is analogous to the above proof of Theorem
\ref{thm.dendri.K.ideal}.
\end{proof}

\begin{corollary}
\label{cor.dendri.K.ideal-LR}Let $\operatorname*{st}$ be a permutation
statistic that is LR-shuffle-compatible. Then, $\operatorname*{st}$ is a
shuffle-compatible descent statistic, and the set $\mathcal{K}%
_{\operatorname*{st}}$ is an ideal and a $\left.  \prec\right.  $-ideal and a
$\left.  \succeq\right.  $-ideal of $\operatorname*{QSym}$.
\end{corollary}

\begin{proof}
[Proof of Corollary \ref{cor.dendri.K.ideal-LR} (sketched).] Proposition
\ref{prop.LRcomp.head} yields that $\operatorname*{st}$ is
head-graft-compatible and shuffle-compatible. Proposition
\ref{prop.LRcomp.equivs} shows that $\operatorname*{st}$ is
left-shuffle-compatible, right-shuffle-compatible and head-graft-compatible
(since $\operatorname*{st}$ is LR-shuffle-compatible). Hence, Proposition
\ref{prop.des-stat.hgc} shows that $\operatorname*{st}$ is a descent
statistic. Thus, Theorem \ref{thm.dendri.K.ideal} yields that $\mathcal{K}%
_{\operatorname*{st}}$ is an $\left.  \prec\right.  $-ideal of
$\operatorname*{QSym}$ (since $\operatorname*{st}$ is
left-shuffle-compatible). Likewise, Theorem \ref{thm.dendri.K.ideal-R} yields
that $\mathcal{K}_{\operatorname*{st}}$ is an $\left.  \succeq\right.  $-ideal
of $\operatorname*{QSym}$ (since $\operatorname*{st}$ is
right-shuffle-compatible). Finally, Proposition \ref{prop.K.ideal} yields that
$\mathcal{K}_{\operatorname*{st}}$ is an ideal of $\operatorname*{QSym}$
(since $\operatorname*{st}$ is a shuffle-compatible descent statistic). This
proves Corollary \ref{cor.dendri.K.ideal-LR}.
\end{proof}

A converse of Corollary \ref{cor.dendri.K.ideal-LR} also holds:

\begin{corollary}
\label{cor.dendri.K.ideal-LRi}Let $\operatorname*{st}$ be a descent statistic
such that $\mathcal{K}_{\operatorname*{st}}$ is a $\left.  \prec\right.
$-ideal and a $\left.  \succeq\right.  $-ideal of $\operatorname*{QSym}$.
Then, $\operatorname*{st}$ is LR-shuffle-compatible and shuffle-compatible.
\end{corollary}

\begin{proof}
[Proof of Corollary \ref{cor.dendri.K.ideal-LR} (sketched).] Theorem
\ref{thm.dendri.K.ideal} yields that $\operatorname*{st}$ is
left-shuffle-compatible (since $\mathcal{K}_{\operatorname*{st}}$ is an
$\left.  \prec\right.  $-ideal of $\operatorname*{QSym}$). Likewise, Theorem
\ref{thm.dendri.K.ideal-R} yields that $\operatorname*{st}$ is
right-shuffle-compatible (since $\mathcal{K}_{\operatorname*{st}}$ is an
$\left.  \succeq\right.  $-ideal of $\operatorname*{QSym}$). Hence, Corollary
\ref{cor.LRcomp.two} shows that $\operatorname*{st}$ is LR-shuffle-compatible.
Thus, Proposition \ref{prop.LRcomp.head} yields that $\operatorname*{st}$ is
head-graft-compatible and shuffle-compatible. This proves Corollary
\ref{cor.dendri.K.ideal-LR}.
\end{proof}

As a consequence of Theorem \ref{thm.dendri.K.ideal} and Theorem
\ref{thm.dendri.K.ideal-R}, we can see that any descent statistic that is
weakly left-shuffle-compatible and weakly right-shuffle-compatible must
automatically be shuffle-compatible\footnote{\textit{Proof.} Let
$\operatorname*{st}$ be a descent statistic that is weakly
left-shuffle-compatible and weakly right-shuffle-compatible. We must prove
that $\operatorname*{st}$ is shuffle-compatible.
\par
The implication B$\Longrightarrow$C in Theorem \ref{thm.dendri.K.ideal} shows
that the set $\mathcal{K}_{\operatorname*{st}}$ is an $\left.  \prec\right.
$-ideal of $\operatorname*{QSym}$. Similarly, the set $\mathcal{K}%
_{\operatorname*{st}}$ is an $\left.  \succeq\right.  $-ideal of
$\operatorname*{QSym}$. Hence, Proposition \ref{prop.ideal-crit3} (applied to
$M=\mathcal{K}_{\operatorname*{st}}$) yields that $\mathcal{K}%
_{\operatorname*{st}}$ is an ideal of $\operatorname*{QSym}$. By Proposition
\ref{prop.K.ideal}, this shows that $\operatorname*{st}$ is
shuffle-compatible.}. Note that this is only true for descent statistics! As
far as arbitrary permutation statistics are concerned, this is false; for
example, the number of inversions is weakly left-shuffle-compatible and weakly
right-shuffle-compatible but not shuffle-compatible.

Recall that every permutation statistic that is left-shuffle-compatible and
right-shuffle-compatible must automatically be LR-shuffle-compatible (by
Corollary \ref{cor.LRcomp.two}) and therefore also shuffle-compatible (by
Corollary \ref{cor.LRcomp.back}) and head-graft-compatible (again by Corollary
\ref{cor.LRcomp.back}) and therefore a descent statistic (by Proposition
\ref{prop.des-stat.hgc}).

\begin{corollary}
\label{cor.dendri.Epk}The descent statistic $\operatorname*{Epk}$ is
left-shuffle-compatible and right-shuffle-compatible.
\end{corollary}

Corollary \ref{cor.dendri.Epk} follows by combining Theorem
\ref{thm.LRcomp.Pks} \textbf{(c)} with Theorem \ref{prop.LRcomp.equivs}. But
we can also give a proof using Theorem \ref{thm.dendri.K.ideal}:

\begin{proof}
[Proof of Corollary \ref{cor.dendri.Epk}.]To prove that $\operatorname*{Epk}$
is left-shuffle-compatible, combine Theorem \ref{thm.dendri.K.ideal} with
Theorem \ref{thm.Epk.dend}. Similarly for right-shuffle-compatibility.
\end{proof}

Using Theorem \ref{thm.dendri.4.1}, we can state an analogue of Theorem
\ref{thm.4.3}. Let us first define the notion of dendriform algebras:

\begin{definition}
\textbf{(a)} A \textit{dendriform algebra} over a field $\mathbf{k}$ means a
$\mathbf{k}$-algebra $A$ equipped with two further $\mathbf{k}$-bilinear
binary operations $\left.  \prec\right.  $ and $\left.  \succeq\right.  $
(these are operations, not relations, despite the symbols) from $A\times A$ to
$A$ that satisfy the four rules%
\begin{align*}
a\left.  \prec\right.  b+a\left.  \succeq\right.  b  &  =ab;\\
\left(  a\left.  \prec\right.  b\right)  \left.  \prec\right.  c  &  =a\left.
\prec\right.  \left(  bc\right)  ;\\
\left(  a\left.  \succeq\right.  b\right)  \left.  \prec\right.  c  &
=a\left.  \succeq\right.  \left(  b\left.  \prec\right.  c\right)  ;\\
a\left.  \succeq\right.  \left(  b\left.  \succeq\right.  c\right)   &
=\left(  ab\right)  \left.  \succeq\right.  c
\end{align*}
for all $a,b,c\in A$. (Depending on the situation, it is useful to also impose
a few axioms that relate the unity $1$ of the $\mathbf{k}$-algebra $A$ with
the operations $\left.  \prec\right.  $ and $\left.  \succeq\right.  $. For
example, we could require $1\left.  \prec\right.  a=a$ for each $a\in A$. For
what we are going to do in the following, it does not matter whether we make
this requirement.)

\textbf{(b)} If $A$ and $B$ are two dendriform algebras over $\mathbf{k}$,
then a \textit{dendriform algebra homomorphism} from $A$ to $B$ means a
$\mathbf{k}$-algebra homomorphism $\phi:A\rightarrow B$ preserving the
operations $\left.  \prec\right.  $ and $\left.  \succeq\right.  $ (that is,
satisfying $\phi\left(  a\left.  \prec\right.  b\right)  =\phi\left(
a\right)  \left.  \prec\right.  \phi\left(  b\right)  $ and $\phi\left(
a\left.  \succeq\right.  b\right)  =\phi\left(  a\right)  \left.
\succeq\right.  \phi\left(  b\right)  $ for all $a,b\in A$). (Some authors
only require it to be a $\mathbf{k}$-linear map instead of being a
$\mathbf{k}$-algebra homomorphism; this boils down to the question whether
$\phi\left(  1\right)  $ must be $1$ or not. This does not make a difference
for us here.)
\end{definition}

Thus, $\operatorname*{QSym}$ (with its two operations $\left.  \prec\right.  $
and $\left.  \succeq\right.  $) becomes a dendriform algebra over $\mathbb{Q}$.

Notice that if $A$ and $B$ are two dendriform algebras over $\mathbf{k}$, then
the kernel of any dendriform algebra homomorphism $A\rightarrow B$ is an
$\left.  \prec\right.  $-ideal and a $\left.  \succeq\right.  $-ideal of $A$.
Conversely, if $A$ is a dendriform algebra over $\mathbf{k}$, and $I$ is
simultaneously an $\left.  \prec\right.  $-ideal and a $\left.  \succeq
\right.  $-ideal of $A$, then $A/I$ canonically becomes a dendriform algebra,
and the canonical projection $A\rightarrow A/I$ becomes a dendriform algebra homomorphism.

Therefore, Theorem \ref{thm.dendri.K.ideal} and Theorem
\ref{thm.dendri.K.ideal-R} (and the $\mathcal{A}_{\operatorname*{st}}%
\cong\operatorname*{QSym}/\mathcal{K}_{\operatorname*{st}}$ isomorphism from
Proposition \ref{prop.K.ideal}) yield the following:

\begin{corollary}
\label{cor.dendri.quotient-dendri}If a descent statistic $\operatorname*{st}$
is left-shuffle-compatible and right-shuffle-compatible, then its shuffle
algebra $\mathcal{A}_{\operatorname*{st}}$ canonically becomes a dendriform algebra.
\end{corollary}

We furthermore have the following analogue of Theorem~\ref{thm.4.3}, which
easily follows from Theorem \ref{thm.dendri.K.ideal} and Theorem
\ref{thm.dendri.K.ideal-R}:

\begin{theorem}
\label{thm.dendri.4.3}Let $\operatorname*{st}$ be a descent statistic.

\textbf{(a)} The descent statistic $\operatorname*{st}$ is
left-shuffle-compatible and right-shuffle-compatible if and only if there
exist a dendriform algebra $A$ with
basis $\left(  u_{\alpha}\right)  $ (indexed by $\operatorname*{st}%
$-equivalence classes $\alpha$ of compositions)
and a dendriform algebra homomorphism
$\phi_{\operatorname*{st}}:\operatorname*{QSym} \rightarrow A$
with the property that whenever $\alpha$ is an
$\operatorname{st}$-equivalence class of compositions, we have
\[
\phi_{\operatorname*{st}}\left(  F_{L}\right)  =u_{\alpha}%
\ \ \ \ \ \ \ \ \ \ \text{for each }L\in\alpha.
\]

\textbf{(b)} In this case, the $\mathbb{Q}$-linear map%
\[
\mathcal{A}_{\operatorname*{st}}\rightarrow A,\ \ \ \ \ \ \ \ \ \ \left[
\pi\right]  _{\operatorname*{st}}\mapsto u_{\alpha},
\]
where $\alpha$ is the $\operatorname*{st}$-equivalence class of the
composition $\operatorname*{Comp}\pi$, is an isomorphism of dendriform
algebras $\mathcal{A}_{\operatorname*{st}}\rightarrow A$.
\end{theorem}

\begin{question}
Can the $\mathbb{Q}$-algebra $\operatorname*{Pow}\mathcal{N}$ from Definition
\ref{def.GammaZ} be endowed with two binary operations $\left.  \prec\right.
$ and $\left.  \succeq\right.  $ that make it into a dendriform algebra? Can
we then find an analogue of Proposition \ref{prop.prod1} along the following lines?

Let $\left(  P,\gamma\right)  $, $\left(  Q,\delta\right)  $ and $\left(
P\sqcup Q,\varepsilon\right)  $ be as in Proposition \ref{prop.prod1}. Assume
that each of the posets $P$ and $Q$ has a minimum element; denote these
elements by $\min P$ and $\min Q$, respectively. Define two posets $P\left.
\prec\right.  Q$ and $P\left.  \succeq\right.  Q$ as in Proposition
\ref{prop.dendri.4.1.lem}. Then, we hope to have%
\begin{align*}
\Gamma_{\mathcal{Z}}\left(  P,\gamma\right)  \left.  \prec\right.
\Gamma_{\mathcal{Z}}\left(  Q,\delta\right)   &  =\Gamma_{\mathcal{Z}}\left(
P\left.  \prec\right.  Q,\varepsilon\right)  \ \ \ \ \ \ \ \ \ \ \text{and}\\
\Gamma_{\mathcal{Z}}\left(  P,\gamma\right)  \left.  \succeq\right.
\Gamma_{\mathcal{Z}}\left(  Q,\delta\right)   &  =\Gamma_{\mathcal{Z}}\left(
P\left.  \succeq\right.  Q,\varepsilon\right)  .
\end{align*}


Ideally, this would be a generalization of Proposition
\ref{prop.dendri.4.1.lem}.
\end{question}

\subsection{Criteria for $\mathcal{K}_{\operatorname*{st}}$ to be a stack
ideal}

We have so far studied the combinatorial significance of when the kernel
$\mathcal{K}_{\operatorname*{st}}$ of a statistic $\operatorname*{st}$ is a
$\left.  \prec\right.  $-ideal or a $\left.  \succeq\right.  $-ideal of
$\operatorname*{QSym}$. What about $\bel$-ideals and $\tvi$-ideals? It turns
out that the answer to this question is given (on the level of compositions)
by the following (easily verified) proposition:

\begin{proposition}
\label{prop.bel-tvi-comp}Let $\operatorname*{st}$ be a descent statistic.

\textbf{(a)} The set $\mathcal{K}_{\operatorname*{st}}$ is a left $\bel$-ideal
of $\operatorname*{QSym}$ if and only if $\operatorname*{st}$ has the
following property: If $J$ and $K$ are two $\operatorname*{st}$-equivalent
nonempty compositions, and if $G$ is any nonempty composition, then $G\odot J$
and $G\odot K$ are $\operatorname*{st}$-equivalent.

\textbf{(b)} The set $\mathcal{K}_{\operatorname*{st}}$ is a right
$\bel$-ideal of $\operatorname*{QSym}$ if and only if $\operatorname*{st}$ has
the following property: If $J$ and $K$ are two $\operatorname*{st}$-equivalent
nonempty compositions, and if $G$ is any nonempty composition, then $J\odot G$
and $K\odot G$ are $\operatorname*{st}$-equivalent.

\textbf{(c)} The set $\mathcal{K}_{\operatorname*{st}}$ is a left $\tvi$-ideal
of $\operatorname*{QSym}$ if and only if $\operatorname*{st}$ has the
following property: If $J$ and $K$ are two $\operatorname*{st}$-equivalent
nonempty compositions, and if $G$ is any nonempty composition, then $\left[
G,J\right]  $ and $\left[  G,K\right]  $ are $\operatorname*{st}$-equivalent.

\textbf{(d)} The set $\mathcal{K}_{\operatorname*{st}}$ is a right
$\tvi$-ideal of $\operatorname*{QSym}$ if and only if $\operatorname*{st}$ has
the following property: If $J$ and $K$ are two $\operatorname*{st}$-equivalent
nonempty compositions, and if $G$ is any nonempty composition, then $\left[
J,G\right]  $ and $\left[  K,G\right]  $ are $\operatorname*{st}$-equivalent.
\end{proposition}

Proposition \ref{prop.bel-tvi-comp} allows us to give a new proof of Theorem
\ref{thm.Epk.dend}, which makes no use of Proposition \ref{prop.K.Epk.F}.
Instead, it will rely on analyzing $\operatorname*{Epk}\left(  \left[
A,B\right]  \right)  $ and $\operatorname*{Epk}\left(  A\odot B\right)  $ when
$A$ and $B$ are two nonempty compositions.

First, we introduce a notation: If $S$ is a set of integers, and $p$ is an
integer, then $S+p$ shall denote the set $\left\{  s+p\ \mid\ s\in S\right\}
$.

We shall use the following simple lemma:

\begin{lemma}
\label{lem.Epk.AB}Let $A$ and $B$ be two nonempty compositions. Let
$n=\left\vert A\right\vert $.

\textbf{(a)} We have $\operatorname*{Epk}\left(  \left[  A,B\right]  \right)
=\left(  \operatorname*{Epk}A\right)  \cup\left(  \left(  \operatorname*{Epk}%
B+n\right)  \setminus\left\{  n+1\right\}  \right)  $.

\textbf{(b)} We have $\operatorname*{Epk}\left(  A\odot B\right)  =\left(
\left(  \operatorname*{Epk}A\right)  \setminus\left\{  n\right\}  \right)
\cup\left(  \operatorname*{Epk}B+n\right)  $.
\end{lemma}

\begin{proof}
[Proof of Lemma \ref{lem.Epk.AB}.]Let $m=\left\vert B\right\vert $. Consider
any $n$-permutation $\alpha=\left(  \alpha_{1},\alpha_{2},\ldots,\alpha
_{n}\right)  $ satisfying $\operatorname*{Comp}\alpha=A$. (Such $\alpha$
exists, since $n=\left\vert A\right\vert $.) Consider any $m$-permutation
$\beta=\left(  \beta_{1},\beta_{2},\ldots,\beta_{m}\right)  $ satisfying
$\operatorname*{Comp}\beta=B$. (Such $\beta$ exists, since $m=\left\vert
B\right\vert $.) From $\operatorname*{Comp}\alpha=A$, we obtain
$\operatorname*{Epk}\alpha=\operatorname*{Epk}A$. Similarly,
$\operatorname*{Epk}\beta=\operatorname*{Epk}B$.

\textbf{(a)} WLOG assume that $\alpha_{i}>\beta_{j}$ for all $i\in\left[
n\right]  $ and $j\in\left[  m\right]  $. (Indeed, we can achieve this by
choosing a positive integer $g$ that is larger than each entry of $\beta$, and
adding $g$ to each entry of $\alpha$.) Thus, in particular, the entries of
$\alpha$ are distinct from the entries of $\beta$. Also, $\alpha_{n}>\beta
_{1}$ (since $\alpha_{i}>\beta_{j}$ for all $i\in\left[  n\right]  $ and
$j\in\left[  m\right]  $).

Let $\gamma$ be the $\left(  n+m\right)  $-permutation $\left(  \alpha
_{1},\alpha_{2},\ldots,\alpha_{n},\beta_{1},\beta_{2},\ldots,\beta_{m}\right)
$. Then, the descents of $\gamma$ are obtained as follows:

\begin{itemize}
\item Each descent of $\alpha$ is a descent of $\gamma$.

\item The number $n$ is a descent of $\gamma$ (since $\alpha_{n}>\beta_{1}$).

\item Adding $n$ to each descent of $\beta$ yields a descent of $\gamma$ (that
is, if $i$ is a descent of $\beta$, then $i+n$ is a descent of $\gamma$).
\end{itemize}

These are all the descents of $\gamma$. Thus,%
\[
\operatorname*{Des}\gamma=\operatorname*{Des}\alpha\cup\left\{  n\right\}
\cup\left(  \operatorname*{Des}\beta+n\right)  .
\]
Hence,%
\begin{align*}
\operatorname*{Comp}\left(  \operatorname*{Des}\gamma\right)   &
=\operatorname*{Comp}\left(  \operatorname*{Des}\alpha\cup\left\{  n\right\}
\cup\left(  \operatorname*{Des}\beta+n\right)  \right) \\
&  =\left[  \operatorname*{Comp}\left(  \operatorname*{Des}\alpha\right)
,\operatorname*{Comp}\left(  \operatorname*{Des}\beta\right)  \right]
\end{align*}
(because of how $\operatorname*{Comp}S$ is defined for a set $S$). Since
$\operatorname*{Comp}\left(  \operatorname*{Des}\pi\right)
=\operatorname*{Comp}\pi$ for any permutation $\pi$, this rewrites as%
\[
\operatorname*{Comp}\gamma=\left[  \operatorname*{Comp}\alpha
,\operatorname*{Comp}\beta\right]  .
\]
In view of $\operatorname*{Comp}\alpha=A$ and $\operatorname*{Comp}\beta=B$,
this rewrites as $\operatorname*{Comp}\gamma=\left[  A,B\right]  $. Thus,
$\operatorname*{Epk}\left(  \left[  A,B\right]  \right)  =\operatorname*{Epk}%
\gamma$.

On the other hand, recall again that $\gamma=\left(  \alpha_{1},\alpha
_{2},\ldots,\alpha_{n},\beta_{1},\beta_{2},\ldots,\beta_{m}\right)  $ and
$\alpha_{n}>\beta_{1}$. Thus, the exterior peaks of $\gamma$ are obtained as follows:

\begin{itemize}
\item Each exterior peak of $\alpha$ is an exterior peak of $\gamma$. (This
includes $n$, if $n$ is an exterior peak of $\alpha$, because $\alpha
_{n}>\beta_{1}$.)

\item Adding $n$ to each exterior peak of $\beta$ yields an exterior peak of
$\gamma$, except for the number $n+1$, which is \textbf{not} an exterior peak
of $\gamma$ (since $\alpha_{n}>\beta_{1}$).
\end{itemize}

These are all the exterior peaks of $\gamma$. Thus,%
\[
\operatorname*{Epk}\gamma=\left(  \operatorname*{Epk}\alpha\right)
\cup\left(  \left(  \operatorname*{Epk}\beta+n\right)  \setminus\left\{
n+1\right\}  \right)  .
\]
In view of $\operatorname*{Epk}\alpha=\operatorname*{Epk}A$,
$\operatorname*{Epk}\beta=\operatorname*{Epk}B$ and $\operatorname*{Epk}%
\gamma=\operatorname*{Epk}\left(  \left[  A,B\right]  \right)  $, this
rewrites as
\[
\operatorname*{Epk}\left(  \left[  A,B\right]  \right)  =\left(
\operatorname*{Epk}A\right)  \cup\left(  \left(  \operatorname*{Epk}%
B+n\right)  \setminus\left\{  n+1\right\}  \right)  .
\]
This proves Lemma \ref{lem.Epk.AB} \textbf{(a)}.

\textbf{(b)} WLOG assume that $\alpha_{i}<\beta_{j}$ for all $i\in\left[
n\right]  $ and $j\in\left[  m\right]  $. (Indeed, we can achieve this by
choosing a positive integer $g$ that is larger than each entry of $\alpha$,
and adding $g$ to each entry of $\beta$.) Thus, in particular, the entries of
$\alpha$ are distinct from the entries of $\beta$. Also, $\alpha_{n}<\beta
_{1}$ (since $\alpha_{i}<\beta_{j}$ for all $i\in\left[  n\right]  $ and
$j\in\left[  m\right]  $).

Let $\gamma$ be the $\left(  n+m\right)  $-permutation $\left(  \alpha
_{1},\alpha_{2},\ldots,\alpha_{n},\beta_{1},\beta_{2},\ldots,\beta_{m}\right)
$. Then, the descents of $\gamma$ are obtained as follows:

\begin{itemize}
\item Each descent of $\alpha$ is a descent of $\gamma$.

\item Adding $n$ to each descent of $\beta$ yields a descent of $\gamma$ (that
is, if $i$ is a descent of $\beta$, then $i+n$ is a descent of $\gamma$).
\end{itemize}

These are all the descents of $\gamma$ (in particular, $n$ is \textbf{not} a
descent of $\gamma$, since $\alpha_{n}<\beta_{1}$). Thus,%
\[
\operatorname*{Des}\gamma=\operatorname*{Des}\alpha\cup\left(
\operatorname*{Des}\beta+n\right)  .
\]
Hence,%
\begin{align*}
\operatorname*{Comp}\left(  \operatorname*{Des}\gamma\right)   &
=\operatorname*{Comp}\left(  \operatorname*{Des}\alpha\cup\left(
\operatorname*{Des}\beta+n\right)  \right) \\
&  =\operatorname*{Comp}\left(  \operatorname*{Des}\alpha\right)
\odot\operatorname*{Comp}\left(  \operatorname*{Des}\beta\right)
\end{align*}
(because of how $\operatorname*{Comp}S$ is defined for a set $S$). Since
$\operatorname*{Comp}\left(  \operatorname*{Des}\pi\right)
=\operatorname*{Comp}\pi$ for any permutation $\pi$, this rewrites as%
\[
\operatorname*{Comp}\gamma=\operatorname*{Comp}\alpha\odot\operatorname*{Comp}%
\beta.
\]
In view of $\operatorname*{Comp}\alpha=A$ and $\operatorname*{Comp}\beta=B$,
this rewrites as $\operatorname*{Comp}\gamma=A\odot B$. Thus,
$\operatorname*{Epk}\left(  A\odot B\right)  =\operatorname*{Epk}\gamma$.

On the other hand, recall again that $\gamma=\left(  \alpha_{1},\alpha
_{2},\ldots,\alpha_{n},\beta_{1},\beta_{2},\ldots,\beta_{m}\right)  $ and
$\alpha_{n}<\beta_{1}$. Thus, the exterior peaks of $\gamma$ are obtained as follows:

\begin{itemize}
\item Each exterior peak of $\alpha$ is an exterior peak of $\gamma$, except
for the number $n$, which is \textbf{not} an exterior peak of $\gamma$ (since
$\alpha_{n}<\beta_{1}$).

\item Adding $n$ to each exterior peak of $\beta$ yields an exterior peak of
$\gamma$. (This includes $n+1$, if $1$ is an exterior peak of $\beta$, because
$\alpha_{n}<\beta_{1}$.)
\end{itemize}

These are all the exterior peaks of $\gamma$. Thus,%
\[
\operatorname*{Epk}\gamma=\left(  \left(  \operatorname*{Epk}\alpha\right)
\setminus\left\{  n\right\}  \right)  \cup\left(  \operatorname*{Epk}%
\beta+n\right)  .
\]
In view of $\operatorname*{Epk}\alpha=\operatorname*{Epk}A$,
$\operatorname*{Epk}\beta=\operatorname*{Epk}B$ and $\operatorname*{Epk}%
\gamma=\operatorname*{Epk}\left(  A\odot B\right)  $, this rewrites as
\[
\operatorname*{Epk}\left(  A\odot B\right)  =\left(  \left(
\operatorname*{Epk}A\right)  \setminus\left\{  n\right\}  \right)  \cup\left(
\operatorname*{Epk}B+n\right)  .
\]
This proves Lemma \ref{lem.Epk.AB} \textbf{(b)}.
\end{proof}

We can now easily prove Theorem \ref{thm.Epk.dend} again:

\begin{proof}
[Second proof of Theorem \ref{thm.Epk.dend} (sketched).]Let
$A=\operatorname*{QSym}$. Corollary \ref{cor.Epk.ideal} shows that
$\mathcal{K}_{\operatorname*{Epk}}$ is an ideal of $\operatorname*{QSym}$.

We now argue the following claim:\footnote{These claims are numbered Claim 2,
Claim 4, Claim 6 and Claim 8, in order to match the numbering of the
corresponding claims in the first proof of Theorem \ref{thm.Epk.dend} above.}

\begin{statement}
\textit{Claim 2:} We have $A\tvi\mathcal{K}_{\operatorname*{Epk}}%
\subseteq\mathcal{K}_{\operatorname*{Epk}}$.
\end{statement}

[\textit{Proof of Claim 2:} We must show that $\mathcal{K}%
_{\operatorname*{Epk}}$ is a left $\tvi$-ideal of $\operatorname*{QSym}$.
According to Proposition \ref{prop.bel-tvi-comp} \textbf{(c)}, this boils down
to proving that if $J$ and $K$ are two $\operatorname*{Epk}$-equivalent
nonempty compositions, and if $G$ is any nonempty composition, then $\left[
G,J\right]  $ and $\left[  G,K\right]  $ are $\operatorname*{Epk}$-equivalent.

So let $J$ and $K$ be two $\operatorname*{Epk}$-equivalent nonempty
compositions. Thus, $\left\vert J\right\vert =\left\vert K\right\vert >0$ and
$\operatorname*{Epk}J=\operatorname*{Epk}K$.

Define a positive integer $n$ by $n=\left\vert G\right\vert $. Lemma
\ref{lem.Epk.AB} \textbf{(a)} (applied to $A=G$ and $B=J$) yields
\begin{equation}
\operatorname*{Epk}\left(  \left[  G,J\right]  \right)  =\left(
\operatorname*{Epk}G\right)  \cup\left(  \left(  \operatorname*{Epk}%
J+n\right)  \setminus\left\{  n+1\right\}  \right)  .
\label{pf.thm.Epk.dend.2nd.c2.pf.4}%
\end{equation}
Similarly,%
\begin{equation}
\operatorname*{Epk}\left(  \left[  G,K\right]  \right)  =\left(
\operatorname*{Epk}G\right)  \cup\left(  \left(  \operatorname*{Epk}%
K+n\right)  \setminus\left\{  n+1\right\}  \right)  .
\label{pf.thm.Epk.dend.2nd.c2.pf.5}%
\end{equation}
The right hand sides of (\ref{pf.thm.Epk.dend.2nd.c2.pf.4}) and
(\ref{pf.thm.Epk.dend.2nd.c2.pf.5}) are equal (since $\operatorname*{Epk}%
J=\operatorname*{Epk}K$). Hence, the left hand sides are equal as well. In
other words, $\operatorname*{Epk}\left(  \left[  G,J\right]  \right)
=\operatorname*{Epk}\left(  \left[  G,K\right]  \right)  $. Combining this
with%
\[
\left\vert \left[  G,J\right]  \right\vert =\left\vert G\right\vert
+\underbrace{\left\vert J\right\vert }_{=\left\vert K\right\vert }=\left\vert
G\right\vert +\left\vert K\right\vert =\left\vert \left[  G,K\right]
\right\vert ,
\]
we conclude that $\left[  G,J\right]  $ and $\left[  G,K\right]  $ are
$\operatorname*{Epk}$-equivalent. As we have said, this concludes the proof of
Claim 2.]

Similarly to Claim 2, we can show the following three claims:

\begin{statement}
\textit{Claim 4:} We have $\mathcal{K}_{\operatorname*{Epk}}\tvi
A\subseteq\mathcal{K}_{\operatorname*{Epk}}$.
\end{statement}

\begin{statement}
\textit{Claim 6:} We have $A\bel\mathcal{K}_{\operatorname*{Epk}}%
\subseteq\mathcal{K}_{\operatorname*{Epk}}$.
\end{statement}

\begin{statement}
\textit{Claim 8:} We have $\mathcal{K}_{\operatorname*{Epk}}\bel
A\subseteq\mathcal{K}_{\operatorname*{Epk}}$.
\end{statement}

(Of course, in proving Claims 4, 6 and 8, we need to use the other three parts
of Proposition \ref{prop.bel-tvi-comp} instead of Proposition
\ref{prop.bel-tvi-comp} \textbf{(c)}, and we occasionally need to use Lemma
\ref{lem.Epk.AB} \textbf{(b)} instead of Lemma \ref{lem.Epk.AB} \textbf{(a)}.)

Combining Claim 2 and Claim 4, we conclude that $\mathcal{K}%
_{\operatorname*{Epk}}$ is a $\tvi$-ideal of $A$.

Combining Claim 6 and Claim 8, we conclude that $\mathcal{K}%
_{\operatorname*{Epk}}$ is a $\bel$-ideal of $A$.

Thus, Theorem \ref{thm.ideal-crit2} \textbf{(c)} (applied to $M=\mathcal{K}%
_{\operatorname*{Epk}}$) shows that $\mathcal{K}_{\operatorname*{Epk}}$ is a
$\left.  \prec\right.  $-ideal and a $\left.  \succeq\right.  $-ideal of
$\operatorname*{QSym}$. This proves Theorem \ref{thm.Epk.dend} again.
\end{proof}

\subsection{\label{subsect.dendri.other-stats}Left/right-shuffle-compatibility
of other statistics}

Let us now briefly analyze the kernels $\mathcal{K}_{\operatorname*{st}}$ of
some other descent statistics, following the same approach that we took in our
above second proof of Theorem \ref{thm.Epk.dend} again. Much of what follows
will merely reproduce results from Section \ref{sect.LR}.

\subsubsection{The descent set $\operatorname*{Des}$}

First of all, the following is obvious:

\begin{proposition}
\label{prop.Des-set.dend}The ideal $\mathcal{K}_{\operatorname*{Des}}$ of
$\operatorname*{QSym}$ is the trivial ideal $0$, and is a $\tvi$-ideal, a
$\bel$-ideal, a $\left.  \prec\right.  $-ideal and a $\left.  \succeq\right.
$-ideal of $\operatorname*{QSym}$.
\end{proposition}

\begin{corollary}
\label{cor.dendri.Des-set}The descent statistic $\operatorname*{Des}$ is
left-shuffle-compatible and right-shuffle-compatible.
\end{corollary}

\begin{proof}
[Proof of Corollary \ref{cor.dendri.Des-set}.]Corollary
\ref{cor.dendri.Des-set} can be derived from Proposition
\ref{prop.Des-set.dend} in the same way as Corollary \ref{cor.dendri.Epk} was
derived from Theorem \ref{thm.Epk.dend}.
\end{proof}

\subsubsection{The descent number $\operatorname*{des}$}

The permutation statistic $\operatorname*{des}$ (called the \textit{descent
number}) is defined as follows: For each permutation $\pi$, we set
$\operatorname*{des}\pi=\left\vert \operatorname*{Des}\pi\right\vert $ (that
is, $\operatorname*{des}\pi$ is the number of all descents of $\pi$). It was
proven in \cite[Theorem 4.6 \textbf{(a)}]{part1} that this statistic
$\operatorname*{des}$ is shuffle-compatible. Furthermore, $\operatorname*{des}%
$ is clearly a descent statistic. Hence, Proposition \ref{prop.K.ideal}
(applied to $\operatorname*{st}=\operatorname*{des}$) shows that
$\mathcal{K}_{\operatorname*{des}}$ is an ideal of $\operatorname*{QSym}$. We
now claim the following:

\begin{proposition}
\label{prop.des.dend}The ideal $\mathcal{K}_{\operatorname*{des}}$ of
$\operatorname*{QSym}$ is a $\tvi$-ideal, a $\bel$-ideal, a $\left.
\prec\right.  $-ideal and a $\left.  \succeq\right.  $-ideal of
$\operatorname*{QSym}$.
\end{proposition}

\begin{corollary}
\label{cor.dendri.des}The descent statistic $\operatorname*{des}$ is
left-shuffle-compatible and right-shuffle-compatible.
\end{corollary}

The proofs rely on the following fact (similar to Lemma \ref{lem.Epk.AB}):

\begin{lemma}
\label{lem.des.AB}Let $A$ and $B$ be two nonempty compositions. Let
$n=\left\vert A\right\vert $.

\textbf{(a)} We have $\operatorname*{des}\left(  \left[  A,B\right]  \right)
=\operatorname*{des}A+\operatorname*{des}B+1$.

\textbf{(b)} We have $\operatorname*{des}\left(  A\odot B\right)
=\operatorname*{des}A+\operatorname*{des}B$.
\end{lemma}

\begin{proof}
[Proof of Lemma \ref{lem.des.AB}.]If $I$ is a nonempty composition, then
$\operatorname*{des}I$ equals the length of $I$ minus $1$. Lemma
\ref{lem.des.AB} follows easily from this.
\end{proof}

\begin{proof}
[Proof of Proposition \ref{prop.des.dend}.]Analogous to the above second proof
of Theorem \ref{thm.Epk.dend}, but using Lemma \ref{lem.des.AB} instead of
Lemma \ref{lem.Epk.AB}.
\end{proof}

\begin{proof}
[Proof of Corollary \ref{cor.dendri.des}.]Corollary \ref{cor.dendri.des} can
be derived from Proposition \ref{prop.des.dend} in the same way as Corollary
\ref{cor.dendri.Epk} was derived from Theorem \ref{thm.Epk.dend}.
\end{proof}

\subsubsection{The major index $\operatorname*{maj}$}

The permutation statistic $\operatorname*{maj}$ (called the \textit{major
index}) is defined as follows: For each permutation $\pi$, we set
$\operatorname*{maj}\pi=\sum_{i\in\operatorname*{Des}\pi}i$ (that is,
$\operatorname*{maj}\pi$ is the sum of all descents of $\pi$). It was proven
in \cite[Theorem 3.1 \textbf{(a)}]{part1} that this statistic
$\operatorname*{maj}$ is shuffle-compatible. Furthermore, $\operatorname*{maj}%
$ is clearly a descent statistic. Hence, Proposition \ref{prop.K.ideal}
(applied to $\operatorname*{st}=\operatorname*{maj}$) shows that
$\mathcal{K}_{\operatorname*{maj}}$ is an ideal of $\operatorname*{QSym}$. We
now claim the following:

\begin{proposition}
\label{prop.maj.dend}The ideal $\mathcal{K}_{\operatorname*{maj}}$ of
$\operatorname*{QSym}$ is a right $\tvi$-ideal and a right $\bel$-ideal, but
neither a $\left.  \prec\right.  $-ideal nor a $\left.  \succeq\right.
$-ideal of $\operatorname*{QSym}$.
\end{proposition}

\begin{corollary}
\label{cor.dendri.maj}The descent statistic $\operatorname*{maj}$ is neither
left-shuffle-compatible nor right-shuffle-compatible.
\end{corollary}

The proofs rely on the following fact (similar to Lemma \ref{lem.Epk.AB}):

\begin{lemma}
\label{lem.maj.AB}Let $A$ and $B$ be two nonempty compositions. Let
$n=\left\vert A\right\vert $.

\textbf{(a)} We have $\operatorname*{maj}\left(  \left[  A,B\right]  \right)
=\operatorname*{maj}A+\operatorname*{maj}B+n\cdot\left(  \operatorname*{des}%
B+1\right)  $.

\textbf{(b)} We have $\operatorname*{maj}\left(  A\odot B\right)
=\operatorname*{maj}A+\operatorname*{maj}B+n\cdot\operatorname*{des}B$.
\end{lemma}

\begin{proof}
[Proof of Lemma \ref{lem.maj.AB}.]If $I=\left(  i_{1},i_{2},\ldots
,i_{k}\right)  $ is a nonempty composition, then
\begin{align*}
\operatorname*{maj}I  &  =i_{1}+\left(  i_{1}+i_{2}\right)  +\left(
i_{1}+i_{2}+i_{3}\right)  +\cdots+\left(  i_{1}+i_{2}+\cdots+i_{k-1}\right) \\
&  =\left(  k-1\right)  i_{1}+\left(  k-2\right)  i_{2}+\cdots+\left(
k-k\right)  i_{k}.
\end{align*}
Lemma \ref{lem.maj.AB} follows easily from this.
\end{proof}

\begin{proof}
[Proof of Proposition \ref{prop.maj.dend}.]To prove that $\mathcal{K}%
_{\operatorname*{maj}}$ is a right $\tvi$-ideal of $\operatorname*{QSym}$, we
proceed as in the proof of Claim 2 in the second proof of Theorem
\ref{thm.Epk.dend}, but using Lemma \ref{lem.maj.AB} instead of Lemma
\ref{lem.Epk.AB}. Similarly, we can show that $\mathcal{K}%
_{\operatorname*{maj}}$ is a right $\bel$-ideal of $\operatorname*{QSym}$.

To prove that $\mathcal{K}_{\operatorname*{maj}}$ is not a $\left.
\prec\right.  $-ideal of $\operatorname*{QSym}$ (and not even a left $\left.
\prec\right.  $-ideal of $\operatorname*{QSym}$), it suffices to find some
$m\in\mathcal{K}_{\operatorname*{maj}}$ and some $a\in\operatorname*{QSym}$
such that $a\left.  \prec\right.  m\notin\mathcal{K}_{\operatorname*{maj}}$.
For example, we can take $m=F_{\left(  1,1,2\right)  }-F_{\left(  3,1\right)
}$ and $a=F_{\left(  1\right)  }$; then, $a\left.  \prec\right.  m=F_{\left(
1,1,1,2\right)  }-F_{\left(  1,3,1\right)  }\notin\mathcal{K}%
_{\operatorname*{maj}}$. The same values of $m$ and $a$ also satisfy $a\left.
\succeq\right.  m\notin\mathcal{K}_{\operatorname*{maj}}$, $m\left.
\prec\right.  a\notin\mathcal{K}_{\operatorname*{maj}}$ and $m\left.
\succeq\right.  a\notin\mathcal{K}_{\operatorname*{maj}}$; thus,
$\mathcal{K}_{\operatorname*{maj}}$ is not a $\left.  \succeq\right.  $-ideal
of $\operatorname*{QSym}$ either. Proposition \ref{prop.maj.dend} is now proven.
\end{proof}

\begin{proof}
[Proof of Corollary \ref{cor.dendri.maj}.]Again, this follows from Proposition
\ref{prop.maj.dend}.
\end{proof}

\subsubsection{The joint statistic $\left(  \operatorname*{des}%
,\operatorname*{maj}\right)  $}

The next permutation statistic we shall study is the so-called joint statistic
$\left(  \operatorname*{des},\operatorname*{maj}\right)  $. This statistic is
defined as the permutation statistic that sends each permutation $\pi$ to the
ordered pair $\left(  \operatorname*{des}\pi,\operatorname*{maj}\pi\right)  $.
(Calling it $\left(  \operatorname*{des},\operatorname*{maj}\right)  $ is thus
a slight abuse of notation.) It was proven in \cite[Theorem 4.5 \textbf{(a)}%
]{part1} that this statistic $\left(  \operatorname*{des},\operatorname*{maj}%
\right)  $ is shuffle-compatible. Furthermore, $\left(  \operatorname*{des}%
,\operatorname*{maj}\right)  $ is clearly a descent statistic. Hence,
Proposition \ref{prop.K.ideal} (applied to $\operatorname*{st}=\left(
\operatorname*{des},\operatorname*{maj}\right)  $) shows that $\mathcal{K}%
_{\left(  \operatorname*{des},\operatorname*{maj}\right)  }$ is an ideal of
$\operatorname*{QSym}$. We now claim the following:

\begin{proposition}
\label{prop.(des,maj).dend}The ideal $\mathcal{K}_{\left(  \operatorname*{des}%
,\operatorname*{maj}\right)  }$ of $\operatorname*{QSym}$ is a $\tvi$-ideal, a
$\bel$-ideal, a $\left.  \prec\right.  $-ideal and a $\left.  \succeq\right.
$-ideal of $\operatorname*{QSym}$.
\end{proposition}

\begin{corollary}
\label{cor.dendri.(des,maj)}The descent statistic $\left(  \operatorname*{des}%
,\operatorname*{maj}\right)  $ is left-shuffle-compatible and right-shuffle-compatible.
\end{corollary}

\begin{proof}
[Proof of Proposition \ref{prop.(des,maj).dend}.]Analogous to the above second
proof of Theorem \ref{thm.Epk.dend}, but using Lemma \ref{lem.des.AB} together
with Lemma \ref{lem.maj.AB} instead of Lemma \ref{lem.Epk.AB}.
\end{proof}

\begin{proof}
[Proof of Corollary \ref{cor.dendri.(des,maj)}.]Corollary
\ref{cor.dendri.(des,maj)} can be derived from Proposition
\ref{prop.(des,maj).dend} in the same way as Corollary \ref{cor.dendri.Epk}
was derived from Theorem \ref{thm.Epk.dend}.
\end{proof}

\subsubsection{The left peak set $\operatorname*{Lpk}$}

Recall the permutation statistic $\operatorname*{Lpk}$ (the left peak set)
defined in Definition \ref{def.Des-et-al}. It was proven in \cite[Theorem 4.9
\textbf{(a)}]{part1} that this statistic $\operatorname*{Lpk}$ is
shuffle-compatible. Furthermore, $\operatorname*{Lpk}$ is clearly a descent
statistic. Hence, Proposition \ref{prop.K.ideal} (applied to
$\operatorname*{st}=\operatorname*{Lpk}$) shows that $\mathcal{K}%
_{\operatorname*{Lpk}}$ is an ideal of $\operatorname*{QSym}$. We now claim
the following:

\begin{proposition}
\label{prop.Lpk.dend}The ideal $\mathcal{K}_{\operatorname*{Lpk}}$ of
$\operatorname*{QSym}$ is a left $\tvi$-ideal, a $\bel$-ideal, a $\left.
\prec\right.  $-ideal and a $\left.  \succeq\right.  $-ideal of
$\operatorname*{QSym}$.
\end{proposition}

\begin{corollary}
\label{cor.dendri.Lpk}The descent statistic $\operatorname*{Lpk}$ is
left-shuffle-compatible and right-shuffle-compatible.
\end{corollary}

The proofs rely on the following fact (similar to Lemma \ref{lem.Epk.AB}):

\begin{lemma}
\label{lem.Lpk.AB}Let $A$ and $B$ be two nonempty compositions. Let
$n=\left\vert A\right\vert $.

\textbf{(a)} We have $\operatorname*{Lpk}\left(  \left[  A,B\right]  \right)
=\left(  \operatorname*{Lpk}A\right)  \cup\left(  \left(  \operatorname*{Lpk}%
B+n\right)  \setminus\left\{  n+1\right\}  \right)  \cup\left\{
n\ \mid\ n-1\notin\operatorname*{Des}A\right\}  $.

\textbf{(b)} We have $\operatorname*{Lpk}\left(  A\odot B\right)  =\left(
\operatorname*{Lpk}A\right)  \cup\left(  \operatorname*{Lpk}B+n\right)  $.
\end{lemma}

\begin{proof}
[Proof of Lemma \ref{lem.Lpk.AB}.]Not unlike the proof of Lemma
\ref{lem.Epk.AB} (but left to the reader).
\end{proof}

\begin{proof}
[Proof of Proposition \ref{prop.Lpk.dend}.]Analogous to the above second proof
of Theorem \ref{thm.Epk.dend}, but using Lemma \ref{lem.Lpk.AB} instead of
Lemma \ref{lem.Epk.AB}. This time, however, the analogue of Claim 4 will be
false (i.e., we don't have $\mathcal{K}_{\operatorname*{Lpk}}\tvi
A\subseteq\mathcal{K}_{\operatorname*{Lpk}}$), because the formula for
$\operatorname*{Lpk}\left(  \left[  A,B\right]  \right)  $ in Lemma
\ref{lem.Lpk.AB} \textbf{(a)} depends on $\operatorname*{Des}A$. Thus,
$\mathcal{K}_{\operatorname*{Lpk}}$ is merely a left $\tvi$-ideal, not a
$\tvi$-ideal. (But this does not prevent us from applying Theorem
\ref{thm.ideal-crit2} \textbf{(c)}, because that theorem does not require
$M\tvi A\subseteq M$.)
\end{proof}

\begin{proof}
[Proof of Corollary \ref{cor.dendri.Lpk}.]Corollary \ref{cor.dendri.Lpk} can
be derived from Proposition \ref{prop.Lpk.dend} in the same way as Corollary
\ref{cor.dendri.Epk} was derived from Theorem \ref{thm.Epk.dend}.
\end{proof}

\subsubsection{The right peak set $\operatorname*{Rpk}$}

Recall the permutation statistic $\operatorname*{Rpk}$ (the right peak set)
defined in Definition \ref{def.Des-et-al}. It follows from \cite[Theorem 4.9
\textbf{(a)} and Theorem 3.5]{part1} that this statistic $\operatorname*{Rpk}$
is shuffle-compatible (since $\operatorname{Lpk}$ and $\operatorname{Rpk}$ are
$r$-equivalent, using the terminology of \cite{part1}). Furthermore,
$\operatorname*{Rpk}$ is clearly a descent statistic. Hence, Proposition
\ref{prop.K.ideal} (applied to $\operatorname*{st}=\operatorname*{Rpk}$) shows
that $\mathcal{K}_{\operatorname*{Rpk}}$ is an ideal of $\operatorname*{QSym}%
$. We now claim the following:

\begin{proposition}
\label{prop.Rpk.dend}The ideal $\mathcal{K}_{\operatorname*{Rpk}}$ of
$\operatorname*{QSym}$ is a $\tvi$-ideal, a right $\bel$-ideal, a left
$\left.  \prec\right.  $-ideal and a left $\left.  \succeq\right.  $-ideal,
but neither a $\left.  \prec\right.  $-ideal nor a $\left.  \succeq\right.
$-ideal of $\operatorname*{QSym}$.
\end{proposition}

\begin{corollary}
\label{cor.dendri.Rpk}The descent statistic $\operatorname*{Rpk}$ is neither
left-shuffle-compatible nor right-shuffle-compatible.
\end{corollary}

The proofs rely on the following fact (similar to Lemma \ref{lem.Epk.AB}):

\begin{lemma}
\label{lem.Rpk.AB}Let $A$ and $B$ be two nonempty compositions. Let
$n=\left\vert A\right\vert $ and $m=\left\vert B\right\vert $.

\textbf{(a)} We have $\operatorname*{Rpk}\left(  \left[  A,B\right]  \right)
=\left(  \operatorname*{Rpk}A\right)  \cup\left(  \operatorname*{Rpk}%
B+n\right)  $.

\textbf{(b)} We have $\operatorname*{Rpk}\left(  A\odot B\right)  =\left(
\left(  \operatorname*{Rpk}A\right)  \setminus\left\{  n\right\}  \right)
\cup\left(  \operatorname*{Rpk}B+n\right)  \cup\left\{  n+1\ \mid
\ 1\in\operatorname*{Des}B\text{ or }m=1\right\}  $.
\end{lemma}

\begin{proof}
[Proof of Lemma \ref{lem.Rpk.AB}.]Not unlike the proof of Lemma
\ref{lem.Epk.AB} (but left to the reader).
\end{proof}

\begin{proof}
[Proof of Proposition \ref{prop.Rpk.dend}.]To prove that $\mathcal{K}%
_{\operatorname*{Rpk}}$ is a $\tvi$-ideal and a right $\bel$-ideal, we proceed
as in the above second proof of Theorem \ref{thm.Epk.dend}, but using Lemma
\ref{lem.Rpk.AB} instead of Lemma \ref{lem.Epk.AB}. This time, however, the
analogue of Claim 6 will be false (i.e., we don't have $A\bel\mathcal{K}%
_{\operatorname*{Rpk}}\subseteq\mathcal{K}_{\operatorname*{Rpk}}$), because
the formula for $\operatorname*{Rpk}\left(  A\odot B\right)  $ in Lemma
\ref{lem.Rpk.AB} \textbf{(b)} depends on $\operatorname*{Des}B$. Thus,
$\mathcal{K}_{\operatorname*{Rpk}}$ is merely a right $\bel$-ideal, not a
$\bel$-ideal. This prevents us from applying Theorem \ref{thm.ideal-crit2}
\textbf{(c)}. However, we can apply Theorem \ref{thm.ideal-crit2} \textbf{(b)}
instead, and obtain $\operatorname*{QSym}\left.  \succeq\right.
\mathcal{K}_{\operatorname*{Rpk}}\subseteq\mathcal{K}_{\operatorname*{Rpk}}$.
In other words, $\mathcal{K}_{\operatorname*{Rpk}}$ is a left $\left.
\succeq\right.  $-ideal of $\operatorname*{QSym}$. Using
(\ref{eq.dendriform.1}), we thus easily see that $\mathcal{K}%
_{\operatorname*{Rpk}}$ is a left $\left.  \prec\right.  $-ideal of
$\operatorname*{QSym}$ as well.

To prove that $\mathcal{K}_{\operatorname*{Rpk}}$ is not a $\left.
\prec\right.  $-ideal of $\operatorname*{QSym}$ (and not even a right $\left.
\prec\right.  $-ideal of $\operatorname*{QSym}$), it suffices to find some
$m\in\mathcal{K}_{\operatorname*{Rpk}}$ and some $a\in\operatorname*{QSym}$
such that $m\left.  \prec\right.  a\notin\mathcal{K}_{\operatorname*{Rpk}}$.
For example, we can take $m=F_{\left(  1,2\right)  }-F_{\left(  3\right)  }$
and $a=F_{\left(  1\right)  }$; then,
\[
m\left.  \prec\right.  a=F_{\left(  3,2\right)  }+F_{\left(  2,3\right)
}+F_{\left(  2,2,1\right)  }-F_{\left(  1,2,2\right)  }-F_{\left(
1,1,3\right)  }-F_{\left(  1,1,2,1\right)  }\notin\mathcal{K}%
_{\operatorname*{Rpk}}.
\]
The same values of $m$ and $a$ also satisfy $m\left.  \succeq\right.
a\notin\mathcal{K}_{\operatorname*{Rpk}}$; thus, $\mathcal{K}%
_{\operatorname*{Rpk}}$ is not a $\left.  \succeq\right.  $-ideal of
$\operatorname*{QSym}$ either. Proposition \ref{prop.Rpk.dend} is now proven.
\end{proof}

\begin{proof}
[Proof of Corollary \ref{cor.dendri.Rpk}.]Follows from Proposition
\ref{prop.Rpk.dend}.
\end{proof}

\subsubsection{The peak set $\operatorname*{Pk}$}

Recall the permutation statistic $\operatorname*{Pk}$ (the peak set) defined
in Definition \ref{def.Des-et-al}. It was proven in \cite[Theorem 4.7
\textbf{(a)}]{part1} that this statistic $\operatorname*{Pk}$ is
shuffle-compatible. Furthermore, $\operatorname*{Pk}$ is clearly a descent
statistic. Hence, Proposition \ref{prop.K.ideal} (applied to
$\operatorname*{st}=\operatorname*{Pk}$) shows that $\mathcal{K}%
_{\operatorname*{Pk}}$ is an ideal of $\operatorname*{QSym}$. We now claim the following:

\begin{proposition}
\label{prop.Pk.dend}The ideal $\mathcal{K}_{\operatorname*{Pk}}$ of
$\operatorname*{QSym}$ is a left $\tvi$-ideal, a right $\bel$-ideal, a left
$\left.  \prec\right.  $-ideal and a left $\left.  \succeq\right.  $-ideal,
but neither a $\left.  \prec\right.  $-ideal nor a $\left.  \succeq\right.
$-ideal of $\operatorname*{QSym}$.
\end{proposition}

\begin{corollary}
\label{cor.dendri.Pk}The descent statistic $\operatorname*{Pk}$ is neither
left-shuffle-compatible nor right-shuffle-compatible.
\end{corollary}

The proofs rely on the following fact (similar to Lemma \ref{lem.Epk.AB}):

\begin{lemma}
\label{lem.Pk.AB}Let $A$ and $B$ be two nonempty compositions. Let
$n=\left\vert A\right\vert $ and $m=\left\vert B\right\vert $.

\textbf{(a)} We have $\operatorname*{Pk}\left(  \left[  A,B\right]  \right)
=\left(  \operatorname*{Pk}A\right)  \cup\left(  \operatorname*{Pk}B+n\right)
\cup\left\{  n\ \mid\ n-1\notin\operatorname*{Des}A\text{ and }n>1\right\}  $.

\textbf{(b)} We have $\operatorname*{Pk}\left(  A\odot B\right)  =\left(
\operatorname*{Pk}A\right)  \cup\left(  \operatorname*{Pk}B+n\right)
\cup\left\{  n+1\ \mid\ 1\in\operatorname*{Des}B\right\}  $.
\end{lemma}

\begin{proof}
[Proof of Lemma \ref{lem.Pk.AB}.]Not unlike the proof of Lemma
\ref{lem.Epk.AB} (but left to the reader).
\end{proof}

\begin{proof}
[Proof of Proposition \ref{prop.Pk.dend}.]To prove that $\mathcal{K}%
_{\operatorname*{Pk}}$ is a left $\tvi$-ideal and a right $\bel$-ideal, we
proceed as in the above second proof of Theorem \ref{thm.Epk.dend}, but using
Lemma \ref{lem.Pk.AB} instead of Lemma \ref{lem.Epk.AB}. This time, however,
the analogues of Claim 4 and Claim 6 will be false (i.e., neither
$\mathcal{K}_{\operatorname*{Pk}}\tvi A\subseteq\mathcal{K}%
_{\operatorname*{Pk}}$ nor $A\bel\mathcal{K}_{\operatorname*{Pk}}%
\subseteq\mathcal{K}_{\operatorname*{Pk}}$ will hold), because the formula for
$\operatorname*{Pk}\left(  \left[  A,B\right]  \right)  $ in Lemma
\ref{lem.Pk.AB} \textbf{(a)} depends on $\operatorname*{Des}A$ whereas the
formula for $\operatorname*{Pk}\left(  A\odot B\right)  $ in Lemma
\ref{lem.Pk.AB} \textbf{(b)} depends on $\operatorname*{Des}B$. Again, this
prevents us from applying Theorem \ref{thm.ideal-crit2} \textbf{(c)}. However,
we can apply Theorem \ref{thm.ideal-crit2} \textbf{(b)} instead, and obtain
$\operatorname*{QSym}\left.  \succeq\right.  \mathcal{K}_{\operatorname*{Pk}%
}\subseteq\mathcal{K}_{\operatorname*{Pk}}$. In other words, $\mathcal{K}%
_{\operatorname*{Pk}}$ is a left $\left.  \succeq\right.  $-ideal of
$\operatorname*{QSym}$. Using (\ref{eq.dendriform.1}), we thus easily see that
$\mathcal{K}_{\operatorname*{Pk}}$ is a left $\left.  \prec\right.  $-ideal of
$\operatorname*{QSym}$ as well.

To prove that $\mathcal{K}_{\operatorname*{Pk}}$ is not a $\left.
\prec\right.  $-ideal of $\operatorname*{QSym}$ (and not even a right $\left.
\prec\right.  $-ideal of $\operatorname*{QSym}$), it suffices to find some
$m\in\mathcal{K}_{\operatorname*{Pk}}$ and some $a\in\operatorname*{QSym}$
such that $m\left.  \prec\right.  a\notin\mathcal{K}_{\operatorname*{Pk}}$.
For example, we can take $m=F_{\left(  1,2\right)  }-F_{\left(  3\right)  }$
and $a=F_{\left(  1\right)  }$; then,
\[
m\left.  \prec\right.  a=F_{\left(  3,2\right)  }+F_{\left(  2,3\right)
}+F_{\left(  2,2,1\right)  }-F_{\left(  1,2,2\right)  }-F_{\left(
1,1,3\right)  }-F_{\left(  1,1,2,1\right)  }\notin\mathcal{K}%
_{\operatorname*{Pk}}.
\]
The same values of $m$ and $a$ also satisfy $m\left.  \succeq\right.
a\notin\mathcal{K}_{\operatorname*{Pk}}$; thus, $\mathcal{K}%
_{\operatorname*{Pk}}$ is not a $\left.  \succeq\right.  $-ideal of
$\operatorname*{QSym}$ either. Proposition \ref{prop.Pk.dend} is now proven.
\end{proof}

\begin{proof}
[Proof of Corollary \ref{cor.dendri.Pk}.]Follows from Proposition
\ref{prop.Pk.dend}.
\end{proof}
\end{verlong}

\begin{thebibliography}{99999999}                                                                                         %


\bibitem[EbMaPa07]{EbMaPa07}%
\href{https://doi.org/10.1016/j.jalgebra.2007.12.013}{Kurusch Ebrahimi-Fard,
Dominique Manchon, Fr\'{e}d\'{e}ric Patras, \textit{New identities in
dendriform algebras}, Journal of Algebra 320 (2008), pp. 708--727}.

\bibitem[GesZhu17]{part1}Ira M. Gessel, Yan Zhuang, \textit{Shuffle-compatible
permutation statistics}, Advances in Mathematics, Volume 332, 9 July 2018, pp.
85--141.\newline Also available at
\texttt{\href{http://arxiv.org/abs/1706.00750v3}{arXiv:1706.00750v3}}.

\bibitem[Greene88]{Greene88}%
\href{https://doi.org/10.1016/0097-3165(88)90018-0}{Curtis Greene,
\textit{Posets of shuffles}, Journal of Combinatorial Theory, Series A, Volume
47, Issue 2, March 1988, pp. 191--206}.

\bibitem[Grinbe16]{dimcr}Darij Grinberg, \textit{Dual immaculate creation
operators and a dendriform algebra structure on the quasisymmetric functions},
version 6, \texttt{\href{https://arxiv.org/abs/1410.0079v6}{arXiv:1410.0079v6}%
}. (Version 5 has been published in:
\href{https://cms.math.ca/10.4153/CJM-2016-018-8?abfmt=ltx}{Canad. J. Math.
\textbf{69} (2017), 21--53}.)

\begin{vershort}


\bibitem[Grinbe18]{verlong}%
\href{http://www.cip.ifi.lmu.de/~grinberg/algebra/gzshuf2-long.pdf}{Darij
Grinberg, \textit{Shuffle-compatible permutation statistics II: the exterior
peak set [detailed version]}, detailed version of the present paper}. Also
available as an ancillary file at
\href{https://arxiv.org/abs/1806.?????v1}{arXiv:1806.?????v1}.
\end{vershort}

\begin{verlong}


\bibitem[Grinbe18]{vershort}%
\href{http://www.cip.ifi.lmu.de/~grinberg/algebra/gzshuf2.pdf}{Darij Grinberg,
\textit{Shuffle-compatible permutation statistics II: the exterior peak set},
standard version of the present paper}. Also available as
\href{https://arxiv.org/abs/1806.?????v1}{arXiv:1806.?????v1}.
\end{verlong}

\bibitem[GriRei18]{HopfComb}Darij Grinberg, Victor Reiner, \textit{Hopf
algebras in Combinatorics}, version of 11 May 2018,
\texttt{\href{http://www.arxiv.org/abs/1409.8356v5}{arXiv:1409.8356v5}}.
\newline See also
\url{http://www.cip.ifi.lmu.de/~grinberg/algebra/HopfComb-sols.pdf} for a
version that gets updated.

\bibitem[HsiPet10]{HsiPet10}Samuel K. Hsiao, T. Kyle Petersen, \textit{Colored
Posets and Colored Quasisymmetric Functions}, Ann. Comb. 14 (2010), pp.
251--289,\newline\url{https://doi.org/10.1007/s00026-010-0059-0} . See
\href{http://www.arxiv.org/abs/math/0610984v1}{\texttt{arXiv:math/0610984v1}}
for a preprint.

\bibitem[Peters05]{Peters05}T. Kyle Petersen, \textit{Enriched }$\mathit{P}%
$\textit{-partitions and peak algebras}, Advances in Mathematics 209 (2007),
pp. 561--610,\newline\url{https://doi.org/10.1016/j.aim.2006.05.016} . See
\texttt{\href{https://arxiv.org/abs/math/0508041v1}{arXiv:math/0508041v1}} for
a preprint.

\bibitem[Peters06]{Peters06}T. Kyle Petersen,
\textit{Descents, Peaks, and $P$-partitions},
thesis at Brandeis University,
2006.
\url{http://people.brandeis.edu/~gessel/homepage/students/petersenthesis.pdf}

\bibitem[SageMath]{SageMath}\href{http://www.sagemath.org}{The Sage
Developers, \textit{SageMath, the Sage Mathematics Software System (Version
8.0)}, 2017.}

\bibitem[Stanle72]{Stanle72}Richard P. Stanley, \textit{Ordered Structures and
Partitions}, Memoirs of the American Mathematical Society, No. 119, American
Mathematical Society, Providence, R.I., 1972. \newline\url{http://www-math.mit.edu/~rstan/pubs/pubfiles/9.pdf}

\bibitem[Stanle11]{Stanley-EC1}Richard Stanley, \textit{Enumerative
Combinatorics, volume 1}, 2nd edition, Cambridge University Press 2012. A
preprint is available at \url{http://math.mit.edu/~rstan/ec/} .

\bibitem[Stembr97]{Stembr97}%
\href{http://www.ams.org/journals/tran/1997-349-02/S0002-9947-97-01804-7/}{John
R. Stembridge, \textit{Enriched P-partitions}, Trans. Amer. Math. Soc.
\textbf{349} (1997), no. 2, pp. 763--788}.
\end{thebibliography}


\end{document}