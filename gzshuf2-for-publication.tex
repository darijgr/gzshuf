% This is the final version of the paper
% "Shuffle-compatible Permutation Statistics II: the Exterior Peak Set"
% submitted for typesetting at the Electronic Journal of Combinatorics
% on 2018-10-06 by Darij Grinberg.

\documentclass[12pt]{article}
\usepackage{e-jc}
\usepackage[all,cmtip]{xy}
\usepackage{amsmath}
\usepackage{color}
\usepackage[breaklinks=true,colorlinks=true,citecolor=black,linkcolor=black,urlcolor=blue]{hyperref}
\usepackage{needspace}

\dateline{Jun 19, 2018}{Oct 6, 2018}{TBD}

\MSC{05E05, 05A05, 06A11}

\Copyright{The author. Released under the CC BY license (International 4.0).}

\theoremstyle{definition}
  \newtheorem{convention}[theorem]{Convention}

\newenvironment{statement}{\begin{quote}}{\end{quote}}
\newcommand{\NN}{\mathbb{N}}
\newcommand{\ZZ}{\mathbb{Z}}
\newcommand{\xx}{\mathbf{x}}
\newcommand{\are}{\ar@{-}}
\let\sumnonlimits\sum
\let\prodnonlimits\prod
\renewcommand{\sum}{\sumnonlimits\limits}
\renewcommand{\prod}{\prodnonlimits\limits}

\title{Shuffle-compatible permutation statistics II: the exterior peak set}

\author{Darij Grinberg\\
\small College of Science \& Engineering\\[-0.8ex]
\small University of Minnesota\\[-0.8ex] 
\small Minneapolis, MN, U.S.A.\\
\small\tt darijgrinberg@gmail.com}

\begin{document}

%\tableofcontents

\maketitle

\begin{abstract}
This is a continuation of the work ``Shuffle-compatible permutation statistics'' by Gessel and Zhuang (but can be read independently from the latter). We study the shuffle-compatibility of permutation statistics -- a concept introduced by Gessel and Zhuang, although various instances of it have appeared throughout the literature before. We prove that (as Gessel and Zhuang have conjectured) the exterior peak set statistic ($\operatorname{Epk}$) is shuffle-compatible. We furthermore introduce the concept of an ``LR-shuffle-compatible'' statistic, which is stronger than shuffle-compatibility. We prove that $\operatorname{Epk}$ and a few other statistics are LR-shuffle-compatible. Furthermore, we connect these concepts with the quasisymmetric functions, in particular the dendriform structure on them.
\end{abstract}

\section*{***}

This paper is a continuation of the work \cite{part1} by Gessel and Zhuang
(but can be read independently from the latter). It is devoted to the study of
shuffle-compatibility of permutation statistics -- a concept introduced in
\cite{part1}, although various instances of it have appeared throughout the
literature before.

In Section \ref{sect.notations}, we introduce the notations that we will need
throughout this paper. In Section \ref{sect.Zenri}, we prove that the exterior
peak set statistic $\operatorname{Epk}$ is shuffle-compatible (Theorem
\ref{thm.Epk.sh-co-a}), as conjectured by Gessel and Zhuang in \cite{part1}.
In Section \ref{sect.LR}, we introduce the concept of an \textquotedblleft
LR-shuffle-compatible\textquotedblright\ statistic, which is stronger than
shuffle-compatibility. We give a sufficient criterion for it and use it to
show that $\operatorname{Epk}$ and some other statistics are
LR-shuffle-compatible.

The last three sections relate all of this to quasisymmetric functions; these
sections are only brief summaries, and we refer to \cite{verlong} for the
details. In Section \ref{sect.Descent}, we recall the concept of descent
statistics introduced in \cite{part1} and its connection to quasisymmetric
functions. Motivated by this connection, in Section \ref{sect.kernel}, we
define the kernel of a descent statistic, and study this kernel for
$\operatorname{Epk}$, giving two explicit generating sets for this kernel. In
Section \ref{sect.dendri}, we extend the quasisymmetric functions connection
to the concept of LR-shuffle-compatible statistics, and relate it to
dendriform algebras.

\subsection{Remark on alternative versions}

This paper also has a detailed version \cite{verlong}, which includes some
proofs that have been omitted from the present version as well as more details
on some other proofs and further results in Sections \ref{sect.Descent},
\ref{sect.kernel} and \ref{sect.dendri}.

\section{\label{sect.notations}Notations and definitions}

Let us first introduce the definitions and notations that we will use in the
rest of this paper. Many of these definitions appear in \cite{part1} already;
we have tried to deviate from the notations of \cite{part1} as little as possible.

\subsection{Permutations and other basic concepts}

\begin{definition}
We let $\mathbb{N}=\left\{  0,1,2,3,\ldots\right\}  $ and $\mathbb{P}=\left\{
1,2,3,\ldots\right\}  $. Both of these sets are understood to be equipped with
their standard total order.
Elements of $\mathbb{P}$ will be called \textit{letters} (despite being
numbers).
\end{definition}

\begin{definition}
Let $n\in\mathbb{Z}$. We shall use the notation $\left[  n\right]  $ for the
totally ordered set $\left\{  1,2,\ldots,n\right\}  $ (with the usual order
relation inherited from $\mathbb{Z}$). Note that $\left[  n\right]
=\varnothing$ when $n\leq0$.
\end{definition}

\begin{definition}
\label{def.perm}Let $n\in\mathbb{N}$. An $n$\textit{-permutation} shall mean a
word with $n$ letters, which are distinct and belong to $\mathbb{P}$.
Equivalently, an $n$-permutation shall be regarded as an injective map
$\left[  n\right]  \rightarrow\mathbb{P}$ (the image of $i$ under this map
being the $i$-th letter of the word).
\end{definition}

For example, $\left(  3,6,4\right)  $ and $\left(  9,1,2\right)  $ are
$3$-permutations, but $\left(  2,1,2\right)  $ is not.

\begin{definition}
A \textit{permutation} is defined to be an $n$-permutation for some
$n\in\mathbb{N}$. If $\pi$ is an $n$-permutation for some $n\in\mathbb{N}$,
then the number $n$ is called the \textit{size} of the
permutation $\pi$ and is denoted by $\left\vert \pi\right\vert $. A
permutation is said to be \textit{nonempty} if it is nonempty as a word (i.e.,
if its size is $>0$).
\end{definition}

Note that the meaning of \textquotedblleft permutation\textquotedblright\ we
have just defined is unusual (most authors define a permutation to be a
bijection from a set to itself); we are following \cite{part1} in defining
permutations this way.

\begin{definition}
Let $n\in\mathbb{N}$. Two $n$-permutations $\alpha$ and $\beta$ are said to be
\textit{order-isomorphic} if they have the following property: For every two
integers $i,j\in\left[  n\right]  $, we have $\alpha\left(  i\right)
<\alpha\left(  j\right)  $ if and only if $\beta\left(  i\right)
<\beta\left(  j\right)  $.
\end{definition}

\begin{definition}
\textbf{(a)} A \textit{permutation statistic} is a map $\operatorname{st}$
from the set of all permutations to an arbitrary set that has the following
property: Whenever $\alpha$ and $\beta$ are two order-isomorphic permutations,
we have $\operatorname{st} \alpha = \operatorname{st} \beta$.

\textbf{(b)} Let $\operatorname{st}$ be a permutation statistic. Two
permutations $\alpha$ and $\beta$ are said to be
$\operatorname{st}$\textit{-equivalent} if they satisfy
$\left\vert \alpha\right\vert = \left\vert \beta\right\vert $ and
$\operatorname{st}\alpha = \operatorname{st}\beta$.
The relation
\textquotedblleft$\operatorname{st}$-equivalent\textquotedblright\ is
an equivalence relation; its equivalence
classes are called $\operatorname{st}$\textit{-equivalence classes}.
\end{definition}

\begin{remark}
Let $n\in\mathbb{N}$. Let us call an $n$-permutation $\pi$ \textit{standard}
if its letters are $1,2,\ldots,n$ (in some order). The standard $n$%
-permutations are in bijection with the $n!$ permutations of the set $\left\{
1,2,\ldots,n\right\}  $ in the usual sense of this word (i.e., the bijections
from this set to itself).

It is easy to see that for each $n$-permutation $\sigma$, there exists a
\textbf{unique} standard $n$-permutation $\pi$ order-isomorphic to $\sigma$.
Thus, a permutation statistic is uniquely determined by its values on standard
permutations. Consequently, we can view permutation statistics as statistics
defined on standard permutations, i.e., on permutations in the usual sense of
the word.
\end{remark}

The word \textquotedblleft permutation statistic\textquotedblright\ is often
abbreviated as \textquotedblleft statistic\textquotedblright.

\subsection{Some examples of permutation statistics}

\begin{definition}
\label{def.Des-et-al}Let $n\in\mathbb{N}$. Let $\pi=\left(  \pi_{1},\pi
_{2},\ldots,\pi_{n}\right)  $ be an $n$-permutation.

\textbf{(a)} The \textit{descents} of $\pi$ are the elements $i\in\left[
n-1\right]  $ satisfying $\pi_{i}>\pi_{i+1}$.

\textbf{(b)} The \textit{descent set} of $\pi$ is defined to be the set of all
descents of $\pi$. This set is denoted by $\operatorname{Des}\pi$, and is
always a subset of $\left[  n-1\right]  $.

\textbf{(c)} The \textit{peaks} of $\pi$ are the elements $i\in\left\{
2,3,\ldots,n-1\right\}  $ satisfying $\pi_{i-1}<\pi_{i}>\pi_{i+1}$.

\textbf{(d)} The \textit{peak set} of $\pi$ is defined to be the set of all
peaks of $\pi$. This set is denoted by $\operatorname*{Pk}\pi$, and is always
a subset of $\left\{  2,3,\ldots,n-1\right\}  $.

\textbf{(e)} The \textit{left peaks} of $\pi$ are the elements $i\in\left[
n-1\right]  $ satisfying $\pi_{i-1}<\pi_{i}>\pi_{i+1}$, where we set $\pi
_{0}=0$.

\textbf{(f)} The \textit{left peak set} of $\pi$ is defined to be the set of
all left peaks of $\pi$. This set is denoted by $\operatorname*{Lpk}\pi$, and
is always a subset of $\left[  n-1\right]  $. It is easy to see that (for
$n\geq2$) we have%
\[
\operatorname*{Lpk}\pi=\operatorname*{Pk}\pi\cup\left\{  1\ \mid\ \pi_{1}%
>\pi_{2}\right\}  .
\]
(The strange notation ``$\left\{  1\ \mid\ \pi_{1}>\pi_{2}\right\}  $''
means the set of all numbers $1$ satisfying $\pi_1 > \pi_2$.
In other words, it is the $1$-element
set $\left\{  1\right\}  $ if $\pi_{1}>\pi_{2}$, and the empty set
$\varnothing$ otherwise.)

\textbf{(g)} The \textit{right peaks} of $\pi$ are the elements $i\in\left\{
2,3,\ldots,n\right\}  $ satisfying $\pi_{i-1}<\pi_{i}>\pi_{i+1}$, where we set
$\pi_{n+1}=0$.

\textbf{(h)} The \textit{right peak set} of $\pi$ is defined to be the set of
all right peaks of $\pi$. This set is denoted by $\operatorname*{Rpk}\pi$, and
is always a subset of $\left\{  2,3,\ldots,n\right\}  $. It is easy to see
that (for $n\geq2$) we have%
\[
\operatorname*{Rpk}\pi=\operatorname*{Pk}\pi\cup\left\{  n\ \mid\ \pi
_{n-1}<\pi_{n}\right\}  .
\]


\textbf{(i)} The \textit{exterior peaks} of $\pi$ are the elements
$i\in\left[  n\right]  $ satisfying $\pi_{i-1}<\pi_{i}>\pi_{i+1}$, where we
set $\pi_{0}=0$ and $\pi_{n+1}=0$.

\textbf{(j)} The \textit{exterior peak set} of $\pi$ is defined to be the set
of all exterior peaks of $\pi$. This set is denoted by
$\operatorname{Epk} \pi$, and is always a subset of $\left[  n\right]  $.
It is easy to see that (for $n\geq2$) we have
\begin{align*}
\operatorname{Epk}\pi &  =\operatorname*{Pk}\pi\cup\left\{  1\ \mid\ \pi
_{1}>\pi_{2}\right\}  \cup\left\{  n\ \mid\ \pi_{n-1}<\pi_{n}\right\} \\
&  =\operatorname*{Lpk}\pi\cup\operatorname*{Rpk}\pi
\end{align*}
(where, again, $\left\{  n\ \mid\ \pi_{n-1}<\pi_{n}\right\}$ is the
$1$-element set $\left\{n\right\}$ if $\pi_{n-1}<\pi_{n}$,
and otherwise is the empty set).

(For $n=1$, we have $\operatorname{Epk}\pi=\left\{  1\right\}  $.)
\end{definition}

For example, the $6$-permutation $\pi=\left(  4,1,3,9,6,8\right)  $ has%
\begin{align*}
\operatorname{Des}\pi &  =\left\{  1,4\right\}
,\ \ \ \ \ \ \ \ \ \ \operatorname*{Pk}\pi=\left\{  4\right\}  ,\\
\operatorname*{Lpk}\pi &  =\left\{  1,4\right\}
,\ \ \ \ \ \ \ \ \ \ \operatorname*{Rpk}\pi=\left\{  4,6\right\}
,\ \ \ \ \ \ \ \ \ \ \operatorname{Epk}\pi=\left\{  1,4,6\right\}  .
\end{align*}
For another example, the $6$-permutation $\pi=\left(  1,4,3,2,9,8\right)  $
has%
\begin{align*}
\operatorname{Des}\pi &  =\left\{  2,3,5\right\}
,\ \ \ \ \ \ \ \ \ \ \operatorname*{Pk}\pi=\left\{  2,5\right\}  ,\\
\operatorname*{Lpk}\pi &  =\left\{  2,5\right\}
,\ \ \ \ \ \ \ \ \ \ \operatorname*{Rpk}\pi=\left\{  2,5\right\}
,\ \ \ \ \ \ \ \ \ \ \operatorname{Epk}\pi=\left\{  2,5\right\}  .
\end{align*}


Notice that Definition \ref{def.Des-et-al} actually defines several
permutation statistics. For example, Definition \ref{def.Des-et-al}
\textbf{(b)} defines the permutation statistic $\operatorname{Des}$, whose
codomain is the set of all subsets of $\mathbb{P}$. Likewise, Definition
\ref{def.Des-et-al} \textbf{(d)} defines the permutation statistic
$\operatorname*{Pk}$, and Definition \ref{def.Des-et-al} \textbf{(f)} defines
the permutation statistic $\operatorname*{Lpk}$, whereas Definition
\ref{def.Des-et-al} \textbf{(h)} defines the permutation statistic
$\operatorname*{Rpk}$. The main permutation statistic that we will study in
this paper is $\operatorname{Epk}$, which is defined in Definition
\ref{def.Des-et-al} \textbf{(j)}; its codomain is the set of all subsets of
$\mathbb{P}$.

The following simple fact expresses the set $\operatorname{Epk}\pi$
corresponding to an $n$-permutation $\pi$ in terms of $\operatorname{Des}\pi$:

\begin{proposition}
\label{prop.Epk.through-Des}Let $n$ be a positive integer. Let $\pi$ be an
$n$-permutation. Then,%
\[
\operatorname{Epk}\pi=\left(  \operatorname{Des}\pi\cup\left\{  n\right\}
\right)  \setminus\left(  \operatorname{Des}\pi+1\right)  ,
\]
where $\operatorname{Des}\pi+1$ denotes the set $\left\{  i+1\ \mid
\ i\in\operatorname{Des}\pi\right\}  $.
\end{proposition}

\begin{proof}
[Proof of Proposition \ref{prop.Epk.through-Des}.]The rather easy proof can be
found in the detailed version \cite{verlong} of this paper.
\end{proof}

\subsection{Shuffles and shuffle-compatibility}

\begin{definition}
\label{def.shuffles}Let $\pi$ and $\sigma$ be two permutations.

\textbf{(a)} We say that $\pi$ and $\sigma$ are \textit{disjoint} if no letter
appears in both $\pi$ and $\sigma$.

\textbf{(b)} Assume that $\pi$ and $\sigma$ are disjoint. Set $m=\left\vert
\pi\right\vert $ and $n=\left\vert \sigma\right\vert $. Let $\tau$ be an
$\left(  m+n\right)  $-permutation. Then, we say that $\tau$ is a
\textit{shuffle} of $\pi$ and $\sigma$ if both $\pi$ and $\sigma$ are
subsequences of $\tau$.

\textbf{(c)} We let $S\left(  \pi,\sigma\right)  $ be the set of all shuffles
of $\pi$ and $\sigma$.
\end{definition}

For example, the permutations $\left(  3,1\right)  $ and $\left(
6,2,9\right)  $ are disjoint, whereas the permutations $\left(  3,1,2\right)
$ and $\left(  6,2,9\right)  $ are not. The shuffles of the two disjoint
permutations $\left(  3,1\right)  $ and $\left(  2,6\right)  $ are%
\begin{align*}
&  \left(  3,1,2,6\right)  ,\ \ \ \ \ \ \ \ \ \ \left(  3,2,1,6\right)
,\ \ \ \ \ \ \ \ \ \ \left(  3,2,6,1\right)  ,\\
&  \left(  2,3,1,6\right)  ,\ \ \ \ \ \ \ \ \ \ \left(  2,3,6,1\right)
,\ \ \ \ \ \ \ \ \ \ \left(  2,6,3,1\right)  .
\end{align*}

If $\pi$ and $\sigma$ are two disjoint permutations, and if $\tau$ is a shuffle
of $\pi$ and $\sigma$, then each letter of $\tau$ must be either a letter of
$\pi$ or a letter of $\sigma$.
(This follows easily from the pigeonhole principle.)

If $\pi$ and $\sigma$ are two disjoint permutations, then $S\left(  \pi
,\sigma\right)  =S\left(  \sigma,\pi\right)  $ is an $\dbinom{m+n}{m}$-element
set, where $m=\left\vert \pi\right\vert $ and $n=\left\vert \sigma\right\vert
$.

Definition \ref{def.shuffles} \textbf{(b)} is used, e.g., in \cite{Greene88}.
From the point of view of combinatorics on words, it is somewhat naive, as it
fails to properly generalize to the case when the words $\pi$ and $\sigma$ are
no longer disjoint\footnote{In this general case, it is best to define a
shuffle of two words $\pi=\left(  \pi_{1},\pi_{2},\ldots,\pi_{m}\right)  $ and
$\sigma=\left(  \sigma_{1},\sigma_{2},\ldots,\sigma_{n}\right)  $ as a word of
the form $\left(  \gamma_{\eta\left(  1\right)  },\gamma_{\eta\left(
2\right)  },\ldots,\gamma_{\eta\left(  m+n\right)  }\right)  $, where $\left(
\gamma_{1},\gamma_{2},\ldots,\gamma_{m+n}\right)  $ is the word $\left(
\pi_{1},\pi_{2},\ldots,\pi_{m},\sigma_{1},\sigma_{2},\ldots,\sigma_{n}\right)
$, and where $\eta$ is some permutation of the set $\left\{  1,2,\ldots
,m+n\right\}  $ (that is, a bijection from this set to itself) satisfying
$\eta^{-1}\left(  1\right)  <\eta^{-1}\left(  2\right)  <\cdots<\eta
^{-1}\left(  m\right)  $ (this causes the letters $\pi_{1},\pi_{2},\ldots
,\pi_{m}$ to appear in the word $\left(  \gamma_{\eta\left(  1\right)
},\gamma_{\eta\left(  2\right)  },\ldots,\gamma_{\eta\left(  m+n\right)
}\right)  $ in this order) and $\eta^{-1}\left(  m+1\right)  <\eta^{-1}\left(
m+2\right)  <\cdots<\eta^{-1}\left(  m+n\right)  $ (this causes the letters
$\sigma_{1},\sigma_{2},\ldots,\sigma_{n}$ to appear in the word $\left(
\gamma_{\eta\left(  1\right)  },\gamma_{\eta\left(  2\right)  },\ldots
,\gamma_{\eta\left(  m+n\right)  }\right)  $ in this order). Furthermore, the
proper generalization of $S\left(  \pi,\sigma\right)  $ to this case would be
a multiset, not a mere set.}. But we will not be considering this general
case, since our results do not seem to straightforwardly extend to it
(although we might have to look more closely); thus, Definition
\ref{def.shuffles} will suffice for us.

\begin{definition}
\textbf{(a)} If $a_{1},a_{2},\ldots,a_{k}$ are finitely many arbitrary
objects, then \newline $\left\{  a_{1},a_{2},\ldots,a_{k}\right\}
_{\operatorname*{multi}}$ denotes the \textbf{multiset} whose elements are
$a_{1},a_{2},\ldots,a_{k}$ (each appearing with the multiplicity with which it
appears in the list $\left(  a_{1},a_{2},\ldots,a_{k}\right)  $).

\textbf{(b)} Let $\left(  a_{i}\right)  _{i\in I}$ be a finite family of
arbitrary objects. Then, $\left\{  a_{i}\ \mid\ i\in I\right\}
_{\operatorname*{multi}}$ denotes the \textbf{multiset} whose elements are the
elements of this family (each appearing with the multiplicity with which it
appears in the family).
\end{definition}

For example, $\left\{  k^{2}\ \mid\ k\in\left\{  -2,-1,0,1,2\right\}
\right\}  _{\operatorname*{multi}}$ is the multiset that contains the element
$4$ twice, the element $1$ twice, and the element $0$ once (and no other
elements). This multiset can also be written in the form $\left\{
4,1,0,1,4\right\}  _{\operatorname*{multi}}$, or in the form $\left\{
0,1,1,4,4\right\}  _{\operatorname*{multi}}$.

\begin{definition}
Let $\operatorname{st}$ be a permutation statistic. We say that
$\operatorname{st}$ is \textit{shuffle-compatible} if and only if it has the
following property: For any two disjoint permutations $\pi$ and $\sigma$, the
multiset%
\[
\left\{  \operatorname{st} \tau \ \mid\ \tau\in S\left(
\pi,\sigma\right)  \right\}  _{\operatorname*{multi}}%
\]
depends only on $\operatorname{st} \pi$, $\operatorname{st} \sigma$,
$\left\vert \pi\right\vert $ and $\left\vert \sigma\right\vert $.
\end{definition}

In other words, a permutation statistic $\operatorname{st}$ is
shuffle-compatible if and only if it has the following property:

\begin{itemize}
\item If $\pi$ and $\sigma$ are two disjoint permutations, and if $\pi
^{\prime}$ and $\sigma^{\prime}$ are two disjoint permutations, and if these
permutations satisfy
\begin{align*}
\operatorname{st} \pi    &  =\operatorname{st}\left(
\pi^{\prime}\right)  ,\ \ \ \ \ \ \ \ \ \ \operatorname{st}
\sigma =\operatorname{st}\left(  \sigma^{\prime}\right)  ,\\
\left\vert \pi\right\vert  &  =\left\vert \pi^{\prime}\right\vert
\ \ \ \ \ \ \ \ \ \ \text{and}\ \ \ \ \ \ \ \ \ \ \left\vert \sigma\right\vert
=\left\vert \sigma^{\prime}\right\vert ,
\end{align*}
then
\[
\left\{  \operatorname{st} \tau   \ \mid\ \tau\in S\left(
\pi,\sigma\right)  \right\}  _{\operatorname*{multi}}=\left\{
\operatorname{st} \tau   \ \mid\ \tau\in S\left(  \pi^{\prime
},\sigma^{\prime}\right)  \right\}  _{\operatorname*{multi}}.
\]

\end{itemize}

The notion of a shuffle-compatible permutation statistic was coined by Gessel
and Zhuang in \cite{part1}, where various statistics were analyzed for their
shuffle-compatibility. In particular, it was shown in \cite{part1} that the
statistics $\operatorname{Des}$, $\operatorname*{Pk}$, $\operatorname*{Lpk}$
and $\operatorname*{Rpk}$ are shuffle-compatible. Our next goal is to prove
the same for the statistic $\operatorname{Epk}$.

\section{\label{sect.Zenri}Extending enriched $P$-partitions and the exterior
peak set}

We are going to define \textit{$\mathcal{Z}$-enriched }$P$\textit{-partitions}%
, which are a straightforward generalization of the notions of
\textquotedblleft$P$-partitions\textquotedblright\ \cite{Stanle72},
\textquotedblleft enriched $P$-partitions\textquotedblright\ \cite[\S 2]{Stembr97}
and \textquotedblleft left enriched $P$-partitions\textquotedblright
\ \cite{Peters05}. We will then consider a new particular case
of this notion, which leads to a proof of the shuffle-compatibility of
$\operatorname{Epk}$ conjectured in \cite{part1}
(Theorem~\ref{thm.Epk.sh-co-a} below).

We remark that Bruce Sagan and Duff Baker-Jarvis are currently working on an
alternative, bijective approach to the shuffle-compatibility of permutation
statistics, which may lead to a different proof of this fact.

\subsection{Lacunar sets}

First, let us briefly study \textit{lacunar sets}, a class of subsets
of $\mathbb{Z}$ that are closely connected to exterior peaks.
We start with the definition:

\begin{definition}
A set $S$ of integers is said to be \textit{lacunar} if each $s\in S$
satisfies $s+1\notin S$.
\end{definition}

In other words, a set of integers is lacunar if and only if it contains no two
consecutive integers. For example, the set $\left\{  2, 5, 7\right\}  $ is
lacunar, while the set $\left\{  2, 5, 6\right\}  $ is not.

Lacunar sets of integers are also called \textit{sparse} sets in some of the
literature (though the latter word has several competing meanings).

\begin{definition}
\label{def.lac.Ln}
Let $n\in\mathbb{N}$. We define a set $\mathbf{L}_{n}$ of subsets of $\left[
n\right]  $ as follows:

\begin{itemize}
\item If $n$ is positive, then $\mathbf{L}_{n}$ shall mean the set of all
nonempty lacunar subsets of $\left[  n\right]  $.

\item If $n=0$, then $\mathbf{L}_{n}$ shall mean the set $\left\{
\varnothing\right\}  $.
\end{itemize}
\end{definition}

For example,%
\begin{align*}
\mathbf{L}_{0}  &  =\left\{  \varnothing\right\}
;\ \ \ \ \ \ \ \ \ \ \mathbf{L}_{1}=\left\{  \left\{  1\right\}  \right\}
;\ \ \ \ \ \ \ \ \ \ \mathbf{L}_{2}
=\left\{ \left\{ 1 \right\} , \left\{  2\right\}  \right\}  ;\\
\mathbf{L}_{3}  &  =\left\{  \left\{  1\right\}  ,\left\{  2\right\}
,\left\{  3\right\}  ,\left\{  1,3\right\}  \right\}  .
\end{align*}


\begin{proposition}
\label{prop.lac.fib}Let $\left(  f_{0},f_{1},f_{2},\ldots\right)  $ be the
Fibonacci sequence (defined by $f_{0}=0$ and $f_{1}=1$ and the recursive
relation $f_{m}=f_{m-1}+f_{m-2}$ for all $m\geq2$). Let $n$ be a positive
integer. Then, $\left\vert \mathbf{L}_{n}\right\vert =f_{n+2}-1$.
\end{proposition}

\begin{proof}
[Proof of Proposition \ref{prop.lac.fib}.]Recall that $\mathbf{L}_{n}$ is the
set of all nonempty lacunar subsets of $\left[  n\right]  $ (since $n$ is
positive). Thus, $\left\vert \mathbf{L}_{n}\right\vert $ is the number of all
lacunar subsets of $\left[  n\right]  $ minus $1$ (since the empty set
$\varnothing$, which is clearly a lacunar subset of $\left[  n\right]  $, is
withheld from the count). But a known fact (see, e.g., \cite[Exercise 1.35
\textbf{a.}]{Stanley-EC1}) says that the number of lacunar subsets of $\left[
n\right]  $ is $f_{n+2}$. Combining the preceding two sentences, we conclude
that $\left\vert \mathbf{L}_{n}\right\vert =f_{n+2}-1$. This proves
Proposition \ref{prop.lac.fib}.
\end{proof}

The following observation is easy:

\begin{proposition}
\label{prop.Epk-lac}Let $n\in\mathbb{N}$. Let $\pi$ be an $n$-permutation.
Then, $\operatorname{Epk}\pi\in\mathbf{L}_{n}$.
\end{proposition}

\begin{proof}
[Proof of Proposition \ref{prop.Epk-lac}.]If $n=0$, then the statement is
obvious (since in this case, we have $\operatorname{Epk}\pi=\varnothing
\in\mathbf{L}_{0}$). Thus, WLOG assume that $n\neq0$. Hence, $n$ is positive.
Hence, $\mathbf{L}_{n}$ is the set of all nonempty lacunar subsets of $\left[
n\right]  $ (by the definition of $\mathbf{L}_{n}$).

The set $\operatorname{Epk}\pi$ is lacunar (since two consecutive integers
cannot both be exterior peaks of $\pi$), and is also nonempty (since $\pi
^{-1}\left(  n\right)  $ is an exterior peak of $\pi$).
Therefore, $\operatorname{Epk}\pi$ is a nonempty lacunar subset of
$\left[  n\right]  $. In other words, $\operatorname{Epk}\pi\in\mathbf{L}%
_{n}$ (since $\mathbf{L}_{n}$ is the set of all nonempty lacunar subsets of
$\left[  n\right]  $). This proves Proposition \ref{prop.Epk-lac}.
\end{proof}

Proposition \ref{prop.Epk-lac} actually has a sort of converse:

\begin{proposition}
\label{prop.when-Epk}Let $n\in\mathbb{N}$. Let $\Lambda$ be a subset of
$\left[  n\right]  $. Then, there exists an $n$-permutation $\pi$ satisfying
$\Lambda=\operatorname{Epk}\pi$ if and only if $\Lambda\in\mathbf{L}_{n}$.
\end{proposition}

\begin{proof}
[Proof of Proposition \ref{prop.when-Epk}.]Omitted; see \cite{verlong} for a proof.
\end{proof}

Next, let us introduce a total order on the finite subsets of $\mathbb{Z}$:

\begin{definition}
\label{def.order-on-P}
\textbf{(a)} Let $\mathbf{P}$ be the set of all finite subsets of $\mathbb{Z}$.

\textbf{(b)} If $A$ and $B$ are any two sets, then $A\bigtriangleup B$ shall
denote the \textit{symmetric difference} of $A$ and $B$. This is the set
$\left(  A\cup B\right)  \setminus\left(  A\cap B\right)  =\left(  A\setminus
B\right)  \cup\left(  B\setminus A\right)  $. It is well-known that the binary
operation $\bigtriangleup$ on sets is associative.

If $A$ and $B$ are two
distinct sets, then the set $A\bigtriangleup B$ is nonempty. Also, if
$A\in\mathbf{P}$ and $B\in\mathbf{P}$, then $A\bigtriangleup B\in\mathbf{P}$.
Thus, if $A$ and $B$ are two distinct sets in $\mathbf{P}$, then $\min\left(
A\bigtriangleup B\right)  \in\mathbb{Z}$ is well-defined.

\textbf{(c)} We define a binary relation $<$ on $\mathbf{P}$ as follows: For
any $A\in\mathbf{P}$ and $B\in\mathbf{P}$, we let $A<B$ if and only if $A\neq
B$ and $\min\left(  A\bigtriangleup B\right)  \in A$. (This definition makes
sense, because the condition $A\neq B$ ensures that $\min\left(
A\bigtriangleup B\right)  $ is well-defined.)
\end{definition}

Note that this relation $<$ is similar to the relation $<$ in \cite[Lemma
4.3]{ABN-peaks}.

\begin{proposition}
\label{prop.lac.order}The relation $<$ on $\mathbf{P}$ is the smaller relation
of a total order on $\mathbf{P}$.
\end{proposition}



\begin{proof}
[Proof of Proposition \ref{prop.lac.order}.]See \cite{verlong} for this
straightforward argument (or imitate \cite[proof of Lemma 4.3]{ABN-peaks}).
\end{proof}

In the following, we shall regard the set $\mathbf{P}$ as a totally ordered
set, equipped with the order from Proposition \ref{prop.lac.order}. Thus, for
example, two sets $A$ and $B$ in $\mathbf{P}$ satisfy $A\geq B$ if and only if
either $A=B$ or $B<A$.

\begin{definition}
\label{def.S+1}Let $S$ be a subset of $\mathbb{Z}$. Then, we define a new
subset $S+1$ of $\mathbb{Z}$ by setting
\[
S+1=\left\{  i+1\ \mid\ i\in S\right\}  =\left\{  j\in\mathbb{Z}\ \mid\ j-1\in
S\right\}  .
\]
Note that $S+1\in\mathbf{P}$ if $S\in\mathbf{P}$.
\end{definition}

For example, $\left\{  2,5\right\}  +1=\left\{  3,6\right\}  $. Note that a
subset $S$ of $\mathbb{Z}$ is lacunar if and only if $S\cap\left(  S+1\right)
=\varnothing$.

\begin{proposition}
\label{prop.lac.RL}Let $\Lambda\in\mathbf{P}$ and $R\in\mathbf{P}$ be such
that the set $R$ is lacunar and $R\subseteq\Lambda\cup\left(  \Lambda
+1\right)  $. Then, $R\geq\Lambda$ (with respect to the total order on
$\mathbf{P}$).
\end{proposition}

\begin{proof}
[Proof of Proposition \ref{prop.lac.RL}.]Assume the contrary. Thus,
$R<\Lambda$ (since $\mathbf{P}$ is totally ordered). In other words,
$R\neq\Lambda$ and $\min\left(  R\bigtriangleup\Lambda\right)  \in R$ (by the
definition of the relation $<$). Let $\mu=\min\left(  R\bigtriangleup
\Lambda\right)  $. Thus, $\mu=\min\left(  R\bigtriangleup\Lambda\right)  \in
R\subseteq\Lambda\cup\left(  \Lambda+1\right)  $.

We have $\mu=\min\left(  R\bigtriangleup\Lambda\right)  \in R\bigtriangleup
\Lambda=\left(  R\cup\Lambda\right)  \setminus\left(  R\cap\Lambda\right)  $.
Hence, $\mu\notin R\cap\Lambda$. If we had $\mu\in\Lambda$, then we would have
$\mu\in R\cap\Lambda$ (since $\mu\in R$ and $\mu\in\Lambda$), which would
contradict $\mu\notin R\cap\Lambda$. Thus, we cannot have $\mu\in\Lambda$.
Hence, $\mu\notin\Lambda$. Combining $\mu\in\Lambda\cup\left(  \Lambda
+1\right)  $ with $\mu\notin\Lambda$, we obtain $\mu\in\left(  \Lambda
\cup\left(  \Lambda+1\right)  \right)  \setminus\Lambda\subseteq\Lambda+1$. In
other words, $\mu-1\in\Lambda$.

Every $x\in R\bigtriangleup\Lambda$ satisfies $x\geq\min\left(
R\bigtriangleup\Lambda\right)  $. Hence, if we had $\mu-1\in R\bigtriangleup
\Lambda$, then we would have $\mu-1\geq\min\left(  R\bigtriangleup
\Lambda\right)  =\mu$, which would contradict $\mu-1<\mu$. Thus, we cannot
have $\mu-1\in R\bigtriangleup\Lambda$. Thus, $\mu-1\notin R\bigtriangleup
\Lambda$. Combining this with $\mu-1\in\Lambda$, we obtain $\mu-1\in
\Lambda\setminus\left(  R\bigtriangleup\Lambda\right)  =R\cap\Lambda$ (since
every two sets $X$ and $Y$ satisfy $Y\setminus\left(  X\bigtriangleup
Y\right)  =X\cap Y$). Thus, $\mu-1\in R\cap\Lambda\subseteq R$.

But the set $R$ is lacunar. In other words, each $s\in R$ satisfies $s+1\notin
R$ (by the definition of \textquotedblleft lacunar\textquotedblright).
Applying this to $s=\mu-1$, we obtain $\left(  \mu-1\right)  +1\notin R$
(since $\mu-1\in R$). This contradicts $\left(  \mu-1\right)  +1=\mu\in R$.
This contradiction shows that our assumption was wrong; hence, Proposition
\ref{prop.lac.RL} is proven.
\end{proof}

\subsection{\label{subsect.Zenri.gen}$\mathcal{Z}$-enriched $\left(
P,\gamma\right)  $-partitions}

\begin{convention}
By abuse of notation, we will often use the same notation for a poset
$P=\left(  X,\leq\right)  $ and its ground set $X$ when there is no danger of
confusion. In particular, if $x$ is some object, then \textquotedblleft$x\in
P$\textquotedblright\ shall mean \textquotedblleft$x\in X$\textquotedblright.
\end{convention}

\begin{definition}
A \textit{labeled poset} means a pair $\left(  P,\gamma\right)  $ consisting
of a finite poset $P=\left(  X,\leq\right)  $ and an injective map
$\gamma:X\rightarrow A$ for some totally ordered set $A$. The injective map
$\gamma$ is called the \textit{labeling} of the labeled poset $\left(
P,\gamma\right)  $. The poset $P$ is called the \textit{ground poset} of the
labeled poset $\left(  P,\gamma\right)  $.
\end{definition}

\begin{convention}
Let $\mathcal{N}$ be a totally ordered set, whose (strict) order relation will
be denoted by $\prec$. Let $+$ and $-$ be two distinct symbols. Let
$\mathcal{Z}$ be a subset of the set $\mathcal{N}\times\left\{  +,-\right\}
$. For each $q=\left(  n,s\right)  \in\mathcal{Z}$, we denote the element
$n\in\mathcal{N}$ by $\left\vert q\right\vert $, and we call the element
$s\in\left\{  +,-\right\}  $ the \textit{sign} of $q$. If $n\in\mathcal{N}$,
then we will denote the two elements $\left(  n,+\right)  $ and $\left(
n,-\right)  $ of $\mathcal{N}\times\left\{  +,-\right\}  $ by $+n$ and $-n$, respectively.

We equip the set $\mathcal{Z}$ with a total order, whose (strict) order
relation $\prec$ is defined by%
\[
\left(  n,s\right)  \prec\left(  n^{\prime},s^{\prime}\right)  \text{ if and
only if either }n\prec n^{\prime}\text{ or }\left(  n=n^{\prime}\text{ and
}s=-\text{ and }s^{\prime}=+\right)  .
\]
Let $\operatorname*{Pow}\mathcal{N}$ be the ring of all formal power series
over $\mathbb{Q}$ in the indeterminates $x_{n}$ for $n\in\mathcal{N}$.

We fix $\mathcal{N}$ and $\mathcal{Z}$ throughout Subsection
\ref{subsect.Zenri.gen}. That is, any result in this subsection is tacitly
understood to begin with \textquotedblleft Let $\mathcal{N}$ be a totally
ordered set, whose (strict) order relation will be denoted by $\prec$, and let
$\mathcal{Z}$ be a subset of the set $\mathcal{N}\times\left\{  +,-\right\}
$\textquotedblright; and the notations of this convention shall always be
in place throughout this Subsection.

Whenever $\prec$ denotes some strict order, the corresponding weak order will
be denoted by $\preccurlyeq$. (Thus, $a \preccurlyeq b$ means ``$a \prec b$ or
$a = b$''.)
\end{convention}

\begin{definition}
\label{def.ambivPp}Let $\left(  P,\gamma\right)  $ be a labeled poset. A
\textit{$\mathcal{Z}$-enriched }$\left(  P,\gamma\right)  $\textit{-partition}
means a map $f:P\rightarrow\mathcal{Z}$ such that for all $x<y$ in $P$, the
following conditions hold:

\begin{enumerate}
\item[\textbf{(i)}] We have $f\left(  x\right)  \preccurlyeq f\left(
y\right)  $.

\item[\textbf{(ii)}] If $f\left(  x\right)  =f\left(  y\right)  =+n$ for some
$n\in\mathcal{N}$, then $\gamma\left(  x\right)  <\gamma\left(  y\right)  $.

\item[\textbf{(iii)}] If $f\left(  x\right)  =f\left(  y\right)  =-n$ for some
$n\in\mathcal{N}$, then $\gamma\left(  x\right)  >\gamma\left(  y\right)  $.
\end{enumerate}

(Of course, this concept depends on $\mathcal{N}$ and $\mathcal{Z}$, but these
will always be clear from the context.)
\end{definition}

\Needspace{4cm}

\begin{example}
\label{exa.ambivPp.diamond}Let $P$ be the poset with the following Hasse
diagram:%
\[%
%TCIMACRO{\TeXButton{Hasse diagram}{\xymatrix{
%& b \are[dl] \are[dr] \\
%c \are[dr] & & d \are[dl] \\
%& a
%}}}%
%BeginExpansion
\xymatrix{
& b \are[dl] \are[dr] \\
c \are[dr] & & d \are[dl] \\
& a
}%
%EndExpansion
\]
(that is, the ground set of $P$ is $\left\{ a, b, c, d \right\}$, and
its order relation is given by $a < c < b$ and $a < d < b$).
Let $\gamma:P\rightarrow\mathbb{Z}$ be a map that satisfies
$\gamma\left(  a\right)  <\gamma\left(  b\right)  <\gamma\left(  c\right)
<\gamma\left(  d\right)  $ (for example, $\gamma$ could be the map that sends
$a,b,c,d$ to $2,3,5,7$, respectively).
Then, $\left(P, \gamma\right)$ is a labeled poset.
A $\mathcal{Z}$-enriched $\left(
P,\gamma\right)  $-partition is a map $f:P\rightarrow\mathcal{Z}$ satisfying
the following conditions:

\begin{enumerate}
\item[\textbf{(i)}] We have $f\left(  a\right)  \preccurlyeq f\left(
c\right)  \preccurlyeq f\left(  b\right)  $ and $f\left(  a\right)
\preccurlyeq f\left(  d\right)  \preccurlyeq f\left(  b\right)  $.

\item[\textbf{(ii)}] We cannot have $f\left(  c\right)  =f\left(  b\right)
=+n$ with $n\in\mathcal{N}$. \newline
We cannot have $f\left(  d\right)
=f\left(  b\right)  =+n$ with $n\in\mathcal{N}$.

\item[\textbf{(iii)}] We cannot have $f\left(  a\right)  =f\left(  c\right)
=-n$ with $n\in\mathcal{N}$. \newline
We cannot have $f\left(  a\right)
=f\left(  d\right)  =-n$ with $n\in\mathcal{N}$.
\end{enumerate}

For example, if $\mathcal{N}=\mathbb{P}$ (the totally ordered set of positive
integers, with its usual ordering) and $\mathcal{Z}=\mathcal{N}\times\left\{
+,-\right\}  $, then the map $f:P\rightarrow\mathcal{Z}$ sending $a,b,c,d$ to
$+2,-3,+2,-3$ (respectively) is a $\mathcal{Z}$-enriched $\left(
P,\gamma\right)  $-partition. Notice that the total ordering on $\mathcal{Z}$
in this case is given by%
\[
-1\prec+1\prec-2\prec+2\prec-3\prec+3\prec\cdots,
\]
rather than by the familiar total order on $\mathbb{Z}$.
\end{example}

The concept of a \textquotedblleft$\mathcal{Z}$-enriched $\left(
P,\gamma\right)  $-partition\textquotedblright\ generalizes three notions in
existing literature: that of a \textquotedblleft$\left(  P,\gamma\right)
$-partition\textquotedblright, that of an \textquotedblleft enriched $\left(
P,\gamma\right)  $-partition\textquotedblright, and that of a
\textquotedblleft left enriched $\left(  P,\gamma\right)  $%
-partition\textquotedblright\footnote{The ideas behind these three concepts
are due to Stanley \cite{Stanle72}, Stembridge \cite[\S 2]{Stembr97} and
Petersen \cite{Peters05}, respectively, but the precise definitions are not
standardized across the literature. We define a \textquotedblleft$\left(
P,\gamma\right)  $-partition\textquotedblright\ as in \cite[\S 1.1]{Stembr97};
this definition differs noticeably from Stanley's (in particular, Stanley
requires $f\left(  x\right)  \succcurlyeq f\left(  y\right)  $ instead of
$f\left(  x\right)  \preccurlyeq f\left(  y\right)  $, but the differences do
not end here). We define an \textquotedblleft enriched $\left(  P,\gamma
\right)  $-partition\textquotedblright\ as in \cite[\S 2]{Stembr97}. Finally,
we define a \textquotedblleft left enriched $\left(  P,\gamma\right)
$-partition\textquotedblright\ to be a $\mathcal{Z}$-enriched $\left(
P,\gamma\right)  $-partition where $\mathcal{N}=\mathbb{N}$ and $\mathcal{Z}%
=\left(  \mathcal{N}\times\left\{  +,-\right\}  \right)  \setminus\left\{
-0\right\}  $; this definition is equivalent to Petersen's
\cite[Definition 3.4.1]{Peters06} up to some differences of notation
(in particular, Petersen assumes that the ground set of $P$ is already
a subset of $\mathbb{P}$, and that the labeling $\gamma$ is the
canonical inclusion map $ P \to \mathbb{P}$; also, he identifies the elements
$+0, -1, +1, -2, +2, \ldots$ of
$\left(  \mathcal{N}\times\left\{  +,-\right\}  \right)  \setminus\left\{
-0\right\}  $ with the integers
$0, -1, +1, -2, +2, \ldots$, respectively).
Note that the definition Petersen gives in \cite[Definition 4.1]{Peters05}
is incorrect, and the one in \cite[Definition 3.4.1]{Peters06} is probably
his intent.}:

\begin{example}
\label{exa.ambivPp.abc}\textbf{(a)} If $\mathcal{N}=\mathbb{P}$ (the totally
ordered set of positive integers) and $\mathcal{Z}=\mathcal{N}\times\left\{
+\right\}  =\left\{  +n\ \mid\ n\in\mathcal{N}\right\}  $, then the
$\mathcal{Z}$-enriched $\left(  P,\gamma\right)  $-partitions are simply the
$\left(  P,\gamma\right)  $-partitions into $\mathcal{N}$, composed with the
canonical bijection $\mathcal{N}\rightarrow\mathcal{Z},\ n\mapsto\left(
+n\right)  $.

\textbf{(b)} If $\mathcal{N}=\mathbb{P}$ (the totally ordered set of positive
integers) and $\mathcal{Z}=\mathcal{N}\times\left\{  +,-\right\}  $, then the
$\mathcal{Z}$-enriched $\left(  P,\gamma\right)  $-partitions are the enriched
$\left(  P,\gamma\right)  $-partitions.

\textbf{(c)} If $\mathcal{N}=\mathbb{N}$ (the totally ordered set of
nonnegative integers) and $\mathcal{Z}=\left(  \mathcal{N}\times\left\{
+,-\right\}  \right)  \setminus\left\{  -0\right\}  $, then the $\mathcal{Z}%
$-enriched $\left(  P,\gamma\right)  $-partitions are the left enriched
$\left(  P,\gamma\right)  $-partitions. Note that $+0$ and $-0$ here stand for
the pairs $\left(  0,+\right)  $ and $\left(  0,-\right)  $; thus, they are
not equal.
\end{example}

\begin{definition}
If $\left(  P,\gamma\right)  $ is a labeled poset, then $\mathcal{E}\left(
P,\gamma\right)  $ shall denote the set of all $\mathcal{Z}$-enriched $\left(
P,\gamma\right)  $-partitions.
\end{definition}

\begin{definition}
Let $P$ be any finite poset. Then, $\mathcal{L}\left(  P\right)  $ shall
denote the set of all linear extensions of $P$. A linear extension of $P$
shall be understood simultaneously as a totally ordered set extending $P$ and
as a list $\left(  w_{1},w_{2},\ldots,w_{n}\right)  $ of all elements of $P$
such that no two integers $i<j$ satisfy $w_{i}\geq w_{j}$ in $P$.
\end{definition}

Let us prove some basic facts about $\mathcal{Z}$-enriched $\left(
P,\gamma\right)  $-partitions, straightforwardly generalizing classical
results proven by Stanley and Gessel (for the case of \textquotedblleft
plain\textquotedblright\ $\left(  P,\gamma\right)  $-partitions), Stembridge
\cite[Lemma 2.1]{Stembr97}
(for enriched $\left(  P,\gamma\right)  $-partitions) and
Petersen \cite[Lemma 3.4.1]{Peters06} (for left
enriched $\left(  P,\gamma\right)  $-partitions):

\begin{proposition}
\label{prop.fund-lem}For any labeled poset $\left(  P,\gamma\right)  $, we
have%
\[
\mathcal{E}\left(  P,\gamma\right)  =\bigsqcup_{w\in\mathcal{L}\left(
P\right)  }\mathcal{E}\left(  w,\gamma\right)  .
\]

\end{proposition}

\begin{proof}
[Proof of Proposition \ref{prop.fund-lem}.]This is analogous to the proof of
\cite[Lemma 2.1]{Stembr97}.
See \cite{verlong} for details.
\end{proof}

\begin{definition}
\label{def.GammaZ}Let $\left(  P,\gamma\right)  $ be a labeled poset. We
define a power series $\Gamma_{\mathcal{Z}}\left(  P,\gamma\right)
\in\operatorname*{Pow}\mathcal{N}$ by%
\[
\Gamma_{\mathcal{Z}}\left(  P,\gamma\right)  =\sum_{f\in\mathcal{E}\left(
P,\gamma\right)  }\prod_{p\in P}x_{\left\vert f\left(  p\right)  \right\vert
}.
\]
This is easily seen to be convergent in the usual topology on
$\operatorname*{Pow}\mathcal{N}$.
(Indeed, for every monomial $\mathfrak{m}$ in $\operatorname*{Pow}\mathcal{N}$,
there exist at most $\left|P\right|! \cdot 2^{\left|P\right|}$ many
$f \in \mathcal{E} \left(P, \gamma\right)$ satisfying
$\prod_{p\in P}x_{\left\vert f\left(  p\right)  \right\vert } = \mathfrak{m}$.)
\end{definition}

\begin{corollary}
\label{cor.fund-lem}For any labeled poset $\left(  P,\gamma\right)  $, we have%
\[
\Gamma_{\mathcal{Z}}\left(  P,\gamma\right)  =\sum_{w\in\mathcal{L}\left(
P\right)  }\Gamma_{\mathcal{Z}}\left(  w,\gamma\right)  .
\]

\end{corollary}

\begin{proof}
[Proof of Corollary \ref{cor.fund-lem}.]Follows straight from Proposition
\ref{prop.fund-lem}.
\end{proof}

\begin{definition}
Let $P$ be any set. Let $A$ be a totally ordered set. Let $\gamma:P\rightarrow
A$ and $\delta:P\rightarrow A$ be two maps. We say that $\gamma$ and $\delta$
are \textit{order-isomorphic} if the following holds: For every pair $\left(
p,q\right)  \in P\times P$, we have $\gamma\left(  p\right)  \leq\gamma\left(
q\right)  $ if and only if $\delta\left(  p\right)  \leq\delta\left(
q\right)  $.
\end{definition}

\begin{lemma} \label{lem.ord-eq-Ppar}
Let $\left(P, \alpha\right)$ and $\left(P, \beta\right)$ be two labeled posets
with the same ground poset $P$.
Assume that the maps $\alpha$ and $\beta$ are order-isomorphic. Then:

\textbf{(a)} We have
$\mathcal{E}\left(P, \alpha\right) = \mathcal{E}\left(P, \beta\right)$.

\textbf{(b)} We have
$\Gamma_{\mathcal{Z}}\left(P, \alpha\right) = \Gamma_{\mathcal{Z}}\left(P, \beta\right)$.
\end{lemma}

\begin{proof}[Proof of Lemma~\ref{lem.ord-eq-Ppar}.]
\textbf{(a)} If $x$ and $y$ are two elements of $P$, then we have the following
equivalences:
\begin{align*}
\left( \alpha\left(x\right) \leq \alpha\left(y\right) \right)
  \ &\Longleftrightarrow \ %
  \left( \beta\left(x\right) \leq \beta\left(y\right) \right) ; \\
\left( \alpha\left(x\right) > \alpha\left(y\right) \right)
  \ &\Longleftrightarrow \ %
  \left( \beta\left(x\right) > \beta\left(y\right) \right) ; \\
\left( \alpha\left(x\right) < \alpha\left(y\right) \right)
  \ &\Longleftrightarrow \ %
  \left( \beta\left(x\right) < \beta\left(y\right) \right) .
\end{align*}
(Indeed, the first of these equivalences holds because $\alpha$ and $\beta$
are order-isomorphic; the second is the contrapositive of the first; the
third is obtained from the second by swapping $x$ with $y$.)

Hence, the conditions ``$\alpha\left(x\right) > \alpha\left(y\right)$''
and ``$\alpha\left(x\right) < \alpha\left(y\right)$'' in the definition
of a $\mathcal{Z}$-enriched $\left(P, \alpha\right)$-partition are
equivalent to the conditions
``$\beta\left(x\right) > \beta\left(y\right)$''
and ``$\beta\left(x\right) < \beta\left(y\right)$'' in the definition
of a $\mathcal{Z}$-enriched $\left(P, \beta\right)$-partition.
Therefore, the $\mathcal{Z}$-enriched $\left(P, \alpha\right)$-partitions
are precisely the
$\mathcal{Z}$-enriched $\left(P, \beta\right)$-partitions.
In other words,
$\mathcal{E}\left(P, \alpha\right) = \mathcal{E}\left(P, \beta\right)$.
This proves Lemma~\ref{lem.ord-eq-Ppar} \textbf{(a)}.

\textbf{(b)} Lemma~\ref{lem.ord-eq-Ppar} \textbf{(b)} follows from
Lemma~\ref{lem.ord-eq-Ppar} \textbf{(a)}.
\end{proof}

Let us recall the notion of the disjoint union of two posets:

\begin{definition}
\textbf{(a)} Let $P$ and $Q$ be two sets.
The \textit{disjoint union} of $P$ and $Q$ is the set
$\left( \left\{0\right\} \times P \right)
              \cup \left( \left\{1\right\} \times Q \right)$.
This set is denoted by $P \sqcup Q$,
and comes with two canonical injections
\begin{align*}
\iota_0 &: P \to P \sqcup Q, \qquad  p \mapsto \left(0, p\right), \qquad \text{and} \\
\iota_1 &: Q \to P \sqcup Q, \qquad  q \mapsto \left(1, q\right).
\end{align*}
The images of these two injections are disjoint, and their
union is $P \sqcup Q$.

If $f : P \sqcup Q \to X$ is any map, then the
\textit{restriction of $f$ to $P$} is understood to be the
map $f \circ \iota_0 : P \to X$, whereas the
\textit{restriction of $f$ to $Q$} is understood to be the
map $f \circ \iota_1 : Q \to X$.
(Of course, this notation is ambiguous when $P = Q$.)

When the sets $P$ and $Q$ are already disjoint, it is common
to identify their disjoint union $P \sqcup Q$ with their
union $P \cup Q$ via the map
\[
P \sqcup Q \to P \cup Q, \qquad \left(i, r\right) \mapsto r .
\]
Under this identification, the restriction of a map
$f : P \sqcup Q \to X$ to $P$ becomes identical with the
(literal) restriction $f\mid_P$ of the map $f : P \cup Q \to X$
(and similarly for the restrictions to $Q$).

\textbf{(b)} Let $P$ and $Q$ be two posets.
The \textit{disjoint union} of the posets $P$ and $Q$ is the poset
$P \sqcup Q$ whose ground set is the disjoint union $P \sqcup Q$,
and whose order relation is defined by the following rules:
\begin{itemize}
\item If $p$ and $p^{\prime}$ are two elements of $P$, then
      $\left(0, p\right) < \left(0, p^{\prime}\right)$ in $P \sqcup Q$
      if and only if $p < p^{\prime}$ in $P$.
\item If $q$ and $q^{\prime}$ are two elements of $Q$, then
      $\left(1, q\right) < \left(1, q^{\prime}\right)$ in $P \sqcup Q$
      if and only if $q < q^{\prime}$ in $Q$.
\item If $p \in P$ and $q \in Q$,
      then the elements $\left(0, p\right)$ and $\left(1, q\right)$
      of $P \sqcup Q$ are incomparable.
\end{itemize}
\end{definition}

\begin{proposition}
\label{prop.prod1}Let $\left(  P,\gamma\right)  $ and $\left(  Q,\delta
\right)  $ be two labeled posets. Let $\left(  P\sqcup Q,\varepsilon\right)  $
be a labeled poset whose ground poset $P\sqcup Q$ is the disjoint union of $P$
and $Q$, and whose labeling $\varepsilon$ is such that the restriction of
$\varepsilon$ to $P$ is order-isomorphic to $\gamma$ and such that the
restriction of $\varepsilon$ to $Q$ is order-isomorphic to $\delta$. Then,%
\[
\Gamma_{\mathcal{Z}}\left(  P,\gamma\right)  \Gamma_{\mathcal{Z}}\left(
Q,\delta\right)  =\Gamma_{\mathcal{Z}}\left(  P\sqcup Q,\varepsilon\right)  .
\]

\end{proposition}

\begin{proof}
[Proof of Proposition \ref{prop.prod1}.]We WLOG assume that the ground sets
$P$ and $Q$ are disjoint; thus, we can identify $P\sqcup Q$ with the union
$P\cup Q$.
Let us make this identification.

The restriction $\varepsilon\mid_P$ of $\varepsilon$ to $P$ is order-isomorphic
to $\gamma$. Hence, Lemma \ref{lem.ord-eq-Ppar} \textbf{(a)} (applied to
$\alpha = \varepsilon\mid_P$ and $\beta = \gamma$) yields
$\mathcal{E}\left(P, \varepsilon\mid_P\right)
= \mathcal{E}\left(P, \gamma\right)$.
Similarly,
$\mathcal{E}\left(Q, \varepsilon\mid_Q\right)
= \mathcal{E}\left(Q, \delta\right)$.

It is easy to see that a map $f : P \sqcup Q \to \mathcal{Z}$
is a $\mathcal{Z}$-enriched $\left(P \sqcup Q, \varepsilon\right)$-partition
if and only if $f \mid_P$ is a $\mathcal{Z}$-enriched
$\left(  P,\varepsilon\mid_P\right)$-partition and
$f \mid_Q$ is a $\mathcal{Z}$-enriched
$\left(  Q,\varepsilon\mid_Q\right)$-partition.

Therefore, the map
\begin{align*}
\mathcal{E}\left(  P\sqcup Q,\varepsilon\right)   &  \rightarrow
\mathcal{E}\left(  P,\varepsilon\mid_P\right)
\times\mathcal{E}\left(  Q,\varepsilon\mid_Q\right)
, \\
f  &  \mapsto\left(  f\mid_{P},f\mid_{Q}\right)
\end{align*}
is a bijection (this is easy to see). In other words, the map
\begin{align}
\mathcal{E}\left(  P\sqcup Q,\varepsilon\right)   &  \rightarrow
\mathcal{E}\left(  P,\gamma\right)  \times\mathcal{E}\left(  Q,\delta\right)
,\nonumber\\
f  &  \mapsto\left(  f\mid_{P},f\mid_{Q}\right)  \label{pf.prop.prod1.bij}%
\end{align}
is a bijection (since $\mathcal{E}\left(P, \varepsilon\mid_P\right)
= \mathcal{E}\left(P, \gamma\right)$
and
$\mathcal{E}\left(Q, \varepsilon\mid_Q\right)
= \mathcal{E}\left(Q, \delta\right)$).
Now, the definition of
$\Gamma_{\mathcal{Z}}\left(  P\sqcup Q,\varepsilon\right)$
yields
\begin{align*}
\Gamma_{\mathcal{Z}}\left(  P\sqcup Q,\varepsilon\right)   &  =\sum
_{f\in\mathcal{E}\left(  P\sqcup Q,\varepsilon\right)  }\underbrace{\prod
_{p\in P\sqcup Q}x_{\left\vert f\left(  p\right)  \right\vert }}_{=\left(
\prod_{p\in P}x_{\left\vert f\left(  p\right)  \right\vert }\right)  \left(
\prod_{p\in Q}x_{\left\vert f\left(  p\right)  \right\vert }\right)  }\\
&  = \sum_{f\in\mathcal{E}\left(  P\sqcup Q,\varepsilon\right)  }
\underbrace{
 \left( \prod_{p\in P}x_{\left\vert f\left(  p\right)  \right\vert }\right)
}_{
 =
 \prod_{p\in P}x_{\left\vert \left(f\mid_P\right) \left(  p\right)  \right\vert }
}
\underbrace{
 \left( \prod_{p\in Q}x_{\left\vert f\left(  p\right)  \right\vert }\right)
}_{
 =
 \prod_{p\in Q}x_{\left\vert \left(f\mid_Q\right) \left(  p\right)  \right\vert }
}
\\
&  = \sum_{f\in\mathcal{E}\left(  P\sqcup Q,\varepsilon\right)  }
 \left( \prod_{p\in P}x_{\left\vert \left(f\mid_P\right) \left(  p\right)  \right\vert }
 \right)
 \left(
 \prod_{p\in Q}x_{\left\vert \left(f\mid_Q\right) \left(  p\right)  \right\vert }
 \right)
\\
&  =
\sum_{\left(g, h\right) \in \mathcal{E}\left(  P,\gamma\right) 
            \times \mathcal{E}\left(  Q,\delta\right) }
\left( \prod_{p\in P}x_{\left\vert g\left(  p\right)  \right\vert }\right)
\left(\prod_{p\in Q}x_{\left\vert
h\left(  p\right)  \right\vert }\right) \\
&  \ \ \ \ \ \ \ \ \ \ \left(
\begin{array}
[c]{c}%
\text{here, we have substituted }\left(  g,h\right)  \text{ for }\left(
f\mid_{P},f\mid_{Q}\right)  \text{,}\\
\text{since the map (\ref{pf.prop.prod1.bij}) is a bijection}%
\end{array}
\right) \\
&  =\underbrace{\left(  \sum_{g\in\mathcal{E}\left(  P,\gamma\right)  }%
\prod_{p\in P}x_{\left\vert g\left(  p\right)  \right\vert }\right)
}_{= \sum_{f\in\mathcal{E}\left(  P,\gamma\right)  }%
\prod_{p\in P}x_{\left\vert f\left(  p\right)  \right\vert }
= \Gamma_{\mathcal{Z}}\left(  P,\gamma\right)  }
\ \cdot \ \underbrace{\left(
\sum_{h\in\mathcal{E}\left(  Q,\delta\right)  }\prod_{p\in Q}x_{\left\vert
h\left(  p\right)  \right\vert }\right)  }_{
= \sum_{f\in\mathcal{E}\left(  Q,\delta\right)  }\prod_{p\in Q}x_{\left\vert
f\left(  p\right)  \right\vert }
= \Gamma_{\mathcal{Z}}\left( Q,\delta\right)  }\\
&  =\Gamma_{\mathcal{Z}}\left(  P,\gamma\right)  \Gamma_{\mathcal{Z}}\left(
Q,\delta\right)  .
\end{align*}
This proves Proposition \ref{prop.prod1}.
\end{proof}

\begin{definition}
Let $n\in\mathbb{N}$. Let $\pi$ be any $n$-permutation. (Recall that we have
defined the concept of an \textquotedblleft$n$-permutation\textquotedblright%
\ in Definition \ref{def.perm}.) Then, $\left(  \left[  n\right]  ,\pi\right)
$ is a labeled poset (in fact, $\pi$ is an injective map $\left[  n\right]
\rightarrow\left\{  1,2,3,\ldots\right\}  $, and thus can be considered a
labeling). We define $\Gamma_{\mathcal{Z}}\left(  \pi\right)  $ to be the
power series $\Gamma_{\mathcal{Z}}\left(  \left[  n\right]  ,\pi\right)  $.
\end{definition}

Let us recall the concept of a ``poset homomorphism'':

\begin{definition}
Let $P$ and $Q$ be two posets.
A map $f : P \to Q$ is said to be a \textit{poset homomorphism} if
for any two elements $x$ and $y$ of $P$ satisfying $x \leq y$ in $P$,
we have $f\left(x\right) \leq f\left(y\right)$ in $Q$.
\end{definition}

It is well-known that if $U$ and $V$ are any two finite totally ordered sets
of the same size, then there is a unique poset isomorphism $U \to V$.
Thus, if $w$ is a finite totally ordered set with $n$ elements, then there
is a unique poset isomorphism $w \to \left[ n \right]$.
Now, we claim the following:

\begin{proposition}
\label{prop.Gamma=Gamma}Let $w$ be a finite totally ordered set with ground
set $W$. Let $n=\left\vert W\right\vert $. Let $\overline{w}$ be the unique
poset isomorphism $w\rightarrow\left[  n\right]  $. Let $\gamma:W\rightarrow
\left\{  1,2,3,\ldots\right\}  $ be any injective map. Then, $\Gamma
_{\mathcal{Z}}\left(  w,\gamma\right)  =\Gamma_{\mathcal{Z}}\left(
\gamma\circ\overline{w}^{-1}\right)  $.
\end{proposition}

\begin{proof}
[Proof of Proposition \ref{prop.Gamma=Gamma}.]Clearly,
$\left( w , \gamma \right)$ is a labeled poset (since $\gamma$ is injective).
The map $\gamma\circ\overline
{w}^{-1}:\left[  n\right]  \rightarrow\left\{  1,2,3,\ldots\right\}  $ is an
injective map, thus an $n$-permutation. Hence, $\Gamma_{\mathcal{Z}}\left(
\gamma\circ\overline{w}^{-1}\right)  $ is well-defined, and its definition
yields
$\Gamma_{\mathcal{Z}}\left(  \gamma\circ\overline{w}^{-1}\right)
=\Gamma_{\mathcal{Z}}\left(  \left[  n\right]  ,\gamma\circ\overline{w}%
^{-1}\right)  $.
But $\overline{w}$ is a poset isomorphism $w \to \left[ n \right]$, and
thus is an isomorphism of labeled posets\footnote{We define the notion of
an ``isomorphism of labeled posets'' in the obvious way:
If $\left(P, \alpha\right)$ and $\left(Q, \beta\right)$ are two labeled
posets, then a \textit{homomorphism of labeled posets} from
$\left(P, \alpha\right)$ to $\left(Q, \beta\right)$ means a poset
homomorphism $f : P \to Q$ satisfying $\alpha = \beta \circ f$.
A \textit{isomorphism of labeled posets} is an invertible homomorphism
of labeled posets whose inverse also is a homomorphism of labeled posets.
Note that this definition of an isomorphism is not equivalent to the
definition given in \cite[Section 1.1]{Stembr97}.}
from $\left(  w,\gamma\right)$ to $\left(  \left[
n\right]  ,\gamma\circ\overline{w}^{-1}\right)$.
Hence,
\begin{align*}
\mathcal{E}\left(  w,\gamma\right)   &  \rightarrow\mathcal{E}\left(  \left[
n\right]  ,\gamma\circ\overline{w}^{-1}\right)  ,\\
f  &  \mapsto f\circ\overline{w}^{-1}%
\end{align*}
is a bijection (since any isomorphism of labeled posets induces a bijection
between their $\mathcal{Z}$-enriched $\left(P,\gamma\right)$-partitions).
Furthermore, it satisfies $\prod_{p\in w}x_{\left\vert
f\left(  p\right)  \right\vert }=\prod_{p\in\left[  n\right]  }x_{\left\vert
\left(  f\circ\overline{w}^{-1}\right)  \left(  p\right)  \right\vert }$ for
each $f\in\mathcal{E}\left(  w,\gamma\right)  $. Hence, $\Gamma_{\mathcal{Z}%
}\left(  w,\gamma\right)  =\Gamma_{\mathcal{Z}}\left(  \left[  n\right]
,\gamma\circ\overline{w}^{-1}\right)  =\Gamma_{\mathcal{Z}}\left(  \gamma
\circ\overline{w}^{-1}\right)  $.
\end{proof}

For the following corollary, let us recall that a bijective poset
homomorphism is not necessarily an isomorphism of posets (since its
inverse may and may not be a poset homomorphism).

\begin{corollary}
\label{cor.fund-lem2}Let $\left(  P,\gamma\right)  $ be a labeled poset. Let
$n=\left\vert P\right\vert $. Then,%
\[
\Gamma_{\mathcal{Z}}\left(  P,\gamma\right)  =\sum_{\substack{x:P\rightarrow
\left[  n\right]  \\\text{bijective poset}\\\text{homomorphism}}%
}\Gamma_{\mathcal{Z}}\left(  \gamma\circ x^{-1}\right)  .
\]

\end{corollary}

\begin{proof}
[Proof of Corollary \ref{cor.fund-lem2}.]For each totally ordered set $w$ with
ground set $P$, we let $\overline{w}$ be the unique poset isomorphism
$w\rightarrow\left[  n\right]  $. If $w$ is a linear extension of $P$, then
this map $\overline{w}$ is also a bijective poset homomorphism $P\rightarrow
\left[  n\right]  $ (since every poset homomorphism $w\rightarrow\left[
n\right]  $ is also a poset homomorphism $P\rightarrow\left[  n\right]  $).
Thus, for each $w\in\mathcal{L}\left(  P\right)  $, we have defined a
bijective poset homomorphism $\overline{w}:P\rightarrow\left[  n\right]  $. We
thus have defined a map%
\begin{align}
\mathcal{L}\left(  P\right)    & \rightarrow\left\{  \text{bijective poset
homomorphisms }P\rightarrow\left[  n\right]  \right\}  ,\nonumber\\
w  & \mapsto\overline{w}.\label{pf.cor.fund-lem2.bij}%
\end{align}
This map is injective (indeed, a linear extension $w\in\mathcal{L}\left(
P\right)  $ can be uniquely reconstructed from $\overline{w}$) and surjective
(because if $x$ is a bijective poset homomorphism $P\rightarrow\left[
n\right]  $, then the linear extension $w\in\mathcal{L}\left(  P\right)  $
defined (as a list) by $w=\left(  x^{-1}\left(  1\right)  ,x^{-1}\left(
2\right)  ,\ldots,x^{-1}\left(  n\right)  \right)  $ satisfies $x=\overline
{w}$). Hence, this map is a bijection. 

Corollary \ref{cor.fund-lem} yields%
\begin{equation}
\Gamma_{\mathcal{Z}}\left(  P,\gamma\right)  =\sum_{w\in\mathcal{L}\left(
P\right)  }\underbrace{\Gamma_{\mathcal{Z}}\left(  w,\gamma\right)
}_{\substack{=\Gamma_{\mathcal{Z}}\left(  \gamma\circ\overline{w}^{-1}\right)
\\\text{(by Proposition \ref{prop.Gamma=Gamma})}}}=\sum_{w\in\mathcal{L}%
\left(  P\right)  }\Gamma_{\mathcal{Z}}\left(  \gamma\circ\overline{w}%
^{-1}\right)  .\label{pf.cor.fund-lem2.1}%
\end{equation}
But recall that the map (\ref{pf.cor.fund-lem2.bij}) is a bijection. Thus, we
can substitute $x$ for $\overline{w}$ in the sum $\sum_{w\in\mathcal{L}\left(
P\right)  }\Gamma_{\mathcal{Z}}\left(  \gamma\circ\overline{w}^{-1}\right)  $,
obtaining%
\[
\sum_{w\in\mathcal{L}\left(  P\right)  }\Gamma_{\mathcal{Z}}\left(
\gamma\circ\overline{w}^{-1}\right)  =\sum_{\substack{x:P\rightarrow\left[
n\right]  \\\text{bijective poset}\\\text{homomorphism}}}\Gamma_{\mathcal{Z}%
}\left(  \gamma\circ x^{-1}\right)  .
\]
Hence, (\ref{pf.cor.fund-lem2.1}) becomes%
\[
\Gamma_{\mathcal{Z}}\left(  P,\gamma\right)  =\sum_{w\in\mathcal{L}\left(
P\right)  }\Gamma_{\mathcal{Z}}\left(  \gamma\circ\overline{w}^{-1}\right)
=\sum_{\substack{x:P\rightarrow\left[  n\right]  \\\text{bijective
poset}\\\text{homomorphism}}}\Gamma_{\mathcal{Z}}\left(  \gamma\circ
x^{-1}\right)  .
\]
This proves Corollary \ref{cor.fund-lem2}.
\end{proof}

\begin{corollary}
\label{cor.prod2}Let $n\in\mathbb{N}$ and $m\in\mathbb{N}$. Let $\pi$ be an
$n$-permutation and let $\sigma$ be an $m$-permutation such that $\pi$ and
$\sigma$ are disjoint. Then,%
\[
\Gamma_{\mathcal{Z}}\left(  \pi\right)  \Gamma_{\mathcal{Z}}\left(
\sigma\right)  =\sum_{\tau\in S\left(  \pi,\sigma\right)  }\Gamma
_{\mathcal{Z}}\left(  \tau\right)  .
\]

\end{corollary}

\begin{proof}
[Proof of Corollary \ref{cor.prod2}.]
Consider the disjoint union $\left[  n\right]  \sqcup\left[  m\right]  $ of
the posets $\left[  n\right]  $ and $\left[  m\right]  $. (Note that this
disjoint union cannot be identified with the union $\left[  n\right]
\cup\left[  m\right]  $.) Let $\varepsilon$ be the map $\left[  n\right]
\sqcup\left[  m\right]  \rightarrow\left\{  1,2,3,\ldots\right\}  $ whose
restriction to $\left[  n\right]  $ is $\pi$ and whose restriction to $\left[
m\right]  $ is $\sigma$. This map $\varepsilon$ is injective, since $\pi$ and
$\sigma$ are disjoint permutations. Thus, $\left(  \left[  n\right]
\sqcup\left[  m\right]  ,\varepsilon\right)  $ is a labeled poset.

The following two observations are easy to show (see \cite{verlong}
for detailed proofs):

\begin{statement}
\textit{Observation 1:} If $x$ is a bijective poset homomorphism $\left[
n\right]  \sqcup\left[  m\right]  \rightarrow\left[  n+m\right]  $, then
$\varepsilon\circ x^{-1}\in S\left(  \pi,\sigma\right)  $.
\end{statement}

\begin{statement}
\textit{Observation 2:} If $\tau\in S\left(  \pi,\sigma\right)  $, then there
exists a unique bijective poset homomorphism $x:\left[  n\right]
\sqcup\left[  m\right]  \rightarrow\left[  n+m\right]  $ satisfying
$\varepsilon\circ x^{-1}=\tau$.
\end{statement}

Now, the map%
\begin{align*}
\left\{  \text{bijective poset homomorphisms }x:\left[  n\right]
\sqcup\left[  m\right]  \rightarrow\left[  n+m\right]  \right\}   &
\rightarrow S\left(  \pi,\sigma\right)  ,\\
x  &  \mapsto\varepsilon\circ x^{-1}%
\end{align*}
is well-defined (by Observation 1) and is a bijection (by Observation 2).
Hence, we can substitute $\varepsilon\circ x^{-1}$ for $\tau$ in the sum
$\sum_{\tau\in S\left(  \pi,\sigma\right)  }\Gamma_{\mathcal{Z}}\left(
\tau\right)  $. We thus obtain%
\begin{equation}
\sum_{\tau\in S\left(  \pi,\sigma\right)  }\Gamma_{\mathcal{Z}}\left(
\tau\right)  =\sum_{\substack{x:\left[  n\right]  \sqcup\left[  m\right]
\rightarrow\left[  n+m\right]  \\\text{bijective poset}\\\text{homomorphism}%
}}\Gamma_{\mathcal{Z}}\left(  \varepsilon\circ x^{-1}\right)  .
\label{pf.cor.prod2.sum=sum}%
\end{equation}


The definition of $\Gamma_{\mathcal{Z}}\left(  \pi\right)  $ yields
$\Gamma_{\mathcal{Z}}\left(  \pi\right)  =\Gamma_{\mathcal{Z}}\left(  \left[
n\right]  ,\pi\right)  $. The definition of $\Gamma_{\mathcal{Z}}\left(
\sigma\right)  $ yields $\Gamma_{\mathcal{Z}}\left(  \sigma\right)
=\Gamma_{\mathcal{Z}}\left(  \left[  m\right]  ,\sigma\right)  $. Multiplying
these two equalities, we obtain
\begin{align*}
\Gamma_{\mathcal{Z}}\left(  \pi\right)  \Gamma_{\mathcal{Z}}\left(
\sigma\right)   &  =\Gamma_{\mathcal{Z}}\left(  \left[  n\right]  ,\pi\right)
\Gamma_{\mathcal{Z}}\left(  \left[  m\right]  ,\sigma\right)  =\Gamma
_{\mathcal{Z}}\left(  \left[  n\right]  \sqcup\left[  m\right]  ,\varepsilon
\right) \\
&  \ \ \ \ \ \ \ \ \ \ \left(
\begin{array}
[c]{c}%
\text{by Proposition \ref{prop.prod1}, applied to }P=\left[  n\right]  \text{,
}\gamma=\pi\text{,}\\
Q=\left[  m\right]  \text{ and }\delta=\sigma
\end{array}
\right) \\
&  =\sum_{\substack{x:\left[  n\right]  \sqcup\left[  m\right]  \rightarrow
\left[  n+m\right]  \\\text{bijective poset}\\\text{homomorphism}}%
}\Gamma_{\mathcal{Z}}\left(  \varepsilon\circ x^{-1}\right) \\
&\ \ \ \ \ \ \ \ \ \ \left(
\begin{array}
[c]{c}%
\text{by Corollary \ref{cor.fund-lem2}, applied to}
\left[  n\right]  \sqcup\left[  m\right]  \text{, }\varepsilon \\
\text{ and
}n+m\text{ instead of }P\text{, }\gamma\text{ and }n
\end{array}
\right) \\
&  =\sum_{\tau\in S\left(  \pi,\sigma\right)  }\Gamma_{\mathcal{Z}}\left(
\tau\right)  \ \ \ \ \ \ \ \ \ \ \left(  \text{by (\ref{pf.cor.prod2.sum=sum}%
)}\right)  .
\end{align*}
This proves Corollary \ref{cor.prod2}.
\end{proof}

\subsection{Exterior peaks}

So far we have been doing general nonsense. Let us now specialize to a
situation that is connected to exterior peaks.

\begin{convention}
From now on, we set $\mathcal{N}=\left\{  0,1,2,\ldots\right\}  \cup\left\{
\infty\right\}  $, with total order given by $0\prec1\prec2\prec\cdots
\prec\infty$, and we set
\begin{align*}
\mathcal{Z}  &  =\left(  \mathcal{N}\times\left\{  +,-\right\}  \right)
\setminus\left\{  -0,+\infty\right\} \\
&  =\left\{  +0\right\}  \cup\left\{  +n\ \mid\ n\in\left\{  1,2,3,\ldots
\right\}  \right\}  \cup\left\{  -n\ \mid\ n\in\left\{  1,2,3,\ldots\right\}
\right\}  \cup\left\{  -\infty\right\}  .
\end{align*}
Recall that the total order on $\mathcal{Z}$ has%
\[
+0\prec-1\prec+1\prec-2\prec+2\prec\cdots\prec-\infty.
\]

\end{convention}

\begin{definition}
Let $S$ be a subset of $\mathbb{Z}$.
A map $\chi$ from $S$ to a totally ordered set $K$ is
said to be \textit{V-shaped} if there exists some $t\in S$ such that the map
$\chi\mid_{\left\{  s\in S\ \mid\ s\leq t\right\}  }$ is strictly decreasing
while the map $\chi\mid_{\left\{  s\in S\ \mid\ s\geq t\right\}  }$ is
strictly increasing. Notice that this $t\in S$ is uniquely determined in this
case; namely, it is the unique $k\in S$ that minimizes $\chi\left(  k\right)
$.
\end{definition}

Thus, roughly speaking, a map from a subset of $\ZZ$ to a totally
ordered set is \textit{V-shaped}
if and only if it is strictly decreasing up until a certain value of its
argument, and then strictly increasing afterwards.
For example, the $6$-permutation $\left(5, 1, 2, 3, 4\right)$ is V-shaped
(keep in mind that we regard $n$-permutations as injective maps
$\left[n\right] \to \mathbb{P}$), whereas the $4$-permutation
$\left(3, 1, 4, 2\right)$ is not.

\begin{definition}
Let $n\in\mathbb{N}$.

\begin{enumerate}
\item[\textbf{(a)}] Let $f:\left[  n\right]  \rightarrow\mathcal{Z}$ be any
map. Then, $\left\vert f\right\vert $ shall denote the map $\left[  n\right]
\rightarrow\mathcal{N},\ i\mapsto\left\vert f\left(  i\right)  \right\vert $.

\item[\textbf{(b)}] Let $g:\left[  n\right]  \rightarrow\mathcal{N}$ be any
map. Then, we define a monomial $\mathbf{x}_{g}$ in $\operatorname*{Pow}%
\mathcal{N}$ by $\mathbf{x}_{g}=\prod_{i=1}^{n}x_{g\left(  i\right)  }$.
\end{enumerate}
\end{definition}

Using this definition, we can rewrite the definition of $\Gamma_{\mathcal{Z}%
}\left(  \pi\right)  $ as follows:

\begin{proposition}
\label{prop.Gamma.rewr}Let $n\in\mathbb{N}$. Let $\pi$ be any $n$-permutation.
Then,%
\begin{equation}
\Gamma_{\mathcal{Z}}\left(  \pi\right)  =\sum_{f\in\mathcal{E}\left(  \left[
n\right]  ,\pi\right)  }\prod_{p\in\left[  n\right]  }x_{\left\vert f\left(
p\right)  \right\vert }=\sum_{f\in\mathcal{E}\left(  \left[  n\right]
,\pi\right)  }\mathbf{x}_{\left\vert f\right\vert }.\label{eq.Gamma.rewr}%
\end{equation}

\end{proposition}

\begin{proof}
[Proof of Proposition \ref{prop.Gamma.rewr}.] Easy consequence of the
definitions (see \cite{verlong} for details).
\end{proof}

\begin{definition}
\label{def.amen}
Let $n\in\mathbb{N}$. Let $g:\left[  n\right]  \rightarrow\mathcal{N}$ be any
map. Let $\pi$ be an $n$-permutation. We shall say that $g$ is $\pi
$\textit{-amenable} if it has the following properties:

\begin{enumerate}
\item[\textbf{(i')}] The map $\pi\mid_{g^{-1}\left(  0\right)  }$ is strictly
increasing. (This allows the case when $g^{-1}\left(  0\right)  =\varnothing$.)

\item[\textbf{(ii')}] For each $h\in g\left(  \left[  n\right]  \right)
\cap\left\{  1,2,3,\ldots\right\}  $, the map $\pi\mid_{g^{-1}\left(
h\right)  }$ is V-shaped.

\item[\textbf{(iii')}] The map $\pi\mid_{g^{-1}\left(  \infty\right)  }$ is
strictly decreasing. (This allows the case when $g^{-1}\left(  \infty\right)
=\varnothing$.)

\item[\textbf{(iv')}] The map $g$ is weakly increasing.
\end{enumerate}
\end{definition}

\begin{proposition}
\label{prop.Epk-formula}Let $n\in\mathbb{N}$. Let $\pi$ be any $n$%
-permutation. Then,%
\[
\Gamma_{\mathcal{Z}}\left(  \pi\right)  =\sum_{\substack{g:\left[  n\right]
\rightarrow\mathcal{N}\\\text{is }\pi\text{-amenable}}}2^{\left\vert g\left(
\left[  n\right]  \right)  \cap\left\{  1,2,3,\ldots\right\}  \right\vert
}\mathbf{x}_{g}.
\]

\end{proposition}

\begin{proof}
[Proof of Proposition \ref{prop.Epk-formula} (sketched).]
The claim will immediately follow from (\ref{eq.Gamma.rewr})
once we have shown the following two observations:

\begin{statement}
\textit{Observation 1:} If $f\in\mathcal{E}\left(  \left[  n\right]
,\pi\right)  $, then the map $\left\vert f\right\vert :\left[  n\right]
\rightarrow\mathcal{N}$ is $\pi$-amenable.
\end{statement}

\begin{statement}
\textit{Observation 2:} If $g:\left[  n\right]  \rightarrow\mathcal{N}$ is a
$\pi$-amenable map, then there exist precisely $2^{\left\vert g\left(  \left[
n\right]  \right)  \cap\left\{  1,2,3,\ldots\right\}  \right\vert }$ maps
$f\in\mathcal{E}\left(  \left[  n\right]  ,\pi\right)  $ satisfying
$\left\vert f\right\vert =g$.
\end{statement}

It thus remains to prove these two observations. Let us do this:

[\textit{Proof of Observation 1:} Let $f\in\mathcal{E}\left(  \left[
n\right]  ,\pi\right)  $. Thus, $f$ is a $\mathcal{Z}$-enriched $\left(
\left[  n\right]  ,\pi\right)  $-partition. In other words, $f$ is a map
$\left[  n\right]  \rightarrow\mathcal{Z}$ such that for all $x<y$ in $\left[
n\right]  $, the following conditions hold:

\begin{enumerate}
\item[\textbf{(i)}] We have $f\left(  x\right)  \preccurlyeq f\left(
y\right)  $.

\item[\textbf{(ii)}] If $f\left(  x\right)  =f\left(  y\right)  =+h$ for some
$h\in\mathcal{N}$, then $\pi\left(  x\right)  <\pi\left(  y\right)  $.

\item[\textbf{(iii)}] If $f\left(  x\right)  =f\left(  y\right)  =-h$ for some
$h\in\mathcal{N}$, then $\pi\left(  x\right)  >\pi\left(  y\right)  $.
\end{enumerate}

(This is due to the
definition of a $\mathcal{Z}$-enriched $\left(  \left[  n\right]  ,\pi\right)
$-partition.)

Condition \textbf{(i)} shows that the map $f$ is weakly increasing. Condition
\textbf{(ii)} shows that for each $h\in\mathcal{N}$, the map $\pi\mid
_{f^{-1}\left(  +h\right)  }$ is strictly increasing. Condition \textbf{(iii)}
shows that for each $h\in\mathcal{N}$, the map $\pi\mid_{f^{-1}\left(
-h\right)  }$ is strictly decreasing.

Now, set $g=\left\vert f\right\vert $. Then, $g^{-1}\left(  0\right)
=f^{-1}\left(  +0\right)  $ (since $-0\notin\mathcal{Z}$). But the map
$\pi\mid_{f^{-1}\left(  +0\right)  }$ is strictly increasing\footnote{because
for each $h\in\mathcal{N}$, the map $\pi\mid_{f^{-1}\left(  +h\right)  }$ is
strictly increasing}. Thus, the map $\pi\mid_{g^{-1}\left(  0\right)  }$ is
strictly increasing (since $g^{-1}\left(  0\right)  =f^{-1}\left(  +0\right)
$). Hence, Property \textbf{(i')} in Definition~\ref{def.amen} holds.
Similarly, Property \textbf{(iii')} in that definition also holds.

Now, fix $h\in g\left(  \left[  n\right]  \right)  \cap\left\{  1,2,3,\ldots
\right\}  $. Then, the set $g^{-1}\left(  h\right)  $ is nonempty (since $h\in
g\left(  \left[  n\right]  \right)  $), and can be written as the union of its
two disjoint subsets $f^{-1}\left(  +h\right)  $ and $f^{-1}\left(  -h\right)
$. Furthermore, each element of $f^{-1}\left(  -h\right)  $ is smaller than
each element of $f^{-1}\left(  +h\right)  $ (since $f$ is weakly increasing),
and we know that the map $\pi\mid_{f^{-1}\left(  -h\right)  }$ is strictly
decreasing while the map $\pi\mid_{f^{-1}\left(  +h\right)  }$ is strictly
increasing. Hence, the map $\pi\mid_{g^{-1}\left(  h\right)  }$ is strictly
decreasing up until some value of its argument (namely, either the largest
element of $f^{-1}\left(  -h\right)  $, or the smallest
element of $f^{-1}\left(  +h\right)  $, depending on which of these two
elements has the smaller image under $\pi$), and then strictly increasing
from there on. In other words, the map $\pi\mid_{g^{-1}\left(  h\right)  }$ is
V-shaped. Thus, Property \textbf{(ii')} in Definition~\ref{def.amen} holds.
Finally, Property \textbf{(iv')} in Definition~\ref{def.amen} holds because
$f$ is weakly increasing. We have
thus checked all four properties in Definition~\ref{def.amen};
thus, $g$ is $\pi$-amenable.
In other words, $\left| f \right|$ is $\pi$-amenable
(since $g = \left| f \right|$).
This proves Observation 1.]

[\textit{Proof of Observation 2:} Let $g:\left[  n\right]  \rightarrow
\mathcal{N}$ be a $\pi$-amenable map. Consider a map $f\in\mathcal{E}\left(
\left[  n\right]  ,\pi\right)  $ satisfying $\left\vert f\right\vert =g$. We
are wondering to what extent the map $f$ is determined by $g$ and $\pi$.

Everything that we said in the proof of Observation 1 still holds in
our situation (since $g = \left| f \right|$).

In order to determine the map $f$, it clearly suffices to determine the sets
$f^{-1}\left(  q\right)  $ for all $q\in\mathcal{Z}$. In other words, it
suffices to determine the set $f^{-1}\left(  +0\right)  $, the set
$f^{-1}\left(  -\infty\right)  $ and the sets $f^{-1}\left(  +h\right)  $ and
$f^{-1}\left(  -h\right)  $ for all $h\in\left\{  1,2,3,\ldots\right\}  $.

Recall from the proof of Observation 1 that $g^{-1}\left(  0\right)
=f^{-1}\left(  +0\right)  $. Thus, $f^{-1}\left(  +0\right)  $ is uniquely
determined by $g$. Similarly, $f^{-1}\left(  -\infty\right)  $ is uniquely
determined by $g$. Thus, we can focus on the remaining sets $f^{-1}\left(
+h\right)  $ and $f^{-1}\left(  -h\right)  $ for $h\in\left\{  1,2,3,\ldots
\right\}  $.

Fix $h\in\left\{  1,2,3,\ldots\right\}  $. Recall that the set $g^{-1}\left(
h\right)  $ is the union of its two disjoint subsets $f^{-1}\left(  +h\right)
$ and $f^{-1}\left(  -h\right)  $. Thus, $f^{-1}\left(  +h\right)  $ and
$f^{-1}\left(  -h\right)  $ are complementary subsets of $g^{-1}\left(
h\right)  $. If $g^{-1}\left(  h\right)  =\varnothing$, then this uniquely
determines $f^{-1}\left(  +h\right)  $ and $f^{-1}\left(  -h\right)  $. Thus,
we focus only on the case when $g^{-1}\left(  h\right)  \neq\varnothing$.

So assume that $g^{-1}\left(  h\right)  \neq\varnothing$. Hence, $h\in
g\left(  \left[  n\right]  \right)  $, so that $h\in g\left(  \left[
n\right]  \right)  \cap\left\{  1,2,3,\ldots\right\}  $. Since the map $g$ is
$\pi$-amenable, we thus conclude that the map $\pi\mid_{g^{-1}\left(
h\right)  }$ is V-shaped (by Property \textbf{(ii')} in
Definition~\ref{def.amen}).

The map $g$ is weakly increasing (by Property \textbf{(iv')} in
Definition~\ref{def.amen}). Hence,
$g^{-1}\left(  h\right)  $ is an interval of $\left[  n\right]  $. Let
$\alpha\in\mathbb{Z}$ and $\gamma\in\mathbb{Z}$ be such that $g^{-1}\left(
h\right)  =\left[  \alpha,\gamma\right]  $ (where $\left[  \alpha
,\gamma\right]  $ means the interval $\left\{  \alpha,\alpha+1,\ldots
,\gamma\right\}  $).

As in the proof of Observation 1, we can see that each element of
$f^{-1}\left(  -h\right)  $ is smaller than each element of $f^{-1}\left(
+h\right)  $. Since the union of $f^{-1}\left(  -h\right)  $ and
$f^{-1}\left(  +h\right)  $ is $g^{-1}\left(  h\right)  =\left[  \alpha
,\gamma\right]  $, we thus conclude that (roughly speaking) the
sets $f^{-1}\left(  -h\right)  $ and $f^{-1}\left(  +h\right)  $
partition the interval $g^{-1}\left(h\right)$ in two (possibly empty)
sub-intervals such that the interval $f^{-1}\left(  -h\right)  $ lies
completely to the left of $f^{-1}\left(  +h\right)  $.
Hence, there exists some $\beta\in\left[
\alpha-1,\gamma\right]  $ such that $f^{-1}\left(  -h\right)  =\left[
\alpha,\beta\right]  $ and $f^{-1}\left(  +h\right)  =\left[  \beta
+1,\gamma\right]  $. Consider this $\beta$. Clearly, $f^{-1}\left(  -h\right)
$ and $f^{-1}\left(  +h\right)  $ are uniquely determined by $\beta$; we just
need to find out which values $\beta$ can take.

As in the proof of Observation 1, we can see that the map $\pi\mid
_{f^{-1}\left(  -h\right)  }$ is strictly decreasing while the map $\pi
\mid_{f^{-1}\left(  +h\right)  }$ is strictly increasing. Let $k$ be the
element of $g^{-1}\left(  h\right)  $ minimizing $\pi\left(  k\right)  $.
Then, the map $\pi$ is strictly decreasing on the set $\left\{  u\in
g^{-1}\left(  h\right)  \ \mid\ u\leq k\right\}  $ and strictly increasing on
the set $\left\{  u\in g^{-1}\left(  h\right)  \ \mid\ u\geq k\right\}  $
(since the map $\pi\mid_{g^{-1}\left(  h\right)  }$ is V-shaped).

The map $\pi\mid_{f^{-1}\left(  -h\right)  }$ is strictly decreasing. In other
words, the map $\pi$ is strictly decreasing on the set $f^{-1}\left(
-h\right)  =\left[  \alpha,\beta\right]  $. On the other hand, the map $\pi$
is strictly increasing on the set $\left\{  u\in g^{-1}\left(  h\right)
\ \mid\ u\geq k\right\}  $. Hence, the two sets $\left[  \alpha,\beta\right]
$ and $\left\{  u\in g^{-1}\left(  h\right)  \ \mid\ u\geq k\right\}  $ cannot
have more than one element in common
(since $\pi$ is strictly decreasing on one
and strictly increasing on the other). Thus, $k\geq\beta$. A similar argument
shows that $k\leq\beta+1$. Combining these inequalities, we obtain
$k\in\left\{  \beta,\beta+1\right\}  $, so that $\beta\in\left\{
k,k-1\right\}  $. This shows that $\beta$ can take only two values: $k$ and
$k-1$.

Now, let us forget that we fixed $h$.
We have shown that for each $h\in g\left(
\left[  n\right]  \right)  \cap\left\{  1,2,3,\ldots\right\}  $, the sets
$f^{-1}\left(  +h\right)  $ and $f^{-1}\left(  -h\right)  $ are uniquely
determined once the integer $\beta$ is chosen, and that this integer $\beta$
can be chosen in two ways. (As we have seen, all other values of $h$ do not
matter.) Thus, in total, the map $f$ is uniquely determined up to $\left\vert
g\left(  \left[  n\right]  \right)  \cap\left\{  1,2,3,\ldots\right\}
\right\vert $ decisions, where each decision allows choosing from two values.
Thus, there are at most $2^{\left\vert g\left(  \left[  n\right]  \right)
\cap\left\{  1,2,3,\ldots\right\}  \right\vert }$ maps $f\in\mathcal{E}\left(
\left[  n\right]  ,\pi\right)  $ satisfying $\left\vert f\right\vert =g$.
Working the above argument backwards, we see that each way of making these
decisions
actually leads to a map $f\in\mathcal{E}\left(  \left[  n\right]  ,\pi\right)
$ satisfying $\left\vert f\right\vert =g$; thus, there are \textbf{exactly}
$2^{\left\vert g\left(  \left[  n\right]  \right)  \cap\left\{  1,2,3,\ldots
\right\}  \right\vert }$ maps $f\in\mathcal{E}\left(  \left[  n\right]
,\pi\right)  $ satisfying $\left\vert f\right\vert =g$. This proves
Observation 2.]
\end{proof}

Now, let us observe that if $g:\left[  n\right]  \rightarrow\mathcal{N}$ is a
weakly increasing map (for some $n\in\mathbb{N}$), then the fibers of $g$
(that is, the subsets $g^{-1}\left(  h\right)  $ of $\left[  n\right]  $ for
various $h\in\mathcal{N}$) are intervals of $\left[  n\right]  $ (possibly
empty). Of course, when these fibers are nonempty, they have smallest elements
and largest elements. We shall next study these elements more closely.

\begin{definition}
\label{def.fiberends}Let $n\in\mathbb{N}$. Let $g:\left[  n\right]
\rightarrow\mathcal{N}$ be any map. We define a subset $\operatorname*{FE}%
\left(  g\right)  $ of $\left[  n\right]  $ as follows:%
\begin{align*}
\operatorname*{FE}\left(  g\right)   &  =\left\{  \min\left(  g^{-1}\left(
h\right)  \right)  \ \mid\ h\in\left\{  1,2,3,\ldots,\infty\right\}  \text{
with }g^{-1}\left(  h\right)  \neq\varnothing\right\} \\
&  \ \ \ \ \ \ \ \ \ \ \cup\left\{  \max\left(  g^{-1}\left(  h\right)
\right)  \ \mid\ h\in\left\{  0,1,2,3,\ldots\right\}  \text{ with }%
g^{-1}\left(  h\right)  \neq\varnothing\right\}  .
\end{align*}
In other words, $\operatorname*{FE}\left(  g\right)  $ is the set comprising
the smallest elements of all nonempty fibers of $g$ except for $g^{-1}\left(
0\right)  $ as well as the largest elements of all nonempty fibers of $g$
except for $g^{-1}\left(  \infty\right)  $. We shall refer to the elements of
$\operatorname*{FE}\left(  g\right)  $ as the \textit{fiber-ends} of $g$.
\end{definition}

\begin{lemma}
\label{lem.FE.exist}Let $n\in\mathbb{N}$. Let $\Lambda\in\mathbf{L}_{n}$.
Then, there exists a weakly increasing map $g:\left[  n\right]  \rightarrow
\mathcal{N}$ such that $\operatorname*{FE}\left(  g\right)  =\left(
\Lambda\cup\left(  \Lambda+1\right)  \right)  \cap\left[  n\right]  $.
\end{lemma}

\begin{proof}
[Proof of Lemma \ref{lem.FE.exist} (sketched).]If $n=0$, then Lemma
\ref{lem.FE.exist} holds for obvious reasons. Thus, WLOG assume that $n\neq0$.
Hence, $n$ is a positive integer. Thus, $\mathbf{L}_{n}$ is the set of all
nonempty lacunar subsets of $\left[  n\right]  $ (by the definition of
$\mathbf{L}_{n}$). Therefore, from $\Lambda\in\mathbf{L}_{n}$, we conclude
that $\Lambda$ is a nonempty lacunar subset of $\left[  n\right]  $. Write
this subset $\Lambda$ in the form $\Lambda=\left\{  j_{1}<j_{2}<\cdots
<j_{k}\right\}  $. Thus, $k\geq1$ (since $\Lambda$ is nonempty).

From $\left\{  j_{1}<j_{2}<\cdots<j_{k}\right\}  =\Lambda\subseteq\left[
n\right]  $, we obtain $0<j_{1}<j_{2}<\cdots<j_{k}\leq n$. Thus,
$1<j_{1}+1<j_{2}+1<\cdots<j_{k}+1\leq n+1$. Hence, the $k-1$ numbers
$j_{1}+1,j_{2}+1,\ldots,j_{k-1}+1$ all belong to the set $\left[  n\right]  $,
whereas the number $j_{k}+1$ only belongs to this set if $j_{k}<n$. Hence,%
\begin{align*}
& \left\{  j_{1}+1,j_{2}+1,\ldots,j_{k}+1\right\}  \cap\left[  n\right]  \\
& =%
\begin{cases}
\left\{  j_{1}+1,j_{2}+1,\ldots,j_{k-1}+1,j_{k}+1\right\}  , & \text{if }%
j_{k}<n;\\
\left\{  j_{1}+1,j_{2}+1,\ldots,j_{k-1}+1\right\}  , & \text{if }j_{k}=n
\end{cases}
.
\end{align*}
But $\Lambda=\left\{  j_{1}<j_{2}<\cdots<j_{k}\right\}  =\left\{  j_{1}%
,j_{2},\ldots,j_{k}\right\}  $ and thus \newline $\Lambda+1=\left\{
j_{1}+1,j_{2}+1,\ldots,j_{k}+1\right\}  $. Hence,%
\begin{align}
\left(  \Lambda+1\right)  \cap\left[  n\right]    & =\left\{  j_{1}%
+1,j_{2}+1,\ldots,j_{k}+1\right\}  \cap\left[  n\right]  \nonumber\\
& =%
\begin{cases}
\left\{  j_{1}+1,j_{2}+1,\ldots,j_{k-1}+1,j_{k}+1\right\}  , & \text{if }%
j_{k}<n;\\
\left\{  j_{1}+1,j_{2}+1,\ldots,j_{k-1}+1\right\}  , & \text{if }j_{k}=n
\end{cases}
.\label{pf.lem.FE.exist.short.2}%
\end{align}


Now, consider the map $g:\left[  n\right]  \rightarrow\mathcal{N}$ defined by%
\begin{align*}
g\left(  x\right)   &  =%
\begin{cases}
\left(  \text{the number of all }\lambda\in\Lambda\text{ such that }%
\lambda<x\right)  , & \text{if }x\leq j_{k};\\
\infty, & \text{if }x>j_{k}%
\end{cases}
\\
&  \ \ \ \ \ \ \ \ \ \ \text{for each }x\in\left[  n\right]  .
\end{align*}
Thus,%
\begin{align*}
&  \left(  g\left(  1\right)  ,g\left(  2\right)  ,\ldots,g\left(  n\right)
\right)  \\
&  =\left(  \underbrace{0,0,\ldots,0}_{j_{1}\text{ entries}}%
,\underbrace{1,1,\ldots,1}_{j_{2}-j_{1}\text{ entries}},\underbrace{2,2,\ldots
,2}_{j_{3}-j_{2}\text{ entries}},\ldots,\underbrace{k-1,k-1,\ldots,k-1}%
_{j_{k}-j_{k-1}\text{ entries}},\underbrace{\infty,\infty,\ldots,\infty
}_{n-j_{k}\text{ entries}}\right)  .
\end{align*}
The $n$-tuple on the right hand side of this equality consists of a block of
$0$'s, followed by a block of $1$'s, followed by a block of $2$'s, and so on,
all the way up to a block of $\left(  k-1\right)  $'s, which is then followed
by a block of $\infty$'s. The first $k$ of these blocks are nonempty (since
$0<j_{1}<j_{2}<\cdots<j_{k}$). The last block is nonempty if $j_{k}<n$, and
empty if $j_{k}=n$. Thus, the map $g$ is weakly increasing, and its nonempty
fibers are
\begin{align*}
g^{-1}\left(  0\right)   &  =\left\{  1,2,\ldots,j_{1}\right\}  ,\\
g^{-1}\left(  1\right)   &  =\left\{  j_{1}+1,j_{1}+2,\ldots,j_{2}\right\}
,\\
g^{-1}\left(  2\right)   &  =\left\{  j_{2}+1,j_{2}+2,\ldots,j_{3}\right\}
,\\
&  \vdots,\\
g^{-1}\left(  k-1\right)   &  =\left\{  j_{k-1}+1,j_{k-1}+2,\ldots
,j_{k}\right\}  ,\\
g^{-1}\left(  \infty\right)   &  =\left\{  j_{k}+1,j_{k}+2,\ldots,n\right\}  ,
\end{align*}
except that $g^{-1}\left(  \infty\right)  $ is empty when $j_{k}=n$.

Hence,
the definition of $\operatorname*{FE}\left(  g\right)  $ yields%
\begin{align*}
&  \operatorname*{FE}\left(  g\right)  \\
&  =\underbrace{\left\{  \min\left(  g^{-1}\left(  h\right)  \right)
\ \mid\ h\in\left\{  1,2,3,\ldots,\infty\right\}  \text{ with }g^{-1}\left(
h\right)  \neq\varnothing\right\}  }_{=%
\begin{cases}
\left\{  \min\left(  g^{-1}\left(  1\right)  \right)  ,\min\left(
g^{-1}\left(  2\right)  \right)  ,\ldots,\min\left(  g^{-1}\left(  k-1\right)
\right)  ,\min\left(  g^{-1}\left(  \infty\right)  \right)  \right\}  , &
\text{if }j_{k}<n;\\
\left\{  \min\left(  g^{-1}\left(  1\right)  \right)  ,\min\left(
g^{-1}\left(  2\right)  \right)  ,\ldots,\min\left(  g^{-1}\left(  k-1\right)
\right)  \right\}  , & \text{if }j_{k}=n
\end{cases}
}\\
&  \ \ \ \ \ \ \ \ \ \ \cup\underbrace{\left\{  \max\left(  g^{-1}\left(
h\right)  \right)  \ \mid\ h\in\left\{  0,1,2,3,\ldots\right\}  \text{ with
}g^{-1}\left(  h\right)  \neq\varnothing\right\}  }_{=\left\{  \max\left(
g^{-1}\left(  0\right)  \right)  ,\max\left(  g^{-1}\left(  1\right)  \right)
,\max\left(  g^{-1}\left(  2\right)  \right)  ,\ldots,\max\left(
g^{-1}\left(  k-1\right)  \right)  \right\}  }\\
&  =\underbrace{%
\begin{cases}
\left\{  \min\left(  g^{-1}\left(  1\right)  \right)  ,\min\left(
g^{-1}\left(  2\right)  \right)  ,\ldots,\min\left(  g^{-1}\left(  k-1\right)
\right)  ,\min\left(  g^{-1}\left(  \infty\right)  \right)  \right\}  , &
\text{if }j_{k}<n;\\
\left\{  \min\left(  g^{-1}\left(  1\right)  \right)  ,\min\left(
g^{-1}\left(  2\right)  \right)  ,\ldots,\min\left(  g^{-1}\left(  k-1\right)
\right)  \right\}  , & \text{if }j_{k}=n
\end{cases}
}_{\substack{=%
\begin{cases}
\left\{  j_{1}+1,j_{2}+1,\ldots,j_{k-1}+1,j_{k}+1\right\}  , & \text{if }%
j_{k}<n;\\
\left\{  j_{1}+1,j_{2}+1,\ldots,j_{k-1}+1\right\}  , & \text{if }j_{k}=n
\end{cases}
\\\text{(since }\min\left(  g^{-1}\left(  h\right)  \right)  =j_{h}+1\text{
for each }h\in\left\{  1,2,\ldots,k-1\right\}  \text{,}\\\text{and since }%
\min\left(  g^{-1}\left(  \infty\right)  \right)  =j_{k}+1\text{ if }%
j_{k}<n\text{)}}}\\
&  \ \ \ \ \ \ \ \ \ \ \cup\underbrace{\left\{  \max\left(  g^{-1}\left(
0\right)  \right)  ,\max\left(  g^{-1}\left(  1\right)  \right)  ,\max\left(
g^{-1}\left(  2\right)  \right)  ,\ldots,\max\left(  g^{-1}\left(  k-1\right)
\right)  \right\}  }_{\substack{=\left\{  j_{1},j_{2},\ldots,j_{k}\right\}
\\\text{(since }\max\left(  g^{-1}\left(  h-1\right)  \right)  =j_{h}\text{
for each }h\in\left\{  1,2,\ldots,k\right\}  \text{)}}}\\
&  =\underbrace{%
\begin{cases}
\left\{  j_{1}+1,j_{2}+1,\ldots,j_{k-1}+1,j_{k}+1\right\}  , & \text{if }%
j_{k}<n;\\
\left\{  j_{1}+1,j_{2}+1,\ldots,j_{k-1}+1\right\}  , & \text{if }j_{k}=n
\end{cases}
}_{\substack{=\left(  \Lambda+1\right)  \cap\left[  n\right]  \\\text{(by
(\ref{pf.lem.FE.exist.short.2}))}}}\cup\underbrace{\left\{  j_{1},j_{2}%
,\ldots,j_{k}\right\}  }_{\substack{=\Lambda=\Lambda\cap\left[  n\right]
\\\text{(since }\Lambda\subseteq\left[  n\right]  \text{)}}}\\
&  =\left(  \left(  \Lambda+1\right)  \cap\left[  n\right]  \right)
\cup\left(  \Lambda\cap\left[  n\right]  \right)  =\left(  \Lambda\cap\left[
n\right]  \right)  \cup\left(  \left(  \Lambda+1\right)  \cap\left[  n\right]
\right)  =\left(  \Lambda\cup\left(  \Lambda+1\right)  \right)  \cap\left[
n\right]  .
\end{align*}
Altogether, we have now shown that our map $g:\left[  n\right]  \rightarrow
\mathcal{N}$ is weakly increasing and satisfies $\operatorname*{FE}\left(
g\right)  =\left(  \Lambda\cup\left(  \Lambda+1\right)  \right)  \cap\left[
n\right]  $. Hence, such a map $g$ exists. Thus, Lemma \ref{lem.FE.exist} is proven.
\end{proof}

\begin{proposition}
\label{prop.Epk.fiberends}Let $n\in\mathbb{N}$.
Let $\pi$ be an $n$-permutation.
Let $g:\left[  n\right]
\rightarrow\mathcal{N}$ be any weakly increasing map. Then, the map $g$ is
$\pi$-amenable if and only if $\operatorname{Epk}\pi\subseteq
\operatorname*{FE}\left(  g\right)  $.
\end{proposition}

\begin{proof}
[Proof of Proposition \ref{prop.Epk.fiberends}.] The map $g$ is weakly
increasing. Thus, all nonempty fibers $g^{-1}\left(  h\right)  $ of $g$ are
intervals of $\left[  n\right]  $. Recall that $g$ is $\pi$-amenable if and
only if the four Properties \textbf{(i')}, \textbf{(ii')}, \textbf{(iii')} and
\textbf{(iv')} in Definition \ref{def.amen} hold. Consider these four
properties. Since Property \textbf{(iv')} automatically holds (since we
assumed $g$ to be weakly increasing), we thus only need to discuss the other three:

\begin{itemize}
\item Property \textbf{(i')} is equivalent to the statement that every
exterior peak of $\pi$ that lies in the fiber $g^{-1}\left(  0\right)  $ must
be the largest element of this fiber. (Indeed, a strictly increasing map is
characterized by having no exterior peaks except for the largest element
of its domain.)

\item Property \textbf{(ii')} is equivalent to the statement that every peak
of $\pi$ that lies in a fiber $g^{-1}\left(  h\right)  $ with $h\in g\left(
\left[  n\right]  \right)  \cap\left\{  1,2,3,\ldots\right\}  $ must be either
the smallest or the largest element of this fiber (because the restriction
$\pi\mid_{g^{-1}\left(  h\right)  }$ is V-shaped if and only if no peak of
$\pi$ appears in the interior of this fiber\footnote{The \textit{interior} of
an interval $\left\{  a,a+1,\ldots,b\right\}  $ of $\left[  n\right]  $ is
defined to be the interval $\left\{  a+1,a+2,\ldots,b-1\right\}  $. (This is
an empty interval if $a+1>b-1$.)}). Moreover, we can replace the word
\textquotedblleft peak\textquotedblright\ by \textquotedblleft exterior
peak\textquotedblright\ in this sentence (since the exterior peaks $1$ and $n$
must automatically be the smallest and the largest element of whatever fibers
they belong to).

\item Property \textbf{(iii')} is equivalent to the statement that every
exterior peak of $\pi$ that lies in the fiber $g^{-1}\left(  \infty\right)  $
must be the smallest element of this fiber. (Indeed, a strictly decreasing map
is characterized by having no exterior peaks except for the smallest element
of its domain.)
\end{itemize}

Combining all of these insights, we conclude that the four Properties
\textbf{(i')}, \textbf{(ii')}, \textbf{(iii')} and \textbf{(iv')} hold if and
only if every exterior peak of $\pi$ is a fiber-end of $g$. In other words,
$g$ is $\pi$-amenable if and only if every exterior peak of $\pi$ is a
fiber-end of $g$ (since $g$ is $\pi$-amenable if and only if the four
Properties \textbf{(i')}, \textbf{(ii')}, \textbf{(iii')} and \textbf{(iv')}
hold). In other words, $g$ is $\pi$-amenable if and only if
$\operatorname{Epk}\pi\subseteq\operatorname*{FE}\left(  g\right)  $. This
proves Proposition \ref{prop.Epk.fiberends}.
\end{proof}


We can rewrite Proposition \ref{prop.Epk-formula} as follows, exhibiting its
analogy with \cite[Proposition 2.2]{Stembr97}:

\begin{proposition}
\label{prop.Epk-formula2}Let $n\in\mathbb{N}$. Let $\pi$ be any $n$%
-permutation. Then,%
\[
\Gamma_{\mathcal{Z}}\left(  \pi\right)  =\sum_{\substack{g:\left[  n\right]
\rightarrow\mathcal{N}\text{ is}\\\text{weakly increasing;}%
\\\operatorname{Epk}\pi\subseteq\operatorname*{FE}\left(  g\right)
}}2^{\left\vert g\left(  \left[  n\right]  \right)  \cap\left\{
1,2,3,\ldots\right\}  \right\vert }\mathbf{x}_{g}.
\]

\end{proposition}

\begin{proof}
[Proof of Proposition \ref{prop.Epk-formula2}.]Each $\pi$-amenable map
$g:\left[  n\right]  \rightarrow\mathcal{N}$ is weakly increasing (because of
Property \textbf{(iv')} in Definition \ref{def.amen}). Hence, Proposition
\ref{prop.Epk.fiberends} yields that the $\pi$-amenable maps $g:\left[
n\right]  \rightarrow\mathcal{N}$ are precisely the weakly increasing maps
$g:\left[  n\right]  \rightarrow\mathcal{N}$ satisfying
$\operatorname{Epk} \pi\subseteq\operatorname*{FE}\left(  g\right)  $.
Thus, Proposition \ref{prop.Epk-formula2} follows from
Proposition \ref{prop.Epk-formula}.
\end{proof}


\begin{definition}
\label{def.KnL}Let $n\in\mathbb{N}$. If $\Lambda$ is any subset of $\left[
n\right]  $, then we define a power series $K_{n,\Lambda}^{\mathcal{Z}}%
\in\operatorname*{Pow}\mathcal{N}$ by%
\begin{equation}
K_{n,\Lambda}^{\mathcal{Z}}=\sum_{\substack{g:\left[  n\right]  \rightarrow
\mathcal{N}\text{ is}\\\text{weakly increasing;}\\\Lambda\subseteq
\operatorname*{FE}\left(  g\right)  }}2^{\left\vert g\left(  \left[  n\right]
\right)  \cap\left\{  1,2,3,\ldots\right\}  \right\vert }\mathbf{x}_{g}.
\label{eq.def.KnL.1}%
\end{equation}
Thus, if $\pi$ is an $n$-permutation, then
Proposition \ref{prop.Epk-formula2} shows that
\begin{equation}
\Gamma_{\mathcal{Z}}\left(  \pi\right)
=K_{n, \operatorname{Epk} \pi}^{\mathcal{Z}}.
\label{eq.def.KnL.2}%
\end{equation}
\end{definition}

\begin{remark}
\label{rmk.KnL.concrete}Let $n\in\mathbb{N}$. Let $\Lambda$ be any subset of
$\left[  n\right]  $. It is easy to see that if $g:\left[  n\right]
\rightarrow\mathcal{N}$ is a weakly increasing map, and if $i\in\left[
n\right]  $, then $i\in\operatorname*{FE}\left(  g\right)  $ holds if and only
if we don't have $g\left(  i-1\right)  =g\left(  i\right)  =g\left(
i+1\right)  $, where we use the convention that $g\left(  0\right)  =0$ and
$g\left(  n+1\right)  =\infty$. Hence, a weakly increasing map $g:\left[
n\right]  \rightarrow\mathcal{N}$ satisfies $\Lambda\subseteq
\operatorname*{FE}\left(  g\right)  $ if and only if no $i\in\Lambda$
satisfies $g\left(  i-1\right)  =g\left(  i\right)  =g\left(  i+1\right)  $,
where we use the convention that $g\left(  0\right)  =0$ and $g\left(
n+1\right)  =\infty$. Thus, (\ref{eq.def.KnL.1}) can be rewritten as follows:%
\begin{align}
K_{n,\Lambda}^{\mathcal{Z}}  &  =\sum_{\substack{g:\left[  n\right]
\rightarrow\mathcal{N}\text{ is}\\\text{weakly increasing;}\\\text{no }%
i\in\Lambda\text{ satisfies }g\left(  i-1\right)  =g\left(  i\right)
=g\left(  i+1\right)  \\\text{(where we set }g\left(  0\right)  =0\text{ and
}g\left(  n+1\right)  =\infty\text{)}}}2^{\left\vert g\left(  \left[
n\right]  \right)  \cap\left\{  1,2,3,\ldots\right\}  \right\vert }%
\mathbf{x}_{g}\nonumber\\
&  =\sum_{\substack{\left(  g_{1},g_{2},\ldots,g_{n}\right)  \in
\mathcal{N}^{n};\\0\preccurlyeq g_{1}\preccurlyeq g_{2}\preccurlyeq
\cdots\preccurlyeq g_{n}\preccurlyeq\infty;\\\text{no }i\in\Lambda\text{
satisfies }g_{i-1}=g_{i}=g_{i+1}\\\text{(where we set }g_{0}=0\text{ and
}g_{n+1}=\infty\text{)}}}2^{\left\vert \left\{  g_{1},g_{2},\ldots
,g_{n}\right\}  \cap\left\{  1,2,3,\ldots\right\}  \right\vert }x_{g_{1}%
}x_{g_{2}}\cdots x_{g_{n}} \label{eq.rmk.KnL.concrete.2}%
\end{align}
(here, we have substituted $\left(  g_{1},g_{2},\ldots,g_{n}\right)  $ for
$\left(  g\left(  1\right)  ,g\left(  2\right)  ,\ldots,g\left(  n\right)
\right)  $ in the sum). For example,
\begin{align*}
K_{3,\left\{  1,3\right\}  }^{\mathcal{Z}}  &  =\sum_{\substack{\left(
g_{1},g_{2},g_{3}\right)  \in\mathcal{N}^{3};\\0\preccurlyeq g_{1}\preccurlyeq
g_{2}\preccurlyeq g_{3}\preccurlyeq\infty;\\\text{no }i\in\left\{
1,3\right\}  \text{ satisfies }g_{i-1}=g_{i}=g_{i+1}\\\text{(where we set
}g_{0}=0\text{ and }g_{4}=\infty\text{)}}}2^{\left\vert \left\{  g_{1}%
,g_{2},g_{3}\right\}  \cap\left\{  1,2,3,\ldots\right\}  \right\vert }%
x_{g_{1}}x_{g_{2}}x_{g_{3}}\\
&  =\sum_{\substack{\left(  g_{1},g_{2},g_{3}\right)  \in\mathcal{N}%
^{3};\\0\preccurlyeq g_{1}\preccurlyeq g_{2}\preccurlyeq g_{3}\preccurlyeq
\infty;\\\text{neither }0=g_{1}=g_{2}\text{ nor }g_{2}=g_{3}=\infty\text{
holds}}}2^{\left\vert \left\{  g_{1},g_{2},g_{3}\right\}  \cap\left\{
1,2,3,\ldots\right\}  \right\vert }x_{g_{1}}x_{g_{2}}x_{g_{3}}.
\end{align*}


As a consequence of (\ref{eq.rmk.KnL.concrete.2}), we see that if we
substitute $0$ for $x_{0}$ and for $x_{\infty}$, then $K_{n,\Lambda
}^{\mathcal{Z}}$ becomes the power series%
\begin{align*}
&  \sum_{\substack{\left(  g_{1},g_{2},\ldots,g_{n}\right)  \in\mathcal{N}%
^{n};\\0\preccurlyeq g_{1}\preccurlyeq g_{2}\preccurlyeq\cdots\preccurlyeq
g_{n}\preccurlyeq\infty;\\\text{no }i\in\Lambda\text{ satisfies }g_{i-1}%
=g_{i}=g_{i+1}\\\text{(where we set }g_{0}=0\text{ and }g_{n+1}=\infty
\text{);}\\\text{none of the }g_{i}\text{ equals }0\text{ or }\infty
}}2^{\left\vert \left\{  g_{1},g_{2},\ldots,g_{n}\right\}  \cap\left\{
1,2,3,\ldots\right\}  \right\vert }x_{g_{1}}x_{g_{2}}\cdots x_{g_{n}}\\
&  =\sum_{\substack{\left(  g_{1},g_{2},\ldots,g_{n}\right)  \in\left\{
1,2,3,\ldots\right\}  ^{n};\\g_{1}\preccurlyeq g_{2}\preccurlyeq
\cdots\preccurlyeq g_{n};\\\text{no }i\in\Lambda\setminus\left\{  1,n\right\}
\text{ satisfies }g_{i-1}=g_{i}=g_{i+1}}}2^{\left\vert \left\{  g_{1}%
,g_{2},\ldots,g_{n}\right\}  \right\vert }x_{g_{1}}x_{g_{2}}\cdots x_{g_{n}}%
\end{align*}
in the indeterminates $x_{1},x_{2},x_{3},\ldots$. This is called the
\textquotedblleft shifted quasi-symmetric function $\Theta_{\Lambda
\setminus\left\{  1,n\right\}  }^{n}\left(  X\right)  $\textquotedblright\ in
\cite[(3.2)]{BilHai95}.
\end{remark}

Corollary \ref{cor.prod2} now leads directly to the following multiplication
rule (an analogue of \cite[(3.1)]{Stembr97}):

\begin{corollary}
\label{cor.KnEpk-prodrule}Let $n\in\mathbb{N}$ and $m\in\mathbb{N}$. Let $\pi$
be an $n$-permutation. Let $\sigma$ be an $m$-permutation such that $\pi$
and $\sigma$ are disjoint. Then,
\[
K_{n,\operatorname{Epk} \pi}^{\mathcal{Z}}
\cdot K_{m,\operatorname{Epk} \sigma}^{\mathcal{Z}}
= \sum_{\tau\in S\left(  \pi,\sigma\right)  }
   K_{n+m,\operatorname{Epk}\tau}^{\mathcal{Z}}.
\]

\end{corollary}

\begin{example}
Applying Corollary \ref{cor.KnEpk-prodrule} to $n=2$, $m=1$, $\pi=\left(
1,2\right)  $ and $\sigma=\left(  3\right)  $, we obtain%
\[
K_{2,\operatorname{Epk}\left(  1,2\right)  }^{\mathcal{Z}}\cdot
K_{1,\operatorname{Epk}\left(  3\right)  }^{\mathcal{Z}}%
=K_{3,\operatorname{Epk}\left(  3,1,2\right)  }^{\mathcal{Z}}%
+K_{3,\operatorname{Epk}\left(  1,3,2\right)  }^{\mathcal{Z}}%
+K_{3,\operatorname{Epk}\left(  1,2,3\right)  }^{\mathcal{Z}}.
\]
In other words,%
\[
K_{2,\left\{  2\right\}  }^{\mathcal{Z}}\cdot K_{1,\left\{  1\right\}
}^{\mathcal{Z}}=K_{3,\left\{  1,3\right\}  }^{\mathcal{Z}}+K_{3,\left\{
2\right\}  }^{\mathcal{Z}}+K_{3,\left\{  3\right\}  }^{\mathcal{Z}}.
\]

\end{example}

\begin{proof}
[Proof of Corollary \ref{cor.KnEpk-prodrule}.]From (\ref{eq.def.KnL.2}), we
obtain
$\Gamma_{\mathcal{Z}}\left(  \pi\right)
=K_{n,\operatorname{Epk}\pi}^{\mathcal{Z}}$.
Similarly, \newline
$\Gamma_{\mathcal{Z}}\left(  \sigma\right)
=K_{m,\operatorname{Epk}\sigma}^{\mathcal{Z}}$. Multiplying these two
equalities, we obtain $\Gamma_{\mathcal{Z}}\left(  \pi\right)  \cdot
\Gamma_{\mathcal{Z}}\left(  \sigma\right)  =K_{n,\operatorname{Epk}\pi
}^{\mathcal{Z}}\cdot K_{m,\operatorname{Epk}\sigma}^{\mathcal{Z}}$. Hence,%
\begin{align*}
K_{n,\operatorname{Epk}\pi}^{\mathcal{Z}}
\cdot K_{m,\operatorname{Epk} \sigma}^{\mathcal{Z}}
&  =\Gamma_{\mathcal{Z}}\left(  \pi\right)  \cdot
\Gamma_{\mathcal{Z}}\left(  \sigma\right)
=\sum_{\tau\in S\left(  \pi
,\sigma\right)  }\underbrace{\Gamma_{\mathcal{Z}}\left(  \tau\right)
}_{\substack{=K_{n+m,\operatorname{Epk}\tau}^{\mathcal{Z}}\\\text{(by
(\ref{eq.def.KnL.2}))}}}\ \ \ \ \ \ \ \ \ \ \left(  \text{by Corollary
\ref{cor.prod2}}\right) \\
&  =\sum_{\tau\in S\left(  \pi,\sigma\right)  }K_{n+m,\operatorname{Epk}\tau
}^{\mathcal{Z}}.
\end{align*}
This proves Corollary \ref{cor.KnEpk-prodrule}.
\end{proof}

Recall Definition~\ref{def.lac.Ln}.

\begin{proposition}
\label{prop.KnL.lindep}Let $n\in\mathbb{N}$. Then, the family%
\[
\left(  K_{n,\Lambda}^{\mathcal{Z}}\right)  _{\Lambda\in \mathbf{L}_n}%
\]
is $\mathbb{Q}$-linearly independent.
\end{proposition}

Our proof of Proposition \ref{prop.KnL.lindep} requires
the following definition:

\begin{definition}
Let $\mathfrak{m}$ be any monomial in $\operatorname*{Pow}\mathcal{N}$ (that
is, a formal commutative product of indeterminates $x_{h}$ with $h\in
\mathcal{N}$). Let $f\in\operatorname*{Pow}\mathcal{N}$. Then, $\left[
\mathfrak{m}\right]  \left(  f\right)  $ shall mean the coefficient of
$\mathfrak{m}$ in the power series $f$. (For example, $\left[  x_{0}^{2}%
x_{3}\right]  \left(  3+5x_{0}^{2}x_{3}+6x_{0}+9x_{\infty}\right)  =5$ and
$\left[  x_{0}^{2}x_{3}\right]  \left(  x_{1}-x_{\infty}\right)  =0$.)
\end{definition}

\begin{lemma}
\label{lem.KnL.coeff}Let $n\in\mathbb{N}$.

\textbf{(a)} If $g$ and $h$ are two weakly increasing maps $\left[  n\right]
\rightarrow\mathcal{N}$, then
\[
\left(\text{we have $\xx_g = \xx_h$ if and only if $g = h$}\right) .
\]


\textbf{(b)} Let $R\in\mathbf{L}_{n}$. Let $h:\left[  n\right]  \rightarrow
\mathcal{N}$ be a weakly increasing map. Then,%
\[
\left[  \mathbf{x}_{h}\right]  \left(  K_{n,R}^{\mathcal{Z}}\right)  =%
\begin{cases}
2^{\left\vert h\left(  \left[  n\right]  \right)  \cap\left\{  1,2,3,\ldots
\right\}  \right\vert }, & \text{if }R\subseteq\operatorname*{FE}\left(
h\right)  ;\\
0, & \text{otherwise}%
\end{cases}
.
\]

\end{lemma}

\begin{proof}
[Proof of Lemma \ref{lem.KnL.coeff}.]\textbf{(a)} A weakly increasing map
$g:\left[  n\right]  \rightarrow\mathcal{N}$ can be uniquely reconstructed
from the \textbf{multiset} $\left\{  g\left(  1\right)  ,g\left(  2\right)
,\ldots,g\left(  n\right)  \right\}  _{\operatorname*{multi}}$ of its values
(because it is weakly increasing, so there is only one way in which these
values can be ordered). Hence, a weakly increasing map $g:\left[  n\right]
\rightarrow\mathcal{N}$ can be uniquely reconstructed from the monomial
$\mathbf{x}_{g}$ (since this monomial
$\mathbf{x}_{g}=x_{g\left(1\right)}x_{g\left(2\right)}\cdots
x_{g\left(n\right)}$
encodes the multiset $\left\{  g\left(  1\right)  ,g\left(
2\right)  ,\ldots,g\left(  n\right)  \right\}  _{\operatorname*{multi}}$). In
other words, if $g$ and $h$ are two weakly increasing maps $\left[  n\right]
\rightarrow\mathcal{N}$, then $\mathbf{x}_{g}=\mathbf{x}_{h}$ holds if and
only if $g=h$. This proves Lemma \ref{lem.KnL.coeff} \textbf{(a)}.

\textbf{(b)} The definition of $K_{n,R}^{\mathcal{Z}}$ yields%
\[
K_{n,R}^{\mathcal{Z}}=\sum_{\substack{g:\left[  n\right]  \rightarrow
\mathcal{N}\text{ is}\\\text{weakly increasing;}\\R\subseteq\operatorname*{FE}%
\left(  g\right)  }}2^{\left\vert g\left(  \left[  n\right]  \right)
\cap\left\{  1,2,3,\ldots\right\}  \right\vert }\mathbf{x}_{g}.
\]
Thus,%
\begin{align*}
\left[  \mathbf{x}_{h}\right]  \left(  K_{n,R}^{\mathcal{Z}}\right)   &
=\left[  \mathbf{x}_{h}\right]  \left(  \sum_{\substack{g:\left[  n\right]
\rightarrow\mathcal{N}\text{ is}\\\text{weakly increasing;}\\R\subseteq
\operatorname*{FE}\left(  g\right)  }}2^{\left\vert g\left(  \left[  n\right]
\right)  \cap\left\{  1,2,3,\ldots\right\}  \right\vert }\mathbf{x}_{g}\right)
\\
&  =\sum_{\substack{g:\left[  n\right]  \rightarrow\mathcal{N}\text{
is}\\\text{weakly increasing;}\\R\subseteq\operatorname*{FE}\left(  g\right)
}}2^{\left\vert g\left(  \left[  n\right]  \right)  \cap\left\{
1,2,3,\ldots\right\}  \right\vert }\underbrace{\left[  \mathbf{x}_{h}\right]
\left(  \mathbf{x}_{g}\right)  }_{\substack{=%
\begin{cases}
1, & \text{if }\mathbf{x}_{g}=\mathbf{x}_{h};\\
0, & \text{if }\mathbf{x}_{g}\neq\mathbf{x}_{h}%
\end{cases}
\\\text{(since }\mathbf{x}_{h}\text{ and }\mathbf{x}_{g}\text{ are two
monomials)}}}\\
&  =\sum_{\substack{g:\left[  n\right]  \rightarrow\mathcal{N}\text{
is}\\\text{weakly increasing;}\\R\subseteq\operatorname*{FE}\left(  g\right)
}}2^{\left\vert g\left(  \left[  n\right]  \right)  \cap\left\{
1,2,3,\ldots\right\}  \right\vert }\underbrace{%
\begin{cases}
1, & \text{if }\mathbf{x}_{g}=\mathbf{x}_{h};\\
0, & \text{if }\mathbf{x}_{g}\neq\mathbf{x}_{h}%
\end{cases}
}_{\substack{=%
\begin{cases}
1, & \text{if }g=h;\\
0, & \text{if }g\neq h
\end{cases}
\\\text{(by Lemma \ref{lem.KnL.coeff} \textbf{(a)})}}}\\
&  =\sum_{\substack{g:\left[  n\right]  \rightarrow\mathcal{N}\text{
is}\\\text{weakly increasing;}\\R\subseteq\operatorname*{FE}\left(  g\right)
}}2^{\left\vert g\left(  \left[  n\right]  \right)  \cap\left\{
1,2,3,\ldots\right\}  \right\vert }%
\begin{cases}
1, & \text{if }g=h;\\
0, & \text{if }g\neq h
\end{cases}
\\
&  =\sum_{\substack{g:\left[  n\right]  \rightarrow\mathcal{N}\text{
is}\\\text{weakly increasing;}\\R\subseteq\operatorname*{FE}\left(  g\right)
;\\g=h}}2^{\left\vert g\left(  \left[  n\right]  \right)  \cap\left\{
1,2,3,\ldots\right\}  \right\vert }.
\end{align*}
The sum on the right hand side of this equality has a unique addend (namely,
its addend for $g=h$, which is $2^{\left\vert h\left(  \left[  n\right]
\right)  \cap\left\{  1,2,3,\ldots\right\}  \right\vert }$) when
$R\subseteq\operatorname*{FE}\left(  h\right)  $; otherwise it is an empty
sum. Hence, this sum simplifies as follows:%
\[
\sum_{\substack{g:\left[  n\right]  \rightarrow\mathcal{N}\text{
is}\\\text{weakly increasing;}\\R\subseteq\operatorname*{FE}\left(  g\right)
;\\g=h}}2^{\left\vert g\left(  \left[  n\right]  \right)  \cap\left\{
1,2,3,\ldots\right\}  \right\vert }=%
\begin{cases}
2^{\left\vert h\left(  \left[  n\right]  \right)  \cap\left\{  1,2,3,\ldots
\right\}  \right\vert }, & \text{if }R\subseteq\operatorname*{FE}\left(
h\right)  ;\\
0, & \text{otherwise}%
\end{cases}
.
\]
Hence,%
\[
\left[  \mathbf{x}_{h}\right]  \left(  K_{n,R}^{\mathcal{Z}}\right)
=\sum_{\substack{g:\left[  n\right]  \rightarrow\mathcal{N}\text{
is}\\\text{weakly increasing;}\\R\subseteq\operatorname*{FE}\left(  g\right)
;\\g=h}}2^{\left\vert g\left(  \left[  n\right]  \right)  \cap\left\{
1,2,3,\ldots\right\}  \right\vert }=%
\begin{cases}
2^{\left\vert h\left(  \left[  n\right]  \right)  \cap\left\{  1,2,3,\ldots
\right\}  \right\vert }, & \text{if }R\subseteq\operatorname*{FE}\left(
h\right)  ;\\
0, & \text{otherwise}%
\end{cases}
.
\]
This proves Lemma \ref{lem.KnL.coeff} \textbf{(b)}.
\end{proof}

\begin{proof}
[First proof of Proposition \ref{prop.KnL.lindep}.]Recall Definition
\ref{def.order-on-P}. In the following, we shall regard the set $\mathbf{P}$
as a totally ordered set, equipped with the order from Proposition
\ref{prop.lac.order}.

Clearly, $\mathbf{L}_{n}\subseteq\mathbf{P}$. Hence, we consider
$\mathbf{L}_{n}$ as a totally ordered set, whose total order is inherited from
$\mathbf{P}$.

Let $\left(  a_{R}\right)  _{R\in\mathbf{L}_{n}}\in\mathbb{Q}^{\mathbf{L}_{n}%
}$ be a family of scalars (in $\mathbb{Q}$) such that $\sum_{R\in
\mathbf{L}_{n}}a_{R}K_{n,R}^{\mathcal{Z}}=0$. We are going to show that
$\left(  a_{R}\right)  _{R\in\mathbf{L}_{n}}=\left(  0\right)  _{R\in
\mathbf{L}_{n}}$.

Indeed, assume the contrary. Thus, $\left(  a_{R}\right)  _{R\in\mathbf{L}%
_{n}}\neq\left(  0\right)  _{R\in\mathbf{L}_{n}}$. Hence, there exists some
$R\in\mathbf{L}_{n}$ such that $a_{R}\neq0$. Let $\Lambda$ be the
\textbf{largest} such $R$ (with respect to the total order on $\mathbf{L}_{n}$
we have introduced above). Hence, $\Lambda$ is an element of $\mathbf{L}_{n}$
and satisfies $a_{\Lambda}\neq0$; but every element $R\in\mathbf{L}_{n}$
satisfying $R>\Lambda$ must satisfy%
\begin{equation}
a_{R}=0. \label{pf.prop.KnL.lindep.larger=0}%
\end{equation}

Lemma \ref{lem.FE.exist} shows that there exists a weakly increasing map
$g:\left[  n\right]  \rightarrow\mathcal{N}$ such that $\operatorname*{FE}%
\left(  g\right)  =\left(  \Lambda\cup\left(  \Lambda+1\right)  \right)
\cap\left[  n\right]  $. Consider this $g$. Combining $\Lambda\subseteq
\Lambda\cup\left(  \Lambda+1\right)  $ with $\Lambda\subseteq\left[  n\right]
$, we obtain
\[
\Lambda\subseteq\left(  \Lambda\cup\left(  \Lambda+1\right)  \right)
\cap\left[  n\right]  =\operatorname*{FE}\left(  g\right)  .
\]


For every $R\in\mathbf{L}_{n}$ satisfying $R\neq\Lambda$, we have%
\begin{equation}
\left[  \mathbf{x}_{g}\right]  \left(  a_{R}K_{n,R}^{\mathcal{Z}}\right)  =0.
\label{pf.prop.KnL.lindep.coeff=0}%
\end{equation}


[\textit{Proof of (\ref{pf.prop.KnL.lindep.coeff=0}):} Let $R\in\mathbf{L}%
_{n}$ be such that $R\neq\Lambda$. We must prove
(\ref{pf.prop.KnL.lindep.coeff=0}).

Assume the contrary. Thus, $\left[  \mathbf{x}_{g}\right]  \left(
a_{R}K_{n,R}^{\mathcal{Z}}\right)  \neq0$. In other words, $a_{R}\left[
\mathbf{x}_{g}\right]  \left(  K_{n,R}^{\mathcal{Z}}\right)  \neq0$. Hence,
$a_{R}\neq0$ and $\left[  \mathbf{x}_{g}\right]  \left(  K_{n,R}^{\mathcal{Z}%
}\right)  \neq0$.

From the definition of $\mathbf{L}_{n}$, it follows easily that every element
of $\mathbf{L}_{n}$ is a lacunar subset of $\left[  n\right]  $. Hence, $R$ is
a lacunar subset of $\left[  n\right]  $ (since $R\in\mathbf{L}_{n}$).

But Lemma \ref{lem.KnL.coeff} \textbf{(b)} (applied to $h=g$) yields%
\[
\left[  \mathbf{x}_{g}\right]  \left(  K_{n,R}^{\mathcal{Z}}\right)  =%
\begin{cases}
2^{\left\vert g\left(  \left[  n\right]  \right)  \cap\left\{  1,2,3,\ldots
\right\}  \right\vert }, & \text{if }R\subseteq\operatorname*{FE}\left(
g\right)  ;\\
0, & \text{otherwise}%
\end{cases}
.
\]
Hence, $\left[  \mathbf{x}_{g}\right]  \left(  K_{n,R}^{\mathcal{Z}}\right)
=0$ if $R\not \subseteq \operatorname*{FE}\left(  g\right)  $. Thus, we cannot
have $R\not \subseteq \operatorname*{FE}\left(  g\right)  $ (since $\left[
\mathbf{x}_{g}\right]  \left(  K_{n,R}^{\mathcal{Z}}\right)  \neq0$).
Therefore, we have
\[
R\subseteq\operatorname*{FE}\left(  g\right)  =\left(  \Lambda\cup\left(
\Lambda+1\right)  \right)  \cap\left[  n\right]  \subseteq\Lambda\cup\left(
\Lambda+1\right)  .
\]
Thus, Proposition \ref{prop.lac.RL} yields that $R\geq\Lambda$. Combining this
with $R\neq\Lambda$, we obtain $R>\Lambda$. Hence,
(\ref{pf.prop.KnL.lindep.larger=0}) yields $a_{R}=0$. This contradicts
$a_{R}\neq0$. This contradiction shows that our assumption was wrong. Hence,
(\ref{pf.prop.KnL.lindep.coeff=0}) is proven.]

On the other hand, Lemma \ref{lem.KnL.coeff} \textbf{(b)} (applied to $h=g$
and $R=\Lambda$) yields%
\[
\left[  \mathbf{x}_{g}\right]  \left(  K_{n,\Lambda}^{\mathcal{Z}}\right)  =%
\begin{cases}
2^{\left\vert g\left(  \left[  n\right]  \right)  \cap\left\{  1,2,3,\ldots
\right\}  \right\vert }, & \text{if }\Lambda\subseteq\operatorname*{FE}\left(
g\right)  ;\\
0, & \text{otherwise}%
\end{cases}
=2^{\left\vert g\left(  \left[  n\right]  \right)  \cap\left\{  1,2,3,\ldots
\right\}  \right\vert }%
\]
(since $\Lambda\subseteq\operatorname*{FE}\left(  g\right)  $).

Now, recall that $\sum_{R\in\mathbf{L}_{n}}a_{R}K_{n,R}^{\mathcal{Z}}=0$.
Hence, $\left[  \mathbf{x}_{g}\right]  \left(  \sum_{R\in\mathbf{L}_{n}}%
a_{R}K_{n,R}^{\mathcal{Z}}\right)  =\left[  \mathbf{x}_{g}\right]  \left(
0\right)  =0$. Therefore,%
\begin{align*}
0 &  =\left[  \mathbf{x}_{g}\right]  \left(  \sum_{R\in\mathbf{L}_{n}}%
a_{R}K_{n,R}^{\mathcal{Z}}\right)  =\sum_{R\in\mathbf{L}_{n}}\left[
\mathbf{x}_{g}\right]  \left(  a_{R}K_{n,R}^{\mathcal{Z}}\right)  \\
&  =\left[  \mathbf{x}_{g}\right]  \left(  a_{\Lambda}K_{n,\Lambda
}^{\mathcal{Z}}\right)  +\sum_{\substack{R\in\mathbf{L}_{n};\\R\neq\Lambda
}}\underbrace{\left[  \mathbf{x}_{g}\right]  \left(  a_{R}K_{n,R}%
^{\mathcal{Z}}\right)  }_{\substack{=0\\\text{(by
(\ref{pf.prop.KnL.lindep.coeff=0}))}}}\ \ \ \ \ \ \ \ \ \ \left(  \text{since
}\Lambda\in\mathbf{L}_{n}\right)  \\
&  =\left[  \mathbf{x}_{g}\right]  \left(  a_{\Lambda}K_{n,\Lambda
}^{\mathcal{Z}}\right)  =a_{\Lambda}\underbrace{\left[  \mathbf{x}_{g}\right]
\left(  K_{n,\Lambda}^{\mathcal{Z}}\right)  }_{=2^{\left\vert g\left(  \left[
n\right]  \right)  \cap\left\{  1,2,3,\ldots\right\}  \right\vert }%
}=\underbrace{a_{\Lambda}}_{\neq0}\underbrace{2^{\left\vert g\left(  \left[
n\right]  \right)  \cap\left\{  1,2,3,\ldots\right\}  \right\vert }}_{\neq
0}\neq0.
\end{align*}
This contradiction shows that our assumption was false. Hence, $\left(
a_{R}\right)  _{R\in\mathbf{L}_{n}}=\left(  0\right)  _{R\in\mathbf{L}_{n}}$
is proven.

Now, forget that we fixed $\left(  a_{R}\right)  _{R\in\mathbf{L}_{n}}$. We
thus have shown that if $\left(  a_{R}\right)  _{R\in\mathbf{L}_{n}}%
\in\mathbb{Q}^{\mathbf{L}_{n}}$ is a family of scalars (in $\mathbb{Q}$) such
that $\sum_{R\in\mathbf{L}_{n}}a_{R}K_{n,R}^{\mathcal{Z}}=0$, then $\left(
a_{R}\right)  _{R\in\mathbf{L}_{n}}=\left(  0\right)  _{R\in\mathbf{L}_{n}}$.
In other words, the family $\left(  K_{n,R}^{\mathcal{Z}}\right)
_{R\in\mathbf{L}_{n}}$ is $\mathbb{Q}$-linearly independent. In other words,
the family $\left(  K_{n,\Lambda}^{\mathcal{Z}}\right)  _{\Lambda\in
\mathbf{L}_{n}}$ is $\mathbb{Q}$-linearly independent. This proves Proposition
\ref{prop.KnL.lindep}.
\end{proof}

A second proof of Proposition \ref{prop.KnL.lindep} can be found in
\cite{verlong}.

\begin{corollary}
\label{cor.KnL.lindep-all}The family%
\[
\left(  K_{n,\Lambda}^{\mathcal{Z}}\right) %
_{n \in \NN; \ \Lambda \in \mathbf{L}_n}
\]
is $\mathbb{Q}$-linearly independent.
\end{corollary}

\begin{proof}
[Proof of Corollary \ref{cor.KnL.lindep-all}.]Follows from Proposition
\ref{prop.KnL.lindep} using gradedness; see \cite{verlong} for details.
\end{proof}

We can now finally prove what we came here for:

\begin{theorem}
\label{thm.Epk.sh-co-a}The permutation statistic $\operatorname{Epk}$ is shuffle-compatible.
\end{theorem}

\begin{proof}
[Proof of Theorem \ref{thm.Epk.sh-co-a}.]We must prove that
$\operatorname{Epk}$ is shuffle-compatible. In other words, we must prove
that for any two disjoint permutations $\pi$ and $\sigma$, the multiset
\newline
$\left\{  \operatorname{Epk} \tau   \ \mid\ \tau\in S\left(
\pi,\sigma\right)  \right\}  _{\operatorname*{multi}}$ depends only on
$\operatorname{Epk} \pi$, $\operatorname{Epk} \sigma$,
$\left\vert \pi\right\vert $ and $\left\vert \sigma \right\vert $.
In other words, we must prove that if $\pi$ and $\sigma$ are
two disjoint permutations, and if $\pi^{\prime}$ and $\sigma^{\prime}$ are two
disjoint permutations satisfying $\operatorname{Epk} \pi 
=\operatorname{Epk}\left(  \pi^{\prime}\right)  $,
$\operatorname{Epk} \sigma
=\operatorname{Epk}\left(  \sigma^{\prime}\right)  $,
$\left\vert \pi\right\vert =\left\vert \pi^{\prime}\right\vert $ and
$\left\vert \sigma\right\vert =\left\vert \sigma^{\prime}\right\vert $, then
the multiset $\left\{  \operatorname{Epk} \tau   \ \mid\ \tau\in
S\left(  \pi,\sigma\right)  \right\}  _{\operatorname*{multi}}$ equals the
multiset $\left\{  \operatorname{Epk} \tau   \ \mid\ \tau\in
S\left(  \pi^{\prime},\sigma^{\prime}\right)  \right\}
_{\operatorname*{multi}}$.

So let $\pi$ and $\sigma$ be two disjoint permutations, and let $\pi^{\prime}$
and $\sigma^{\prime}$ be two disjoint permutations satisfying
$\operatorname{Epk} \pi   =\operatorname{Epk}\left(
\pi^{\prime}\right)  $, $\operatorname{Epk} \sigma 
=\operatorname{Epk}\left(  \sigma^{\prime}\right)  $, $\left\vert
\pi\right\vert =\left\vert \pi^{\prime}\right\vert $ and $\left\vert
\sigma\right\vert =\left\vert \sigma^{\prime}\right\vert $.

Define $n\in\mathbb{N}$ by $n=\left\vert \pi\right\vert =\left\vert
\pi^{\prime}\right\vert $ (this is well-defined, since $\left\vert
\pi\right\vert =\left\vert \pi^{\prime}\right\vert $). Likewise, define
$m\in\mathbb{N}$ by $m=\left\vert \sigma\right\vert =\left\vert \sigma
^{\prime}\right\vert $. Thus, $\pi$ is an $n$-permutation, while $\sigma$ is
an $m$-permutation. Hence, each $\tau\in S\left(  \pi,\sigma\right)  $ is an
$\left(  n+m\right)  $-permutation, and therefore satisfies
$\operatorname{Epk}\tau\in\mathbf{L}_{n+m}$ (by Proposition
\ref{prop.Epk-lac}, applied to $n+m$ and $\tau$ instead of $n$ and $\pi$).
Thus, the multiset $\left\{  \operatorname{Epk} \tau 
\ \mid\ \tau\in S\left(  \pi,\sigma\right)  \right\}  _{\operatorname*{multi}%
}$ consists of elements of $\mathbf{L}_{n+m}$. The same holds for the multiset
$\left\{  \operatorname{Epk} \tau   \ \mid\ \tau\in S\left(
\pi^{\prime},\sigma^{\prime}\right)  \right\}  _{\operatorname*{multi}}$ (for
similar reasons).

Corollary \ref{cor.KnEpk-prodrule} yields
\begin{align*}
&  K_{n,\operatorname{Epk} \pi   }^{\mathcal{Z}}\cdot
K_{m,\operatorname{Epk} \sigma   }^{\mathcal{Z}}\\
&  =\sum_{\tau\in S\left(  \pi,\sigma\right)  }K_{n+m,\operatorname{Epk}\tau
}^{\mathcal{Z}}=\sum_{\substack{\Lambda\in\mathbf{L}_{n+m}}}\underbrace{\sum
_{\substack{\tau\in S\left(  \pi,\sigma\right)  ;\\
\operatorname{Epk} \tau=\Lambda}} K_{n+m,\Lambda}^{\mathcal{Z}}
}_{=\left\vert \left\{  \tau\in
S\left(  \pi,\sigma\right)  \ \mid\ \operatorname{Epk}\tau=\Lambda\right\}
\right\vert K_{n+m,\Lambda}^{\mathcal{Z}}}\\
&  \ \ \ \ \ \ \ \ \ \ \left(  \text{because each }\tau\in S\left(  \pi
,\sigma\right)  \text{ satisfies }\operatorname{Epk}\tau\in\mathbf{L}%
_{n+m}\right)  \\
&  =\sum_{\substack{\Lambda\in\mathbf{L}_{n+m}}}\left\vert \left\{  \tau\in
S\left(  \pi,\sigma\right)  \ \mid\ \operatorname{Epk}\tau=\Lambda\right\}
\right\vert K_{n+m,\Lambda}^{\mathcal{Z}}.
\end{align*}
The same argument (but using $\pi^{\prime}$ and $\sigma^{\prime}$ instead
of $\pi$ and $\sigma$) yields
\[
K_{n,\operatorname{Epk}\left(  \pi^{\prime}\right)  }^{\mathcal{Z}}\cdot
K_{m,\operatorname{Epk}\left(  \sigma^{\prime}\right)  }^{\mathcal{Z}}%
=\sum_{\substack{\Lambda\in\mathbf{L}_{n+m}}}\left\vert \left\{  \tau\in
S\left(  \pi^{\prime},\sigma^{\prime}\right)
\ \mid\ \operatorname{Epk} \tau=\Lambda\right\}  \right\vert
K_{n+m,\Lambda}^{\mathcal{Z}}.
\]
The left-hand sides of these two equalities are identical (since
$\operatorname{Epk} \pi   =\operatorname{Epk}\left(
\pi^{\prime}\right)  $ and $\operatorname{Epk} \sigma 
=\operatorname{Epk}\left(  \sigma^{\prime}\right)  $). Thus, their right-hand
sides must also be identical. In other words, we have%
\begin{align*}
&  \sum_{\substack{\Lambda\in\mathbf{L}_{n+m}}}\left\vert \left\{  \tau\in
S\left(  \pi,\sigma\right)  \ \mid\ \operatorname{Epk}\tau=\Lambda\right\}
\right\vert K_{n+m,\Lambda}^{\mathcal{Z}}\\
&  =\sum_{\substack{\Lambda\in\mathbf{L}_{n+m}}}\left\vert \left\{  \tau\in
S\left(  \pi^{\prime},\sigma^{\prime}\right)
\ \mid\ \operatorname{Epk} \tau =\Lambda\right\}
\right\vert K_{n+m,\Lambda}^{\mathcal{Z}}.
\end{align*}
Since the family $\left(  K_{n+m,\Lambda}^{\mathcal{Z}}\right)  _{\Lambda
\in\mathbf{L}_{n+m}}$ is $\mathbb{Q}$-linearly independent (by Proposition
\ref{prop.KnL.lindep}), this shows that%
\[
\left\vert \left\{  \tau\in S\left(  \pi,\sigma\right)  \ \mid
\ \operatorname{Epk}\tau=\Lambda\right\}  \right\vert =\left\vert \left\{
\tau\in S\left(  \pi^{\prime},\sigma^{\prime}\right)  \ \mid
\ \operatorname{Epk}\tau=\Lambda\right\}  \right\vert
\]
for each $\Lambda\in\mathbf{L}_{n+m}$.
In other words, the multiset $\left\{
\operatorname{Epk} \tau   \ \mid\ \tau\in S\left(  \pi
,\sigma\right)  \right\}  _{\operatorname*{multi}}$ equals the multiset
\newline
$\left\{  \operatorname{Epk} \tau   \ \mid\ \tau\in S\left(
\pi^{\prime},\sigma^{\prime}\right)  \right\}  _{\operatorname*{multi}}$
(because both of these multisets consist of elements of $\mathbf{L}_{n+m}$,
and the previous sentence shows that each of these elements appears with
equal multiplicities in them). This completes our proof of Theorem
\ref{thm.Epk.sh-co-a}.
\end{proof}

We end this section with a tangential remark for readers of
\cite{part1}:

\begin{remark}
\label{rmk.Epk.equivalents}Let us use the notations of \cite{part1}
(specifically, the concept of ``equivalent'' statistics defined
in \cite[Section 3.1]{part1}; and various specific statistics
defined in \cite[Section 2.2]{part1}).
The permutation statistics $\left(
\operatorname*{Lpk},\operatorname{val}\right)  $, $\left(
\operatorname*{Lpk},\operatorname*{udr}\right)  $ and $\left(
\operatorname*{Pk},\operatorname*{udr}\right)  $ are equivalent to
$\operatorname{Epk}$, and therefore are shuffle-compatible.
\end{remark}

\begin{proof}
[Proof of Remark \ref{rmk.Epk.equivalents} (sketched).]If
$\operatorname{st}_1$ and $\operatorname{st}_2$ are two permutation
statistics, then we shall write
$\operatorname{st}_1 \sim\operatorname{st}_2$ to mean \textquotedblleft%
$\operatorname{st}_1$ is equivalent to
$\operatorname{st}_2$\textquotedblright.

The permutation statistic $\operatorname{val}$ is equivalent to
$\operatorname*{epk}$, because of \cite[Lemma 2.1 \textbf{(e)}]{part1}. In
other words, $\operatorname{val}\sim\operatorname*{epk}$. Hence, $\left(
\operatorname*{Lpk},\operatorname{val}\right)  \sim\left(
\operatorname*{Lpk},\operatorname*{epk}\right)  $. But if $\pi$ is an
$n$-permutation, then $\operatorname{Epk}\pi$ can be computed from the
knowledge of $\operatorname*{Lpk}\pi$ and $\operatorname*{epk}\pi$ (indeed,
$\operatorname{Epk}\pi$ differs from $\operatorname*{Lpk}\pi$ only in the
possible element $n$, so that
\[
\operatorname{Epk}\pi=%
\begin{cases}
\operatorname*{Lpk}\pi, & \text{if }\operatorname*{epk}\pi=\left\vert
\operatorname*{Lpk}\pi\right\vert ;\\
\operatorname*{Lpk}\pi\cup\left\{  n\right\}  , & \text{if }%
\operatorname*{epk}\pi\neq\left\vert \operatorname*{Lpk}\pi\right\vert
\end{cases}
\]
) and vice versa (since $\operatorname*{Lpk}\pi=\left(  \operatorname{Epk}
\pi\right)  \setminus\left\{  n\right\}  $ and $\operatorname{epk}
\pi=\left\vert \operatorname{Epk}\pi\right\vert $). Thus, $\left(
\operatorname{Lpk},\operatorname{epk}\right)  \sim\operatorname{Epk}$.
Hence, altogether, we obtain $\left(  \operatorname{Lpk},\operatorname{val}
\right)  \sim\left(  \operatorname*{Lpk},\operatorname*{epk}\right)
\sim\operatorname{Epk}$.

Moreover, \cite[Lemma 2.2 \textbf{(a)}]{part1} shows that
for any permutation $\pi$, the
knowledge of $\operatorname*{Lpk}\pi$ allows us to compute
$\operatorname*{udr}\pi$ from $\operatorname{val}\pi$ and vice versa. Hence,
$\left(  \operatorname*{Lpk},\operatorname*{udr}\right)  \sim\left(
\operatorname*{Lpk},\operatorname{val}\right)  \sim\operatorname{Epk}$.

On the other hand, $\left(  \operatorname*{Pk},\operatorname*{lpk}\right)
\sim\operatorname*{Lpk}$. (This is proven similarly to our proof of $\left(
\operatorname*{Lpk},\operatorname*{epk}\right)  \sim\operatorname{Epk}$.)

Also, $\operatorname*{udr}\sim\left(  \operatorname*{lpk},\operatorname{val}
\right)  $ (indeed, \cite[Lemma 2.2 \textbf{(b)} and \textbf{(c)}]{part1} show
how the value $\left(  \operatorname*{lpk},\operatorname{val}\right)  \left(
\pi\right)  $ can be computed from $\operatorname*{udr}\pi$, whereas
\cite[Lemma 2.2 \textbf{(a)}]{part1} shows the opposite direction). Hence,
$\left(  \operatorname*{Pk},\operatorname*{udr}\right)  \sim\left(
\operatorname*{Pk},\operatorname*{lpk},\operatorname{val}\right)  \sim\left(
\operatorname*{Lpk},\operatorname{val}\right)  $ (since $\left(
\operatorname*{Pk},\operatorname*{lpk}\right)  \sim\operatorname*{Lpk}$).
Therefore, $\left(  \operatorname*{Pk},\operatorname*{udr}\right)  \sim\left(
\operatorname*{Lpk},\operatorname{val}\right)  \sim\operatorname{Epk}$.

We have now shown that the statistics $\left(  \operatorname{Lpk}%
,\operatorname{val}\right)  $, $\left(  \operatorname{Lpk}%
,\operatorname{udr}\right)  $ and $\left(  \operatorname{Pk}%
,\operatorname{udr}\right)  $ are equivalent to $\operatorname{Epk}$. Thus,
\cite[Theorem 3.2]{part1} shows that they are shuffle-compatible (since
$\operatorname{Epk}$ is shuffle-compatible). This proves Remark
\ref{rmk.Epk.equivalents}.
\end{proof}

\begin{question}
Our concept of a \textquotedblleft$\mathcal{Z}$-enriched $\left(
P,\gamma\right)  $-partition\textquotedblright\ generalizes the concept of an
\textquotedblleft enriched $\left(  P,\gamma\right)  $%
-partition\textquotedblright\ by restricting ourselves to a subset
$\mathcal{Z}$ of $\mathcal{N}\times\left\{  +,-\right\}  $. (This does not
sound like much of a generalization when stated like this, but as we have seen
the behavior of the power series $\Gamma_{\mathcal{Z}}\left(  P,\gamma\right)
$ depends strongly on what $\mathcal{Z}$ is, and is not all anticipated by the
$\mathcal{Z}=\mathcal{N}\times\left\{  +,-\right\}  $ case.) A different
generalization of enriched $\left(  P,\gamma\right)  $-partitions (introduced
by Hsiao and Petersen in \cite{HsiPet10}) are the \textit{colored }$\left(
P,\gamma\right)  $\textit{-partitions}, where the two-element set $\left\{
+,-\right\}  $ is replaced by the set $\left\{  1,\omega,\ldots,\omega
^{m-1}\right\}  $ of all $m$-th roots of unity (where $m$ is a chosen positive
integer, and $\omega$ is a fixed primitive $m$-th root of unity). We can play
various games with this concept. The most natural thing to do seems to be to
consider $m$ arbitrary total orders $<_{0},<_{1},\ldots,<_{m-1}$ on the
codomain $A$ of the labeling $\gamma$ (perhaps with some nice properties such
as all intervals being finite) and an arbitrary subset $\mathcal{Z}$ of
$\mathcal{N}\times\left\{  1,\omega,\ldots,\omega^{m-1}\right\}  $, and define
a $\mathcal{Z}$-enriched colored $\left(  P,\gamma\right)  $-partition to be a
map $f:P\rightarrow\mathcal{Z}$ such that every $x<y$ in $P$ satisfy the
following conditions:

\begin{enumerate}
\item[\textbf{(i)}] We have $f\left(  x\right)  \preccurlyeq f\left(
y\right)  $. (Here, the total order on $\mathcal{N}\times\left\{
1,\omega,\ldots,\omega^{m-1}\right\}  $ is defined by%
\[
\left(  n,\omega^{i}\right)  \prec\left(  n^{\prime},\omega^{i^{\prime}%
}\right)  \text{ if and only if either }n\prec n^{\prime}\text{ or }\left(
n=n^{\prime}\text{ and }i<i^{\prime}\right)
\]
(for $i,i^{\prime}\in\left\{  0,1,\ldots,m-1\right\}  $).)

\item[\textbf{(ii)}] If $f\left(  x\right)  =f\left(  y\right)  =\left(
n,\omega^{i}\right)  $ for some $n\in\mathcal{N}$ and $i\in\left\{
0,1,\ldots,m-1\right\}  $, then $\gamma\left(  x\right)  <_{i}\gamma\left(
y\right)  $.
\end{enumerate}

Is this a useful concept, and can it be used to study permutation statistics?
\end{question}



\begin{question}
Corollary \ref{cor.KnEpk-prodrule} provides a formula for rewriting a product
of the form $K_{n,\Lambda}^{\mathcal{Z}}\cdot K_{m,\Omega}^{\mathcal{Z}}$ as a
$\mathbb{Q}$-linear combination of $K_{n+m,\Xi}^{\mathcal{Z}}$'s when
$\Lambda\in\mathbf{L}_{n}$ and $\Omega\in\mathbf{L}_{m}$ (because any such
$\Lambda$ and $\Omega$ can be written as $\Lambda=\operatorname{Epk}\pi$ and
$\Omega=\operatorname{Epk}\sigma$ for appropriate permutations $\pi$ and
$\sigma$). Thus, in particular, any such product belongs to the $\mathbb{Q}%
$-linear span of the $K_{n+m,\Xi}^{\mathcal{Z}}$'s. Is this still true if
$\Lambda$ and $\Omega$ are arbitrary subsets of $\left[  n\right]  $ and
$\left[  m\right]  $ rather than having to belong to $\mathbf{L}_{n}$ and to
$\mathbf{L}_{m}$ ? Computations with SageMath suggest that the answer is
\textquotedblleft yes\textquotedblright. For example,%
\begin{align*}
K_{2,\left\{  1,2\right\}  }^{\mathcal{Z}}\cdot K_{1,\left\{  1\right\}
}^{\mathcal{Z}}  & =K_{3,\left\{  2\right\}  }^{\mathcal{Z}}+2\cdot
K_{3,\left\{  1,3\right\}  }^{\mathcal{Z}}\ \ \ \ \ \ \ \ \ \ \text{and}\\
K_{2,\varnothing}^{\mathcal{Z}}\cdot K_{1,\left\{  1\right\}  }^{\mathcal{Z}}
& =K_{3,\varnothing}^{\mathcal{Z}}+K_{3,\left\{  2\right\}  }^{\mathcal{Z}%
}+K_{3,\left\{  1,3\right\}  }^{\mathcal{Z}}=K_{3,\left\{  1\right\}
}^{\mathcal{Z}}+K_{3,\left\{  2\right\}  }^{\mathcal{Z}}+K_{3,\left\{
3\right\}  }^{\mathcal{Z}}.
\end{align*}
Note that the $\mathbb{Q}$-linear span of the $K_{n+m,\Xi}^{\mathcal{Z}}$'s
for all $\Xi\subseteq\left[  n+m\right]  $ is (generally) larger than that of
the $K_{n+m,\Xi}^{\mathcal{Z}}$'s with $\Xi\in\mathbf{L}_{n+m}$.
\end{question}

\section{\label{sect.LR}LR-shuffle-compatibility}

In this section, we shall introduce the concept of \textquotedblleft
LR-shuffle-compatibility\textquotedblright\ (short for \textquotedblleft
left-and-right-shuffle-compatibility\textquotedblright), which is stronger
than usual shuffle-compatibility. We shall prove that $\operatorname{Epk}$
still is LR-shuffle-compatible, and study some other statistics that are and
some that are not.

\subsection{Left and right shuffles}

We begin by introducing \textquotedblleft left shuffles\textquotedblright\ and
\textquotedblleft right shuffles\textquotedblright. There is a well-known
notion of left and right shuffles of words (see, e.g., the operations $\prec$
and $\succ$ in \cite[Example 1]{EbMaPa07}). Specialized to permutations, it
can be defined in the following simple way:

\begin{definition}
\label{def.LRshuf}Let $\pi$ and $\sigma$ be two disjoint permutations. Then:

\begin{itemize}
\item A \textit{left shuffle} of $\pi$ and $\sigma$ means a shuffle $\tau$ of
$\pi$ and $\sigma$ such that the first letter of $\tau$ is the first letter of
$\pi$. (This makes sense only when $\pi$ is nonempty. Otherwise, there are no
left shuffles of $\pi$ and $\sigma$.)

\item A \textit{right shuffle} of $\pi$ and $\sigma$ means a shuffle $\tau$ of
$\pi$ and $\sigma$ such that the first letter of $\tau$ is the first letter of
$\sigma$. (This makes sense only when $\sigma$ is nonempty. Otherwise, there
are no right shuffles of $\pi$ and $\sigma$.)

\item We let $S_{\prec}\left(  \pi,\sigma\right)  $ denote the set of all left
shuffles of $\pi$ and $\sigma$.

\item We let $S_{\succ}\left(  \pi,\sigma\right)  $ denote the set of all
right shuffles of $\pi$ and $\sigma$.
\end{itemize}
\end{definition}

For example, the left shuffles of the two disjoint permutations $\left(
3,1\right)  $ and $\left(  2,6\right)  $ are%
\[
\left(  3,1,2,6\right)  ,\ \ \ \ \ \ \ \ \ \ \left(  3,2,1,6\right)
,\ \ \ \ \ \ \ \ \ \ \left(  3,2,6,1\right)  ,
\]
whereas their right shuffles are%
\[
\left(  2,3,1,6\right)  ,\ \ \ \ \ \ \ \ \ \ \left(  2,3,6,1\right)
,\ \ \ \ \ \ \ \ \ \ \left(  2,6,3,1\right)  .
\]
The permutations $\left(  {}\right)  $ and $\left(  1,3\right)  $ have only
one right shuffle, which is $\left(  1,3\right)  $, and they have no left shuffles.

Clearly, if $\pi$ and $\sigma$ are two disjoint permutations such that at
least one of $\pi$ and $\sigma$ is nonempty, then the two sets $S_{\prec
}\left(  \pi,\sigma\right)  $ and $S_{\succ}\left(  \pi,\sigma\right)  $ are
disjoint and their union is $S\left(  \pi,\sigma\right)  $ (because every
shuffle of $\pi$ and $\sigma$ is either a left shuffle or a right shuffle, but
not both).

Left and right shuffles have a recursive structure that makes them amenable to
inductive arguments. To state it, we need one more definition:

\begin{definition}
\label{def.LR.pi1}Let $n\in\mathbb{N}$. Let $\pi$ be an $n$-permutation.

\textbf{(a)} For each $i\in\left\{  1,2,\ldots,n\right\}  $, we let $\pi_{i}$
denote the $i$-th entry of $\pi$. Thus, $\pi=\left(  \pi_{1},\pi_{2}%
,\ldots,\pi_{n}\right)  $.

\textbf{(b)} If $a$ is a positive integer that does not appear in $\pi$, then
$a:\pi$ denotes the $\left(  n+1\right)  $-permutation $\left(  a,\pi_{1}%
,\pi_{2},\ldots,\pi_{n}\right)  $.

\textbf{(c)} If $n>0$, then $\pi_{\sim1}$ denotes the $\left(  n-1\right)
$-permutation $\left(  \pi_{2},\pi_{3},\ldots,\pi_{n}\right)  $.
\end{definition}

\begin{proposition}
\label{prop.LR.rec}Let $\pi$ and $\sigma$ be two disjoint permutations.

\textbf{(a)} We have $S_{\prec}\left(  \pi,\sigma\right)  =S_{\succ}\left(
\sigma,\pi\right)  $.

\textbf{(b)} If $\pi$ is nonempty, then the permutations $\pi_{\sim1}$ and
$\pi_{1}:\sigma$ are well-defined and disjoint, and satisfy $S_{\prec}\left(
\pi,\sigma\right)  =S_{\succ}\left(  \pi_{\sim1},\pi_{1}:\sigma\right)  $.

\textbf{(c)} If $\sigma$ is nonempty, then the permutations $\sigma_{\sim1}$
and $\sigma_{1}:\pi$ are well-defined and disjoint, and satisfy $S_{\succ
}\left(  \pi,\sigma\right)  =S_{\prec}\left(  \sigma_{1}:\pi,\sigma_{\sim
1}\right)  $.
\end{proposition}

\begin{proof}
[Proof of Proposition \ref{prop.LR.rec}.]The fairly simple proof is left to
the reader, who can also find it in \cite{verlong}.
\end{proof}

\subsection{LR-shuffle-compatibility}

We shall use the so-called \textit{Iverson bracket notation} for truth values:

\begin{definition}
\label{def.iverson}If $\mathcal{A}$ is any logical statement, then we define
an integer $\left[  \mathcal{A}\right]  \in\left\{  0,1\right\}  $ by%
\[
\left[  \mathcal{A}\right]  =%
\begin{cases}
1, & \text{if }\mathcal{A}\text{ is true};\\
0, & \text{if }\mathcal{A}\text{ is false}%
\end{cases}
.
\]
This integer $\left[  \mathcal{A}\right]  $ is known as the \textit{truth
value} of $\mathcal{A}$.
\end{definition}

Thus, for example, $\left[  4>2\right]  =1$ whereas $\left[  2>4\right]  =0$.

We can now define a notion similar to shuffle-compatibility:

\begin{definition}
\label{def.LRcomp}Let $\operatorname{st}$ be a permutation statistic. We say
that $\operatorname{st}$ is \textit{LR-shuffle-compatible} if and only if it
has the following property: For any two disjoint nonempty permutations $\pi$
and $\sigma$, the multisets%
\[
\left\{  \operatorname{st} \tau   \ \mid\ \tau\in S_{\prec
}\left(  \pi,\sigma\right)  \right\}  _{\operatorname*{multi}}
\ \ \ \ \ \ \ \ \ \ \text{and}\ \ \ \ \ \ \ \ \ \ \left\{  \operatorname{st}
\tau \ \mid\ \tau\in S_{\succ}\left(  \pi,\sigma\right)
\right\}  _{\operatorname*{multi}}
\]
depend only on $\operatorname{st} \pi   $, $\operatorname{st} \sigma$,
$\left\vert \pi\right\vert $, $\left\vert
\sigma\right\vert $ and $\left[  \pi_{1}>\sigma_{1}\right]  $.
\end{definition}

In other words, a permutation statistic $\operatorname{st}$ is
LR-shuffle-compatible if and only if every two disjoint nonempty permutations
$\pi$ and $\sigma$ and every two disjoint nonempty permutations $\pi^{\prime}$
and $\sigma^{\prime}$ satisfying%
\begin{align*}
\operatorname{st} \pi    &  =\operatorname{st}\left(
\pi^{\prime}\right)  ,\ \ \ \ \ \ \ \ \ \ \operatorname{st} \sigma
=\operatorname{st}\left(  \sigma^{\prime}\right)  ,\\
\left\vert \pi\right\vert  &  =\left\vert \pi^{\prime}\right\vert
,\ \ \ \ \ \ \ \ \ \ \left\vert \sigma\right\vert =\left\vert \sigma^{\prime
}\right\vert \ \ \ \ \ \ \ \ \ \ \text{and}\ \ \ \ \ \ \ \ \ \ \left[  \pi
_{1}>\sigma_{1}\right]  =\left[  \pi_{1}^{\prime}>\sigma_{1}^{\prime}\right]
\end{align*}
satisfy
\begin{align*}
\left\{  \operatorname{st} \tau   \ \mid\ \tau\in S_{\prec
}\left(  \pi,\sigma\right)  \right\}  _{\operatorname*{multi}}  &  =\left\{
\operatorname{st} \tau   \ \mid\ \tau\in S_{\prec}\left(
\pi^{\prime},\sigma^{\prime}\right)  \right\}  _{\operatorname*{multi}%
}\ \ \ \ \ \ \ \ \ \ \text{and}\\
\left\{  \operatorname{st} \tau   \ \mid\ \tau\in S_{\succ
}\left(  \pi,\sigma\right)  \right\}  _{\operatorname*{multi}}  &  =\left\{
\operatorname{st} \tau   \ \mid\ \tau\in S_{\succ}\left(
\pi^{\prime},\sigma^{\prime}\right)  \right\}  _{\operatorname*{multi}}.
\end{align*}


For example, the permutation statistic $\operatorname*{Pk}$ is not
LR-shuffle-compatible. Indeed, if we take $\pi=\left(  4,2,3\right)  $,
$\sigma=\left(  1\right)  $, $\pi^{\prime}=\left(  2,3,4\right)  $ and
$\sigma^{\prime}=\left(  1\right)  $, then the equalities%
\begin{align*}
\operatorname*{Pk} \pi &  =\operatorname*{Pk}\left(
\pi^{\prime}\right)  ,\ \ \ \ \ \ \ \ \ \ \operatorname*{Pk} \sigma
=\operatorname*{Pk}\left(  \sigma^{\prime}\right)  ,\\
\left\vert \pi\right\vert  &  =\left\vert \pi^{\prime}\right\vert
,\ \ \ \ \ \ \ \ \ \ \left\vert \sigma\right\vert =\left\vert \sigma^{\prime
}\right\vert \ \ \ \ \ \ \ \ \ \ \text{and}\ \ \ \ \ \ \ \ \ \ \left[  \pi
_{1}>\sigma_{1}\right]  =\left[  \pi_{1}^{\prime}>\sigma_{1}^{\prime}\right]
\end{align*}
are all satisfied, but%
\[
\left\{  \operatorname*{Pk} \tau \ \mid\ \tau\in S_{\succ
}\left(  \pi,\sigma\right)  \right\}  _{\operatorname*{multi}}=\left\{
\underbrace{\operatorname*{Pk}\left(  1,4,2,3\right)  }_{=\left\{  2\right\}
}\right\}  _{\operatorname*{multi}}=\left\{  \left\{  2\right\}  \right\}
_{\operatorname*{multi}}%
\]
is not the same as%
\[
\left\{  \operatorname*{Pk} \tau \ \mid\ \tau\in S_{\succ
}\left(  \pi^{\prime},\sigma^{\prime}\right)  \right\}
_{\operatorname*{multi}}=\left\{  \underbrace{\operatorname*{Pk}\left(
1,2,3,4\right)  }_{=\varnothing}\right\}  _{\operatorname*{multi}}=\left\{
\varnothing\right\}  _{\operatorname*{multi}}.
\]
Similarly, the permutation statistic $\operatorname*{Rpk}$ is not
LR-shuffle-compatible. As we will see in Theorem \ref{thm.LRcomp.Pks} further
below, the three statistics $\operatorname{Des}$, $\operatorname*{Lpk}$ and
$\operatorname{Epk}$ are LR-shuffle-compatible.

\subsection{Head-graft-compatibility}

We shall now define another compatibility concept for a permutation statistic,
which will later prove a useful stepping stone for checking the
LR-shuffle-compatibility of this statistic.

\begin{definition}
\label{def.head-comp}Let $\operatorname{st}$ be a permutation statistic. We
say that $\operatorname{st}$ is \textit{head-graft-compatible} if and only if
it has the following property: For any nonempty permutation $\pi$ and any
letter $a$ that does not appear in $\pi$, the element $\operatorname{st}
\left(  a:\pi\right)  $ depends only on $\operatorname{st} \pi 
$, $\left\vert \pi\right\vert $ and $\left[  a>\pi_{1}\right]  $.
\end{definition}

In other words, a permutation statistic $\operatorname{st}$ is
head-graft-compatible if and only if every nonempty permutation $\pi$, every
letter $a$ that does not appear in $\pi$, every nonempty permutation
$\pi^{\prime}$ and every letter $a^{\prime}$ that does not appear in
$\pi^{\prime}$ satisfying%
\[
\operatorname{st} \pi   =\operatorname{st}\left(  \pi^{\prime
}\right)  ,\ \ \ \ \ \ \ \ \ \ \left\vert \pi\right\vert =\left\vert
\pi^{\prime}\right\vert \ \ \ \ \ \ \ \ \ \ \text{and}%
\ \ \ \ \ \ \ \ \ \ \left[  a>\pi_{1}\right]  =\left[  a^{\prime}>\pi
_{1}^{\prime}\right]
\]
satisfy $\operatorname{st}\left(  a:\pi\right)  =\operatorname{st}\left(
a^{\prime}:\pi^{\prime}\right)  $.

For example, the permutation statistic $\operatorname*{Pk}$ is not
head-graft-compatible, because if we take $\pi=\left(  3,1\right)  $, $a=2$,
$\pi^{\prime}=\left(  3,4\right)  $ and $a^{\prime}=2$, then we do have%
\[
\operatorname*{Pk} \pi =\operatorname*{Pk}\left(  \pi^{\prime
}\right)  ,\ \ \ \ \ \ \ \ \ \ \left\vert \pi\right\vert =\left\vert
\pi^{\prime}\right\vert \ \ \ \ \ \ \ \ \ \ \text{and}%
\ \ \ \ \ \ \ \ \ \ \left[  a>\pi_{1}\right]  =\left[  a^{\prime}>\pi
_{1}^{\prime}\right]
\]
but we don't have $\operatorname*{Pk}\left(  a:\pi\right)  =\operatorname{Pk}%
\left(  a^{\prime}:\pi^{\prime}\right)  $ (in fact, $\operatorname*{Pk}\left(
a:\pi\right)  =\operatorname*{Pk}\left(  2,3,1\right)  =\left\{  2\right\}  $
whereas $\operatorname*{Pk}\left(  a^{\prime}:\pi^{\prime}\right)
=\operatorname*{Pk}\left(  2,3,4\right)  =\varnothing$). Similarly, it can be
shown that $\operatorname*{Rpk}$ is not head-graft-compatible. As we will see
below (in Proposition \ref{prop.head-comp.Pks}), the permutation statistics
$\operatorname{Des}$, $\operatorname*{Lpk}$ and $\operatorname{Epk}$ are
head-graft-compatible; we will analyze a few other statistics in Subsection
\ref{subsect.LR.others}.

\begin{remark}
Let $\operatorname{st}$ be a head-graft-compatible permutation statistic.
Then, it is easy to see that%
\[
\operatorname{st}\left(  3,1,2\right)  =\operatorname{st}\left(
2,1,3\right)  \ \ \ \ \ \ \ \ \ \ \text{and}%
\ \ \ \ \ \ \ \ \ \ \operatorname{st}\left(  2,3,1\right)
=\operatorname{st}\left(  1,3,2\right)  .
\]
Moreover, these are the only restrictions that head-graft-compatibility places
on the values of $\operatorname{st}$ at $3$-permutations.
The restrictions placed on the values of $\operatorname{st}$ at
permutations of length $n > 3$ are more complicated, and depend
on its values on shorter permutations.
\end{remark}

It is usually easy to check if a given permutation statistic is
head-graft-compatible. For example:

\begin{proposition}
\label{prop.head-comp.Pks}\textbf{(a)} The permutation statistic
$\operatorname{Des}$ is head-graft-compatible.

\textbf{(b)} The permutation statistic $\operatorname*{Lpk}$ is head-graft-compatible.

\textbf{(c)} The permutation statistic $\operatorname{Epk}$ is head-graft-compatible.
\end{proposition}

\begin{proof}
[Proof of Proposition \ref{prop.head-comp.Pks}.]In this proof, we shall use
the following notation: If $S$ is a set of integers, and $p$ is an integer,
then $S+p$ shall denote the set $\left\{  s+p\ \mid\ s\in S\right\}  $.

\textbf{(a)} Let $\pi$ be a nonempty permutation. Let $a$ be a letter that
does not appear in $\pi$. We shall express
the element $\operatorname{Des}\left(  a:\pi\right)  $ in terms of
$\operatorname{Des}\pi$, $\left|\pi\right|$
and $\left[  a>\pi_{1}\right]  $.

Let $n=\left\vert \pi\right\vert $. Thus, $\pi=\left(  \pi_{1},\pi
_{2},\ldots,\pi_{n}\right)  $. Therefore, $a:\pi=\left(  a,\pi_{1},\pi_{2}%
,\ldots,\pi_{n}\right)  $. Hence, the descents of $a:\pi$ are obtained as follows:

\begin{itemize}
\item The number $1$ is a descent of $a:\pi$ if and only if $a>\pi_{1}$.

\item Adding $1$ to each descent of $\pi$ yields a descent of $a:\pi$. (That
is, if $i$ is a descent of $\pi$, then $i+1$ is a descent of $a:\pi$.)
\end{itemize}

These are all the descents of $a:\pi$. Thus,%
\begin{equation}
\operatorname{Des}\left(  a:\pi\right)  =\left\{  1\ \mid\ a>\pi_{1}\right\}
\cup\left(  \operatorname{Des}\pi+1\right)  .
\label{pf.prop.head-comp.Pks.a.1}%
\end{equation}
(The strange notation \textquotedblleft$\left\{  1\ \mid\ a>\pi_{1}\right\}
$\textquotedblright\ means exactly what it says: It is the set of all numbers
$1$ satisfying $a>\pi_{1}$. In other words, it is $\left\{  1\right\}  $ if
$a>\pi_{1}$, and $\varnothing$ otherwise.)

The equality (\ref{pf.prop.head-comp.Pks.a.1}) shows that $\operatorname{Des}
\left(  a:\pi\right)  $ depends only on $\operatorname{Des}\pi$, $\left\vert
\pi\right\vert $ and $\left[  a>\pi_{1}\right]  $ (indeed, the truth value
$\left[  a>\pi_{1}\right]  $ determines whether $a>\pi_{1}$ is true). In other
words, $\operatorname{Des}$ is head-graft-compatible (by the definition of
\textquotedblleft head-graft-compatible\textquotedblright). This proves
Proposition \ref{prop.head-comp.Pks} \textbf{(a)}.

\textbf{(b)} Let $\pi$ be a nonempty permutation. Let $a$ be a letter that
does not appear in $\pi$. We shall express
the element $\operatorname*{Lpk}\left(  a:\pi\right)  $ in terms of
$\operatorname*{Lpk}\pi$, $\left|\pi\right|$
and $\left[  a>\pi_{1}\right]  $.

Notice first that $a\neq\pi_{1}$ (since $a$ does not appear in $\pi$). Thus,
$a<\pi_{1}$ is true if and only if $a>\pi_{1}$ is false.

Let $n=\left\vert \pi\right\vert $. Therefore, $\pi=\left(  \pi_{1},\pi
_{2},\ldots,\pi_{n}\right)  $. Thus, $a:\pi=\left(  a,\pi_{1},\pi_{2}%
,\ldots,\pi_{n}\right)  $. Hence, the left peaks of $a:\pi$ are obtained as follows:

\begin{itemize}
\item The number $1$ is a left peak of $a:\pi$ if and only if $a>\pi_{1}$.

\item Adding $1$ to each left peak $i$ of $\pi$ yields a left peak $i+1$ of
$a:\pi$, except for the case when $i=1$ (in which case $i+1=2$ is a left peak
of $a:\pi$ only if $a<\pi_{1}$).
\end{itemize}

These are all the left peaks of $a:\pi$. Thus,%
\begin{equation}
\operatorname*{Lpk}\left(  a:\pi\right)  =\left\{  1\ \mid\ a>\pi_{1}\right\}
\cup%
\begin{cases}
\operatorname*{Lpk}\pi+1, & \text{if }a<\pi_{1};\\
\left(  \operatorname*{Lpk}\pi+1\right)  \setminus\left\{  2\right\}  , &
\text{if not }a<\pi_{1}%
\end{cases}
. \label{pf.prop.head-comp.Pks.b.1}%
\end{equation}


This equality shows that $\operatorname*{Lpk}\left(  a:\pi\right)  $ depends
only on $\operatorname*{Lpk}\pi$, $\left\vert \pi\right\vert $ and $\left[
a>\pi_{1}\right]  $ (indeed, the truth value $\left[  a>\pi_{1}\right]  $
determines whether $a>\pi_{1}$ is true and also determines whether $a<\pi_{1}$
is true\footnote{Indeed, $a<\pi_{1}$ is true if and only if $a>\pi_{1}$ is
false.}). In other words, $\operatorname*{Lpk}$ is head-graft-compatible (by
the definition of \textquotedblleft head-graft-compatible\textquotedblright).
This proves Proposition \ref{prop.head-comp.Pks} \textbf{(b)}.

\textbf{(c)} To obtain a proof of Proposition \ref{prop.head-comp.Pks}
\textbf{(c)}, it suffices to take our above proof of Proposition
\ref{prop.head-comp.Pks} \textbf{(b)} and replace every appearance of
\textquotedblleft left peak\textquotedblright\ and \textquotedblleft%
$\operatorname*{Lpk}$\textquotedblright\ by \textquotedblleft exterior
peak\textquotedblright\ and
\textquotedblleft$\operatorname{Epk}$\textquotedblright.
\end{proof}

\subsection{Proving LR-shuffle-compatibility}

Let us now state a sufficient criterion for the LR-shuffle-compatibility of a statistic:

\begin{theorem}
\label{thm.head-comp.LRcomp}Let $\operatorname{st}$ be a permutation
statistic that is both shuffle-compatible and head-graft-compatible. Then,
$\operatorname{st}$ is LR-shuffle-compatible.
\end{theorem}

Before we prove this theorem, let us introduce some terminology and state an
almost-trivial fact:

\begin{definition}
\textbf{(a)} If $A$ is a finite multiset, and if $g$ is any object, then
$\left\vert A\right\vert _{g}$ means the multiplicity of $g$ in $A$.

\textbf{(b)} If $A$ and $B$ are two finite multisets, then we say that
$B\subseteq A$ if and only if each object $g$ satisfies $\left\vert
B\right\vert _{g}\leq\left\vert A\right\vert _{g}$.

\textbf{(c)} If $A$ and $B$ are two finite multisets satisfying $B\subseteq
A$, then $A-B$ shall denote the \textquotedblleft multiset
difference\textquotedblright\ of $A$ and $B$; this is the finite
multiset $C$ such
that each object $g$ satisfies $\left\vert C\right\vert _{g}=\left\vert
A\right\vert _{g}-\left\vert B\right\vert _{g}$.
\end{definition}

For example, $\left\{  2,3,3\right\}  _{\operatorname*{multi}}\subseteq
\left\{  1,2,2,3,3\right\}  _{\operatorname*{multi}}$ and $\left\{
1,2,2,3,3\right\}  _{\operatorname*{multi}}-\left\{  2,3,3\right\}
_{\operatorname*{multi}}=\left\{  1,2\right\}  _{\operatorname*{multi}}$.

\begin{lemma}
\label{lem.LR.difference}Let $\pi$ and $\sigma$ be two disjoint permutations
such that at least one of $\pi$ and $\sigma$ is nonempty. Let
$\operatorname{st}$ be any permutation statistic. Then:

\textbf{(a)} We have%
\begin{align*}
&  \left\{  \operatorname{st} \tau   \ \mid\ \tau\in S_{\prec
}\left(  \pi,\sigma\right)  \right\}  _{\operatorname*{multi}}\\
&  =\left\{  \operatorname{st} \tau   \ \mid\ \tau\in S\left(
\pi,\sigma\right)  \right\}  _{\operatorname*{multi}}-\left\{
\operatorname{st} \tau   \ \mid\ \tau\in S_{\succ}\left(
\pi,\sigma\right)  \right\}  _{\operatorname*{multi}}.
\end{align*}


\textbf{(b)} We have%
\begin{align*}
&  \left\{  \operatorname{st} \tau   \ \mid\ \tau\in S_{\succ
}\left(  \pi,\sigma\right)  \right\}  _{\operatorname*{multi}}\\
&  =\left\{  \operatorname{st} \tau   \ \mid\ \tau\in S\left(
\pi,\sigma\right)  \right\}  _{\operatorname*{multi}}-\left\{
\operatorname{st} \tau   \ \mid\ \tau\in S_{\prec}\left(
\pi,\sigma\right)  \right\}  _{\operatorname*{multi}}.
\end{align*}

\end{lemma}

\begin{proof}
[Proof of Lemma \ref{lem.LR.difference}.]Recall that the two sets $S_{\prec
}\left(  \pi,\sigma\right)  $ and $S_{\succ}\left(  \pi,\sigma\right)  $ are
disjoint and their union is $S\left(  \pi,\sigma\right)  $. Thus, $S_{\succ
}\left(  \pi,\sigma\right)  \subseteq S\left(  \pi,\sigma\right)  $ and
$S_{\prec}\left(  \pi,\sigma\right)  =S\left(  \pi,\sigma\right)  \setminus
S_{\succ}\left(  \pi,\sigma\right)  $. Hence,%
\begin{align*}
&  \left\{  \operatorname{st} \tau   \ \mid\ \tau\in S_{\prec
}\left(  \pi,\sigma\right)  \right\}  _{\operatorname*{multi}}\\
&  =\left\{  \operatorname{st} \tau   \ \mid\ \tau\in S\left(
\pi,\sigma\right)  \right\}  _{\operatorname*{multi}}-\left\{
\operatorname{st} \tau   \ \mid\ \tau\in S_{\succ}\left(
\pi,\sigma\right)  \right\}  _{\operatorname*{multi}}.
\end{align*}
This proves Lemma \ref{lem.LR.difference} \textbf{(a)}. The proof of Lemma
\ref{lem.LR.difference} \textbf{(b)} is analogous.
\end{proof}

\begin{proof}
[Proof of Theorem \ref{thm.head-comp.LRcomp}.]We shall first show the following:

\begin{statement}
\textit{Claim 1:} Let $\pi$, $\pi^{\prime}$ and $\sigma$ be three nonempty
permutations. Assume that $\pi$ and $\sigma$ are disjoint. Assume that
$\pi^{\prime}$ and $\sigma$ are disjoint. Assume furthermore that%
\[
\operatorname{st} \pi   =\operatorname{st}\left(  \pi^{\prime
}\right)  ,\ \ \ \ \ \ \ \ \ \ \left\vert \pi\right\vert =\left\vert
\pi^{\prime}\right\vert \ \ \ \ \ \ \ \ \ \ \text{and}%
\ \ \ \ \ \ \ \ \ \ \left[  \pi_{1}>\sigma_{1}\right]  =\left[  \pi
_{1}^{\prime}>\sigma_{1}\right]  .
\]
Then,
\begin{align}
&  \left\{  \operatorname{st} \tau   \ \mid\ \tau\in S_{\prec
}\left(  \pi,\sigma\right)  \right\}  _{\operatorname*{multi}}\nonumber\\
&  =\left\{  \operatorname{st} \tau   \ \mid\ \tau\in S_{\prec
}\left(  \pi^{\prime},\sigma\right)  \right\}  _{\operatorname*{multi}}
\label{pf.thm.head-comp.LRcomp.c1.eq3}%
\end{align}
and%
\begin{align}
&  \left\{  \operatorname{st} \tau   \ \mid\ \tau\in S_{\succ
}\left(  \pi,\sigma\right)  \right\}  _{\operatorname*{multi}}\nonumber\\
&  =\left\{  \operatorname{st} \tau   \ \mid\ \tau\in S_{\succ
}\left(  \pi^{\prime},\sigma\right)  \right\}  _{\operatorname*{multi}}.
\label{pf.thm.head-comp.LRcomp.c1.eq4}%
\end{align}

\end{statement}

[\textit{Proof of Claim 1:} We shall prove Claim 1 by induction on $\left\vert
\sigma\right\vert $:

\textit{Induction base:} The case $\left\vert \sigma\right\vert =0$ cannot
happen (because we have assumed $\sigma$ to be nonempty). Thus, Claim 1 is true in
the case $\left\vert \sigma\right\vert =0$. This completes the induction base.

\textit{Induction step:} Let $N$ be a positive integer. Assume (as the
induction hypothesis) that Claim 1 holds when $\left\vert \sigma\right\vert
=N-1$. We must now prove that Claim 1 holds when $\left\vert \sigma\right\vert
=N$.

Indeed, let $\pi$, $\pi^{\prime}$ and $\sigma$ be as in Claim 1, and assume
that $\left\vert \sigma\right\vert =N$. We must prove
(\ref{pf.thm.head-comp.LRcomp.c1.eq3}) and
(\ref{pf.thm.head-comp.LRcomp.c1.eq4}).

Proposition \ref{prop.LR.rec} \textbf{(c)} yields that the permutations
$\sigma_{\sim1}$ and $\sigma_{1}:\pi$ are well-defined and disjoint, and
satisfy%
\begin{equation}
S_{\succ}\left(  \pi,\sigma\right)  =S_{\prec}\left(  \sigma_{1}:\pi
,\sigma_{\sim1}\right)  . \label{pf.thm.head-comp.LRcomp.c1.pf.1}%
\end{equation}
Furthermore, $\left\vert \sigma_{\sim1}\right\vert =\left\vert \sigma
\right\vert -1=N-1$ (since $\left\vert \sigma\right\vert =N$).

Proposition \ref{prop.LR.rec} \textbf{(c)} (applied to $\pi^{\prime}$ instead
of $\pi$) yields that the permutations $\sigma_{\sim1}$ and $\sigma_{1}%
:\pi^{\prime}$ are well-defined and disjoint, and satisfy%
\begin{equation}
S_{\succ}\left(  \pi^{\prime},\sigma\right)  =S_{\prec}\left(  \sigma_{1}%
:\pi^{\prime},\sigma_{\sim1}\right)  . \label{pf.thm.head-comp.LRcomp.c1.pf.2}%
\end{equation}


The letter $\sigma_{1}$ does not appear in the permutation $\pi$ (since $\pi$
and $\sigma$ are disjoint). Similarly, the letter $\sigma_{1}$ does not appear
in the permutation $\pi^{\prime}$. Also, $\left\vert \sigma_{1}:\pi\right\vert
=\underbrace{\left\vert \pi\right\vert }_{=\left\vert \pi^{\prime}\right\vert
}+1=\left\vert \pi^{\prime}\right\vert +1=\left\vert \sigma_{1}:\pi^{\prime
}\right\vert $.

We have $\sigma_{1}\neq\pi_{1}$ (since $\pi$ and $\sigma$ are disjoint). Thus,
the statement $\left(  \sigma_{1}>\pi_{1}\right)  $ is equivalent to $\left(
\text{not }\pi_{1}>\sigma_{1}\right)  $. Hence, $\left[  \sigma_{1}>\pi
_{1}\right]  =\left[  \text{not }\pi_{1}>\sigma_{1}\right]  =1-\left[  \pi
_{1}>\sigma_{1}\right]  $. Similarly, $\left[  \sigma_{1}>\pi_{1}^{\prime
}\right]  =1-\left[  \pi_{1}^{\prime}>\sigma_{1}\right]  $. Hence,%
\[
\left[  \sigma_{1}>\pi_{1}\right]  =1-\underbrace{\left[  \pi_{1}>\sigma
_{1}\right]  }_{=\left[  \pi_{1}^{\prime}>\sigma_{1}\right]  }=1-\left[
\pi_{1}^{\prime}>\sigma_{1}\right]  =\left[  \sigma_{1}>\pi_{1}^{\prime
}\right]  .
\]


Both permutations $\sigma_{1}:\pi$ and $\sigma_{1}:\pi^{\prime}$ begin with
the letter $\sigma_{1}$. Thus, both $\left(  \sigma_{1}:\pi\right)  _{1}$ and
$\left(  \sigma_{1}:\pi^{\prime}\right)  _{1}$ equal $\sigma_{1}$. Hence,
$\left(  \sigma_{1}:\pi\right)  _{1}=\left(  \sigma_{1}:\pi^{\prime}\right)
_{1}$.

The statistic $\operatorname{st}$ is head-graft-compatible. In other words,
for any nonempty permutation $\varphi$ and any letter $a$ that does not appear
in $\varphi$, the element $\operatorname{st}\left(  a:\varphi\right)  $
depends only on $\operatorname{st}\left(  \varphi\right)  $, $\left\vert
\varphi\right\vert $ and $\left[  a>\varphi_{1}\right]  $ (by the definition
of \textquotedblleft head-graft-compatible\textquotedblright). Hence,
if $\varphi$ and $\varphi^{\prime}$ are two nonempty permutations, and if $a$
is any letter that does not appear in $\varphi$ and does not appear in
$\varphi^{\prime}$, and if we have $\operatorname{st} \varphi
=\operatorname{st}\left(  \varphi^{\prime}\right)  $ and $\left\vert
\varphi\right\vert =\left\vert \varphi^{\prime}\right\vert $ and $\left[
a>\varphi_{1}\right]  =\left[  a>\varphi_{1}^{\prime}\right]  $, then
$\operatorname{st}\left(  a:\varphi\right)  =\operatorname{st}\left(
a:\varphi^{\prime}\right)  $. Applying this to $a=\sigma_{1}$, $\varphi=\pi$
and $\varphi^{\prime}=\pi^{\prime}$, we obtain
\[
\operatorname{st}\left(  \sigma_{1}:\pi\right)  =\operatorname{st}\left(
\sigma_{1}:\pi^{\prime}\right)
\]
(since $\operatorname{st} \pi   =\operatorname{st}\left(
\pi^{\prime}\right)  $ and $\left\vert \pi\right\vert =\left\vert \pi^{\prime
}\right\vert $ and $\left[  \sigma_{1}>\pi_{1}\right]  =\left[  \sigma_{1}%
>\pi_{1}^{\prime}\right]  $).

Next, we claim that%
\begin{align}
&\left\{  \operatorname{st} \tau   \ \mid\ \tau\in S_{\prec
}\left(  \sigma_{1}:\pi,\sigma_{\sim1}\right)  \right\}
_{\operatorname*{multi}}
\nonumber\\
&=\left\{
\operatorname{st} \tau   \ \mid\ \tau\in S_{\prec}\left(
\sigma_{1}:\pi^{\prime},\sigma_{\sim1}\right)  \right\}
_{\operatorname*{multi}}.
\label{pf.thm.head-comp.LRcomp.c1.pf.5}%
\end{align}


[\textit{Proof of (\ref{pf.thm.head-comp.LRcomp.c1.pf.5}):} The permutations
$\sigma_{1}:\pi$ and $\sigma_{1}:\pi^{\prime}$ are clearly nonempty. Hence, if
$\sigma_{\sim1}$ is the $0$-permutation $\left(  {}\right)  $, then
$S_{\prec}\left(  \sigma_{1}:\pi,\sigma_{\sim1}\right)  =\left\{  \sigma
_{1}:\pi\right\}  $ and $S_{\prec}\left(  \sigma_{1}:\pi^{\prime},\sigma
_{\sim1}\right)  =\left\{  \sigma_{1}:\pi^{\prime}\right\}  $. Thus, if
$\sigma_{\sim1}$ is the $0$-permutation $\left(  {}\right)  $, then
(\ref{pf.thm.head-comp.LRcomp.c1.pf.5}) follows from%
\begin{align*}
&  \left\{  \operatorname{st} \tau   \ \mid\ \tau\in
\underbrace{S_{\prec}\left(  \sigma_{1}:\pi,\sigma_{\sim1}\right)
}_{=\left\{  \sigma_{1}:\pi\right\}  }\right\}  _{\operatorname*{multi}}\\
&  =\left\{  \operatorname{st} \tau   \ \mid\ \tau\in\left\{
\sigma_{1}:\pi\right\}  \right\}  _{\operatorname*{multi}}=\left\{
\underbrace{\operatorname{st}\left(  \sigma_{1}:\pi\right)  }%
_{=\operatorname{st}\left(  \sigma_{1}:\pi^{\prime}\right)  }\right\}
_{\operatorname*{multi}}\\
&  =\left\{  \operatorname{st}\left(  \sigma_{1}:\pi^{\prime}\right)
\right\}  _{\operatorname*{multi}}=\left\{  \operatorname{st} \tau
\ \mid\ \tau\in\underbrace{\left\{  \sigma_{1}:\pi^{\prime
}\right\}  }_{=S_{\prec}\left(  \sigma_{1}:\pi^{\prime},\sigma_{\sim1}\right)
}\right\}  _{\operatorname*{multi}}\\
&  =\left\{  \operatorname{st} \tau   \ \mid\ \tau\in S_{\prec
}\left(  \sigma_{1}:\pi^{\prime},\sigma_{\sim1}\right)  \right\}
_{\operatorname*{multi}}.
\end{align*}
Thus, for the rest of our proof of (\ref{pf.thm.head-comp.LRcomp.c1.pf.5}), we
WLOG assume that $\sigma_{\sim1}$ is not the $0$-permutation $\left(
{}\right)  $. Thus, $\sigma_{\sim1}$ is nonempty.

But recall that $\left\vert \sigma_{\sim1}\right\vert =N-1$. Hence, the
induction hypothesis allows us to apply Claim 1 to $\sigma_{1}:\pi$,
$\sigma_{1}:\pi^{\prime}$ and $\sigma_{\sim1}$ instead of $\pi$, $\pi^{\prime
}$ and $\sigma$ (because we know that the permutations $\sigma_{\sim1}$ and
$\sigma_{1}:\pi$ are disjoint; that the permutations $\sigma_{\sim1}$ and
$\sigma_{1}:\pi^{\prime}$ are disjoint; that $\operatorname{st}\left(
\sigma_{1}:\pi\right)  =\operatorname{st}\left(  \sigma_{1}:\pi^{\prime
}\right)  $ and $\left\vert \sigma_{1}:\pi\right\vert =\left\vert \sigma
_{1}:\pi^{\prime}\right\vert $; and that $\left[  \underbrace{\left(
\sigma_{1}:\pi\right)  _{1}}_{=\left(  \sigma_{1}:\pi^{\prime}\right)  _{1}%
}>\left(  \sigma_{\sim1}\right)  _{1}\right]  =\left[  \left(  \sigma_{1}%
:\pi^{\prime}\right)  _{1}>\left(  \sigma_{\sim1}\right)  _{1}\right]  $). We
therefore obtain%
\[
\left\{  \operatorname{st} \tau   \ \mid\ \tau\in S_{\prec
}\left(  \sigma_{1}:\pi,\sigma_{\sim1}\right)  \right\}
_{\operatorname*{multi}}=\left\{  \operatorname{st} \tau 
\ \mid\ \tau\in S_{\prec}\left(  \sigma_{1}:\pi^{\prime},\sigma_{\sim
1}\right)  \right\}  _{\operatorname*{multi}}%
\]
and%
\[
\left\{  \operatorname{st} \tau   \ \mid\ \tau\in S_{\succ
}\left(  \sigma_{1}:\pi,\sigma_{\sim1}\right)  \right\}
_{\operatorname*{multi}}=\left\{  \operatorname{st} \tau 
\ \mid\ \tau\in S_{\succ}\left(  \sigma_{1}:\pi^{\prime},\sigma_{\sim
1}\right)  \right\}  _{\operatorname*{multi}}.
\]
The first of these two equalities is precisely
(\ref{pf.thm.head-comp.LRcomp.c1.pf.5}). Thus,
(\ref{pf.thm.head-comp.LRcomp.c1.pf.5}) is proven.]

Now,
\begin{align}
&  \left\{  \operatorname{st} \tau   \ \mid\ \tau\in
\underbrace{S_{\succ}\left(  \pi,\sigma\right)  }_{\substack{=S_{\prec}\left(
\sigma_{1}:\pi,\sigma_{\sim1}\right)  \\\text{(by
(\ref{pf.thm.head-comp.LRcomp.c1.pf.1}))}}}\right\}  _{\operatorname*{multi}%
}\nonumber\\
&  =\left\{  \operatorname{st} \tau   \ \mid\ \tau\in S_{\prec
}\left(  \sigma_{1}:\pi,\sigma_{\sim1}\right)  \right\}
_{\operatorname*{multi}}\nonumber\\
&  =\left\{  \operatorname{st} \tau   \ \mid\ \tau
\in\underbrace{S_{\prec}\left(  \sigma_{1}:\pi^{\prime},\sigma_{\sim1}\right)
}_{\substack{=S_{\succ}\left(  \pi^{\prime},\sigma\right)  \\\text{(by
(\ref{pf.thm.head-comp.LRcomp.c1.pf.2}))}}}\right\}  _{\operatorname*{multi}%
}\ \ \ \ \ \ \ \ \ \ \left(  \text{by (\ref{pf.thm.head-comp.LRcomp.c1.pf.5}%
)}\right) \nonumber\\
&  =\left\{  \operatorname{st} \tau   \ \mid\ \tau\in S_{\succ
}\left(  \pi^{\prime},\sigma\right)  \right\}  _{\operatorname*{multi}}.
\label{pf.thm.head-comp.LRcomp.c1.pf.6}%
\end{align}
This proves (\ref{pf.thm.head-comp.LRcomp.c1.eq4}). It remains to prove
(\ref{pf.thm.head-comp.LRcomp.c1.eq3}).

Lemma \ref{lem.LR.difference} \textbf{(a)} yields%
\begin{align}
&  \left\{  \operatorname{st} \tau   \ \mid\ \tau\in S_{\prec
}\left(  \pi,\sigma\right)  \right\}  _{\operatorname*{multi}}\nonumber\\
&  =\left\{  \operatorname{st} \tau   \ \mid\ \tau\in S\left(
\pi,\sigma\right)  \right\}  _{\operatorname*{multi}}-\left\{
\operatorname{st} \tau   \ \mid\ \tau\in S_{\succ}\left(
\pi,\sigma\right)  \right\}  _{\operatorname*{multi}}.
\label{pf.thm.head-comp.LRcomp.c1.pf.7}%
\end{align}
Lemma \ref{lem.LR.difference} \textbf{(a)} (applied to $\pi^{\prime}$ instead
of $\pi$) yields%
\begin{align}
&  \left\{  \operatorname{st} \tau   \ \mid\ \tau\in S_{\prec
}\left(  \pi^{\prime},\sigma\right)  \right\}  _{\operatorname*{multi}%
}\nonumber\\
&  =\left\{  \operatorname{st} \tau   \ \mid\ \tau\in S\left(
\pi^{\prime},\sigma\right)  \right\}  _{\operatorname*{multi}}-\left\{
\operatorname{st} \tau   \ \mid\ \tau\in S_{\succ}\left(
\pi^{\prime},\sigma\right)  \right\}  _{\operatorname*{multi}}.
\label{pf.thm.head-comp.LRcomp.c1.pf.8}%
\end{align}


But recall that the statistic $\operatorname{st}$ is shuffle-compatible. In
other words, for any two disjoint permutations $\alpha$ and $\beta$, the
multiset%
\[
\left\{  \operatorname{st} \tau   \ \mid\ \tau\in S\left(
\alpha,\beta\right)  \right\}  _{\operatorname*{multi}}%
\]
depends only on $\operatorname{st} \alpha$,
$\operatorname{st} \beta$, $\left\vert \alpha\right\vert $
and $\left\vert \beta\right\vert $ (by the definition of
shuffle-compatibility). In other words, if $\alpha$ and $\beta$ are two
disjoint permutations, and if $\alpha^{\prime}$ and $\beta^{\prime}$ are two
disjoint permutations, and if we have
\[
\operatorname{st} \alpha = \operatorname{st}\left(
\alpha^{\prime}\right)  ,\ \ \ \ \ \ \ \ \ \ \operatorname{st} \beta
=\operatorname{st}\left(  \beta^{\prime}\right)
,\ \ \ \ \ \ \ \ \ \ \left\vert \alpha\right\vert =\left\vert \alpha^{\prime
}\right\vert \ \ \ \ \ \ \ \ \ \ \text{and}\ \ \ \ \ \ \ \ \ \ \left\vert
\beta\right\vert =\left\vert \beta^{\prime}\right\vert ,
\]
then%
\[
\left\{  \operatorname{st} \tau   \ \mid\ \tau\in S\left(
\alpha,\beta\right)  \right\}  _{\operatorname*{multi}}=\left\{
\operatorname{st} \tau   \ \mid\ \tau\in S\left(  \alpha
^{\prime},\beta^{\prime}\right)  \right\}  _{\operatorname*{multi}}.
\]
Applying this to $\alpha=\pi$, $\beta=\sigma$, $\alpha^{\prime}=\pi^{\prime}$
and $\beta^{\prime}=\sigma$, we obtain
\begin{equation}
\left\{  \operatorname{st} \tau   \ \mid\ \tau\in S\left(
\pi,\sigma\right)  \right\}  _{\operatorname*{multi}}=\left\{
\operatorname{st} \tau   \ \mid\ \tau\in S\left(  \pi^{\prime
},\sigma\right)  \right\}  _{\operatorname*{multi}}
\label{pf.thm.head-comp.LRcomp.c1.pf.9}%
\end{equation}
(since $\operatorname{st} \pi   =\operatorname{st}\left(
\pi^{\prime}\right)  $, $\operatorname{st} \sigma 
=\operatorname{st} \sigma   $, $\left\vert \pi\right\vert
=\left\vert \pi^{\prime}\right\vert $ and $\left\vert \sigma\right\vert
=\left\vert \sigma\right\vert $). Now, (\ref{pf.thm.head-comp.LRcomp.c1.pf.7})
becomes%
\begin{align*}
&  \left\{  \operatorname{st} \tau   \ \mid\ \tau\in S_{\prec
}\left(  \pi,\sigma\right)  \right\}  _{\operatorname*{multi}}\\
&  =\underbrace{\left\{  \operatorname{st} \tau   \ \mid
\ \tau\in S\left(  \pi,\sigma\right)  \right\}  _{\operatorname*{multi}}%
}_{\substack{=\left\{  \operatorname{st} \tau   \ \mid\ \tau\in
S\left(  \pi^{\prime},\sigma\right)  \right\}  _{\operatorname*{multi}%
}\\\text{(by (\ref{pf.thm.head-comp.LRcomp.c1.pf.9}))}}}-\underbrace{\left\{
\operatorname{st} \tau   \ \mid\ \tau\in S_{\succ}\left(
\pi,\sigma\right)  \right\}  _{\operatorname*{multi}}}_{\substack{=\left\{
\operatorname{st} \tau   \ \mid\ \tau\in S_{\succ}\left(
\pi^{\prime},\sigma\right)  \right\}  _{\operatorname*{multi}}\\\text{(by
(\ref{pf.thm.head-comp.LRcomp.c1.pf.6}))}}}\\
&  =\left\{  \operatorname{st} \tau   \ \mid\ \tau\in S\left(
\pi^{\prime},\sigma\right)  \right\}  _{\operatorname*{multi}}-\left\{
\operatorname{st} \tau   \ \mid\ \tau\in S_{\succ}\left(
\pi^{\prime},\sigma\right)  \right\}  _{\operatorname*{multi}}\\
&  =\left\{  \operatorname{st} \tau   \ \mid\ \tau\in S_{\prec
}\left(  \pi^{\prime},\sigma\right)  \right\}  _{\operatorname*{multi}%
}\ \ \ \ \ \ \ \ \ \ \left(  \text{by (\ref{pf.thm.head-comp.LRcomp.c1.pf.8}%
)}\right)  .
\end{align*}
Thus, (\ref{pf.thm.head-comp.LRcomp.c1.eq3}) is proven. Hence, we have proven
both (\ref{pf.thm.head-comp.LRcomp.c1.eq3}) and
(\ref{pf.thm.head-comp.LRcomp.c1.eq4}). This shows that Claim 1 holds for our
$\pi$, $\pi^{\prime}$ and $\sigma$. This completes the induction step. Thus,
Claim 1 is proven by induction.]

We shall next derive a \textquotedblleft mirror version\textquotedblright\ of
Claim 1:

\begin{statement}
\textit{Claim 2:} Let $\pi$, $\sigma$ and $\sigma^{\prime}$ be three nonempty
permutations. Assume that $\pi$ and $\sigma$ are disjoint. Assume that $\pi$
and $\sigma^{\prime}$ are disjoint. Assume furthermore that%
\[
\operatorname{st} \sigma   =\operatorname{st}\left(
\sigma^{\prime}\right)  ,\ \ \ \ \ \ \ \ \ \ \left\vert \sigma\right\vert
=\left\vert \sigma^{\prime}\right\vert \ \ \ \ \ \ \ \ \ \ \text{and}%
\ \ \ \ \ \ \ \ \ \ \left[  \pi_{1}>\sigma_{1}\right]  =\left[  \pi_{1}%
>\sigma_{1}^{\prime}\right]  .
\]
Then,
\begin{align*}
&  \left\{  \operatorname{st} \tau   \ \mid\ \tau\in S_{\prec
}\left(  \pi,\sigma\right)  \right\}  _{\operatorname*{multi}}\\
&  =\left\{  \operatorname{st} \tau   \ \mid\ \tau\in S_{\prec
}\left(  \pi,\sigma^{\prime}\right)  \right\}  _{\operatorname*{multi}}%
\end{align*}
and%
\begin{align*}
&  \left\{  \operatorname{st} \tau   \ \mid\ \tau\in S_{\succ
}\left(  \pi,\sigma\right)  \right\}  _{\operatorname*{multi}}\\
&  =\left\{  \operatorname{st} \tau   \ \mid\ \tau\in S_{\succ
}\left(  \pi,\sigma^{\prime}\right)  \right\}  _{\operatorname*{multi}}.
\end{align*}

\end{statement}

[\textit{Proof of Claim 2:} We have $\sigma_{1}\neq\pi_{1}$ (since $\pi$ and
$\sigma$ are disjoint). Thus, the statement $\left(  \sigma_{1}>\pi
_{1}\right)  $ is equivalent to $\left(  \text{not }\pi_{1}>\sigma_{1}\right)
$. Hence, $\left[  \sigma_{1}>\pi_{1}\right]  =\left[  \text{not }\pi
_{1}>\sigma_{1}\right]  =1-\left[  \pi_{1}>\sigma_{1}\right]  $. Similarly,
$\left[  \sigma_{1}^{\prime}>\pi_{1}\right]  =1-\left[  \pi_{1}>\sigma
_{1}^{\prime}\right]  $. Hence,%
\[
\left[  \sigma_{1}>\pi_{1}\right]  =1-\underbrace{\left[  \pi_{1}>\sigma
_{1}\right]  }_{=\left[  \pi_{1}>\sigma_{1}^{\prime}\right]  }=1-\left[
\pi_{1}>\sigma_{1}^{\prime}\right]  =\left[  \sigma_{1}^{\prime}>\pi
_{1}\right]  .
\]
Hence, Claim 1 (applied to $\sigma$, $\sigma^{\prime}$ and $\pi$ instead of
$\pi$, $\pi^{\prime}$ and $\sigma$) shows that%
\begin{align*}
&  \left\{  \operatorname{st} \tau   \ \mid\ \tau\in S_{\prec
}\left(  \sigma,\pi\right)  \right\}  _{\operatorname*{multi}}\\
&  =\left\{  \operatorname{st} \tau   \ \mid\ \tau\in S_{\prec
}\left(  \sigma^{\prime},\pi\right)  \right\}  _{\operatorname*{multi}}%
\end{align*}
and%
\begin{align*}
&  \left\{  \operatorname{st} \tau   \ \mid\ \tau\in S_{\succ
}\left(  \sigma,\pi\right)  \right\}  _{\operatorname*{multi}}\\
&  =\left\{  \operatorname{st} \tau   \ \mid\ \tau\in S_{\succ
}\left(  \sigma^{\prime},\pi\right)  \right\}  _{\operatorname*{multi}}.
\end{align*}
But Proposition \ref{prop.LR.rec} \textbf{(a)} yields $S_{\prec}\left(
\pi,\sigma\right)  =S_{\succ}\left(  \sigma,\pi\right)  $. Similarly,
$S_{\prec}\left(  \pi,\sigma^{\prime}\right)  =S_{\succ}\left(  \sigma
^{\prime},\pi\right)  $. Also, Proposition \ref{prop.LR.rec} \textbf{(a)}
(applied to $\sigma$ and $\pi$ instead of $\pi$ and $\sigma$) yields
$S_{\prec}\left(  \sigma,\pi\right)  =S_{\succ}\left(  \pi,\sigma\right)  $.
Similarly, $S_{\prec}\left(  \sigma^{\prime},\pi\right)  =S_{\succ}\left(
\pi,\sigma^{\prime}\right)  $. Using all these equalities, we find%
\begin{align*}
&  \left\{  \operatorname{st} \tau   \ \mid\ \tau\in
\underbrace{S_{\prec}\left(  \pi,\sigma\right)  }_{=S_{\succ}\left(
\sigma,\pi\right)  }\right\}  _{\operatorname*{multi}}\\
&  =\left\{  \operatorname{st} \tau   \ \mid\ \tau\in S_{\succ
}\left(  \sigma,\pi\right)  \right\}  _{\operatorname*{multi}}=\left\{
\operatorname{st} \tau   \ \mid\ \tau\in\underbrace{S_{\succ
}\left(  \sigma^{\prime},\pi\right)  }_{=S_{\prec}\left(  \pi,\sigma^{\prime
}\right)  }\right\}  _{\operatorname*{multi}}\\
&  =\left\{  \operatorname{st} \tau   \ \mid\ \tau\in S_{\prec
}\left(  \pi,\sigma^{\prime}\right)  \right\}  _{\operatorname*{multi}}%
\end{align*}
and%
\begin{align*}
&  \left\{  \operatorname{st} \tau   \ \mid\ \tau\in
\underbrace{S_{\succ}\left(  \pi,\sigma\right)  }_{=S_{\prec}\left(
\sigma,\pi\right)  }\right\}  _{\operatorname*{multi}}\\
&  =\left\{  \operatorname{st} \tau   \ \mid\ \tau\in S_{\prec
}\left(  \sigma,\pi\right)  \right\}  _{\operatorname*{multi}}=\left\{
\operatorname{st} \tau   \ \mid\ \tau\in\underbrace{S_{\prec
}\left(  \sigma^{\prime},\pi\right)  }_{=S_{\succ}\left(  \pi,\sigma^{\prime
}\right)  }\right\}  _{\operatorname*{multi}}\\
&  =\left\{  \operatorname{st} \tau   \ \mid\ \tau\in S_{\succ
}\left(  \pi,\sigma^{\prime}\right)  \right\}  _{\operatorname*{multi}}.
\end{align*}
Thus, Claim 2 is proven.]

Finally, we are ready to take on the LR-shuffle-compatibility of
$\operatorname{st}$:

\begin{statement}
\textit{Claim 3:} Let $\pi$ and $\sigma$ be two disjoint nonempty
permutations. Let $\pi^{\prime}$ and $\sigma^{\prime}$ be two disjoint
nonempty permutations. Assume that%
\begin{align*}
\operatorname{st} \pi    &  =\operatorname{st}\left(
\pi^{\prime}\right)  ,\ \ \ \ \ \ \ \ \ \ \operatorname{st}
\sigma  =\operatorname{st}\left(  \sigma^{\prime}\right)  ,\\
\left\vert \pi\right\vert  &  =\left\vert \pi^{\prime}\right\vert
,\ \ \ \ \ \ \ \ \ \ \left\vert \sigma\right\vert =\left\vert \sigma^{\prime
}\right\vert \ \ \ \ \ \ \ \ \ \ \text{and}\ \ \ \ \ \ \ \ \ \ \left[  \pi
_{1}>\sigma_{1}\right]  =\left[  \pi_{1}^{\prime}>\sigma_{1}^{\prime}\right]
.
\end{align*}
Then,
\[
\left\{  \operatorname{st} \tau   \ \mid\ \tau\in S_{\prec
}\left(  \pi,\sigma\right)  \right\}  _{\operatorname*{multi}}=\left\{
\operatorname{st} \tau   \ \mid\ \tau\in S_{\prec}\left(
\pi^{\prime},\sigma^{\prime}\right)  \right\}  _{\operatorname*{multi}}%
\]
and%
\[
\left\{  \operatorname{st} \tau   \ \mid\ \tau\in S_{\succ
}\left(  \pi,\sigma\right)  \right\}  _{\operatorname*{multi}}=\left\{
\operatorname{st} \tau   \ \mid\ \tau\in S_{\succ}\left(
\pi^{\prime},\sigma^{\prime}\right)  \right\}  _{\operatorname*{multi}}.
\]

\end{statement}

[\textit{Proof of Claim 3:} We have $\left[  \pi_{1}>\sigma_{1}\right]
=\left[  \pi_{1}^{\prime}>\sigma_{1}^{\prime}\right]  $. Since $\left[
\pi_{1}>\sigma_{1}\right]  =\left[  \pi_{1}^{\prime}>\sigma_{1}^{\prime
}\right]  $ is either $1$ or $0$, we must therefore be in one of the following
two cases:

\textit{Case 1:} We have $\left[  \pi_{1}>\sigma_{1}\right]  =\left[  \pi
_{1}^{\prime}>\sigma_{1}^{\prime}\right]  =1$.

\textit{Case 2:} We have $\left[  \pi_{1}>\sigma_{1}\right]  =\left[  \pi
_{1}^{\prime}>\sigma_{1}^{\prime}\right]  =0$.

Let us first consider Case 1. In this case, we have $\left[  \pi_{1}%
>\sigma_{1}\right]  =\left[  \pi_{1}^{\prime}>\sigma_{1}^{\prime}\right]  =1$.

There clearly exists a positive integer $N$ that is larger than all entries of
$\sigma$ and larger than all entries of $\sigma^{\prime}$. Consider such an
$N$. Let $n=\left\vert \pi\right\vert $; thus, $\pi=\left(  \pi_{1},\pi
_{2},\ldots,\pi_{n}\right)  $. Let $\gamma$ be the permutation $\left(
\pi_{1}+N,\pi_{2}+N,\ldots,\pi_{n}+N\right)  $. This permutation $\gamma$ is
order-isomorphic to $\pi$, but is disjoint from $\sigma$ (since all its
entries are $>N$, while all the entries of $\sigma$ are $<N$) and disjoint
from $\sigma^{\prime}$ (for similar reasons). Also, $\gamma_{1}%
=\underbrace{\pi_{1}}_{>0}+N>N>\sigma_{1}$ (since $N$ is larger than all
entries of $\sigma$), so that $\left[  \gamma_{1}%
>\sigma_{1}\right]  =1$. Similarly, $\left[  \gamma_{1}>\sigma_{1}^{\prime
}\right]  =1$.

The permutation $\gamma$ is order-isomorphic to $\pi$. Thus,
$\operatorname{st} \gamma = \operatorname{st} \pi$ (since $\operatorname{st}$
is a permutation statistic) and
$\left\vert \gamma\right\vert =\left\vert \pi\right\vert $. The permutation
$\gamma$ is furthermore nonempty (since it is order-isomorphic to the nonempty
permutation $\pi$). Also, $\operatorname{st} \gamma
=\operatorname{st} \pi =\operatorname{st}\left(  \pi^{\prime
}\right)  $ and $\left\vert \gamma\right\vert =\left\vert \pi\right\vert
=\left\vert \pi^{\prime}\right\vert $. Moreover, $\left[  \pi_{1}>\sigma
_{1}\right]  =1=\left[  \gamma_{1}>\sigma_{1}\right]  $ and $\left[
\gamma_{1}>\sigma_{1}\right]  =1=\left[  \gamma_{1}>\sigma_{1}^{\prime
}\right]  $ and
$\left[  \gamma_{1}>\sigma^{\prime}_{1}\right]  =1=\left[  \pi
_{1}^{\prime}>\sigma_{1}^{\prime}\right]  $. Hence, Claim 1 (applied to
$\gamma$ instead of $\pi^{\prime}$) yields%
\[
\left\{  \operatorname{st} \tau   \ \mid\ \tau\in S_{\prec
}\left(  \pi,\sigma\right)  \right\}  _{\operatorname*{multi}}=\left\{
\operatorname{st} \tau   \ \mid\ \tau\in S_{\prec}\left(
\gamma,\sigma\right)  \right\}  _{\operatorname*{multi}}%
\]
and%
\begin{equation}
\left\{  \operatorname{st} \tau   \ \mid\ \tau\in S_{\succ
}\left(  \pi,\sigma\right)  \right\}  _{\operatorname*{multi}}=\left\{
\operatorname{st} \tau   \ \mid\ \tau\in S_{\succ}\left(
\gamma,\sigma\right)  \right\}  _{\operatorname*{multi}}.\nonumber
\end{equation}
Furthermore, Claim 2 (applied to $\gamma$ instead of $\pi$) yields%
\[
\left\{  \operatorname{st} \tau   \ \mid\ \tau\in S_{\prec
}\left(  \gamma,\sigma\right)  \right\}  _{\operatorname*{multi}}=\left\{
\operatorname{st} \tau   \ \mid\ \tau\in S_{\prec}\left(
\gamma,\sigma^{\prime}\right)  \right\}  _{\operatorname*{multi}}%
\]
and%
\[
\left\{  \operatorname{st} \tau   \ \mid\ \tau\in S_{\succ
}\left(  \gamma,\sigma\right)  \right\}  _{\operatorname*{multi}}=\left\{
\operatorname{st} \tau   \ \mid\ \tau\in S_{\succ}\left(
\gamma,\sigma^{\prime}\right)  \right\}  _{\operatorname*{multi}}.
\]
Finally, Claim 1 (applied to $\gamma$ and $\sigma^{\prime}$ instead of $\pi$
and $\sigma$) yields%
\[
\left\{  \operatorname{st} \tau   \ \mid\ \tau\in S_{\prec
}\left(  \gamma,\sigma^{\prime}\right)  \right\}  _{\operatorname*{multi}%
}=\left\{  \operatorname{st} \tau   \ \mid\ \tau\in S_{\prec
}\left(  \pi^{\prime},\sigma^{\prime}\right)  \right\}
_{\operatorname*{multi}}%
\]
and%
\begin{equation}
\left\{  \operatorname{st} \tau   \ \mid\ \tau\in S_{\succ
}\left(  \gamma,\sigma^{\prime}\right)  \right\}  _{\operatorname*{multi}%
}=\left\{  \operatorname{st} \tau   \ \mid\ \tau\in S_{\succ
}\left(  \pi^{\prime},\sigma^{\prime}\right)  \right\}
_{\operatorname*{multi}}.\nonumber
\end{equation}
Combining the equalities we have found, we obtain%
\begin{align*}
\left\{  \operatorname{st} \tau   \ \mid\ \tau\in S_{\prec
}\left(  \pi,\sigma\right)  \right\}  _{\operatorname*{multi}}  &  =\left\{
\operatorname{st} \tau   \ \mid\ \tau\in S_{\prec}\left(
\gamma,\sigma\right)  \right\}  _{\operatorname*{multi}}\\
&  =\left\{  \operatorname{st} \tau   \ \mid\ \tau\in S_{\prec
}\left(  \gamma,\sigma^{\prime}\right)  \right\}  _{\operatorname*{multi}}\\
&  =\left\{  \operatorname{st} \tau   \ \mid\ \tau\in S_{\prec
}\left(  \pi^{\prime},\sigma^{\prime}\right)  \right\}
_{\operatorname*{multi}}.
\end{align*}
The same argument (but with the symbols \textquotedblleft$S_{\prec}%
$\textquotedblright\ and \textquotedblleft$S_{\succ}$\textquotedblright%
\ interchanged) yields
\[
\left\{  \operatorname{st} \tau   \ \mid\ \tau\in S_{\succ
}\left(  \pi,\sigma\right)  \right\}  _{\operatorname*{multi}}=\left\{
\operatorname{st} \tau   \ \mid\ \tau\in S_{\succ}\left(
\pi^{\prime},\sigma^{\prime}\right)  \right\}  _{\operatorname*{multi}}.
\]
Thus, Claim 3 is proven in Case 1.

Let us now consider Case 2. In this case, we have $\left[  \pi_{1}>\sigma
_{1}\right]  =\left[  \pi_{1}^{\prime}>\sigma_{1}^{\prime}\right]  =0$.

There clearly exists a positive integer $N$ that is larger than all entries of
$\pi$ and larger than all entries of $\pi^{\prime}$. Consider such an $N$. Set
$m=\left\vert \sigma\right\vert $. Thus, $\sigma=\left(  \sigma_{1},\sigma
_{2},\ldots,\sigma_{m}\right)  $.
Let $\delta$ be the permutation $\left(  \sigma_{1}+N,\sigma_{2}%
+N,\ldots,\sigma_{m}+N\right)  $. This permutation $\delta$ is
order-isomorphic to $\sigma$, but is disjoint from $\pi$ (since all its
entries are $>N$, while all the entries of $\pi$ are $<N$) and disjoint from
$\pi^{\prime}$ (for similar reasons). Also, $\delta_{1}=\underbrace{\sigma
_{1}}_{>0}+N>N>\pi_{1}$ (since $N$ is larger than all entries of $\pi$),
so that we don't have $\pi_{1}>\delta_{1}$. Thus,
$\left[  \pi_{1}>\delta_{1}\right]  =0$. Similarly, $\left[  \pi_{1}^{\prime
}>\delta_{1}\right]  =0$.

The permutation $\delta$ is order-isomorphic to $\sigma$. Thus,
$\operatorname{st} \delta = \operatorname{st} \sigma$
(since $\operatorname{st}$ is a permutation statistic) and
$\left\vert \delta\right\vert =\left\vert \sigma\right\vert $. The permutation
$\delta$ is furthermore nonempty (since it is order-isomorphic to the nonempty
permutation $\sigma$). Also, $\operatorname{st} \delta
=\operatorname{st} \sigma   =\operatorname{st}\left(
\sigma^{\prime}\right)  $ and $\left\vert \delta\right\vert =\left\vert
\sigma\right\vert =\left\vert \sigma^{\prime}\right\vert $. Moreover, $\left[
\pi_{1}>\sigma_{1}\right]  =0=\left[  \pi_{1}>\delta_{1}\right]  $ and
$\left[  \pi_{1}>\delta_{1}\right]  =0=\left[  \pi_{1}^{\prime}>\delta
_{1}\right]  $ and $\left[  \pi_{1}^{\prime}>\delta_{1}\right]  =0=\left[
\pi_{1}^{\prime}>\sigma_{1}^{\prime}\right]  $. Hence, Claim 2 (applied to
$\delta$ instead of $\sigma^{\prime}$) yields%
\[
\left\{  \operatorname{st} \tau   \ \mid\ \tau\in S_{\prec
}\left(  \pi,\sigma\right)  \right\}  _{\operatorname*{multi}}=\left\{
\operatorname{st} \tau   \ \mid\ \tau\in S_{\prec}\left(
\pi,\delta\right)  \right\}  _{\operatorname*{multi}}%
\]
and%
\begin{equation}
\left\{  \operatorname{st} \tau   \ \mid\ \tau\in S_{\succ
}\left(  \pi,\sigma\right)  \right\}  _{\operatorname*{multi}}=\left\{
\operatorname{st} \tau   \ \mid\ \tau\in S_{\succ}\left(
\pi,\delta\right)  \right\}  _{\operatorname*{multi}}.\nonumber
\end{equation}
Furthermore, Claim 1 (applied to $\delta$ instead of $\sigma$) yields%
\[
\left\{  \operatorname{st} \tau   \ \mid\ \tau\in S_{\prec
}\left(  \pi,\delta\right)  \right\}  _{\operatorname*{multi}}=\left\{
\operatorname{st} \tau   \ \mid\ \tau\in S_{\prec}\left(
\pi^{\prime},\delta\right)  \right\}  _{\operatorname*{multi}}%
\]
and%
\[
\left\{  \operatorname{st} \tau   \ \mid\ \tau\in S_{\succ
}\left(  \pi,\delta\right)  \right\}  _{\operatorname*{multi}}=\left\{
\operatorname{st} \tau   \ \mid\ \tau\in S_{\succ}\left(
\pi^{\prime},\delta\right)  \right\}  _{\operatorname*{multi}}.
\]
Finally, Claim 2 (applied to $\pi^{\prime}$ and $\delta$ instead of $\pi$ and
$\sigma$) yields%
\[
\left\{  \operatorname{st} \tau   \ \mid\ \tau\in S_{\prec
}\left(  \pi^{\prime},\delta\right)  \right\}  _{\operatorname*{multi}%
}=\left\{  \operatorname{st} \tau   \ \mid\ \tau\in S_{\prec
}\left(  \pi^{\prime},\sigma^{\prime}\right)  \right\}
_{\operatorname*{multi}}%
\]
and%
\begin{equation}
\left\{  \operatorname{st} \tau   \ \mid\ \tau\in S_{\succ
}\left(  \pi^{\prime},\delta\right)  \right\}  _{\operatorname*{multi}%
}=\left\{  \operatorname{st} \tau   \ \mid\ \tau\in S_{\succ
}\left(  \pi^{\prime},\sigma^{\prime}\right)  \right\}
_{\operatorname*{multi}}.\nonumber
\end{equation}
Combining the equalities we have found, we obtain%
\begin{align*}
\left\{  \operatorname{st} \tau   \ \mid\ \tau\in S_{\prec
}\left(  \pi,\sigma\right)  \right\}  _{\operatorname*{multi}}  &  =\left\{
\operatorname{st} \tau   \ \mid\ \tau\in S_{\prec}\left(
\pi,\delta\right)  \right\}  _{\operatorname*{multi}}\\
&  =\left\{  \operatorname{st} \tau   \ \mid\ \tau\in S_{\prec
}\left(  \pi^{\prime},\delta\right)  \right\}  _{\operatorname*{multi}}\\
&  =\left\{  \operatorname{st} \tau   \ \mid\ \tau\in S_{\prec
}\left(  \pi^{\prime},\sigma^{\prime}\right)  \right\}
_{\operatorname*{multi}}.
\end{align*}
The same argument (but with the symbols \textquotedblleft$S_{\prec}%
$\textquotedblright\ and \textquotedblleft$S_{\succ}$\textquotedblright%
\ interchanged) yields
\[
\left\{  \operatorname{st} \tau   \ \mid\ \tau\in S_{\succ
}\left(  \pi,\sigma\right)  \right\}  _{\operatorname*{multi}}=\left\{
\operatorname{st} \tau   \ \mid\ \tau\in S_{\succ}\left(
\pi^{\prime},\sigma^{\prime}\right)  \right\}  _{\operatorname*{multi}}.
\]
Thus, Claim 3 is proven in Case 2.

We have now proven Claim 3 in each of the two Cases 1 and 2. Hence, Claim 3
always holds.]

Claim 3 says that for any two disjoint nonempty permutations $\pi$ and
$\sigma$, the multisets%
\[
\left\{  \operatorname{st} \tau   \ \mid\ \tau\in S_{\prec
}\left(  \pi,\sigma\right)  \right\}  _{\operatorname*{multi}}%
\ \ \ \ \ \ \ \ \ \ \text{and}\ \ \ \ \ \ \ \ \ \ \left\{  \operatorname{st}
\tau \ \mid\ \tau\in S_{\succ}\left(  \pi,\sigma\right)
\right\}  _{\operatorname*{multi}}
\]
depend only on $\operatorname{st} \pi   $, $\operatorname{st} \sigma$,
$\left\vert \pi\right\vert $, $\left\vert
\sigma\right\vert $ and $\left[  \pi_{1}>\sigma_{1}\right]  $. In other words,
the statistic $\operatorname{st}$ is LR-shuffle-compatible (by the definition
of \textquotedblleft LR-shuffle-compatible\textquotedblright). This proves
Theorem \ref{thm.head-comp.LRcomp}.
\end{proof}

Combining Theorem \ref{thm.head-comp.LRcomp} with Proposition
\ref{prop.head-comp.Pks}, we obtain the following:

\begin{theorem}
\label{thm.LRcomp.Pks}\textbf{(a)} The permutation statistic
$\operatorname{Des}$ is LR-shuffle-compatible.

\textbf{(b)} The permutation statistic $\operatorname*{Lpk}$ is LR-shuffle-compatible.

\textbf{(c)} The permutation statistic $\operatorname{Epk}$ is LR-shuffle-compatible.
\end{theorem}

\begin{proof}
[Proof of Theorem \ref{thm.LRcomp.Pks}.]\textbf{(a)} The permutation statistic
$\operatorname{Des}$ is shuffle-compatible (by \cite[\S 2.4]{part1}) and
head-graft-compatible (by Proposition \ref{prop.head-comp.Pks} \textbf{(a)}).
Thus, Theorem \ref{thm.head-comp.LRcomp} (applied to $\operatorname{st}
=\operatorname{Des}$) shows that the permutation statistic
$\operatorname{Des}$ is LR-shuffle-compatible. This proves Theorem
\ref{thm.LRcomp.Pks} \textbf{(a)}.

\textbf{(b)} The permutation statistic $\operatorname*{Lpk}$ is
shuffle-compatible (by \cite[Theorem 4.9 \textbf{(a)}]{part1}) and
head-graft-compatible (by Proposition \ref{prop.head-comp.Pks} \textbf{(b)}).
Thus, Theorem \ref{thm.head-comp.LRcomp} (applied to $\operatorname{st}
=\operatorname{Lpk}$) shows that the permutation statistic
$\operatorname{Lpk}$ is LR-shuffle-compatible. This proves Theorem
\ref{thm.LRcomp.Pks} \textbf{(b)}.

\textbf{(c)} The permutation statistic $\operatorname{Epk}$ is
shuffle-compatible (by Theorem \ref{thm.Epk.sh-co-a}) and
head-graft-compatible (by Proposition \ref{prop.head-comp.Pks} \textbf{(c)}).
Thus, Theorem \ref{thm.head-comp.LRcomp} (applied to $\operatorname{st}
=\operatorname{Epk}$) shows that the permutation statistic
$\operatorname{Epk}$ is LR-shuffle-compatible. This proves Theorem
\ref{thm.LRcomp.Pks} \textbf{(c)}.
\end{proof}

\subsection{\label{subsect.LR.others}Some other statistics}

The question of LR-shuffle-compatibility can be asked about any statistic. We
have so far answered it for $\operatorname{Des}$, $\operatorname*{Pk}$,
$\operatorname*{Lpk}$, $\operatorname*{Rpk}$ and $\operatorname{Epk}$. In
this section, we shall analyze it for some further statistics.

\subsubsection{The descent number $\operatorname{des}$}

The permutation statistic $\operatorname{des}$ (called the \textit{descent
number}) is defined as follows: For each permutation $\pi$, we set
$\operatorname{des}\pi=\left\vert \operatorname{Des}\pi\right\vert $ (that
is, $\operatorname{des}\pi$ is the number of all descents of $\pi$). It was
proven in \cite[Theorem 4.6 \textbf{(a)}]{part1} that this statistic
$\operatorname{des}$ is shuffle-compatible. We now claim the following:

\begin{proposition}
\label{prop.LRcomp.des}The permutation statistic $\operatorname{des}$ is
head-graft-compatible and LR-shuffle-compatible.
\end{proposition}

\begin{proof}
[Proof of Proposition \ref{prop.LRcomp.des}.]From
(\ref{pf.prop.head-comp.Pks.a.1}), we easily obtain the following: If $\pi$ is
a nonempty permutation, and if $a$ is a letter that does not appear in $\pi$,
then%
\[
\operatorname{des}\left(  a:\pi\right)  =\operatorname{des}\pi+\left[
a>\pi_{1}\right]  .
\]
Thus, $\operatorname{des}\left(  a:\pi\right)  $ depends only on
$\operatorname{des}\pi$, $\left\vert \pi\right\vert $ and $\left[  a>\pi
_{1}\right]  $. In other words, $\operatorname{des}$ is head-graft-compatible
(by the definition of \textquotedblleft
head-graft-compatible\textquotedblright). Hence,
Theorem \ref{thm.head-comp.LRcomp}
(applied to $\operatorname{st}=\operatorname{des}$) shows that
the permutation statistic $\operatorname{des}$ is
LR-shuffle-compatible. This proves Proposition \ref{prop.LRcomp.des}.
\end{proof}

\subsubsection{The major index $\operatorname*{maj}$}

The permutation statistic $\operatorname*{maj}$ (called the \textit{major
index}) is defined as follows: For each permutation $\pi$, we set
$\operatorname*{maj}\pi=\sum_{i\in\operatorname{Des}\pi}i$ (that is,
$\operatorname*{maj}\pi$ is the sum of all descents of $\pi$). It was proven
in \cite[Theorem 3.1 \textbf{(a)}]{part1} that this statistic
$\operatorname*{maj}$ is shuffle-compatible.

However, $\operatorname*{maj}$ is neither head-graft-compatible nor
LR-shuffle-compatible. For example, if we take $\pi=\left(  5,4,2,3\right)  $,
$a=1$, $\pi^{\prime}=\left(  3,4,5,2\right)  $ and $a^{\prime}=1$, then we do
have%
\[
\operatorname*{maj} \pi =\operatorname*{maj}\left(  \pi
^{\prime}\right)  ,\ \ \ \ \ \ \ \ \ \ \left\vert \pi\right\vert =\left\vert
\pi^{\prime}\right\vert \ \ \ \ \ \ \ \ \ \ \text{and}%
\ \ \ \ \ \ \ \ \ \ \left[  a>\pi_{1}\right]  =\left[  a^{\prime}>\pi
_{1}^{\prime}\right]
\]
but we don't have $\operatorname*{maj}\left(  a:\pi\right)
=\operatorname*{maj}\left(  a^{\prime}:\pi^{\prime}\right)  $. Thus,
$\operatorname*{maj}$ is not head-graft-compatible. Using Corollary
\ref{cor.LRcomp.back} below, this entails that
$\operatorname*{maj}$ is not LR-shuffle-compatible.

\subsubsection{The joint statistic $\left(  \operatorname{des}
,\operatorname*{maj}\right)  $}

The next permutation statistic we shall study is the so-called joint statistic
$\left(  \operatorname{des},\operatorname*{maj}\right)  $. This statistic is
defined as the permutation statistic that sends each permutation $\pi$ to the
ordered pair $\left(  \operatorname{des}\pi,\operatorname*{maj}\pi\right)  $.
(Calling it $\left(  \operatorname{des},\operatorname*{maj}\right)  $ is thus
a slight abuse of notation.) It was proven in \cite[Theorem 4.5 \textbf{(a)}%
]{part1} that this statistic $\left(  \operatorname{des},\operatorname{maj}%
\right)  $ is shuffle-compatible. We now claim the following:

\begin{proposition}
\label{prop.LRcomp.desmaj}The permutation statistic $\left(
\operatorname{des},\operatorname*{maj}\right)  $ is head-graft-compatible and LR-shuffle-compatible.
\end{proposition}

\begin{proof}
[Proof of Proposition \ref{prop.LRcomp.desmaj}.]From
(\ref{pf.prop.head-comp.Pks.a.1}), we easily obtain the following: If $\pi$ is
a nonempty permutation, and if $a$ is a letter that does not appear in $\pi$,
then%
\begin{align*}
\operatorname{des}\left(  a:\pi\right)
&  =\operatorname{des}\pi+\left[ a>\pi_{1}\right]
\ \ \ \ \ \ \ \ \ \ \text{and}\\
\operatorname{maj}\left(  a:\pi\right)
&  =\operatorname{des} \pi + \operatorname{maj}\pi +\left[  a>\pi_{1}\right]  .
\end{align*}
Thus, $\left(  \operatorname{des},\operatorname*{maj}\right)  \left(
a:\pi\right)  $ depends only on $\left(  \operatorname{des}
,\operatorname*{maj}\right)  \left(  \pi\right)  $, $\left\vert \pi\right\vert
$ and $\left[  a>\pi_{1}\right]  $. In other words, $\left(
\operatorname{des},\operatorname*{maj}\right)  $ is head-graft-compatible (by
the definition of \textquotedblleft head-graft-compatible\textquotedblright).
Hence, Theorem \ref{thm.head-comp.LRcomp} (applied to $\operatorname{st}
=\left(  \operatorname{des},\operatorname*{maj}\right)  $) shows that the
permutation statistic $\left(  \operatorname{des},\operatorname*{maj}\right)
$ is LR-shuffle-compatible. This proves Proposition \ref{prop.LRcomp.desmaj}.
\end{proof}

\subsubsection{The comajor index $\operatorname*{comaj}$}

The permutation statistic $\operatorname*{comaj}$ (called the \textit{comajor
index}) is defined as follows: For each permutation $\pi$, we set
$\operatorname*{comaj}\pi=\sum_{k\in\operatorname{Des}\pi}\left(  n-k\right)
$, where $n=\left\vert \pi\right\vert $. It was proven in \cite[\S 3.2]{part1}
that this statistic $\operatorname*{comaj}$ is shuffle-compatible. We now
claim the following:

\begin{proposition}
\label{prop.LRcomp.comaj}The permutation statistic $\operatorname*{comaj}$ is
head-graft-compatible and \newline LR-shuffle-compatible.
\end{proposition}

\begin{proof}
[Proof of Proposition \ref{prop.LRcomp.comaj}.]From
(\ref{pf.prop.head-comp.Pks.a.1}), we easily obtain the following: If $\pi$ is
a nonempty permutation, and if $a$ is a letter that does not appear in $\pi$,
then%
\[
\operatorname*{comaj}\left(  a:\pi\right)  =\operatorname*{comaj}\pi+\left[
a>\pi_{1}\right]  \cdot\left\vert \pi\right\vert .
\]
Thus, $\operatorname*{comaj}\left(  a:\pi\right)  $ depends only on
$\operatorname*{comaj}\pi$, $\left\vert \pi\right\vert $ and $\left[
a>\pi_{1}\right]  $. In other words, $\operatorname*{comaj}$ is
head-graft-compatible (by the definition of \textquotedblleft
head-graft-compatible\textquotedblright). Hence, Theorem
\ref{thm.head-comp.LRcomp} (applied to $\operatorname{st}
=\operatorname*{comaj}$) shows that the permutation statistic
$\operatorname*{comaj}$ is LR-shuffle-compatible. This proves Proposition
\ref{prop.LRcomp.comaj}.
\end{proof}

\subsection{Left- and right-shuffle-compatibility}

In this section, we shall study two notions closely related to LR-shuffle-compatibility:

\begin{definition}
\label{def.LR.left-right}Let $\operatorname{st}$ be a permutation statistic.

\textbf{(a)} We say that $\operatorname{st}$ is
\textit{left-shuffle-compatible} if for any two disjoint nonempty permutations
$\pi$ and $\sigma$ having the property that $\pi_{1}>\sigma_{1}$, the multiset
$\left\{  \operatorname{st} \tau   \ \mid\ \tau\in S_{\prec
}\left(  \pi,\sigma\right)  \right\}  _{\operatorname*{multi}}$ depends only
on $\operatorname{st} \pi$, $\operatorname{st} \sigma$,
$\left\vert \pi\right\vert $ and $\left\vert \sigma \right\vert $.

\textbf{(b)} We say that $\operatorname{st}$ is
\textit{right-shuffle-compatible} if for any two disjoint nonempty
permutations $\pi$ and $\sigma$ having the property that $\pi_{1}>\sigma_{1}$,
the multiset $\left\{  \operatorname{st} \tau   \ \mid\ \tau\in
S_{\succ}\left(  \pi,\sigma\right)  \right\}  _{\operatorname*{multi}}$
depends only on $\operatorname{st} \pi$, $\operatorname{st} \sigma$,
$\left\vert \pi\right\vert $ and $\left\vert \sigma\right\vert $.
\end{definition}

For a shuffle-compatible permutation statistic, these two notions are
equivalent to the notions of LR-shuffle-compatibility and
head-graft-compatibility, as the following proposition reveals:

\begin{proposition}
\label{prop.LRcomp.equivs}Let $\operatorname{st}$ be a shuffle-compatible
permutation statistic. Then, the following assertions are equivalent:

\begin{itemize}
\item \textit{Assertion }$\mathcal{A}_{1}$\textit{:} The statistic
$\operatorname{st}$ is LR-shuffle-compatible.

\item \textit{Assertion }$\mathcal{A}_{2}$\textit{:} The statistic
$\operatorname{st}$ is left-shuffle-compatible.

\item \textit{Assertion }$\mathcal{A}_{3}$\textit{:} The statistic
$\operatorname{st}$ is right-shuffle-compatible.

\item \textit{Assertion }$\mathcal{A}_{4}$\textit{:} The statistic
$\operatorname{st}$ is head-graft-compatible.
\end{itemize}
\end{proposition}

\begin{proof}
[Proof of Proposition \ref{prop.LRcomp.equivs}.]Omitted; see \cite{verlong}.
\end{proof}

Note that on their own, the properties of left-shuffle-compatibility and
right-shuffle-compatibility are not equivalent. For example, the permutation
statistic that sends each nonempty permutation $\pi$ to the truth value
$\left[  \pi_{1}>\pi_{i}\text{ for all }i>1\right]  $ (and the $0$-permutation
$\left(  {}\right)  $ to $0$) is right-shuffle-compatible (because in the
definition of right-shuffle-compatibility, all the $\operatorname{st}
\tau$ will be $0$), but not left-shuffle-compatible.

\subsection{Properties of compatible statistics}

Let us state some more facts on compatibility properties. We refer to
\cite{verlong} for their proofs. We begin with a converse to Theorem
\ref{thm.head-comp.LRcomp}:

\begin{corollary}
\label{cor.LRcomp.back}Let $\operatorname{st}$ be a LR-shuffle-compatible
permutation statistic. Then, $\operatorname{st}$ is shuffle-compatible,
left-shuffle-compatible, right-shuffle-compatible and head-graft-compatible.
\end{corollary}

\begin{corollary}
\label{cor.LRcomp.two}Let $\operatorname{st}$ be a permutation statistic that
is left-shuffle-compatible and right-shuffle-compatible. Then,
$\operatorname{st}$ is LR-shuffle-compatible.
\end{corollary}

\section{\label{sect.Descent}Descent statistics and quasisymmetric functions}

In this section, we shall recall the concepts of descent statistics and their
shuffle algebras (introduced in \cite{part1}), and apply them to
$\operatorname{Epk}$.

\subsection{Compositions}

\begin{definition}
A \textit{composition} is a finite list of positive integers. If $I=\left(
i_{1},i_{2},\ldots,i_{n}\right)  $ is a composition, then the nonnegative
integer $i_{1}+i_{2}+\cdots+i_{n}$ is called the \textit{size} of $I$ and is
denoted by $\left\vert I\right\vert $; we furthermore say that $I$ is a
\textit{composition of }$\left\vert I\right\vert $.
\end{definition}

\begin{definition}
\label{def.comps-to-sets}Let $n\in\mathbb{N}$. For each composition $I=\left(
i_{1},i_{2},\ldots,i_{k}\right)  $ of $n$, we define a subset
$\operatorname{Des}I$ of $\left[  n-1\right]  $ by%
\begin{align*}
\operatorname{Des}I  &  =\left\{  i_{1},i_{1}+i_{2},i_{1}+i_{2}+i_{3}%
,\ldots,i_{1}+i_{2}+\cdots+i_{k-1}\right\} \\
&  =\left\{  i_{1}+i_{2}+\cdots+i_{s}\ \mid\ s\in\left[  k-1\right]  \right\}
.
\end{align*}


On the other hand, for each subset $A=\left\{  a_{1}<a_{2}<\cdots
<a_{k}\right\}  $ of $\left[  n-1\right]  $, we define a composition
$\operatorname*{Comp}A$ of $n$ by%
\[
\operatorname*{Comp}A=\left(  a_{1},a_{2}-a_{1},a_{3}-a_{2},\ldots
,a_{k}-a_{k-1},n-a_{k}\right)  .
\]
(The definition of $\operatorname*{Comp}A$ should be understood to give
$\operatorname*{Comp}A=\left(  n\right)  $ if $A=\varnothing$ and $n > 0$,
and to give
$\operatorname*{Comp}A=\left( \right)  $ if $A=\varnothing$ and $n = 0$.
Note that
$\operatorname*{Comp}A$ depends not only on the set $A$ itself, but also on
$n$. We hope that $n$ will always be clear from the context when we use this notation.)

We thus have defined a map $\operatorname{Des}$ (from the set of all
compositions of $n$ to the set of all subsets of $\left[  n-1\right]  $) and a
map $\operatorname*{Comp}$ (in the opposite direction). These two maps are
mutually inverse bijections.
\end{definition}

\begin{definition}
\label{def.CompDes}Let $n\in\mathbb{N}$. Let $\pi=\left(  \pi_{1},\pi
_{2},\ldots,\pi_{n}\right)  $ be an $n$-permutation. The \textit{descent
composition} of $\pi$ is defined to be the composition $\operatorname*{Comp}%
\left(  \operatorname{Des}\pi\right)  $ of $n$. This composition is denoted
by $\operatorname*{Comp}\pi$.
\end{definition}

For example, the $6$-permutation $\pi=\left(  4,1,3,9,6,8\right)  $ has
$\operatorname*{Comp}\pi=\left(  1,3,2\right)  $. For another example, the
$6$-permutation $\pi=\left(  1,4,3,2,9,8\right)  $ has $\operatorname*{Comp}%
\pi=\left(  2,1,2,1\right)  $.

Definition \ref{def.CompDes} defines the permutation statistic
$\operatorname*{Comp}$, whose codomain is the set of all compositions.

\subsection{Descent statistics}

\begin{definition}
Let $\operatorname{st}$ be a permutation statistic. We say that
$\operatorname{st}$ is a \textit{descent statistic} if and only if
$\operatorname{st}\pi$ (for $\pi$ a permutation) depends only on the descent
composition $\operatorname*{Comp}\pi$ of $\pi$. In other words,
$\operatorname{st}$ is a descent statistic if and only if every two
permutations $\pi$ and $\sigma$ satisfying $\operatorname*{Comp}%
\pi=\operatorname*{Comp}\sigma$ satisfy $\operatorname{st}\pi
=\operatorname{st}\sigma$.
\end{definition}

Equivalently, a permutation statistic $\operatorname{st}$ is a descent
statistic if and only if every two permutations $\pi$ and $\sigma$ satisfying
$\left\vert \pi\right\vert =\left\vert \sigma\right\vert $ and
$\operatorname{Des}\pi=\operatorname{Des}\sigma$ satisfy $\operatorname{st}
\pi=\operatorname{st}\sigma$. (This is indeed equivalent, because for two
permutations $\pi$ and $\sigma$, the condition \newline $\left(  \left\vert
\pi\right\vert =\left\vert \sigma\right\vert \text{ and }\operatorname{Des}
\pi=\operatorname{Des}\sigma\right)  $ is equivalent to $\left(
\operatorname*{Comp}\pi=\operatorname*{Comp}\sigma\right)  $.)

For example, the permutation statistic $\operatorname{Des}$ is a descent
statistic, because each permutation $\pi$ satisfies $\operatorname{Des}
\pi=\operatorname{Des}\left(  \operatorname*{Comp}\pi\right)  $. Also,
$\operatorname*{Pk}$ is a descent statistic, since each permutation $\pi$
satisfies%
\[
\operatorname*{Pk}\pi=\left(  \operatorname{Des}\pi\right)  \setminus\left(
\left\{  1\right\}  \cup\left(  \operatorname{Des}\pi+1\right)  \right)  ,
\]
where $\operatorname{Des}\pi+1$ denotes the set $\left\{  i+1\ \mid
\ i\in\operatorname{Des}\pi\right\}  $ (and, as we have just said,
$\operatorname{Des}\pi$ can be recovered from $\operatorname*{Comp}\pi$).
Furthermore, $\operatorname{Epk}$ is a descent statistic, since each
$n$-permutation $\pi$ (for a positive integer $n$) satisfies%
\[
\operatorname{Epk}\pi=\left(  \operatorname{Des}\pi\cup\left\{  n\right\}
\right)  \setminus\left(  \operatorname{Des}\pi+1\right)
\]
(and both $\operatorname{Des}\pi$ and $n$ can be recovered from
$\operatorname*{Comp}\pi$). The permutation statistics $\operatorname*{Lpk}$
and $\operatorname*{Rpk}$ (and, of course, $\operatorname*{Comp}$) are descent
statistics as well, as one can easily check.

In \cite[Corollary 1.6]{Oguz18}, Ezgi Kantarc{\i} O\u{g}uz has demonstrated that
not every shuffle-compatible permutation statistic is a descent statistic.
However, this changes if we require LR-shuffle-compatibility, because of
Corollary \ref{cor.LRcomp.back} and of the following fact:

\begin{proposition}
\label{prop.des-stat.hgc}Every head-graft-compatible permutation statistic is
a descent statistic.
\end{proposition}

\begin{proof}
[Proof of Proposition \ref{prop.des-stat.hgc}.]See \cite{verlong}.
\end{proof}

\begin{definition}
\label{def.des-stat.stI}Let $\operatorname{st}$ be a descent statistic. Then,
we can regard $\operatorname{st}$ as a map from the set of all compositions
(rather than from the set of all permutations). Namely, for any composition
$I$, we define $\operatorname{st}I$ (an element of the codomain of
$\operatorname{st}$) by setting%
\[
\operatorname{st}I=\operatorname{st}\pi\ \ \ \ \ \ \ \ \ \ \text{for any
permutation }\pi\text{ satisfying }\operatorname*{Comp}\pi=I.
\]
This is well-defined (because for every composition $I$, there exists at least
one permutation $\pi$ satisfying $\operatorname*{Comp}\pi=I$, and all such
permutations $\pi$ have the same value of $\operatorname{st}\pi$). In the
following, we shall regard every descent statistic $\operatorname{st}$
simultaneously as a map from the set of all permutations and as a map from the
set of all compositions.
\end{definition}

Note that this definition leads to a new interpretation of
$\operatorname{Des}I$ for a composition $I$: It is now defined as
$\operatorname{Des}\pi$ for any permutation $\pi$ satisfying
$\operatorname*{Comp}\pi=I$. This could clash with the old meaning of
$\operatorname{Des}I$ introduced in Definition \ref{def.comps-to-sets}.
Fortunately, these two meanings of $\operatorname{Des}I$ are exactly the
same, so there is no conflict of notation.

However, Definition \ref{def.des-stat.stI} causes an ambiguity for expressions
like \textquotedblleft$\operatorname{Des}\left(  i_{1},i_{2},\ldots
,i_{n}\right)  $\textquotedblright: Here, the \textquotedblleft$\left(
i_{1},i_{2},\ldots,i_{n}\right)  $\textquotedblright\ might be understood
either as a permutation, or as a composition, and the resulting descent sets
$\operatorname{Des}\left(  i_{1},i_{2},\ldots,i_{n}\right)  $ are not the
same. A similar ambiguity occurs for any descent statistic
$\operatorname{st}$ instead of $\operatorname{Des}$.
We hope that this ambiguity will not arise
in this paper due to our explicit typecasting of permutations and
compositions; but the reader should be warned that it can arise if one takes
the notation too literally.

\begin{definition}
\label{def.des-stat.eq-comp}Let $\operatorname{st}$ be a descent statistic.

\textbf{(a)} Two compositions $J$ and $K$ are said to be
$\operatorname{st}$\textit{-equivalent} if and only if they
have the same size and satisfy
$\operatorname{st}J=\operatorname{st}K$. Equivalently, two compositions $J$
and $K$ are $\operatorname{st}$-equivalent if and only if there exist two
$\operatorname{st}$-equivalent permutations $\pi$ and $\sigma$ satisfying
$J=\operatorname*{Comp}\pi$ and $K=\operatorname*{Comp}\sigma$.

\textbf{(b)} The relation \textquotedblleft$\operatorname{st}$%
-equivalent\textquotedblright\ is an equivalence relation on compositions; its
equivalence classes are called $\operatorname{st}$\textit{-equivalence
classes of compositions}.
\end{definition}

\subsection{Quasisymmetric functions}

We now recall the definition of quasisymmetric functions; see \cite[Chapter
5]{HopfComb} (and various other modern textbooks) for more details about this:

\begin{definition}
\textbf{(a)}
Consider the ring of power series $\mathbb{Q}\left[  \left[  x_{1},x_{2}%
,x_{3},\ldots\right]  \right]  $ in infinitely many commuting indeterminates
over $\mathbb{Q}$.
A power series $f\in\mathbb{Q}\left[  \left[  x_{1}%
,x_{2},x_{3},\ldots\right]  \right]  $ is said to be \textit{quasisymmetric}
if it has the following property:
\begin{itemize}
\item For any positive integers $a_{1}%
,a_{2},\ldots,a_{k}$ and any two strictly increasing sequences $\left(
i_{1}<i_{2}<\cdots<i_{k}\right)  $ and $\left(  j_{1}<j_{2}<\cdots
<j_{k}\right)  $ of positive integers, the coefficient of $x_{i_{1}}^{a_{1}%
}x_{i_{2}}^{a_{2}}\cdots x_{i_{k}}^{a_{k}}$ in $f$ equals the coefficient of
$x_{j_{1}}^{a_{1}}x_{j_{2}}^{a_{2}}\cdots x_{j_{k}}^{a_{k}}$ in $f$.
\end{itemize}

\textbf{(b)}
A \textit{quasisymmetric function} is a quasisymmetric power series
$f\in\mathbb{Q}\left[  \left[  x_{1},x_{2},x_{3},\ldots\right]  \right]  $
that has bounded degree (i.e., there exists an $N\in\mathbb{N}$ such that each
monomial appearing in $f$ has degree $\leq N$).

\textbf{(c)} The quasisymmetric functions
form a $\mathbb{Q}$-subalgebra of $\mathbb{Q}\left[  \left[  x_{1},x_{2}%
,x_{3},\ldots\right]  \right]  $; this $\mathbb{Q}$-subalgebra is denoted by
$\operatorname*{QSym}$ and called the \textit{ring of quasisymmetric
functions} over $\mathbb{Q}$.
This $\mathbb{Q}$-algebra $\operatorname{QSym}$ is graded (in the obvious way,
i.e., by the degree of a monomial).
\end{definition}

The $\mathbb{Q}$-algebra $\operatorname*{QSym}$ has much interesting structure
(e.g., it is a Hopf algebra), some of which we will introduce later when we
need it. One simple yet crucial feature of $\operatorname*{QSym}$ that we will
immediately use is the \textit{fundamental basis} of $\operatorname*{QSym}$:

\begin{definition}
For any composition $\alpha$, we define the \textit{fundamental quasisymmetric
function} $F_{\alpha}$ to be the power series%
\[
\sum_{\substack{i_{1}\leq i_{2}\leq\cdots\leq i_{n};\\i_{j}<i_{j+1}\text{ for
each }j\in\operatorname{Des}\alpha}}x_{i_{1}}x_{i_{2}}\cdots x_{i_{n}}%
\in\operatorname*{QSym},
\]
where $n=\left\vert \alpha\right\vert $ is the size of $\alpha$. The family
$\left(  F_{\alpha}\right)  _{\alpha\text{ is a composition}}$ is a basis of
the $\mathbb{Q}$-vector space $\operatorname*{QSym}$; it is known as the
\textit{fundamental basis} of $\operatorname*{QSym}$.
\end{definition}

The fundamental quasisymmetric function $F_{\alpha}$ is denoted
by $L_{\alpha}$ in \cite[\S 5.2]{HopfComb}.

The multiplication of fundamental quasisymmetric functions is intimately
related to shuffles of permutations:

\begin{proposition}
\label{prop.4.1.rewr}Let $\pi$ and $\sigma$ be two disjoint permutations.
Then,%
\[
F_{\operatorname*{Comp}\pi}F_{\operatorname*{Comp}\sigma}=\sum_{\chi\in
S\left(  \pi,\sigma\right)  }F_{\operatorname*{Comp}\chi}.
\]

\end{proposition}

Proposition \ref{prop.4.1.rewr} is a restatement of \cite[Theorem 4.1]{part1},
and is proven in \cite[(5.2.6)]{HopfComb} (which makes the additional
requirement that the letters of $\pi$ are $1,2,\ldots,\left\vert
\pi\right\vert $ and the letters of $\sigma$ are $\left\vert \pi\right\vert
+1,\left\vert \pi\right\vert +2,\ldots,\left\vert \pi\right\vert +\left\vert
\sigma\right\vert $; but this requirement is not used in the proof and thus
can be dropped).

\subsection{Shuffle algebras}

Any shuffle-compatible permutation statistic $\operatorname{st}$ gives rise
to a \textit{shuffle algebra} $\mathcal{A}_{\operatorname{st}}$, defined as follows:

\begin{definition}
\label{def.Ast}Let $\operatorname{st}$ be a shuffle-compatible permutation
statistic. For each permutation $\pi$, let $\left[  \pi\right]
_{\operatorname{st}}$ denote the $\operatorname{st}$-equivalence class of
$\pi$.

Let $\mathcal{A}_{\operatorname{st}}$ be the free $\mathbb{Q}$-vector space
whose basis is the set of all $\operatorname{st}$-equivalence classes of
permutations. We define a multiplication on $\mathcal{A}_{\operatorname{st}}$
by setting%
\[
\left[  \pi\right]  _{\operatorname{st}}\left[  \sigma\right]
_{\operatorname{st}}=\sum_{\tau\in S\left(  \pi,\sigma\right)  }\left[
\tau\right]  _{\operatorname{st}}%
\]
for any two disjoint permutations $\pi$ and $\sigma$. It is easy to see that
this multiplication is well-defined and associative, and turns $\mathcal{A}%
_{\operatorname{st}}$ into a $\mathbb{Q}$-algebra whose unity is the
$\operatorname{st}$-equivalence class of the $0$-permutation
$\left( \right)$.
This
$\mathbb{Q}$-algebra is denoted by $\mathcal{A}_{\operatorname{st}}$, and is
called the \textit{shuffle algebra} of $\operatorname{st}$. It is a graded
$\mathbb{Q}$-algebra; its $n$-th graded component (for each $n\in\mathbb{N}$)
is spanned by the $\operatorname{st}$-equivalence classes of all $n$-permutations.
\end{definition}

This definition originates in \cite[\S 3.1]{part1}.
The following fact is implicit in \cite{part1}:

\begin{proposition}
\label{prop.Ast.alg}Let $\operatorname{st}$ be a shuffle-compatible descent statistic.

There is a surjective $\mathbb{Q}$-algebra homomorphism
$p_{\operatorname{st}}:\operatorname*{QSym}\rightarrow\mathcal{A}%
_{\operatorname{st}}$ that satisfies
\[
p_{\operatorname{st}}\left(  F_{\operatorname*{Comp}\pi}\right)  =\left[
\pi\right]  _{\operatorname{st}}\ \ \ \ \ \ \ \ \ \ \text{for every
permutation }\pi.
\]

\end{proposition}

A central result, connecting shuffle-compatibility of a descent
statistic with $\operatorname*{QSym}$, is \cite[Theorem 4.3]{part1},
which we restate as follows:

\begin{theorem}
\label{thm.4.3}Let $\operatorname{st}$ be a descent statistic.

\textbf{(a)} The descent statistic $\operatorname{st}$ is shuffle-compatible
if and only if there exist a $\mathbb{Q}$-algebra $A$ with basis $\left(
u_{\alpha}\right)  $ (indexed by $\operatorname{st}$-equivalence classes
$\alpha$ of compositions) and a $\mathbb{Q}$-algebra homomorphism
$\phi_{\operatorname{st}}:\operatorname*{QSym}\rightarrow A$ with the
property that whenever $\alpha$ is an $\operatorname{st}$-equivalence class
of compositions, we have%
\[
\phi_{\operatorname{st}}\left(  F_{L}\right)  =u_{\alpha}%
\ \ \ \ \ \ \ \ \ \ \text{for each }L\in\alpha.
\]


\textbf{(b)} In this case, the $\mathbb{Q}$-linear map%
\[
\mathcal{A}_{\operatorname{st}}\rightarrow A,\ \ \ \ \ \ \ \ \ \ \left[
\pi\right]  _{\operatorname{st}}\mapsto u_{\alpha},
\]
where $\alpha$ is the $\operatorname{st}$-equivalence class of the
composition $\operatorname*{Comp}\pi$, is a $\mathbb{Q}$-algebra isomorphism
$\mathcal{A}_{\operatorname{st}}\rightarrow A$.
\end{theorem}

Proofs of Proposition \ref{prop.Ast.alg} and Theorem \ref{thm.4.3}
(independent of \cite{part1}) can be found in \cite{verlong}.

\subsection{The shuffle algebra of $\operatorname{Epk}$}

Theorem \ref{thm.Epk.sh-co-a} yields that the permutation statistic
$\operatorname{Epk}$ is shuffle-compatible. Hence, the shuffle algebra
$\mathcal{A}_{\operatorname{Epk}}$ is well-defined. We have little to say
about it:

\begin{theorem}
\label{thm.Epk.AEpk}\textbf{(a)} The shuffle algebra $\mathcal{A}%
_{\operatorname{Epk}}$ is a graded quotient algebra of $\operatorname*{QSym}$.

\textbf{(b)} Define the Fibonacci sequence
$\left(  f_{0},f_{1},f_{2},\ldots\right)  $ as in
Proposition~\ref{prop.lac.fib}.
Let $n$ be a positive integer. The $n$-th graded component of
$\mathcal{A}_{\operatorname{Epk}}$ has dimension $f_{n+2}-1$.
\end{theorem}

\begin{proof}
[Proof of Theorem \ref{thm.Epk.AEpk}.]See \cite{verlong}.
\end{proof}

We can describe $\mathcal{A}_{\operatorname{Epk}}$ using the notations
of Section \ref{sect.Zenri}:

\begin{definition}
Let $\Pi_{\mathcal{Z}}$ be the $\mathbb{Q}$-vector subspace of
$\operatorname*{Pow}\mathcal{N}$ spanned by the family $\left(  K_{n,\Lambda
}^{\mathcal{Z}}\right)  _{n \in \NN;\  \Lambda \in \mathbf{L}_n}  $.
Then, $\Pi_{\mathcal{Z}}$ is also the $\mathbb{Q}$-vector subspace of
$\operatorname*{Pow}\mathcal{N}$ spanned by the family $\left(
K_{n,\operatorname{Epk}\pi}^{\mathcal{Z}}\right)  _{n\in\mathbb{N}%
;\ \pi\text{ is an }n\text{-permutation}}$ (by Proposition \ref{prop.when-Epk}%
). In other words, $\Pi_{\mathcal{Z}}$ is also the $\mathbb{Q}$-vector
subspace of $\operatorname*{Pow}\mathcal{N}$ spanned by the family $\left(
\Gamma_{\mathcal{Z}}\left(  \pi\right)  \right)  _{n\in\mathbb{N};\ \pi\text{
is an }n\text{-permutation}}$ (because of (\ref{eq.def.KnL.2})). Hence,
Corollary \ref{cor.prod2} shows that $\Pi_{\mathcal{Z}}$ is closed under
multiplication. Since furthermore $\Gamma_{\mathcal{Z}}\left(
\left( \right) \right)  =1$ (for the $0$-permutation
$\left( \right)$), we can thus conclude that $\Pi
_{\mathcal{Z}}$ is a $\mathbb{Q}$-subalgebra of $\operatorname*{Pow}%
\mathcal{N}$.
\end{definition}

\begin{theorem}
\label{thm.Epk.sh-co}The $\mathbb{Q}$-linear map
\[
\mathcal{A}_{\operatorname{Epk}} \to \Pi_{\mathcal{Z}}, \qquad
\left[  \pi\right]  _{\operatorname{Epk}}\mapsto
K_{n,\operatorname{Epk} \pi}^{\mathcal{Z}}
\]
is a $\mathbb{Q}$-algebra isomorphism.
\end{theorem}

\begin{proof}
[Proof of Theorem \ref{thm.Epk.sh-co}.]See \cite{verlong}.
\end{proof}

\section{\label{sect.kernel}The kernel of the map $\operatorname*{QSym}%
\rightarrow\mathcal{A}_{\operatorname{Epk}}$}

\subsection{The kernel of a descent statistic}

Now, we shall focus on a feature of shuffle-compatible descent statistics that
seems to have been overlooked so far: their kernels.

All proofs in this section are omitted; they can be found in \cite{verlong}.

\begin{definition}
Let $\operatorname{st}$ be a descent statistic. Then, $\mathcal{K}%
_{\operatorname{st}}$ shall mean the $\mathbb{Q}$-vector subspace of
$\operatorname*{QSym}$ spanned by all elements of the form $F_{J}-F_{K}$,
where $J$ and $K$ are two $\operatorname{st}$-equivalent compositions. (See
Definition \ref{def.des-stat.eq-comp} \textbf{(a)} for the definition of
\textquotedblleft$\operatorname{st}$-equivalent
compositions\textquotedblright.) We shall refer to $\mathcal{K}%
_{\operatorname{st}}$ as the \textit{kernel} of $\operatorname{st}$.
\end{definition}

The following basic linear-algebraic lemma will be useful:

\begin{lemma}
\label{lem.K.fist}Let $\operatorname{st}$ be a descent statistic. Let $A$ be
a $\mathbb{Q}$-vector space with basis $\left(  u_{\alpha}\right)  $ indexed
by $\operatorname{st}$-equivalence classes $\alpha$ of compositions. Let
$\phi_{\operatorname{st}}:\operatorname*{QSym}\rightarrow A$ be a
$\mathbb{Q}$-linear map with the property that whenever $\alpha$ is an
$\operatorname{st}$-equivalence class of compositions, we have%
\begin{equation}
\phi_{\operatorname{st}}\left(  F_{L}\right)  =u_{\alpha}%
\ \ \ \ \ \ \ \ \ \ \text{for each }L\in\alpha. \label{pf.prop.K.ideal.dir1.1}%
\end{equation}


Then, $\operatorname*{Ker}\left(  \phi_{\operatorname{st}}\right)
=\mathcal{K}_{\operatorname{st}}$.
\end{lemma}

Theorem~\ref{thm.4.3} easily yields the following fact:

\begin{proposition}
\label{prop.K.ideal}Let $\operatorname{st}$ be a descent statistic. Then,
$\operatorname{st}$ is shuffle-compatible if and only if $\mathcal{K}%
_{\operatorname{st}}$ is an ideal of $\operatorname*{QSym}$. Furthermore, in
this case, $\mathcal{A}_{\operatorname{st}}\cong\operatorname*{QSym}%
/\mathcal{K}_{\operatorname{st}}$ as $\mathbb{Q}$-algebras.
\end{proposition}

\begin{corollary}
\label{cor.Epk.ideal}The kernel $\mathcal{K}_{\operatorname{Epk}}$ of the
descent statistic $\operatorname{Epk}$ is an ideal of $\operatorname*{QSym}$.
\end{corollary}

We can study the kernel of any descent statistic; in particular, the case of
shuffle-compatible descent statistics appears interesting. Since
$\operatorname*{QSym}$ is isomorphic to a polynomial ring (as an algebra), it
has many ideals, which are rather hopeless to classify or tame; but the ones
obtained as kernels of shuffle-compatible descent statistics might be worth discussing.

\subsection{An F-generating set of $\mathcal{K}_{\operatorname{Epk}}$}

Let us now focus on $\mathcal{K}_{\operatorname{Epk}}$, the kernel of
$\operatorname{Epk}$.

\begin{proposition}
\label{prop.K.Epk.F}If $J=\left(  j_{1},j_{2},\ldots,j_{m}\right)  $ and $K$
are two compositions, then we shall write $J\rightarrow K$ if there exists an
$\ell\in\left\{  2,3,\ldots,m\right\}  $ such that $j_{\ell}>2$ and
\[
K=\left(
j_{1},j_{2},\ldots,j_{\ell-1},1,j_{\ell}-1,j_{\ell+1},j_{\ell+2},\ldots
,j_{m}\right) .
\]
(In other words, we write $J\rightarrow K$ if $K$ can be
obtained from $J$ by \textquotedblleft splitting\textquotedblright\ some entry
$j_{\ell}>2$ into two consecutive entries\footnotemark\ $1$ and $j_{\ell}-1$,
provided that this entry was not the first entry -- i.e., we had $\ell>1$ --
and that this entry was greater than $2$.)

The ideal $\mathcal{K}_{\operatorname{Epk}}$ of $\operatorname*{QSym}$ is
spanned (as a $\mathbb{Q}$-vector space) by all differences of the form
$F_{J}-F_{K}$, where $J$ and $K$ are two compositions satisfying $J\rightarrow
K$.
\end{proposition}

\footnotetext{The word \textquotedblleft consecutive\textquotedblright\ here
means \textquotedblleft in consecutive positions of $J$\textquotedblright, not
\textquotedblleft consecutive integers\textquotedblright. So two consecutive
entries of $J$ are two entries of the form $j_{p}$ and $j_{p+1}$ for some
$p\in\left\{  1,2,\ldots,m-1\right\}  $.}

\begin{example}
We have $\left(  2,1,4,4\right)  \rightarrow\left(  2,1,1,3,4\right)  $, since
the composition $\left(  2,1,1,3,4\right)  $ is obtained from $\left(
2,1,4,4\right)  $ by splitting the third entry (which is $4>2$) into two
consecutive entries $1$ and $3$.

Similarly, $\left(  2,1,4,4\right)  \rightarrow\left(  2,1,4,1,3\right)  $.

But we do not have $\left(  3,1\right)  \rightarrow\left(  1,2,1\right)  $,
because splitting the first entry of the composition is not allowed in the
definition of the relation $\rightarrow$. Also, we do not have $\left(
1,2,1\right)  \rightarrow\left(  1,1,1,1\right)  $, because the entry we are
splitting must be $>2$.

Two compositions $J$ and $K$ satisfying $J\rightarrow K$ must necessarily
satisfy $\left\vert J\right\vert =\left\vert K\right\vert $.

Here are all relations $\rightarrow$ between compositions of size $4$:%
\[
\left(  1,3\right)  \rightarrow\left(  1,1,2\right)  .
\]


Here are all relations $\rightarrow$ between compositions of size $5$:%
\begin{align*}
\left(  1,4\right)   &  \rightarrow\left(  1,1,3\right)  ,\\
\left(  1,3,1\right)   &  \rightarrow\left(  1,1,2,1\right)  ,\\
\left(  1,1,3\right)   &  \rightarrow\left(  1,1,1,2\right)  ,\\
\left(  2,3\right)   &  \rightarrow\left(  2,1,2\right)  .
\end{align*}
There are no relations $\rightarrow$ between compositions of size $\leq3$.
\end{example}


\subsection{An M-generating set of $\mathcal{K}_{\operatorname{Epk}}$}

Another characterization of the ideal $\mathcal{K}_{\operatorname{Epk}}$ of
$\operatorname*{QSym}$ can be obtained using the monomial basis of
$\operatorname*{QSym}$. Let us first recall how said basis is defined:

For any composition $\alpha=\left(  \alpha_{1},\alpha_{2},\ldots,\alpha_{\ell
}\right)  $, we let%
\[
M_{\alpha}=\sum_{i_{1}<i_{2}<\cdots<i_{\ell}}x_{i_{1}}^{\alpha_{1}}x_{i_{2}%
}^{\alpha_{2}}\cdots x_{i_{\ell}}^{\alpha_{\ell}}%
\]
(where the sum is over all strictly increasing $\ell$-tuples $\left(
i_{1},i_{2},\ldots,i_{\ell}\right)  $ of positive integers). This power series
$M_{\alpha}$ belongs to $\operatorname*{QSym}$. The family $\left(  M_{\alpha
}\right)  _{\alpha\text{ is a composition}}$ is a basis of the $\mathbb{Q}%
$-vector space $\operatorname*{QSym}$; it is called the \textit{monomial
basis} of $\operatorname*{QSym}$.

\begin{proposition}
\label{prop.K.Epk.M}If $J=\left(  j_{1},j_{2},\ldots,j_{m}\right)  $ and $K$
are two compositions, then we shall write $J\underset{M}{\rightarrow}K$ if
there exists an $\ell\in\left\{  2,3,\ldots,m\right\}  $ such that $j_{\ell
}>2$ and
\[
K=\left(  j_{1},j_{2},\ldots,j_{\ell-1},2,j_{\ell}-2,j_{\ell
+1},j_{\ell+2},\ldots,j_{m}\right) .
\]
(In other words, we write
$J\underset{M}{\rightarrow}K$ if $K$ can be obtained from $J$ by
\textquotedblleft splitting\textquotedblright\ some entry $j_{\ell}>2$ into
two consecutive entries $2$ and $j_{\ell}-2$, provided that this entry was not
the first entry -- i.e., we had $\ell>1$ -- and that this entry was greater
than $2$.)

The ideal $\mathcal{K}_{\operatorname{Epk}}$ of $\operatorname*{QSym}$ is
spanned (as a $\mathbb{Q}$-vector space) by all sums of the form $M_{J}+M_{K}%
$, where $J$ and $K$ are two compositions satisfying
$J\underset{M}{\rightarrow}K$.
\end{proposition}

\begin{example}
We have $\left(  2,1,4,4\right)  \underset{M}{\rightarrow}\left(
2,1,2,2,4\right)  $, since the composition $\left(  2,1,2,2,4\right)  $ is
obtained from $\left(  2,1,4,4\right)  $ by splitting the third entry (which
is $4>2$) into two consecutive entries $2$ and $2$.

Similarly, $\left(  2,1,4,4\right)  \underset{M}{\rightarrow}\left(
2,1,4,2,2\right)  $ and $\left(  2,1,5,4\right)  \underset{M}{\rightarrow
}\left(  2,1,2,3,4\right)  $.

But we do not have $\left(  3,1\right)  \underset{M}{\rightarrow}\left(
2,1,1\right)  $, because splitting the first entry of the composition is not
allowed in the definition of the relation $\underset{M}{\rightarrow}$.

Two compositions $J$ and $K$ satisfying $J\underset{M}{\rightarrow}K$ must
necessarily satisfy $\left\vert J\right\vert =\left\vert K\right\vert $.

Here are all relations $\underset{M}{\rightarrow}$ between compositions of
size $4$:%
\[
\left(  1,3\right)  \underset{M}{\rightarrow}\left(  1,2,1\right)  .
\]


Here are all relations $\underset{M}{\rightarrow}$ between compositions of
size $5$:%
\begin{align*}
\left(  1,4\right)   &  \underset{M}{\rightarrow}\left(  1,2,2\right)  ,\\
\left(  1,3,1\right)   &  \underset{M}{\rightarrow}\left(  1,2,1,1\right)  ,\\
\left(  1,1,3\right)   &  \underset{M}{\rightarrow}\left(  1,1,2,1\right)  ,\\
\left(  2,3\right)   &  \underset{M}{\rightarrow}\left(  2,2,1\right)  .
\end{align*}
There are no relations $\underset{M}{\rightarrow}$ between compositions of
size $\leq3$.
\end{example}

\begin{question}
It is worth analyzing the kernels of other known descent statistics
(shuffle-compatible or not).
Let us say that a descent statistic $\operatorname{st}$ is
\textit{M-binomial}
if its kernel $\mathcal{K}_{\operatorname{st}}$ can be spanned by
elements of the form $\lambda M_{J}+\mu M_{K}$ with
$\lambda, \mu \in \mathbb{Q}$ and compositions $J,K$.
Then, Proposition \ref{prop.K.Epk.M} yields that $\operatorname{Epk}$
is M-binomial.
It is easy to see that the statistics $\operatorname{Des}$ and
$\operatorname{des}$ are M-binomial as well.
Computations using SageMath suggest that the statistics
$\operatorname{Lpk}$, $\operatorname{Rpk}$, $\operatorname{Pk}$,
$\operatorname{Val}$,
$\operatorname{pk}$, $\operatorname{lpk}$, $\operatorname{rpk}$
and $\operatorname{val}$
(see \cite{part1} for some of their definitions) are M-binomial, too
(at least for compositions of size $\leq 9$);
this would be nice to prove.
On the other hand, the statistics
$\operatorname{maj}$, $\left(\operatorname{des},\operatorname{maj}\right)$
and $\left(\operatorname{val},\operatorname{des}\right)$ (again, see
\cite{part1} for definitions) are not M-binomial.
\end{question}

\section{\label{sect.dendri}Dendriform structures}

Next, we shall recall the \textit{dendriform operations }$\left.
\prec\right.  $ and $\left.  \succeq\right.  $ on $\operatorname*{QSym}$
studied in \cite{dimcr}, and we shall connect these operations back to
LR-shuffle-compatibility. Since we consider this somewhat tangential to the
present paper, we merely summarize the main results here; more can be found in
\cite{verlong}.

\subsection{Two operations on $\operatorname*{QSym}$}

We begin with some definitions. We will use some notations from \cite{dimcr},
but we set $\mathbf{k}=\mathbb{Q}$ because we are working over the ring
$\mathbb{Q}$ in this paper. Monomials always mean formal expressions of the
form $x_{1}^{a_{1}}x_{2}^{a_{2}}x_{3}^{a_{3}}\cdots$ with $a_{1}+a_{2}%
+a_{3}+\cdots<\infty$ (see \cite[Section 2]{dimcr} for details). If
$\mathfrak{m}$ is a monomial, then $\operatorname*{Supp}\mathfrak{m}$ will
denote the finite subset
\[
\left\{  i\in\left\{  1,2,3,\ldots\right\}  \ \mid\ \text{the exponent with
which }x_{i}\text{ occurs in }\mathfrak{m}\text{ is }>0\right\}
\]
of $\left\{  1,2,3,\ldots\right\}  $. Next, we define two binary operations
\begin{align*}
&  \left.  \prec\right.  \ \left(  \text{called \textquotedblleft dendriform
less-than\textquotedblright; but it's an operation, not a relation}\right)
,\\
&  \left.  \succeq\right.  \ \left(  \text{called \textquotedblleft dendriform
greater-or-equal\textquotedblright; but it's an operation, not a
relation}\right)  ,
\end{align*}
on the ring $\mathbf{k}\left[  \left[  x_{1},x_{2},x_{3},\ldots\right]
\right]  $ of power series by first defining how they act on monomials:%
\begin{align*}
\mathfrak{m}\left.  \prec\right.  \mathfrak{n}  &  =\left\{
\begin{array}
[c]{l}%
\mathfrak{m}\cdot\mathfrak{n},\ \ \ \ \ \ \ \ \ \ \text{if }\min\left(
\operatorname*{Supp}\mathfrak{m}\right)  <\min\left(  \operatorname*{Supp}%
\mathfrak{n}\right)  ;\\
0,\ \ \ \ \ \ \ \ \ \ \text{if }\min\left(  \operatorname*{Supp}%
\mathfrak{m}\right)  \geq\min\left(  \operatorname*{Supp}\mathfrak{n}\right)
\end{array}
\right.  ;\\
\mathfrak{m}\left.  \succeq\right.  \mathfrak{n}  &  =\left\{
\begin{array}
[c]{l}%
\mathfrak{m}\cdot\mathfrak{n},\ \ \ \ \ \ \ \ \ \ \text{if }\min\left(
\operatorname*{Supp}\mathfrak{m}\right)  \geq\min\left(  \operatorname*{Supp}%
\mathfrak{n}\right)  ;\\
0,\ \ \ \ \ \ \ \ \ \ \text{if }\min\left(  \operatorname*{Supp}%
\mathfrak{m}\right)  <\min\left(  \operatorname*{Supp}\mathfrak{n}\right)
\end{array}
\right.  ;
\end{align*}
and then requiring that they all be $\mathbf{k}$-bilinear and continuous (so
their action on pairs of arbitrary power series can be computed by
\textquotedblleft opening the parentheses\textquotedblright). These operations
$\left.  \prec\right.  $ and $\left.  \succeq\right.  $ restrict to the subset
$\operatorname*{QSym}$ of $\mathbf{k}\left[  \left[  x_{1},x_{2},x_{3}%
,\ldots\right]  \right]  $ (this is proven in \cite[detailed version, Section
3]{dimcr}). They furthermore satisfy the following relations (which are easy
to verify):

\begin{itemize}
\item For all $a,b,c\in\mathbf{k}\left[  \left[  x_{1},x_{2},x_{3},\ldots\right]
\right]  $, we have
\begin{align*}
a\left.  \prec\right.  b+a\left.  \succeq\right.  b  &  =ab;\\
\left(  a\left.  \prec\right.  b\right)  \left.  \prec\right.  c  &  =a\left.
\prec\right.  \left(  bc\right)  ;\ \ \ \ \ \ \ \ \ \ \left(  a\left.
\succeq\right.  b\right)  \left.  \prec\right.  c=a\left.  \succeq\right.
\left(  b\left.  \prec\right.  c\right)  ;\\
a\left.  \succeq\right.  \left(  b\left.  \succeq\right.  c\right)   &
=\left(  ab\right)  \left.  \succeq\right.  c .
\end{align*}

\item For any $a\in\mathbf{k}\left[  \left[  x_{1},x_{2},x_{3},\ldots\right]
\right]  $, we have%
\begin{align*}
1\left.  \prec\right.  a  &  =0;\ \ \ \ \ \ \ \ \ \ a\left.  \prec\right.
1=a-\varepsilon\left(  a\right)  ; \ \ \ \ \ \ \ \ \ \ % \\
1\left.  \succeq\right.  a   =a;\ \ \ \ \ \ \ \ \ \ a\left.  \succeq\right.
1=\varepsilon\left(  a\right)  ,
\end{align*}
where $\varepsilon\left(  a\right)  $ denotes the constant term of the power
series $a$.
\end{itemize}

The operations $\left.  \prec\right.  $ and $\left.  \succeq\right.  $ are
sometimes called \textquotedblleft restricted products\textquotedblright\ due
to their similarity with the (regular) multiplication of $\operatorname*{QSym}%
$. In particular, they satisfy the following analogue of Proposition
\ref{prop.4.1.rewr}:

\begin{proposition}
Let $\pi$ and $\sigma$ be two disjoint nonempty permutations. Assume that
$\pi_{1}>\sigma_{1}$. Then,%
\begin{align*}
F_{\operatorname*{Comp}\pi}\left.  \prec\right.  F_{\operatorname*{Comp}%
\sigma}
&=\sum_{\chi\in S_{\prec}\left(  \pi,\sigma\right)  }%
F_{\operatorname*{Comp}\chi}  \qquad \text{ and } \\
F_{\operatorname*{Comp}\pi}\left.  \succeq\right.  F_{\operatorname*{Comp}%
\sigma}
&=\sum_{\chi\in S_{\succ}\left(  \pi,\sigma\right)  }%
F_{\operatorname*{Comp}\chi}.
\end{align*}

\end{proposition}

\subsection{Left- and right-shuffle-compatibility and ideals}

This proposition lets us relate the notions introduced in Definition
\ref{def.LR.left-right} to the operations $\left.  \prec\right.  $ and
$\left.  \succeq\right.  $. To state the precise connection, we need the
following notation:

\begin{definition}
Let $A$ be a $\mathbf{k}$-module equipped with some binary operation $\ast$
(written infix).

\textbf{(a)} If $B$ and $C$ are two $\mathbf{k}$-submodules of $A$, then
$B\ast C$ shall mean the $\mathbf{k}$-submodule of $A$ spanned by all elements
of the form $b\ast c$ with $b\in B$ and $c\in C$.

\textbf{(b)} A $\mathbf{k}$-submodule $M$ of $A$ is said to be a\textit{
}$\ast$\textit{-ideal} if and only if it satisfies $A\ast M\subseteq M$ and
$M\ast A\subseteq M$.
\end{definition}

Now, let us define two further variants of LR-shuffle-compatibility (to be
compared with those introduced in Definition \ref{def.LR.left-right}):

\begin{definition}
Let $\operatorname{st}$ be a permutation statistic.

\textbf{(a)} We say that $\operatorname{st}$ is \textit{weakly
left-shuffle-compatible} if for any two disjoint nonempty permutations $\pi$
and $\sigma$ having the property that%
\[
\text{each entry of }\pi\text{ is greater than each entry of }\sigma,
\]
the multiset $\left\{  \operatorname{st} \tau   \ \mid\ \tau\in
S_{\prec}\left(  \pi,\sigma\right)  \right\}  _{\operatorname*{multi}}$
depends only on $\operatorname{st} \pi$, $\operatorname{st} \sigma$,
$\left\vert \pi\right\vert $ and $\left\vert \sigma\right\vert $.

\textbf{(b)} We say that $\operatorname{st}$ is \textit{weakly
right-shuffle-compatible} if for any two disjoint nonempty permutations $\pi$
and $\sigma$ having the property that%
\[
\text{each entry of }\pi\text{ is greater than each entry of }\sigma,
\]
the multiset $\left\{  \operatorname{st} \tau   \ \mid\ \tau\in
S_{\succ}\left(  \pi,\sigma\right)  \right\}  _{\operatorname*{multi}}$
depends only on $\operatorname{st} \pi$, $\operatorname{st} \sigma$,
$\left\vert \pi\right\vert $ and $\left\vert \sigma\right\vert $.
\end{definition}

Then, the following analogues to the first part of Proposition
\ref{prop.K.ideal} hold:

\begin{theorem}
\label{thm.dendri.K.ideal}Let $\operatorname{st}$ be a descent statistic.
Then, the following three statements are equivalent:

\begin{itemize}
\item \textit{Statement A:} The statistic $\operatorname{st}$ is left-shuffle-compatible.

\item \textit{Statement B:} The statistic $\operatorname{st}$ is weakly left-shuffle-compatible.

\item \textit{Statement C:} The set $\mathcal{K}_{\operatorname{st}}$ is an
$\left.  \prec\right.  $-ideal of $\operatorname*{QSym}$.
\end{itemize}
\end{theorem}

\begin{theorem}
\label{thm.dendri.K.ideal-R}Let $\operatorname{st}$ be a descent statistic.
Then, the following three statements are equivalent:

\begin{itemize}
\item \textit{Statement A:} The statistic $\operatorname{st}$ is right-shuffle-compatible.

\item \textit{Statement B:} The statistic $\operatorname{st}$ is weakly right-shuffle-compatible.

\item \textit{Statement C:} The set $\mathcal{K}_{\operatorname{st}}$ is an
$\left.  \succeq\right.  $-ideal of $\operatorname*{QSym}$.
\end{itemize}
\end{theorem}

\begin{corollary}
\label{cor.dendri.K.ideal-LR}Let $\operatorname{st}$ be a permutation
statistic that is LR-shuffle-compatible. Then, $\operatorname{st}$ is a
shuffle-compatible descent statistic, and the set $\mathcal{K}%
_{\operatorname{st}}$ is an ideal and a $\left.  \prec\right.  $-ideal and a
$\left.  \succeq\right.  $-ideal of $\operatorname*{QSym}$.
\end{corollary}

\begin{corollary}
\label{cor.dendri.K.ideal-LRi}Let $\operatorname{st}$ be a descent statistic
such that $\mathcal{K}_{\operatorname{st}}$ is a $\left.  \prec\right.
$-ideal and a $\left.  \succeq\right.  $-ideal of $\operatorname*{QSym}$.
Then, $\operatorname{st}$ is LR-shuffle-compatible and shuffle-compatible.
\end{corollary}

Corollary \ref{cor.dendri.K.ideal-LR} can (for example) be applied to
$\operatorname{st}=\operatorname{Epk}$, which we know to be
LR-shuffle-compatible (from Theorem \ref{thm.LRcomp.Pks} \textbf{(c)}); the
result is that $\mathcal{K}_{\operatorname{Epk}}$ is an ideal and a $\left.
\prec\right.  $-ideal and a $\left.  \succeq\right.  $-ideal of
$\operatorname*{QSym}$. The same can be said about $\operatorname{Des}$ and
$\operatorname*{Lpk}$ and some other statistics.

Combining Theorem \ref{thm.dendri.K.ideal} with Theorem
\ref{thm.dendri.K.ideal-R}, we can also see that any descent statistic that is
weakly left-shuffle-compatible and weakly right-shuffle-compatible must
automatically be shuffle-compatible (because any $\left.  \prec\right.
$-ideal of $\operatorname*{QSym}$ that is also a $\left.  \succeq\right.
$-ideal of $\operatorname*{QSym}$ is an ideal of $\operatorname*{QSym}$ as
well). Note that this is only true for descent statistics! As far as arbitrary
permutation statistics are concerned, this is false; for example, the number
of inversions is weakly left-shuffle-compatible and weakly
right-shuffle-compatible but not shuffle-compatible.

Let us next define the notion of dendriform algebras:

\begin{definition}
\textbf{(a)} A \textit{dendriform algebra} over a field $\mathbf{k}$ means a
$\mathbf{k}$-algebra $A$ equipped with two further $\mathbf{k}$-bilinear
binary operations $\left.  \prec\right.  $ and $\left.  \succeq\right.  $
(these are operations, not relations, despite the symbols) from $A\times A$ to
$A$ that satisfy the four rules%
\begin{align*}
a\left.  \prec\right.  b+a\left.  \succeq\right.  b  &  =ab;\\
\left(  a\left.  \prec\right.  b\right)  \left.  \prec\right.  c  &  =a\left.
\prec\right.  \left(  bc\right)  ;\\
\left(  a\left.  \succeq\right.  b\right)  \left.  \prec\right.  c  &
=a\left.  \succeq\right.  \left(  b\left.  \prec\right.  c\right)  ;\\
a\left.  \succeq\right.  \left(  b\left.  \succeq\right.  c\right)   &
=\left(  ab\right)  \left.  \succeq\right.  c
\end{align*}
for all $a,b,c\in A$. (Depending on the situation, it is useful to also impose
a few axioms that relate the unity $1$ of the $\mathbf{k}$-algebra $A$ with
the operations $\left.  \prec\right.  $ and $\left.  \succeq\right.  $. For
example, we could require $1\left.  \prec\right.  a=0$ for each $a\in A$. For
what we are going to do, these extra axioms don't matter.)

\textbf{(b)} If $A$ and $B$ are two dendriform algebras over $\mathbf{k}$,
then a \textit{dendriform algebra homomorphism} from $A$ to $B$ means a
$\mathbf{k}$-algebra homomorphism $\phi:A\rightarrow B$ preserving the
operations $\left.  \prec\right.  $ and $\left.  \succeq\right.  $ (that is,
satisfying $\phi\left(  a\left.  \prec\right.  b\right)  =\phi\left(
a\right)  \left.  \prec\right.  \phi\left(  b\right)  $ and $\phi\left(
a\left.  \succeq\right.  b\right)  =\phi\left(  a\right)  \left.
\succeq\right.  \phi\left(  b\right)  $ for all $a,b\in A$). (Some authors
only require it to be a $\mathbf{k}$-linear map instead of being a
$\mathbf{k}$-algebra homomorphism; this boils down to the question whether
$\phi\left(  1\right)  $ must be $1$ or not. This does not make a difference
for us here.)
\end{definition}

Thus, $\operatorname*{QSym}$ (with its two operations $\left.  \prec\right.  $
and $\left.  \succeq\right.  $) becomes a dendriform algebra over $\mathbb{Q}$.

Notice that if $A$ and $B$ are two dendriform algebras over $\mathbf{k}$, then
the kernel of any dendriform algebra homomorphism $A\rightarrow B$ is an
$\left.  \prec\right.  $-ideal and a $\left.  \succeq\right.  $-ideal of $A$.
Conversely, if $A$ is a dendriform algebra over $\mathbf{k}$, and $I$ is
simultaneously a $\left.  \prec\right.  $-ideal and a $\left.  \succeq
\right.  $-ideal of $A$, then $A/I$ canonically becomes a dendriform algebra,
and the canonical projection $A\rightarrow A/I$ becomes a dendriform algebra homomorphism.

Therefore, Corollary \ref{cor.dendri.K.ideal-LR} (and the $\mathcal{A}%
_{\operatorname{st}}\cong\operatorname*{QSym}/\mathcal{K}_{\operatorname{st}%
}$ isomorphism from Proposition \ref{prop.K.ideal}) yields the following:

\begin{corollary}
If a descent statistic $\operatorname{st}$ is LR-shuffle-compatible, then its
shuffle algebra $\mathcal{A}_{\operatorname{st}}$ canonically becomes a
dendriform algebra.
\end{corollary}

We furthermore have the following analogue of Theorem~\ref{thm.4.3}, which
easily follows from Theorem \ref{thm.dendri.K.ideal} and Theorem
\ref{thm.dendri.K.ideal-R}:

\begin{theorem}
\label{thm.dendri.4.3}Let $\operatorname{st}$ be a descent statistic.

\textbf{(a)} The descent statistic $\operatorname{st}$ is
left-shuffle-compatible and right-shuffle-compatible if and only if there
exist a dendriform algebra $A$ with
basis $\left(  u_{\alpha}\right)  $ (indexed by $\operatorname{st}%
$-equivalence classes $\alpha$ of compositions)
and a dendriform algebra homomorphism
$\phi_{\operatorname{st}}:\operatorname*{QSym} \rightarrow A$
with the property that whenever $\alpha$ is an
$\operatorname{st}$-equivalence class of compositions, we have
\[
\phi_{\operatorname{st}}\left(  F_{L}\right)  =u_{\alpha}%
\ \ \ \ \ \ \ \ \ \ \text{for each }L\in\alpha.
\]

\textbf{(b)} In this case, the $\mathbb{Q}$-linear map%
\[
\mathcal{A}_{\operatorname{st}}\rightarrow A,\ \ \ \ \ \ \ \ \ \ \left[
\pi\right]  _{\operatorname{st}}\mapsto u_{\alpha},
\]
where $\alpha$ is the $\operatorname{st}$-equivalence class of the
composition $\operatorname*{Comp}\pi$, is an isomorphism of dendriform
algebras $\mathcal{A}_{\operatorname{st}}\rightarrow A$.
\end{theorem}

\begin{question}
Can the $\mathbb{Q}$-algebra $\operatorname*{Pow}\mathcal{N}$ from Definition
\ref{def.GammaZ} be endowed with two binary operations $\left.  \prec\right.
$ and $\left.  \succeq\right.  $ that make it into a dendriform algebra? Can
we then find an analogue of Proposition \ref{prop.prod1} along the following lines?

Let $\left(  P,\gamma\right)  $, $\left(  Q,\delta\right)  $ and $\left(
P\sqcup Q,\varepsilon\right)  $ be as in Proposition \ref{prop.prod1}. Assume
that each of the posets $P$ and $Q$ has a (global) minimum element; denote these
elements by $\min P$ and $\min Q$, respectively.
Let $P \left. \prec \right. Q$ be the poset obtained by adding
the relation $\min P<\min Q$ to $P\sqcup Q$.
Let $P \left. \succ \right. Q$ be the poset obtained by adding
the relation $\min P>\min Q$ to $P\sqcup Q$.
Then, we hope to have%
\begin{align*}
\Gamma_{\mathcal{Z}}\left(  P,\gamma\right)  \left.  \prec\right.
\Gamma_{\mathcal{Z}}\left(  Q,\delta\right)   &  =\Gamma_{\mathcal{Z}}\left(
P\left.  \prec\right.  Q,\varepsilon\right)  \ \ \ \ \ \ \ \ \ \ \text{and}\\
\Gamma_{\mathcal{Z}}\left(  P,\gamma\right)  \left.  \succeq\right.
\Gamma_{\mathcal{Z}}\left(  Q,\delta\right)   &  =\Gamma_{\mathcal{Z}}\left(
P\left.  \succ\right.  Q,\varepsilon\right)  ,
\end{align*}
assuming a simple condition on $\min P$ and $\min Q$
(say, $\gamma\left(\min P\right) <_{\ZZ} \delta\left(\min Q\right)$).
\end{question}

\subsection*{Acknowledgments}

We thank Yan Zhuang, Ira Gessel and Sara Billey for helpful conversations and
corrections. An anonymous referee has also helpfully pointed out mistakes.
The SageMath computer algebra system \cite{SageMath} has been
used in finding some of the results below.

\begin{thebibliography}{99}                                                                                         %


\bibitem{ABN-peaks}Marcelo Aguiar, Nantel Bergeron and Kathryn Nyman,
\textit{The peak algebra and the descent algebras of types B and D},
Trans. Amer. Math. Soc. \textbf{356} (2004), pp. 2781--2824,
\url{https://doi.org/10.1090/S0002-9947-04-03541-X} .

\bibitem{BilHai95}%
\href{https://doi.org/10.1090/S0894-0347-1995-1290232-1}{Sara Billey, Mark
Haiman, \textit{Schubert polynomials for the classical groups}, J. Amer. Math.
Soc. \textbf{8} (1995), pp. 443--482}.

\bibitem{EbMaPa07}%
\href{https://doi.org/10.1016/j.jalgebra.2007.12.013}{Kurusch Ebrahimi-Fard,
Dominique Manchon, Fr\'{e}d\'{e}ric Patras, \textit{New identities in
dendriform algebras}, Journal of Algebra 320 (2008), pp. 708--727}.

\bibitem{part1}Ira M. Gessel, Yan Zhuang, \textit{Shuffle-compatible
permutation statistics}, Advances in Mathematics, Volume 332, 9 July 2018, pp.
85--141.\newline Also available at
\texttt{\href{http://arxiv.org/abs/1706.00750v3}{arXiv:1706.00750v3}}.

\bibitem{Greene88}%
\href{https://doi.org/10.1016/0097-3165(88)90018-0}{Curtis Greene,
\textit{Posets of shuffles}, Journal of Combinatorial Theory, Series A, Volume
47, Issue 2, March 1988, pp. 191--206}.

\bibitem{dimcr}Darij Grinberg, \textit{Dual immaculate creation
operators and a dendriform algebra structure on the quasisymmetric functions},
version 6, \texttt{\href{https://arxiv.org/abs/1410.0079v6}{arXiv:1410.0079v6}%
}. (Version 5 has been published in:
\href{https://cms.math.ca/10.4153/CJM-2016-018-8?abfmt=ltx}{Canad. J. Math.
\textbf{69} (2017), 21--53}.)

\bibitem{verlong}%
\href{http://www.cip.ifi.lmu.de/~grinberg/algebra/gzshuf2-long.pdf}{Darij
Grinberg, \textit{Shuffle-compatible permutation statistics II: the exterior
peak set [detailed version]}, detailed version of the present paper}. Also
available as an ancillary file at
\href{https://arxiv.org/abs/1806.04114v3}{arXiv:1806.04114v3}.

\bibitem{HopfComb}Darij Grinberg, Victor Reiner, \textit{Hopf
algebras in Combinatorics}, version of 11 May 2018,
\texttt{\href{http://www.arxiv.org/abs/1409.8356v5}{arXiv:1409.8356v5}}.
\newline See also
\url{http://www.cip.ifi.lmu.de/~grinberg/algebra/HopfComb-sols.pdf} for a
version that gets updated.

\bibitem{HsiPet10}Samuel K. Hsiao, T. Kyle Petersen, \textit{Colored
Posets and Colored Quasisymmetric Functions}, Ann. Comb. 14 (2010), pp.
251--289,\newline\url{https://doi.org/10.1007/s00026-010-0059-0} . See
\href{http://www.arxiv.org/abs/math/0610984v1}{\texttt{arXiv:math/0610984v1}}
for a preprint.

\bibitem{Oguz18}Ezgi Kantarc{\i} O\u{g}uz,
\textit{A Counter Example to the Shuffle Compatiblity Conjecture},
\arxiv{1807.01398v1}.

\bibitem{Peters05}T. Kyle Petersen, \textit{Enriched }$\mathit{P}%
$\textit{-partitions and peak algebras}, Advances in Mathematics 209 (2007),
pp. 561--610,\newline\url{https://doi.org/10.1016/j.aim.2006.05.016} . See
\texttt{\href{https://arxiv.org/abs/math/0508041v1}{arXiv:math/0508041v1}} for
a preprint.

\bibitem{Peters06}T. Kyle Petersen,
\textit{Descents, Peaks, and $P$-partitions},
thesis at Brandeis University,
2006.
\url{http://people.brandeis.edu/~gessel/homepage/students/petersenthesis.pdf}

\bibitem{SageMath}\href{http://www.sagemath.org}{The Sage
Developers, \textit{SageMath, the Sage Mathematics Software System (Version
8.0)}, 2017.}

\bibitem{Stanle72}Richard P. Stanley, \textit{Ordered Structures and
Partitions}, Memoirs of the American Mathematical Society, No. 119, American
Mathematical Society, Providence, R.I., 1972. \newline\url{http://www-math.mit.edu/~rstan/pubs/pubfiles/9.pdf}

\bibitem{Stanley-EC1}Richard Stanley, \textit{Enumerative
Combinatorics, volume 1}, 2nd edition, Cambridge University Press 2012. A
preprint is available at \url{http://math.mit.edu/~rstan/ec/} .

\bibitem{Stembr97}%
\href{http://www.ams.org/journals/tran/1997-349-02/S0002-9947-97-01804-7/}{John
R. Stembridge, \textit{Enriched P-partitions}, Trans. Amer. Math. Soc.
\textbf{349} (1997), no. 2, pp. 763--788}.
\end{thebibliography}


\end{document}