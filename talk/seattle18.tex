\documentclass{beamer}
\usepackage{amsmath}
\usepackage{amssymb}
\usepackage{array}
\usepackage{setspace}
\usepackage{graphicx}
%\usepackage{tikz}
%\usetikzlibrary{matrix,arrows,backgrounds,shapes.misc,shapes.geometric,fit}
\usepackage{etex}
\usepackage{amsthm}
\usepackage{color}
\usepackage{wasysym}
\usepackage{allrunes}
\usepackage[all]{xy}
\usepackage{textpos}
\usepackage{ytableau}
%\usetikzlibrary{calc,through,backgrounds}
%\CompileMatrices

\definecolor{grau}{rgb}{.5 , .5 , .5}
\definecolor{dunkelgrau}{rgb}{.35 , .35 , .35}
\definecolor{schwarz}{rgb}{0 , 0 , 0}
\definecolor{violet}{RGB}{143,0,255}
\definecolor{forestgreen}{RGB}{34, 100, 34}

\newcommand{\red}{\color{red}}
\newcommand{\grey}{\color{grau}}
\newcommand{\green}{\color{forestgreen}}
\newcommand{\violet}{\color{violet}}
\newcommand{\blue}{\color{blue}}

\newcommand{\bIf}{\textbf{If} }
\newcommand{\bif}{\textbf{if} }
\newcommand{\bthen}{\textbf{then} }

\newcommand{\EE}{{\mathbf{E}}}
\newcommand{\ZZ}{{\mathbb Z}}
\newcommand{\NN}{{\mathbb N}}
\newcommand{\QQ}{{\mathbb Q}}
\newcommand{\kk}{{\mathbf k}}
\newcommand{\OO}{\operatorname {O}}
\newcommand{\Nm}{\operatorname {N}}
\newcommand{\Par}{\operatorname{Par}}
\newcommand{\Comp}{\operatorname{Comp}}
\newcommand{\Stab}{\operatorname {Stab}}
\newcommand{\id}{\operatorname{id}}
\newcommand{\ev}{\operatorname{ev}}
\newcommand{\Sym}{\operatorname{Sym}}
\newcommand{\Lpk}{\operatorname{Lpk}}
\newcommand{\lpk}{\operatorname{lpk}}
\newcommand{\Rpk}{\operatorname{Rpk}}
\newcommand{\rpk}{\operatorname{rpk}}
\newcommand{\Pk}{\operatorname{Pk}}
\newcommand{\Epk}{\operatorname{Epk}}
\newcommand{\epk}{\operatorname{epk}}
\newcommand{\Des}{\operatorname{Des}}
\newcommand{\des}{\operatorname{des}}
\newcommand{\inv}{\operatorname{inv}}
\newcommand{\maj}{\operatorname{maj}}
\newcommand{\Val}{\operatorname{Val}}
\newcommand{\pk}{\operatorname{pk}}
\newcommand{\st}{\operatorname{st}}
\newcommand{\QSym}{\operatorname{QSym}}
\newcommand{\NSym}{\operatorname{NSym}}
\newcommand{\Mat}{\operatorname{M}}
\newcommand{\bk}{\mathbf{k}}
\newcommand{\Nplus}{\mathbb{N}_{+}}
\newcommand\arxiv[1]{\href{http://www.arxiv.org/abs/#1}{\texttt{arXiv:#1}}}
\newcommand{\Orb}{{\mathcal O}}
\newcommand{\GL}{\operatorname {GL}}
\newcommand{\SL}{\operatorname {SL}}
\newcommand{\Or}{\operatorname {O}}
\newcommand{\im}{\operatorname {Im}}
\newcommand{\Iso}{\operatorname {Iso}}
\newcommand{\Adm}{\operatorname{Adm}}
\newcommand{\Supp}{\operatorname{Supp}}
\newcommand{\Powser}{\QQ\left[\left[x_1,x_2,x_3,\ldots\right]\right]}
\newcommand{\rad}{\operatorname {rad}}
\newcommand{\zero}{\mathbf{0}}
\newcommand{\xx}{\mathbf{x}}
\newcommand{\ord}{\operatorname*{ord}}
\newcommand{\bbK}{{\mathbb{K}}}
\newcommand{\whP}{{\widehat{P}}}
%\newcommand{\Trop}{\operatorname*{Trop}}
%\newcommand{\TropZ}{{\operatorname*{Trop}\mathbb{Z}}}
\newcommand{\rato}{\dashrightarrow}
\newcommand{\lcm}{\operatorname*{lcm}}
\newcommand{\tlab}{\operatorname*{tlab}}
\newcommand{\tvi}{\left. \textarm{\tvimadur} \right.}
\newcommand{\bel}{\left. \textarm{\belgthor} \right.}
\newcommand{\calK}{\mathcal{K}}
\newcommand{\calE}{\mathcal{E}}
\newcommand{\calL}{\mathcal{L}}
\newcommand{\calN}{\mathcal{N}}
\newcommand{\calZ}{\mathcal{Z}}

\newcommand{\fti}[1]{\frametitle{\ \ \ \ \ #1}}
\newenvironment{iframe}[1][]{\begin{frame} \fti{[#1]} \begin{itemize}}{\end{frame}\end{itemize}}

\newcommand{\are}{\ar@{-}}
\newcommand{\arinj}{\ar@{_{(}->}}
\newcommand{\arsurj}{\ar@{->>}}

\newcommand{\set}[1]{\left\{ #1 \right\}}
% $\set{...}$ yields $\left\{ ... \right\}$.
\newcommand{\abs}[1]{\left| #1 \right|}
% $\abs{...}$ yields $\left| ... \right|$.
\newcommand{\tup}[1]{\left( #1 \right)}
% $\tup{...}$ yields $\left( ... \right)$.
\newcommand{\ive}[1]{\left[ #1 \right]}
% $\ive{...}$ yields $\left[ ... \right]$.
\newcommand{\verts}[1]{\operatorname{V}\left( #1 \right)}
% $\verts{...}$ yields $\operatorname{V}\left( ... \right)$.
\newcommand{\edges}[1]{\operatorname{E}\left( #1 \right)}
% $\edges{...}$ yields $\operatorname{E}\left( ... \right)$.
\newcommand{\arcs}[1]{\operatorname{A}\left( #1 \right)}
% $\arcs{...}$ yields $\operatorname{A}\left( ... \right)$.
\newcommand{\underbrack}[2]{\underbrace{#1}_{\substack{#2}}}
% $\underbrack{...1}{...2}$ yields
% $\underbrace{...1}_{\substack{...2}}$. This is useful for doing
% local rewriting transformations on mathematical expressions with
% justifications.

\setbeamertemplate{itemize/enumerate body begin}{\large}
\setbeamertemplate{itemize/enumerate subbody begin}{\large}
\setbeamertemplate{itemize/enumerate subsubbody begin}{\large}

\usepackage{url}%this line and the next are related to hyperlinks
%\usepackage[colorlinks=true, pdfstartview=FitV, linkcolor=blue, citecolor=blue, urlcolor=blur]{hyperref}

\usepackage{color}

%\usetheme{Antibes}
%\usetheme{Bergen}
%\usetheme{Berkeley}
%\usetheme{Berlin}
%\usetheme{Boadilla}
%\usetheme{Copenhagen}
%\usetheme{Darmstadt}
%\usetheme{Dresden}
\usetheme{Frankfurt}
%\usetheme{Goettingen}
%\usetheme{Hannover}
%\usetheme{Ilmenau}
%\usetheme{JuanLesPins}
%\usetheme{Luebeck}
%\usetheme{Madrid}
%\usetheme{Malmoe}
%\usetheme{Marburg}
%\usetheme{Montpellier}
%\usetheme{PaloAlto}
%\usetheme{Pittsburgh}
%\usetheme{Rochester}
%\usetheme{Singapore}
%\usetheme{Szeged}
%\usetheme{Warsaw}

\usefonttheme[onlylarge]{structurebold}
\setbeamerfont*{frametitle}{size=\normalsize,series=\bfseries}
\setbeamertemplate{navigation symbols}{}
\setbeamertemplate{footline}[frame number]
\setbeamertemplate{itemize/enumerate body begin}{}
\setbeamertemplate{itemize/enumerate subbody begin}{\normalsize}
%\setbeamertemplate{section in head/foot shaded}[default][60]
%\setbeamertemplate{subsection in head/foot shaded}[default][60]
\beamersetuncovermixins{\opaqueness<1>{0}}{\opaqueness<2->{15}}

%\usepackage{beamerthemesplit}
\usepackage{epsfig,amsfonts,bbm,mathrsfs}
\usepackage{verbatim} 

% Dark red emphasis
\definecolor{darkred}{rgb}{0.7,0,0} % darkred color
\newcommand{\defn}[1]{{\color{darkred}\emph{#1}}} % emphasis of a definition



\newcommand{\STRUT}{\vrule width 0pt depth 8pt height 0pt}
\newcommand{\ASTRUT}{\vrule width 0pt depth 0pt height 11pt}


\theoremstyle{plain}
\newtheorem{conj}[theorem]{Conjecture}


\setbeamertemplate{headline}{}
%This removes a black stripe from the top of the slides.


\author{Darij Grinberg (UMN)
}
\title[Shuffle-compatibility and exterior peaks]{Ideals of QSym, shuffle-compatibility and exterior peaks}
% https://math.washington.edu/events/2018-02-28/ideals-qsym-shuffle-compatibility-and-exterior-peaks
% With the concept of a shuffle-compatible permutation statistic (arXiv:1706.00750), Gessel and Zhuang have opened up a new direction in the study of quasisymmetric functions. They have shown that various "descent statistics" (e.g., the descent set, the descent number, the major index, the peak set, the peak number) are shuffle-compatible. Every such statistic leads to an ideal of the algebra QSym, and the quotient algebra is often of interest. We shall discuss one particular statistic -- the "exterior peak set" -- whose shuffle-compatibility we prove (it was left open by Gessel and Zhuang). We then proceed to extend the notion of shuffle-compatibility to a stronger notion, which takes the dendriform structure of QSym into account (and which also holds for the exterior peak set).
\date{28 February 2018 \\ University of Washington}

\begin{document}

\frame{\titlepage
\textbf{slides: \red \url{http://www.cip.ifi.lmu.de/~grinberg/algebra/seattle18.pdf}} \\
\textbf{paper: \red \url{http://www.cip.ifi.lmu.de/~grinberg/algebra/gzshuf2.pdf}} \\
\textbf{project: \red \url{https://github.com/darijgr/gzshuf}}
}

% \item \href{http://arxiv.org/abs/1409.8356}{\red Darij Grinberg, Victor Reiner, \textit{Hopf Algebras in Combinatorics}, arXiv:1409.8356}.
% \item M. Hazewinkel, N. Gubareni, V.V. Kirichenko, \textit{Algebras, rings and modules. Lie algebras and Hopf algebras}, AMS 2010.

\begin{frame}
\fti{Section 1}
\begin{center}
{\LARGE \bf Section 1} \\
\noindent\rule[0.5ex]{\linewidth}{1pt}
{\Large \bf Shuffle-compatibility}
\end{center}
\vspace{1cm}
Reference:
\begin{itemize}
\item \href{https://arxiv.org/abs/1706.00750}{\red Ira M. Gessel, Yan Zhuang, \textit{Shuffle-compatible permutation statistics}, arXiv:1706.00750}.
\end{itemize}
\end{frame}

\begin{frame}
\fti{Permutations \& permutation statistics: Definitions 1}

\begin{itemize}

\item This project spins off from a paper by Ira Gessel and Yan Zhuang
({\red \arxiv{1706.00750}}),
which Yan presented here last week. \\
We prove a conjecture (shuffle-compatibility of $\Epk$)
and study a stronger version of shuffle-compatibility.

\pause

\item Let $\NN = \set{0, 1, 2, \ldots}$.

\item For $n \in \NN$, an \defn{$n$-permutation} means a tuple of $n$
distinct positive integers. \\
Example: $\tup{3, 1, 7}$ is a $3$-permutation, but
         $\tup{2, 1, 2}$ is not.

\pause

\item A \defn{permutation} means an $n$-permutation for some $n$. \\ \pause
      If $\pi$ is an $n$-permutation, then \defn{$\abs{\pi} := n$}. \\ \pause
      We say that $\pi$ is \defn{nonempty} if $n > 0$.

\pause

\item If $\pi$ is an $n$-permutation and $i \in \set{1, 2, \ldots, n}$,
      then \defn{$\pi_i$} denotes the $i$-th entry of $\pi$.

\end{itemize}
\end{frame}

\begin{frame}
\fti{Permutations \& permutation statistics: Definitions 2}

\begin{itemize}

\item Two $n$-permutations $\alpha$ and $\beta$ (with the same $n$)
      are \defn{order-equivalent} if all $i, j \in \set{1, 2, \ldots, n}$
      satisfy $\tup{\alpha_i < \alpha_j} \Longleftrightarrow
      \tup{\beta_i < \beta_j}$.

\item Order-equivalence is an equivalence relation on permutations.
      Its equivalence classes are called \defn{order-equivalence
      classes}.

\pause

\item A \defn{permutation statistic} (henceforth just \defn{statistic})
      is a map $\st$ from the set
      of all permutations (to anywhere) that is constant on each
      order-equivalence class. \\
      \textbf{Intuition:} A statistic computes some ``fingerprint''
      of a permutation that only depends on the relative order of its
      letters. \pause \\
      \textbf{Note:} A statistic need not be integer-valued!
      It can be set-valued, or list-valued for example.

\end{itemize}
\end{frame}

\begin{frame}
\fti{Examples of permutation statistics, 1: descents et al}

\begin{itemize}

\item If $\pi$ is an $n$-permutation, then a \defn{descent} of $\pi$
      means an $i \in \set{1, 2, \ldots, n-1}$ such that
      $\pi_i > \pi_{i+1}$.

\item The \defn{descent set $\Des \pi$} of a permutation $\pi$ is
      the set of all descents of $\pi$. \\
      Thus, \defn{$\Des$} is a statistic. \\
      \textbf{Example:} $\Des \tup{3, 1, 5, 2, 4} = \set{1, 3}$.

\pause

\item The \defn{descent number $\des \pi$} of a permutation $\pi$
      is the number of all descents of $\pi$: that is,
      $\des \pi = \abs{\Des \pi}$. \\
      Thus, \defn{$\des$} is a statistic. \\
      \textbf{Example:} $\des \tup{3, 1, 5, 2, 4} = 2$.

\pause

\item The \defn{major index $\maj \pi$} of a permutation $\pi$
      is the \textbf{sum} of all descents of $\pi$. \\
      Thus, \defn{$\maj$} is a statistic. \\
      \textbf{Example:} $\des \tup{3, 1, 5, 2, 4} = 4$.

\pause

\item The \defn{Coxeter length} $\operatorname{inv}$
      (i.e., \defn{number of inversions})
      and the \defn{set of inversions} are statistics, too.

\end{itemize}
\end{frame}

\begin{frame}
\fti{Examples of permutation statistics, 2: peaks}

\begin{itemize}

\item If $\pi$ is an $n$-permutation, then a \defn{peak} of $\pi$
      means an $i \in \set{2, 3, \ldots, n-1}$ such that
      $\pi_{i-1} < \pi_i > \pi_{i+1}$. \\
      (Thus, peaks can only exist if $n \geq 3$. \\
      The name refers to the plot of $\pi$, where peaks are
      local maxima.)

\item The \defn{peak set $\Pk \pi$} of a permutation $\pi$ is
      the set of all peaks of $\pi$. \\
      Thus, \defn{$\Pk$} is a statistic. \\
      \textbf{Examples:}
      \begin{itemize}
      \item $\Pk \tup{3, 1, 5, 2, 4} = \set{3}$.
      \item $\Pk \tup{1, 3, 2, 5, 4, 6} = \set{2, 4}$.
      \item $\Pk \tup{3, 2} = \set{}$.
      \end{itemize}

\pause

\item The \defn{peak number $\pk \pi$} of a permutation $\pi$
      is the number of all peaks of $\pi$: that is,
      $\pk \pi = \abs{\Pk \pi}$. \\
      Thus, \defn{$\pk$} is a statistic. \\
      \textbf{Example:} $\pk \tup{3, 1, 5, 2, 4} = 1$.

\end{itemize}
\vspace{10cm}
\end{frame}

\begin{frame}
\fti{Examples of permutation statistics, 3: left peaks}

\begin{itemize}

\item If $\pi$ is an $n$-permutation, then a \defn{left peak} of $\pi$
      means an $i \in \set{1, 2, \ldots, n-1}$ such that
      $\pi_{i-1} < \pi_i > \pi_{i+1}$, where we set \defn{$\pi_0 = 0$}. \\
      (Thus, left peaks are the same as peaks, except that $1$ counts
      as a left peak if $\pi_1 > \pi_2$.)

\item The \defn{left peak set $\Lpk \pi$} of a permutation $\pi$ is
      the set of all left peaks of $\pi$. \\
      Thus, \defn{$\Lpk$} is a statistic. \\
      \textbf{Examples:}
      \begin{itemize}
      \item $\Lpk \tup{3, 1, 5, 2, 4} = \set{1, 3}$.
      \item $\Lpk \tup{1, 3, 2, 5, 4, 6} = \set{2, 4}$.
      \item $\Lpk \tup{3, 2} = \set{1}$.
      \end{itemize}

\item The \defn{left peak number $\lpk \pi$} of a permutation $\pi$
      is the number of all left peaks of $\pi$: that is,
      $\lpk \pi = \abs{\Lpk \pi}$. \\
      Thus, \defn{$\lpk$} is a statistic. \\
      \textbf{Example:} $\lpk \tup{3, 1, 5, 2, 4} = 2$.

\end{itemize}
\vspace{10cm}
\end{frame}

\begin{frame}
\fti{Examples of permutation statistics, 4: right peaks}

\begin{itemize}

\item If $\pi$ is an $n$-permutation, then a \defn{right peak} of $\pi$
      means an $i \in \set{2, 3, \ldots, n}$ such that
      $\pi_{i-1} < \pi_i > \pi_{i+1}$, where we set \defn{$\pi_{n+1} = 0$}. \\
      (Thus, right peaks are the same as peaks, except that $n$ counts
      as a right peak if $\pi_{n-1} < \pi_n$.)

\item The \defn{right peak set $\Rpk \pi$} of a permutation $\pi$ is
      the set of all right peaks of $\pi$. \\
      Thus, \defn{$\Rpk$} is a statistic. \\
      \textbf{Examples:}
      \begin{itemize}
      \item $\Rpk \tup{3, 1, 5, 2, 4} = \set{3, 5}$.
      \item $\Rpk \tup{1, 3, 2, 5, 4, 6} = \set{2, 4, 6}$.
      \item $\Rpk \tup{3, 2} = \set{}$.
      \end{itemize}

\item The \defn{right peak number $\rpk \pi$} of a permutation $\pi$
      is the number of all right peaks of $\pi$: that is,
      $\rpk \pi = \abs{\Rpk \pi}$. \\
      Thus, \defn{$\rpk$} is a statistic. \\
      \textbf{Example:} $\rpk \tup{3, 1, 5, 2, 4} = 2$.

\end{itemize}
\vspace{10cm}
\end{frame}

\begin{frame}
\fti{Examples of permutation statistics, 5: exterior peaks}

\begin{itemize}

\item If $\pi$ is an $n$-permutation, then an \defn{exterior peak} of $\pi$
      means an $i \in \set{1, 2, \ldots, n}$ such that
      $\pi_{i-1} < \pi_i > \pi_{i+1}$, where we set \defn{$\pi_0 = 0$ and $\pi_{n+1} = 0$}. \\
      (Thus, exterior peaks are the same as peaks, except that
      $1$ counts if $\pi_1 > \pi_2$, and
      $n$ counts if $\pi_{n-1} < \pi_n$.)

\item The \defn{exterior peak set $\Epk \pi$} of a permutation $\pi$ is
      the set of all exterior peaks of $\pi$. \\
      Thus, \defn{$\Epk$} is a statistic. \\
      \textbf{Examples:}
      \begin{itemize}
      \item $\Epk \tup{3, 1, 5, 2, 4} = \set{1, 3, 5}$.
      \item $\Epk \tup{1, 3, 2, 5, 4, 6} = \set{2, 4, 6}$.
      \item $\Epk \tup{3, 2} = \set{1}$.
      \end{itemize}

\item Thus, $\Epk \pi = \Lpk \pi \cup \Rpk \pi$ if $n \geq 2$.

\item The \defn{exterior peak number $\epk \pi$} of a permutation $\pi$
      is the number of all exterior peaks of $\pi$: that is,
      $\epk \pi = \abs{\Epk \pi}$. \\
      Thus, \defn{$\epk$} is a statistic. \\
      \textbf{Example:} $\epk \tup{3, 1, 5, 2, 4} = 3$.

\end{itemize}
\vspace{10cm}
\end{frame}

\begin{frame}
\fti{Shuffles of permutations}

\begin{itemize}

\item Let $\pi$ and $\sigma$ be two permutations.

\item We say that $\pi$ and $\sigma$ are \defn{disjoint} if they have
      no letter in common.

\pause

\item Assume that $\pi$ and $\sigma$ are disjoint. Set
      $m = \abs{\pi}$ and $n = \abs{\sigma}$.
      An $\tup{m+n}$-permutation $\tau$ is called a \defn{shuffle} of
      $\pi$ and $\sigma$ if both $\pi$ and $\sigma$ appear as
      subsequences of $\tau$.

\item We let $S \tup{\pi, \sigma}$ be the set of all shuffles
      of $\pi$ and $\sigma$.

\item \textbf{Example:}
      \begin{align*}
      S \tup{ {\red \tup{4, 1}} , {\blue \tup{2, 5}} }
      &= \{ \tup{ {\red 4}, {\red 1}, {\blue 2}, {\blue 5} } ,
            \tup{ {\red 4}, {\blue 2}, {\red 1}, {\blue 5} } ,
            \tup{ {\red 4}, {\blue 2}, {\blue 5}, {\red 1} } , \\
      & \qquad
            \tup{ {\blue 2}, {\red 4}, {\red 1}, {\blue 5} } ,
            \tup{ {\blue 2}, {\red 4}, {\blue 5}, {\red 1} } ,
            \tup{ {\blue 2}, {\blue 5}, {\red 4}, {\red 1} }  \} .
      \end{align*}

\pause

\item Observe that $\pi$ and $\sigma$ have $\binom{m+n}{m}$
      shuffles, in bijection with $m$-element subsets of
      $\set{1, 2, \ldots, m+n}$.

\end{itemize}
\end{frame}

\begin{frame}
\fti{Shuffle-compatible statistics: definition}

\begin{itemize}

\item A statistic $\st$ is said to be \defn{shuffle-compatible}
      if for any two disjoint permutations $\pi$ and $\sigma$, the
      multiset
      \[
      \set{ \st\tau \mid \tau\in S\tup{\pi, \sigma} }_{\text{multiset}}
      \]
      depends only on $\st \pi$, $\st \sigma$, $\abs{\pi}$ and
      $\abs{\sigma}$.

\pause

\item In other words, $\st$ is shuffle-compatible if and only
      the distribution of
      $\st$ on the set $S \tup{ \pi , \sigma }$
      stays unchaged if $\pi$ and $\sigma$ are replaced by two
      other permutations of the same size and same $\st$-values.
      \pause \\
      In particular, it has to stay unchanged if $\pi$ and $\sigma$
      are replaced by two permutations order-equivalent to them:
      e.g., $\st$ must have the same distribution
      on the three sets
      \[
      S \tup{ {\red \tup{4, 1}} , {\blue \tup{2, 5}} },
      \qquad S \tup{ {\red \tup{2, 1}} , {\blue \tup{3, 5}} } ,
      \qquad S \tup{ {\red \tup{9, 8}} , {\blue \tup{2, 3}} } .
      \]

\end{itemize}

\end{frame}

\begin{frame}
\fti{Shuffle-compatible statistics: results of Gessel and Zhuang}

\begin{itemize}

\item Gessel and Zhuang, in {\red \arxiv{1706.00750}}, prove that
      various important statistics are shuffle-compatible (but some
      are not).

\pause

\item Statistics they show to be \textbf{shuffle-compatible}: $\Des$, $\des$,
      $\maj$, $\Pk$, $\Lpk$, $\Rpk$, $\lpk$, $\rpk$, $\epk$,
      and various others.

\pause

\item Statistics that are \textbf{not shuffle-compatible}:
      $\operatorname{inv}$, $\des + \maj$, $\maj_2$ (sending $\pi$
      to the sum of the squares of its descents),
      $\tup{\Pk, \des}$ (sending $\pi$ to $\tup{\Pk\pi, \des\pi}$),
      and others.

\pause 

\item Their proofs use a mixture of enumerative combinatorics
      (including some known formulas of MacMahon, Stanley, ...),
      quasisymmetric functions, Hopf algebra theory,
      P-partitions (and variants by Stembridge and Petersen),
      Eulerian polynomials (based on earlier work by Zhuang,
      and even earlier work by Foata and Strehl).

\pause

\item The shuffle-compatibility of $\Epk$ is left unproven
      in Gessel/Zhang. Proving this is our first goal.

\end{itemize}

\end{frame}

\begin{frame}
\fti{Left- and right-shuffle-compatibility}

\begin{itemize}

\item We further begin the study of a finer version of
      shuffle-compatibility: ``left- and right-shuffle-compatibility''.

\item Given two disjoint nonempty permutations $\pi$ and $\sigma$,
      \begin{itemize}
      \item
      a \defn{left shuffle} of $\pi$ and $\sigma$ is a shuffle
      of $\pi$ and $\sigma$ that starts with a letter of $\pi$;
      \item
      a \defn{right shuffle} of $\pi$ and $\sigma$ is a shuffle
      of $\pi$ and $\sigma$ that starts with a letter of $\sigma$.
      \end{itemize}

\item We let $S_\prec \tup{\pi, \sigma}$ be the set of all left shuffles of $\pi$
      and $\sigma$. \\
      We let $S_\succ \tup{\pi, \sigma}$ be the set of all right shuffles of $\pi$
      and $\sigma$.

\pause

\item \only<2>{
      A statistic $\st$ is said to be \defn{left-shuffle-compatible}
      if for any two disjoint nonempty permutations $\pi$ and $\sigma$
      such that
      \[
      \text{the first entry of }\pi\text{ is greater than the first entry of }%
      \sigma,
      \]
      the multiset
      \[
      \set{ \st\tau \mid \tau\in S_\prec \tup{\pi, \sigma} }_{\text{multiset}}
      \]
      depends only on $\st \pi$, $\st \sigma$, $\abs{\pi}$ and
      $\abs{\sigma}$.
      }
      \only<3-4>{
      A statistic $\st$ is said to be \defn{right-shuffle-compatible}
      if for any two disjoint nonempty permutations $\pi$ and $\sigma$
      such that
      \[
      \text{the first entry of }\pi\text{ is greater than the first entry of }%
      \sigma,
      \]
      the multiset
      \[
      \set{ \st\tau \mid \tau\in S_\succ \tup{\pi, \sigma} }_{\text{multiset}}
      \]
      depends only on $\st \pi$, $\st \sigma$, $\abs{\pi}$ and
      $\abs{\sigma}$.
      }
      
\pause\pause

\item We'll show that $\Des$, $\des$, $\Lpk$ and $\Epk$ are
      left- and right-shuffle-compatible. \\
      (But not $\maj$ or $\Rpk$.)

\end{itemize}
\vspace{10cm}
\end{frame}

\begin{frame}
\fti{Section 2}
\begin{center}
{\LARGE \bf Section 2} \\
\noindent\rule[0.5ex]{\linewidth}{1pt}
{\Large \bf The algebraic approach: $\QSym$ and kernels}
\end{center}
\vspace{1cm}
Reference:
\begin{itemize}
\item \href{https://arxiv.org/abs/1706.00750}{\red Ira M. Gessel, Yan Zhuang, \textit{Shuffle-compatible permutation statistics}, arXiv:1706.00750}.
\item \href{http://arxiv.org/abs/1409.8356}{\red Darij Grinberg, Victor Reiner, \textit{Hopf Algebras in Combinatorics}, arXiv:1409.8356},
      and various other texts on combinatorial Hopf algebras.
\end{itemize}
\end{frame}

\begin{frame}
\fti{Descent statistics}

\begin{itemize}

\item Gessel and Zhuang prove \textbf{most} of their shuffle-compatibilities
      algebraically. Their methods involve combinatorial Hopf
      algebras ($\QSym$ and $\NSym$).

\item These methods work for \textbf{descent statistics} only.
      What is a descent statistic?

\pause

\item A \defn{descent statistic} is a statistic $\st$ such that
      $\st \pi$ depends only on $\abs{\pi}$ and $\Des\pi$
      (in other words: if $\pi$ and $\sigma$ are two
      $n$-permutations with $\Des\pi = \Des\sigma$, then
      $\st\pi = \st\sigma$). \\
      \textbf{Intuition:} A descent statistic is a statistic
      which ``factors through $\Des$ in each size''.

\end{itemize}
\end{frame}

\begin{frame}
\fti{Compositions \& descent compositions: definitions}

\begin{itemize}
      
\item A \defn{composition} is a finite list of positive integers.
      \\
      A \defn{composition of $n \in \NN$} is a composition whose
      entries sum to $n$. \pause

\item For example, $\left(1,3,2\right)$ is a composition of $6$.

\pause

\item Let $n \in \NN$, and let $\ive{n-1} = \set{1, 2, \ldots, n-1}$.
      \\
      \only<3>{Then, there are mutually inverse bijections
      \begin{align*}
        \Des : \set{\text{compositions of } n}
               &\to \set{\text{subsets of } \ive{n-1} }, \\
               \tup{i_1, i_2, \ldots, i_k}
               &\mapsto \set{i_1 + i_2 + \cdots + i_j \mid 1 \leq j \leq k-1 }
      \end{align*}
      and
      \begin{align*}
        \Comp : \set{\text{subsets of } \ive{n-1} }
               &\to \set{\text{compositions of } n}, \\
               \set{s_1 < s_2 < \cdots < s_k}
               &\mapsto \tup{s_1 - s_0, s_2 - s_1, \ldots, s_{k+1} - s_k }
      \end{align*}
      (using the notations $s_0 = 0$ and $s_{k+1} = n$).}
      \only<4,5,6,7>{Then, there are mutually inverse bijections
      $\Des$ and $\Comp$ between $\set{\text{subsets of } \ive{n-1} }$
      and $\set{\text{compositions of } n}$. \\
      If $\pi$ is an $n$-permutation, then $\Comp\tup{\Des \pi}$ is
      called the \defn{descent composition} of $\pi$, and is written
      \defn{$\Comp\pi$}.
      }

\only<5,6>{
\item Thus, a descent statistic is a statistic $\st$ that factors through
      $\Comp$ (that is, $\st\pi$ depends only on $\Comp\pi$).
}
\only<6,7>{
\item If $\st$ is a descent statistic, then we use the notation
      {$\st \alpha$} (where $\alpha$ is a composition) for $\st \pi$,
      where $\pi$ is any permutation with $\Comp \pi = \alpha$.
}
\only<7>{
\item \textbf{Warning:}
      \begin{align*}
      \Des\tup{\tup{1, 5, 2} \text{ the composition}} &= \set{1, 6} ; \\
      \Des\tup{\tup{1, 5, 2} \text{ the permutation}} &= \set{2} .
      \end{align*}
      Same for other statistics!
      Context must disambiguate.
}
      

\end{itemize}

\vspace{10cm}

\end{frame}

\begin{frame}
\fti{Descent statistics: examples}

\begin{itemize}

\item Almost all of our statistics so far are descent statistics.
     Examples:

\pause

\item $\Des$, $\des$ and $\maj$ are descent statistics.

\pause

\item $\Pk$ is a descent statistic: If $\pi$ is an $n$-permutation,
      then
      \[
      \Pk \pi = \tup{\Des \pi} \setminus \tup{\tup{\Des \pi \cup \set{0}} + 1} ,
      \]
      where for any set $K$ of integers and any integer $a$
      we set \defn{$K + a = \set{k + a \mid k \in K}$}.

\item Similarly, $\Lpk$, $\Rpk$ and $\Epk$ are descent statistics.

\pause

\item $\inv$ is not a descent statistic:
      The permutations $\tup{2,1,3}$ and $\tup{3,1,2}$ have the same
      descents, but different numbers of inversions.

\pause

\item \textbf{Question (Gessel \& Zhang).}
      Is every shuffle-compatible statistic a descent statistic?

\end{itemize}

\end{frame}

\begin{frame}
\fti{Power series \& symmetric functions}

\begin{itemize}

\item Let's now talk about power series, which are crucial to the
      algebraic approach to shuffle-compatibility.

\item Consider the ring $\QQ\left[\left[x_1,x_2,x_3,\ldots\right]\right]$
      of formal power series in countably many indeterminates.

\pause

\item A formal power series $f$ is said to be \defn{bounded-degree} if
      the monomials it contains are bounded (from above) in degree.

\pause

\item A formal power series $f$ is said to be \defn{symmetric} if it
      is invariant under permutations of the indeterminates. \\
      Equivalently, if its coefficients in front of
      $x_{i_1}^{a_1} x_{i_2}^{a_2} \cdots x_{i_k}^{a_k}$ and
      $x_{j_1}^{a_1} x_{j_2}^{a_2} \cdots x_{j_k}^{a_k}$
      are equal whenever $i_1, i_2, \ldots, i_k$ are distinct and
      $j_1, j_2, \ldots, j_k$ are distinct.

\only<1-3>{\item For example:
\begin{itemize}
\item $1 + x_1 + x_2^3$ is bounded-degree but not symmetric.
\item $\left(1+x_1\right)\left(1+x_2\right)\left(1+x_3\right) \cdots $ is symmetric but not bounded-degree.
\end{itemize}}
\only<4>{\item The symmetric bounded-degree power series form a subring
$\Lambda$ of $\QQ\left[\left[x_1,x_2,x_3,\ldots\right]\right]$,
called the \defn{ring of symmetric functions} over $\QQ$.
This talk is not about it.}

\vspace{3cm}
\end{itemize}

\end{frame}

\begin{frame}
\fti{Quasisymmetric functions, part 1: definition}

\begin{itemize}

\item We shall now define the quasisymmetric functions -- a bigger algebra than $\Lambda$, but still with many of its nice properties.

\item A formal power series $f$ (still in $\QQ\left[\left[x_1,x_2,x_3,\ldots\right]\right]$) is said to be \defn{quasisymmetric} if its coefficients in front of $x_{i_1}^{a_1} x_{i_2}^{a_2} \cdots x_{i_k}^{a_k}$ and $x_{j_1}^{a_1} x_{j_2}^{a_2} \cdots x_{j_k}^{a_k}$ are equal whenever $i_1 < i_2 < \cdots < i_k$ and $j_1 < j_2 < \cdots < j_k$.

\item For example:
\begin{itemize}
\item Every symmetric power series is quasisymmetric.

\item $\sum\limits_{i<j} x_i^2 x_j = x_1^2 x_2 + x_1^2 x_3 + x_2^2 x_3 + x_1^2 x_4 + \cdots$ is quasisymmetric, but not symmetric.

\end{itemize}

\pause

\item Let \defn{$\QSym$} be the set of all quasisymmetric bounded-degree power series in $\QQ\left[\left[x_1,x_2,x_3,\ldots\right]\right]$. This is a $\QQ$-subalgebra, called the \defn{ring of quasisymmetric functions} over $\QQ$. (Gessel, 1980s.) \pause

\only<3>{
\item We have $\Lambda \subseteq \QSym \subseteq \QQ\left[\left[x_1,x_2,x_3,\ldots\right]\right]$.
}
\pause

\item The $\QQ$-vector space $\QSym$ has several combinatorial bases.
      We will use two of them: the monomial basis and the
      fundamental basis.

\end{itemize}

\end{frame}

\begin{frame}
\fti{Quasisymmetric functions, part 2: the monomial basis}

\begin{itemize}

\item For every composition
      $\alpha = \left(\alpha_1, \alpha_2, \ldots, \alpha_k\right)$, define
\begin{align*}
M_\alpha &= \sum\limits_{i_1 < i_2 < \cdots < i_k} x_{i_1}^{\alpha_1} x_{i_2}^{\alpha_2} \cdots x_{i_k}^{\alpha_k} \\
&= \text{sum of all monomials whose nonzero exponents } \\
& \qquad \text{are } \alpha_1, \alpha_2, \ldots, \alpha_k \text{ in \textbf{this} order}.
\end{align*}
This is a homogeneous power series of degree \defn{$\abs{\alpha}$} (the \defn{size} of $\alpha$, defined by $\abs{\alpha} := \alpha_1 + \alpha_2 + \cdots + \alpha_k$).

\only<1>{\item Examples:
\begin{itemize}
\item $M_{\left(\right)} = 1$.
\item $M_{\left(1,1\right)} = \sum\limits_{i<j} x_i x_j = x_1 x_2 + x_1 x_3 + x_2 x_3 + x_1 x_4 + x_2 x_4 + \cdots$.
\item $M_{\left(2,1\right)} = \sum\limits_{i<j} x_i^2 x_j = x_1^2 x_2 + x_1^2 x_3 + x_2^2 x_3 + \cdots$.
\item $M_{\left(3\right)} = \sum\limits_i x_i^3 = x_1^3 + x_2^3 + x_3^3 + \cdots$.
\end{itemize}
%Note: $m_{\left(2,1\right)} = M_{\left(2,1\right)} + M_{\left(1,2\right)}$.
}
\only<2-3>{
\item The family $\left(M_\alpha\right)_{\alpha \text{ is a composition}}$ is a basis of the $\QQ$-vector space $\QSym$, called the \defn{monomial basis} (or $M$-basis).
}

\end{itemize}

\vspace{4cm}

\end{frame}

\begin{frame}
\fti{Quasisymmetric functions, part 3: the fundamental basis}

\begin{itemize}

\item For every composition
      $\alpha = \left(\alpha_1, \alpha_2, \ldots, \alpha_k\right)$, define
\begin{align*}
F_\alpha &= \sum\limits_{\substack{i_1 \leq i_2 \leq \cdots \leq i_n;\\ i_j < i_{j+1} \text{ for all } j \in \Des\alpha}} x_{i_1} x_{i_2} \cdots x_{i_n} \\
&= \sum\limits_{\substack{\beta \text{ is a composition of } n; \\ \Des \beta \supseteq \Des \alpha}} M_\beta , \qquad \text{where } n = \abs{\alpha} .
\end{align*}
This is a homogeneous power series of degree $\abs{\alpha}$ again.

\only<1>{\item Examples:
\begin{itemize}
\item $F_{\left(\right)} = 1$.
\item $F_{\left(1,1\right)} = \sum\limits_{i<j} x_i x_j = x_1 x_2 + x_1 x_3 + x_2 x_3 + x_1 x_4 + x_2 x_4 + \cdots$.
\item $F_{\left(2,1\right)} = \sum\limits_{i \leq j < k} x_i x_j x_k$.
\item $F_{\left(3\right)} = \sum\limits_{i \leq j \leq k} x_i x_j x_k$.
\end{itemize}
}
\only<2>{
\item The family $\left(F_\alpha\right)_{\alpha \text{ is a composition}}$ is a basis of the $\QQ$-vector space $\QSym$, called the \defn{fundamental basis} (or $F$-basis). \\
Sometimes, $F_\alpha$ is also denoted $L_\alpha$.
}

\end{itemize}

\vspace{4cm}

\end{frame}

\begin{frame}
\fti{The product formula for the $F_\alpha$}

\begin{itemize}

\item What connects $\QSym$ with shuffles of permutations is the following fact:
      \\
      \textbf{Theorem.} If $\pi$ and $\sigma$ are two disjoint permutations,
      then
      \[
      F_{\Comp \pi} \cdot F_{\Comp \sigma}
      = \sum\limits_{\tau \in S\tup{\pi, \sigma}} F_{\Comp \tau} .
      \]

\pause

\item This theorem yields that $\Des$ is shuffle-compatible. Why?
      \pause
      \begin{itemize}
      \item
      \only<3>{
      Let $\pi, \pi', \sigma, \sigma'$ be permutations with
      $\abs{\pi} = \abs{\pi'}$ and $\abs{\sigma} = \abs{\sigma'}$
      and $\Des\pi = \Des\pi'$ and $\Des\sigma = \Des\sigma'$. \\
      We must prove that
      \begin{align*}
      &\set{ \Des\tau \mid \tau\in S\tup{\pi, \sigma} }_{\text{multiset}} \\
      =
      &\set{ \Des\tau \mid \tau\in S\tup{\pi', \sigma'} }_{\text{multiset}} .
      \end{align*}
      }
      \only<4-7>{
      Let $\pi, \pi', \sigma, \sigma'$ be permutations with
      $\Comp\pi = \Comp\pi'$ and $\Comp\sigma = \Comp\sigma'$. \\
      We must prove that
      }
      \only<4>{
      \begin{align*}
      &\set{ \Comp\tau \mid \tau\in S\tup{\pi, \sigma} }_{\text{multiset}} \\
      =
      &\set{ \Comp\tau \mid \tau\in S\tup{\pi', \sigma'} }_{\text{multiset}}
      \end{align*}
      (this is equivalent to what we just said,
      since $\Comp \pi$ encodes the same
      data as $\Des \pi$ and $\abs{\pi}$ together).
      }
      \only<5>{
      \[
      \sum\limits_{\tau \in S\tup{\pi, \sigma}} F_{\Comp \tau}
      = \sum\limits_{\tau \in S\tup{\pi', \sigma'}} F_{\Comp \tau} 
      \]
      (this is equivalent to what we just said, since the
      $F_\alpha$ for $\alpha$ ranging over all compositions
      are linearly independent).
      }
      \only<6-7>{
      \[
      F_{\Comp \pi} \cdot F_{\Comp \sigma}
      = F_{\Comp \pi'} \cdot F_{\Comp \sigma'}
      \]
      (this is equivalent to what we just said, by the Theorem
      above).
      }
      \only<7>{
      \\ But this follows from assumptions.
      }
      \end{itemize}

\end{itemize}
\vspace{10cm}
\end{frame}

\begin{frame}
\fti{Shuffle-compatibility of $\des$}

\begin{itemize}

\only<1>{
\item The same technique works for some other statistics.
      For example, we can show that $\des$ is shuffle-compatible.
}

\pause

\item For any $n \in \NN$ and $k \in \NN$, define the polynomial
      \[
      f_{n, k} = x^n \dbinom{p - k + n}{n} \in \QQ\ive{p, x}.
      \]

\pause

\item \textbf{Corollary (of preceding Theorem).}
      If $\pi$ and $\sigma$ are two disjoint permutations,
      with $n = \abs{\pi}$ and $m = \abs{\sigma}$,
      then
      \[
      f_{n, \des \pi} \cdot f_{m, \des \sigma}
      = \sum\limits_{\tau \in S\tup{\pi, \sigma}} f_{n+m, \des \tau} .
      \]

\pause

\only<3-4>{
\item \textbf{Proof idea (from Gessel/Zhang).}
      There is a $\QQ$-algebra homomorphism $\QSym \to \QQ\ive{p, x}$
      sending each $g \in \QSym$ to
      $g \tup{ \underbrace{x, x, \ldots, x}_{p \text{ times}} ,
                       0, 0, 0, \ldots }$
      (yes, this can be made sense of).
      \only<4>{
      This is a variant of the \defn{(generic) principal specialization}.
      }
      \pause
      \\ Check that it sends $F_{\Comp \pi}$ to $f_{n, \des \pi}$
      for any $n$-permutation $\pi$.
      Then, apply it to the preceding Theorem.
}

\only<5-8>{
\item This corollary yields that $\des$ is shuffle-compatible. Why?
      \pause
      \begin{itemize}
      \item
      Let $\pi, \pi', \sigma, \sigma'$ be permutations with
      $\abs{\pi} = \abs{\pi'}$ and $\abs{\sigma} = \abs{\sigma'}$
      and $\des\pi = \des\pi'$ and $\des\sigma = \des\sigma'$. \\
      We must prove that
      \only<5>{
      \begin{align*}
      &\set{ \des\tau \mid \tau\in S\tup{\pi, \sigma} }_{\text{multiset}} \\
      =
      &\set{ \des\tau \mid \tau\in S\tup{\pi', \sigma'} }_{\text{multiset}} .
      \end{align*}
      }
      \only<6>{
      \[
      \sum\limits_{\tau \in S\tup{\pi, \sigma}} f_{n+m, \des \tau}
      = \sum\limits_{\tau \in S\tup{\pi', \sigma'}} f_{n+m, \des \tau} ,
      \]
      where $n = \abs{\pi} = \abs{\pi'}$ and $m = \abs{\sigma} = \abs{\sigma'}$
      (this is equivalent to what we just said, since the
      $f_{n, k}$ for $n, k \in \NN$
      are linearly independent).
      }
      \only<7-8>{
      \[
      f_{n, \des \pi} \cdot f_{m, \des \sigma}
      = f_{n, \des \pi'} \cdot f_{m, \des \sigma'}
      \]
      (this is equivalent to what we just said, by the Corollary
      above).
      }
      \only<8>{
      \\ But this follows from assumptions.
      }
      \end{itemize}
}

\end{itemize}
\vspace{10cm}
\end{frame}

\begin{frame}
\fti{The kernel criterion for shuffle-compatibility, 1}

\begin{itemize}

\item The above arguments can be abstracted into a general
      criterion for shuffle-compatibility of a descent statistic
      (Gessel and Zhuang, in {\red \arxiv{1706.00750v2}},
      Section 4.1).
      \\
      $\QSym$ and $\QQ\ive{p, x}$ get replaced by a ``shuffle
      algebra'' with an algebra homomorphism from $\QSym$.

\item We shall give our own variant of the criterion.

\end{itemize}

\end{frame}

\begin{frame}
\fti{The kernel criterion for shuffle-compatibility, 2}

\begin{itemize}

\item If $\st$ is a descent statistic, then two compositions
      $\alpha$ and $\beta$ are said to be \defn{$\st$-equivalent}
      if $\abs{\alpha} = \abs{\beta}$ and $\st\alpha = \st\beta$.
      \\ (Remember: $\st\alpha$ means $\st\pi$ for any permutation
      $\pi$ satisfying $\Comp\pi = \alpha$.)

\pause

\item The \defn{kernel $\calK_{\st}$} of a descent statistic $\st$
      is the $\QQ$-vector subspace of $\QSym$ spanned by all
      differences of the form $F_\alpha - F_\beta$, with $\alpha$
      and $\beta$ being two $\st$-equivalent compositions:
      \[
      \calK_{\st} = \left< F_\alpha - F_\beta \ 
                              \mid \ \abs{\alpha} = \abs{\beta} \text{ and }
                                   \st \alpha = \st \beta \right>_\QQ .
      \]

\pause

\item \textbf{Theorem.} The descent statistic $\st$ is
      shuffle-compatible if and only if $\calK_{\st}$ is an
      ideal of $\QSym$.

\end{itemize}

\end{frame}

\begin{frame}
\fti{Section 3}
\begin{center}
{\LARGE \bf Section 3} \\
\noindent\rule[0.5ex]{\linewidth}{1pt}
{\Large \bf The exterior peak set}
\end{center}
\vspace{1cm}
References:
\begin{itemize}
\item \href{https://github.com/darijgr/gzshuf}{\red Darij Grinberg, \textit{Shuffle-compatible permutation statistics II: the exterior peak set}, draft}.
\item \href{http://www.ams.org/journals/tran/1997-349-02/S0002-9947-97-01804-7/}{\red John R. Stembridge, \textit{Enriched P-partitions}, Trans. Amer. Math. Soc. 349 (1997), no. 2, pp. 763--788}.
\item \href{https://doi.org/10.1016/j.aim.2006.05.0160}{\red T. Kyle Petersen, \textit{Enriched P-partitions and peak algebras}, Adv. in Math. 209 (2007), pp. 561--610}.
\end{itemize}
\end{frame}

\begin{frame}
\fti{Roadmap to $\Epk$}

\begin{itemize}

\item We will now outline our proof that $\Epk$ is shuffle-compatible.

\item The main idea is to imitate the above proof for $\Des$,
      but instead of $F_{\Comp \pi}$ we'll now have
      some different power series (not in $\QSym$).

\pause

\item The idea is not new.
      This is how $\Pk$, $\Lpk$ and $\Rpk$ were proven
      shuffle-compatible.

\pause

\item The main tool is the concept of
      \textbf{$\mathcal{Z}$-enriched $P$-partitions}:
      a generalization of
      \begin{itemize}
      \item $P$-partitions (Stanley 1972);
      \item enriched $P$-partitions (Stembridge 1997);
      \item left enriched $P$-partitions (Petersen 2007),
      \end{itemize}
      which are used in the proofs for $\Des$, $\Pk$
      and $\Lpk$, respectively. \pause
      \\ (Yes, the $F_{\Comp \pi} \cdot F_{\Comp \sigma}$
      theorem we used in proving $\Des$ follows from the
      theory of $P$-partitions.)

\pause

\item The idea is simple, but the proof has technical
      parts I am not showing.

\end{itemize}

\end{frame}

\begin{frame}
\fti{Labeled posets}

\begin{itemize}

\item A \defn{labeled poset} means a pair $\left(  P,\gamma\right)  $ consisting
of a finite poset $P=\left(  X,\leq\right)  $ and an injective map
$\gamma:X\rightarrow A$ into some totally ordered set $A$. The injective map
$\gamma$ is called the \defn {labeling} of the labeled poset $\left(
P,\gamma\right)  $.

\end{itemize}

\end{frame}

\begin{frame}
\fti{$\calN$ and $\calZ$: definitions}

\begin{itemize}

\item Fix a totally ordered set \defn{$\mathcal{N}$}, and denote its strict order
relation by \defn{$\prec$}.

\item Let \defn{$+$} and \defn{$-$} be two distinct symbols. \\
Let \defn{$\mathcal{Z}$} be a subset of the set $\mathcal{N}\times\left\{  +,-\right\}
$.

\item \textbf{Intuition:} $\calN$ is a set of letters that will index our
      indeterminates. \\
      $\calZ$ is a set of ``signed letters'', which are pairs of a letter
      in $\calN$ and a sign in $\set{+,-}$.
      (Not all such pairs must lie in $\calZ$.)

\pause

% \item For each $q=\left(  n,s\right)  \in\mathcal{Z}$, we denote the element
% $n\in\mathcal{N}$ by \defn{$\left\vert q\right\vert $}, and we call the element
% $s\in\left\{  +,-\right\}  $ the \textit{sign} of $q$.

\item If $n\in\mathcal{N}$,
then we will denote the two elements $\left(  n,+\right)  $ and $\left(
n,-\right)  $ of $\mathcal{N}\times\left\{  +,-\right\}  $ by \defn{$+n$} and \defn{$-n$}, respectively.

\pause

\item Let us totally order the set $\mathcal{Z}$ in such a way that the (strict)
order relation $\prec$ satisfies%
\begin{align*}
\left(  n,s\right)  &\prec\left(  n^{\prime},s^{\prime}\right)  \text{ if and
only if either }n\prec n^{\prime} \\
& \qquad \qquad \text{ or }\left(  n=n^{\prime}\text{ and
}s=-\text{ and }s^{\prime}=+\right)  .
\end{align*}

\pause

\item Let $\operatorname*{Pow}\mathcal{N}$ be the ring of all power series over
$\mathbb{Q}$ in the indeterminates $x_{n}$ for $n\in\mathcal{N}$.

\end{itemize}

\end{frame}

\begin{frame}
\fti{$\calN$ and $\calZ$: example}

\begin{itemize}

\item For an example of the setting just introduced, take
$\calN = \NN$ with $\prec$ being the usual order. Then,
\[
\calZ \subseteq \NN \times \set{+, -}
= \set{ -0, +0, -1, +1, -2, +2, \ldots } .
\]
Note: $-0 \neq +0$, since these are shorthands for pairs, not
numbers.

\pause

\item The total order $\prec$ on $\calZ$ is the restriction of
\[
-0 \prec +0 \prec -1 \prec +1 \prec -2 \prec +2 \prec \cdots .
\]

\pause

\item $\operatorname*{Pow}\mathcal{N}
= \QQ\ive{\ive{x_0, x_1, x_2, \ldots}}$.

\end{itemize}

\end{frame}

\begin{frame}
\fti{$\calZ$-enriched $\tup{P,\gamma}$-partitions: definition}

\begin{itemize}

\item Now, let $\left(  P,\gamma\right)  $ be a labeled poset. A
\defn{$\mathcal{Z}$-enriched $\left(  P,\gamma\right)  $-partition}
means a map $f:P\rightarrow\mathcal{Z}$ such that for all $x<y$ in $P$, the
following conditions hold:

\begin{enumerate}
\item[\textbf{(i)}] We have $f\left(  x\right)  \preccurlyeq f\left(
y\right)  $.

\item[\textbf{(ii)}] If $f\left(  x\right)  =f\left(  y\right)  =+n$ for some
$n\in\mathcal{N}$, then $\gamma\left(  x\right)  <\gamma\left(  y\right)  $.

\item[\textbf{(iii)}] If $f\left(  x\right)  =f\left(  y\right)  =-n$ for some
$n\in\mathcal{N}$, then $\gamma\left(  x\right)  >\gamma\left(  y\right)  $.
\end{enumerate}

(Keep in mind: $\calN$ and $\calZ$ are fixed.)

\pause

\item \textbf{(Attempt at) intuition:}
      A $\calZ$-enriched $\tup{P,\gamma}$-partition is a map
      $f : P \to \calZ$ (that is, assigning a signed letter to each poset
      element) which
      \begin{enumerate}
      \item[\textbf{(i)}] is weakly increasing on $P$;
      \item[\textbf{(ii) + (iii)}] is occasionally strictly increasing,
                                   when $\gamma$ and the sign of the
                                   $f$-value ``are out of alignment''.
      \end{enumerate}

\end{itemize}

\end{frame}

\begin{frame}
\fti{$\calZ$-enriched $\tup{P,\gamma}$-partitions: example}

\begin{itemize}

\item Let $P$ be the poset with the following Hasse
diagram:%
\[
\xymatrix@R=1pc{
& b \are[dl] \are[dr] \\
c \are[dr] & & d \are[dl] \\
& a
}%
%EndExpansion
\]
and let $\gamma:P\rightarrow\mathbb{Z}$ be a labeling that satisfies
$\gamma\left(  a\right)  <\gamma\left(  b\right)  <\gamma\left(  c\right)
<\gamma\left(  d\right)  $ (for example, $\gamma$ could be the map that sends
$a,b,c,d$ to $2,3,5,7$, respectively). Then, a $\mathcal{Z}$-enriched $\left(
P,\gamma\right)  $-partition is a map $f:P\rightarrow\mathcal{Z}$ satisfying
the following conditions:

\begin{enumerate}
\item[\textbf{(i)}] We have $f\left(  a\right)  \preccurlyeq f\left(
c\right)  \preccurlyeq f\left(  b\right)  $ and $f\left(  a\right)
\preccurlyeq f\left(  d\right)  \preccurlyeq f\left(  b\right)  $.

\item[\textbf{(ii)}] We cannot have $f\left(  c\right)  =f\left(  b\right)
=+n$ with $n\in\mathcal{N}$. Also, we cannot have $f\left(  d\right)
=f\left(  b\right)  =+n$ with $n\in\mathcal{N}$.

\item[\textbf{(iii)}] We cannot have $f\left(  a\right)  =f\left(  c\right)
=-n$ with $n\in\mathcal{N}$. Also, we cannot have $f\left(  a\right)
=f\left(  d\right)  =-n$ with $n\in\mathcal{N}$.
\end{enumerate}

% For example, if $\mathcal{N}=\mathbb{P}$ (the totally ordered set of positive
% integers, with its usual ordering) and $\mathcal{Z}=\mathcal{N}\times\left\{
% +,-\right\}  $, then the map $f:P\rightarrow\mathcal{Z}$ sending $a,b,c,d$ to
% $+2,-3,+2,-3$ (respectively) is a $\mathcal{Z}$-enriched $\left(
% P,\gamma\right)  $-partition. Notice that the total ordering on $\mathcal{Z}$
% in this case is given by%
% \[
% -1\prec+1\prec-2\prec+2\prec-3\prec+3\prec\cdots,
% \]
% rather than by the familiar total order on $\mathbb{Z}$.

\end{itemize}

\end{frame}

\begin{frame}
\fti{$\calZ$-enriched $\tup{P,\gamma}$-partitions: revisiting the literature}

\begin{itemize}

\item Consider again the case when $\calN = \NN$ with $\prec$ being the usual order.
Let us see what $\calZ$-enriched $\tup{P,\gamma}$-partitions are,
depending on $\calZ$.

\pause

\item If $\calZ = \NN \times \set{+} = \set{+0 \prec +1 \prec +2 \prec \cdots}$,
then the $\calZ$-enriched $\tup{P,\gamma}$-partitions are just
the (usual) $\tup{P,\gamma}$-partitions into $\NN$ (up to renaming
$n$ as $+n$).

\pause
\item If $\calZ = \NN \times \set{+,-}
= \set{-0 \prec +0 \prec -1 \prec +1 \prec -2 \prec +2 \prec \cdots}$, then the
$\calZ$-enriched $\tup{P,\gamma}$-partitions are
Stembridge's enriched $\tup{P,\gamma}$-partitions (up to renaming
$n$ as $n-1$).

\pause
\item If $\calZ = \tup{\NN \times \set{+,-}} \setminus \set{-0}
= \set{+0 \prec -1 \prec +1 \prec -2 \prec +2 \prec \cdots}$, then the
$\calZ$-enriched $\tup{P,\gamma}$-partitions are
Petersen's left enriched $\tup{P,\gamma}$-partitions.

\pause

\item We shall later focus on the case when
$\calN = \NN \cup \set{\infty}$ and
$\calZ = \tup{\calN \times \set{+,-}} \setminus \set{-0, +\infty}$.

\end{itemize}

\end{frame}

\begin{frame}
\fti{$\calE\tup{P,\gamma}$ and $\calL\tup{P}$}

\begin{itemize}

\item A few more notations are needed.

\item If $\left(  P,\gamma\right)  $ is a labeled poset, then
\defn{$\calE\tup{P,\gamma}$}
shall denote the set of all $\mathcal{Z}$-enriched $\left(
P,\gamma\right)  $-partitions.

\pause

\item If $P$ is any poset, then \defn{$\calL\tup{P}$} shall denote the set
of all linear extensions of $P$. \\
A linear extension of $P$ shall be understood
simultaneously as a totally ordered set extending $P$ and as a list $\left(
w_{1},w_{2},\ldots,w_{n}\right)  $ of all elements of $P$ such that no two
integers $i<j$ satisfy $w_{i}\geq w_{j}$ in $P$.

\end{itemize}

\end{frame}

\begin{frame}
\fti{Any $\calE\tup{P,\gamma}$-partition has its favorite linear extension}

\begin{itemize}

\item \textbf{Proposition.} For any labeled poset $\left(  P,\gamma\right)  $, we
have
\[
\mathcal{E}\left(  P,\gamma\right)  =\bigsqcup_{w\in\mathcal{L}\left(
P\right)  }\mathcal{E}\left(  w,\gamma\right)  .
\]

\item This is a generalization of a standard result on $P$-partitions
      (``Stanley's main lemma''), and is proven by the same reasoning.

\end{itemize}

\end{frame}

\begin{frame}
\fti{The power series $\Gamma_{\calZ}\tup{P,\gamma}$}

\begin{itemize}

\item Let $\left(  P,\gamma\right)  $ be a labeled poset. We
define a power series \defn{$\Gamma_{\mathcal{Z}}\left(  P,\gamma\right)
\in\operatorname*{Pow}\mathcal{N}$} by
\[
\Gamma_{\mathcal{Z}}\left(  P,\gamma\right)  =\sum_{f\in\mathcal{E}\left(
P,\gamma\right)  }\prod_{p\in P}x_{\left\vert f\left(  p\right)  \right\vert
}.
\]
Here, \defn{$\abs{f\tup{p}} \in \calN$} is defined to be the first entry of
$f\tup{p}$ (recall: $f\tup{p}$ is a pair of an element of $\calN$
and a sign in $\set{+,-}$).

\pause

\item This generalizes the classical quasisymmetric $P$-partition
enumerators (which give the fundamental basis $F_\alpha$ when
$P$ is totally ordered).

\pause

\only<3>{
\item \textbf{Corollary.} For any labeled poset $\left(  P,\gamma\right)  $, we have%
\[
\Gamma_{\mathcal{Z}}\left(  P,\gamma\right)  =\sum_{w\in\mathcal{L}\left(
P\right)  }\Gamma_{\mathcal{Z}}\left(  w,\gamma\right)  .
\]
}

\only<4>{
\item \textbf{Question.} Where do these $\Gamma_{\calZ}\tup{P,\gamma}$ live
(other than in $\operatorname{Pow}\calN$) ? \\
I don't know a good answer; it should be a generalization of $\QSym$.
\\
Jia Huang's work ({\red \arxiv{1506.02962v2}}) looks relevant.
}

\end{itemize}

\vspace{5cm}

\end{frame}

\begin{frame}
\fti{Disjoint unions give product of $\Gamma$'s}

\begin{itemize}

\item Let $P$ be any set. Let $A$ be a totally ordered set. Let $\gamma:P\rightarrow
A$ and $\delta:P\rightarrow A$ be two maps. We say that $\gamma$ and $\delta$
are \defn{order-equivalent} if the following holds: For every pair $\left(
p,q\right)  \in P\times P$, we have $\gamma\left(  p\right)  \leq\gamma\left(
q\right)  $ if and only if $\delta\left(  p\right)  \leq\delta\left(
q\right)  $.

\pause

\item \textbf{Proposition.} Let $\left(  P,\gamma\right)  $ and $\left(  Q,\delta
\right)  $ be two labeled posets. Let $\left(  P\sqcup Q,\varepsilon\right)  $
be the labeled poset
\begin{itemize}
\item for which $P\sqcup Q$ is the disjoint union of
$P$ and $Q$, and
\item whose labeling $\varepsilon$ is such that the restriction of
$\varepsilon$ to $P$ is order-equivalent to $\gamma$ and such that the
restriction of $\varepsilon$ to $Q$ is order-equivalent to $\delta$.
\end{itemize}
Then,%
\[
\Gamma_{\mathcal{Z}}\left(  P,\gamma\right)  \cdot
\Gamma_{\mathcal{Z}}\left(
Q,\delta\right)  =\Gamma_{\mathcal{Z}}\left(  P\sqcup Q,\varepsilon\right)  .
\]

\item Again, the proof is simple.

\end{itemize}

\end{frame}

\begin{frame}
\fti{From a permutation to a labeled poset}

\begin{itemize}

\item Let $n\in\mathbb{N}$. Write $\ive{n}$ for $\set{1,2,\ldots,n}$. \\
Let $\pi$ be any $n$-permutation.
Consider $\pi$ as an injective map $\ive{n} \to \set{1,2,3,\ldots}$
(sending $i$ to $\pi_i$).
Thus, $\left(  \left[  n\right]  ,\pi\right)$ is a labeled poset.
We define \defn{$\Gamma_{\mathcal{Z}}\left(  \pi\right)  $} to be the
power series $\Gamma_{\mathcal{Z}}\left(  \left[  n\right]  ,\pi\right)  $.

\pause

\only<2>{
\item Explicitly:
\[
\Gamma_{\calZ}\tup{\pi}
=
\sum x_{\abs{j_1}} x_{\abs{j_2}} \cdots x_{\abs{j_n}} ,
\]
where the sum is over all $n$-tuples
$\tup{j_1, j_2, \ldots, j_n} \in \calZ^n$ having the properties
that:
\begin{enumerate}
\item[\textbf{(i)}] $j_1 \preccurlyeq j_2 \preccurlyeq \cdots \preccurlyeq j_n$;
\item[\textbf{(ii)}] if $j_k = j_{k+1} = +s$ for some $s \in \calN$,
then $\pi_k < \pi_{k+1}$;
\item[\textbf{(iii)}] if $j_k = j_{k+1} = -s$ for some $s \in \calN$,
then $\pi_k > \pi_{k+1}$.
\end{enumerate}

\item This $\Gamma_{\calZ} \tup{\pi}$ will serve as an
analogue of $F_{\Comp\pi}$.
}

\pause

\item \textbf{Proposition.}
Let $w$ be a finite totally ordered set with ground
set $W$. Let $n=\left\vert W\right\vert $. Let $\overline{w}$ be the unique
poset isomorphism $w\rightarrow\left[  n\right]  $. Let $\gamma:W\rightarrow
\left\{  1,2,3,\ldots\right\}  $ be any injective map. Then, $\Gamma
_{\mathcal{Z}}\left(  w,\gamma\right)  =\Gamma_{\mathcal{Z}}\left(
\gamma\circ\overline{w}^{-1}\right)  $.

\item Again, this follows the roadmap of classical $P$-partition theory.

\pause

\item \textbf{Corollary.}
Let $\left(  P,\gamma\right)  $ be a labeled poset. Let
$n=\left\vert P\right\vert $. Then,
\[
\Gamma_{\mathcal{Z}}\left(  P,\gamma\right)  =\sum_{\substack{x:P\rightarrow
\left[  n\right]  \\\text{bijective poset}\\\text{homomorphism}}%
}\Gamma_{\mathcal{Z}}\left(  \gamma\circ x^{-1}\right)  .
\]

\pause

\item Thus, the $\Gamma_{\calZ}$ of any labeled poset can be
described in terms of the $\Gamma_{\calZ} \tup{\pi}$.

\end{itemize}

\vspace{5cm}

\end{frame}

\begin{frame}
\fti{The product formula for the $\Gamma_{\calZ}\tup{P,\gamma}$}

\begin{itemize}

\item Combining the above results, we see: \\
      \textbf{Theorem.} Let $\pi$ and $\sigma$ be two disjoint
      permutations. Then,
\[
\Gamma_{\mathcal{Z}}\left(  \pi\right)  \cdot 
\Gamma_{\mathcal{Z}} \left(
\sigma\right)  =\sum_{\tau\in S\left(  \pi,\sigma\right)  }\Gamma
_{\mathcal{Z}}\left(  \tau\right)  .
\]      

\pause

\item This generalizes the
      \[
      F_{\Comp \pi} \cdot F_{\Comp \sigma}
      = \sum\limits_{\tau \in S\tup{\pi, \sigma}} F_{\Comp \tau}
      \]
      formula in $\QSym$
      (which you can recover by setting $\calN = \NN$
      and
      $\calZ = \NN \times \set{+} = \set{+0 \prec +1 \prec +2 \prec \cdots}$).

\item Likewise, you can recover similar results by Stembridge
      and Petersen from this.

\end{itemize}

\end{frame}

\begin{frame}
\fti{Customizing the setting for $\Epk$}

\begin{itemize}

\item Remember: we want to show $\Epk$ is shuffle-compatible.

\item Specialize the above setting as follows:
\begin{itemize}
\item
      Set $\calN = \set{0, 1, 2, \ldots} \cup \set{\infty}$,
      with total order given by $0\prec1\prec2\prec\cdots \prec\infty$.
\item
      Set
      \begin{align*}
\mathcal{Z}  &  =\left(  \mathcal{N}\times\left\{  +,-\right\}  \right)
\setminus\left\{  -0,+\infty\right\} \\
&  =\left\{  +0\right\}  \cup\left\{  +n\ \mid\ n\in\left\{  1,2,3,\ldots
\right\}  \right\}  \\
& \qquad \cup\left\{  -n\ \mid\ n\in\left\{  1,2,3,\ldots\right\}
\right\}  \cup\left\{  -\infty\right\}  .
\end{align*}
Recall that the total order on $\mathcal{Z}$ has
\[
+0\prec-1\prec+1\prec-2\prec+2\prec\cdots\prec-\infty.
\]
\end{itemize}

\end{itemize}

\end{frame}

\begin{frame}
\fti{Fiber-ends}

\begin{itemize}

\item Let $n\in\mathbb{N}$. Let $g:\left[  n\right]
\rightarrow\mathcal{N}$ be any map. We define a subset
\defn{$\operatorname*{FE}\left(  g\right)  $}
of $\left[  n\right]  $ by
\begin{align*}
\operatorname*{FE}\left(  g\right)   &  =\left\{  \min\left(  g^{-1}\left(
h\right)  \right)  \ \mid\ h\in\left\{  1,2,3,\ldots,\infty\right\}  \right\}
\\
&  \ \ \ \ \ \ \ \ \ \ \cup\left\{  \max\left(  g^{-1}\left(  h\right)
\right)  \ \mid\ h\in\left\{  0,1,2,3,\ldots\right\}  \right\}
\end{align*}
(ignore the maxima/minima of empty fibers). \\
In other words, $\operatorname*{FE}\left(  g\right)  $ is the set comprising
\begin{itemize}
\item
the smallest elements of all nonempty fibers of $g$ except for $g^{-1}\left(
0\right)  $ as well as
\item the largest elements of all nonempty fibers of $g$
except for $g^{-1}\left(  \infty\right)  $.
\end{itemize}
% We shall refer to the elements of
% $\operatorname*{FE}\left(  g\right)  $ as the \defn{fiber-ends} of $g$.

\end{itemize}

\end{frame}



\begin{frame}
\fti{$K$-series}

\begin{itemize}

\item Let $n\in\mathbb{N}$. If $\Lambda$ (no connection to
symmetric functions) is any subset of $\left[
n\right]  $, then we define a power series \defn{$K_{n,\Lambda}^{\mathcal{Z}}%
\in\operatorname*{Pow}\mathcal{N}$} by%
\[
K_{n,\Lambda}^{\mathcal{Z}}=\sum_{\substack{g:\left[  n\right]  \rightarrow
\mathcal{N}\text{ is}\\\text{weakly increasing;}\\\Lambda\subseteq
\operatorname*{FE}\left(  g\right)  }}2^{\left\vert g\left(  \left[  n\right]
\right)  \cap\left\{  1,2,3,\ldots\right\}  \right\vert }
x_{g\tup{1}} x_{g\tup{2}} \cdots x_{g\tup{n}} .
\]

\item \textbf{Proposition.}
\[
\Gamma_{\mathcal{Z}}\left(  \pi\right)  =K_{n,\Epk \pi}^{\mathcal{Z}}.
\]

\only<1>{
This is proven by a counting argument (if a map $g$
comes from an $\tup{\ive{n},\pi}$-partition, then
the fibers of $g$ subdivide $\ive{n}$ into intervals
on which $\pi$ is ``V-shaped''; a peak can only occur
at a border between two such intervals).
}

\pause

\item Thus, the product formula above specializes to
\[
K_{n,\operatorname*{Epk}\pi}^{\mathcal{Z}}\cdot K_{m,\operatorname*{Epk}%
\sigma}^{\mathcal{Z}}=\sum_{\tau\in S\left(  \pi,\sigma\right)  }%
K_{n+m,\operatorname*{Epk}\tau}^{\mathcal{Z}}.
\]

\item This formula is used to show that $\Epk$ is shuffle-compatible,
      but we need a bit more:
      we need to show that the ``relevant'' $K_{n,\Lambda}^{\mathcal{Z}}$
      are linearly independent.

\pause

\item Not all $K_{n,\Lambda}^{\mathcal{Z}}$ are linearly independent.
      Rather, we need to pick the right subset.

\end{itemize}

\end{frame}


\begin{frame}
\fti{Lacunar subsets and linear independence}

\begin{itemize}

\item A set $S$ of integers is called \defn{lacunar} if it contains
      no two consecutive integers.

\item \textbf{Well-known fact:} The number of lacunar subsets of
      $\ive{n}$ is the Fibonacci number $f_{n+1}$.

\pause

\item \textbf{Lemma.}
      For each permutation $\pi$, the set $\Epk \pi$ is a nonempty
      lacunar subset of $\ive{n}$. \\
      (And conversely -- although we won't need it --,
      any such subset has the form $\Epk \pi$ for some $\pi$.)

\pause

\item \textbf{Lemma.}
      The family%
\[
\left(  K_{n,\Lambda}^{\mathcal{Z}}\right)  _{n\in\mathbb{N};\ \Lambda
\subseteq\left[  n\right]  \text{ is lacunar and nonempty}}%
\]
is $\mathbb{Q}$-linearly independent.

\item This actually takes work to prove. But once proven, it completes
      the argument for the shuffle-compatibility of $\Epk$.

\end{itemize}

\end{frame}


\begin{frame}
\fti{The kernel $\calK_{\Epk}$}

\begin{itemize}

\item Recall: The \defn{kernel $\calK_{\st}$} of a descent statistic $\st$
      is the $\QQ$-vector subspace of $\QSym$ spanned by all
      differences of the form $F_\alpha - F_\beta$, with $\alpha$
      and $\beta$ being two $\st$-equivalent compositions:
      \[
      \calK_{\st} = \left< F_\alpha - F_\beta \ 
                              \mid \ \abs{\alpha} = \abs{\beta} \text{ and }
                                   \st \alpha = \st \beta \right>_\QQ .
      \]

\pause

\item Since $\Epk$ is shuffle-compatible, its kernel $\calK_{\Epk}$
      is an ideal of $\QSym$.
      How can we describe it?

\item Two ways: using the $F$-basis and using the $M$-basis.

\end{itemize}

\end{frame}


\begin{frame}
\fti{The kernel $\calK_{\Epk}$ in terms of the $F$-basis}

\begin{itemize}

\item If $J=\left(  j_{1},j_{2},\ldots,j_{m}\right)  $ and $K$
are two compositions, then we write \defn{$J\rightarrow K$} if there exists an
$\ell\in\left\{  2,3,\ldots,m\right\}  $ such that $j_{\ell}>2$ and $K=\left(
j_{1},j_{2},\ldots,j_{\ell-1},1,j_{\ell}-1,j_{\ell+1},j_{\ell+2},\ldots
,j_{m}\right)  $. \\
(In other words, we write $J\rightarrow K$ if $K$ can be
obtained from $J$ by \textquotedblleft splitting\textquotedblright\ some
non-initial entry
$j_{\ell}>2$ into two consecutive entries $1$ and $j_{\ell}-1$.)

\item \textbf{Example.} Here are all instances of the $\to$ relation
on compositions of size $\leq 5$:
\begin{align*}
\left(  1,3\right)  &\rightarrow\left(  1,1,2\right)  , \qquad
\left(  1,4\right)    \rightarrow\left(  1,1,3\right)  ,\\
\left(  1,3,1\right)   &\rightarrow\left(  1,1,2,1\right)  ,\qquad
\left(  1,1,3\right)   \rightarrow\left(  1,1,1,2\right)  ,\\
\left(  2,3\right)  & \rightarrow\left(  2,1,2\right)  .
\end{align*}

\item \textbf{Proposition.}
The ideal $\mathcal{K}_{\operatorname*{Epk}}$ of $\operatorname*{QSym}$ is
spanned (as a $\mathbb{Q}$-vector space) by all differences of the form
$F_{J}-F_{K}$, where $J$ and $K$ are two compositions satisfying $J\rightarrow
K$.

\end{itemize}

\end{frame}

\begin{frame}
\fti{The kernel $\calK_{\Epk}$ in terms of the $M$-basis}

\begin{itemize}

\item If $J=\left(  j_{1},j_{2},\ldots,j_{m}\right)  $ and $K$
are two compositions, then we write \defn{$J\underset{M}{\rightarrow}K$} if
there exists an $\ell\in\left\{  2,3,\ldots,m\right\}  $ such that $j_{\ell
}>2$ and $K=\left(  j_{1},j_{2},\ldots,j_{\ell-1},2,j_{\ell}-2,j_{\ell
+1},j_{\ell+2},\ldots,j_{m}\right)  $. (In other words, we write
$J\underset{M}{\rightarrow}K$ if $K$ can be obtained from $J$ by
\textquotedblleft splitting\textquotedblright\ some non-initial entry
$j_{\ell}>2$ into
two consecutive entries $2$ and $j_{\ell}-2$.)

\item \textbf{Example.} Here are all instances of the $\underset{M}{\rightarrow}$ relation
on compositions of size $\leq 5$:
\begin{align*}
\left(  1,3\right)  &\underset{M}{\rightarrow}\left(  1,2,1\right)  ,\qquad
\left(  1,4\right)    \underset{M}{\rightarrow}\left(  1,2,2\right)  ,\\
\left(  1,3,1\right)   &  \underset{M}{\rightarrow}\left(  1,2,1,1\right)  ,\qquad
\left(  1,1,3\right)    \underset{M}{\rightarrow}\left(  1,1,2,1\right)  ,\\
\left(  2,3\right)   &  \underset{M}{\rightarrow}\left(  2,2,1\right)  .
\end{align*}

\item \textbf{Proposition.}
The ideal $\mathcal{K}_{\operatorname*{Epk}}$ of $\operatorname*{QSym}$ is
spanned (as a $\mathbb{Q}$-vector space) by all sums of the form $M_{J}+M_{K}%
$, where $J$ and $K$ are two compositions satisfying
$J\underset{M}{\rightarrow}K$.

\end{itemize}

\end{frame}

\begin{frame}
\fti{What about other statistics?}

\begin{itemize}

\item \textbf{Question.}
      Do other descent statistics allow for similar descriptions of
      $\calK_{\st}$ ?

\end{itemize}

\end{frame}

\begin{frame}
\fti{Section 4}
\begin{center}
{\LARGE \bf Section 4} \\
\noindent\rule[0.5ex]{\linewidth}{1pt}
{\Large \bf Left-/right-shuffle-compatibility}
\end{center}
\vspace{1cm}
References:
\begin{itemize}
\item \href{https://github.com/darijgr/gzshuf}{\red Darij Grinberg, \textit{Shuffle-compatible permutation statistics II: the exterior peak set}, draft}.
\item \href{http://www.cip.ifi.lmu.de/~grinberg/algebra/dimcreation.pdf}{\red Darij Grinberg, \textit{Dual immaculate creation operators and a dendriform algebra structure on the quasisymmetric functions}, Canad. J. Math. 69 (2017), pp. 21--53.}
\end{itemize}
\end{frame}

\begin{frame}
\fti{Left/right-shuffle-compatibility (repeated)}

\begin{itemize}

\item We further begin the study of a finer version of
      shuffle-compatibility: ``left/right-shuffle-compatibility''.

\item Given two disjoint nonempty permutations $\pi$ and $\sigma$,
      \begin{itemize}
      \item
      a \defn{left shuffle} of $\pi$ and $\sigma$ is a shuffle
      of $\pi$ and $\sigma$ that starts with a letter of $\pi$;
      \item
      a \defn{right shuffle} of $\pi$ and $\sigma$ is a shuffle
      of $\pi$ and $\sigma$ that starts with a letter of $\sigma$.
      \end{itemize}

\item We let $S_\prec \tup{\pi, \sigma}$ be the set of all left shuffles of $\pi$
      and $\sigma$. \\
      We let $S_\succ \tup{\pi, \sigma}$ be the set of all right shuffles of $\pi$
      and $\sigma$.

\item A statistic $\st$ is said to be \defn{left-shuffle-compatible}
      if for any two disjoint nonempty permutations $\pi$ and $\sigma$
      such that
      \[
      \text{the first entry of }\pi\text{ is greater than the first entry of }%
      \sigma,
      \]
      the
      multiset
      \[
      \set{ \st\tau \mid \tau\in S_\prec \tup{\pi, \sigma} }_{\text{multiset}}
      \]
      depends only on $\st \pi$, $\st \sigma$, $\abs{\pi}$ and
      $\abs{\sigma}$. \\
      %Similarly define \defn{right-shuffle-compatible}.

\item We show that $\Des$, $\des$, $\Lpk$ and $\Epk$ are
      left- and right-shuffle-compatible.
      (But not $\maj$ or $\Rpk$.)

\end{itemize}

\end{frame}

\begin{frame}
\fti{Dendriform structure on $\QSym$, introduction}

\begin{itemize}

\item This proof will use a \textbf{dendriform algebra} structure on $\QSym$,
      as well as two other operations and a bit of the Hopf algebra
      structure. \\
      I don't know of a combinatorial proof.

\item This structure first appeared in: \\
      {\red \href{http://www.cip.ifi.lmu.de/~grinberg/algebra/dimcreation.pdf}{\red Darij Grinberg, \textit{Dual immaculate creation operators and a dendriform algebra structure on the quasisymmetric functions}, Canad. J. Math. 69 (2017), pp. 21--53.}}
      \\
      But the ideas go back to:
      \begin{itemize}
      \item
      {\red \href{http://www.sciencedirect.com/science/article/pii/0001870877900421}{Gl\^anffrwd P. Thomas, \textit{Frames, Young tableaux, and Baxter sequences}, Advances in Mathematics, Volume 26, Issue 3, December 1977, Pages 275--289}}.
      \item
      {\red \href{http://arxiv.org/abs/math/0510218}{Jean-Christophe Novelli, Jean-Yves Thibon,
\textit{Construction of dendriform trialgebras}, arXiv:math/0510218}}.
      \end{itemize}
      Something similar also appeared in:
      {\red \href{https://doi.org/10.1016/j.jpaa.2009.06.001}{Aristophanes Dimakis, Folkert M\"uller-Hoissen,
      \textit{Quasi-symmetric functions and the KP hierarchy},
      Journal of Pure and Applied Algebra, Volume 214, Issue 4, April 2010, Pages 449--460}}.

\end{itemize}

\vspace{9cm}

\end{frame}

\begin{frame}
\fti{Dendriform structure on $\QSym$, part 1}

\begin{itemize}

\item For any monomial $\mathfrak{m}$, let $\Supp \mathfrak{m}$
denote the set $\left\{i \mid x_i \text{ appears in } \mathfrak{m}\right\}$.

\item \textbf{Example.} $\Supp \left(x_3^5 x_6 x_8\right) = \left\{3,6,8\right\}$.

\only<2>{
\item We define a binary operation $\left.  \prec\right.$ on the
$\QQ$-vector space $\Powser$ as follows:

\begin{itemize}
\item On monomials, it should be given by
\[
\mathfrak{m}\left.  \prec\right.  \mathfrak{n}=\left\{
\begin{array}
[c]{c}%
\mathfrak{m}\cdot\mathfrak{n},\ \ \ \ \ \ \ \ \ \ \text{if }\min\left(
\operatorname*{Supp}\mathfrak{m}\right)  <\min\left(  \operatorname*{Supp}%
\mathfrak{n}\right)  ;\\
0,\ \ \ \ \ \ \ \ \ \ \text{if }\min\left(  \operatorname*{Supp}%
\mathfrak{m}\right)  \geq\min\left(  \operatorname*{Supp}\mathfrak{n}\right)
\end{array}
\right.
\]
for any two monomials $\mathfrak{m}$ and $\mathfrak{n}$.
\item It should be $\QQ$-bilinear.
\item It should be continuous (i.e., its $\QQ$-bilinearity also
applies to infinite $\QQ$-linear combinations).
\end{itemize}

\item Well-definedness is pretty clear.

\item \textbf{Example.} $\left(x_2^2 x_4\right) \left.\prec\right. \left(x_3^2 x_5\right) = x_2^2 x_3^2 x_4 x_5$, but
$\left(x_2^2 x_4\right) \left.\prec\right. \left(x_2^2 x_5\right) = 0$.
}
\only<3>{
\item We define a binary operation $\left.  \succeq\right.$ on the
$\QQ$-module $\Powser$ as follows:

\begin{itemize}
\item On monomials, it should be given by
\[
\mathfrak{m}\left.  \succeq\right.  \mathfrak{n}=\left\{
\begin{array}
[c]{c}%
\mathfrak{m}\cdot\mathfrak{n},\ \ \ \ \ \ \ \ \ \ \text{if }\min\left(
\operatorname*{Supp}\mathfrak{m}\right) \geq \min\left(  \operatorname*{Supp}%
\mathfrak{n}\right)  ;\\
0,\ \ \ \ \ \ \ \ \ \ \text{if }\min\left(  \operatorname*{Supp}%
\mathfrak{m}\right) < \min\left(  \operatorname*{Supp}\mathfrak{n}\right)
\end{array}
\right.
\]
for any two monomials $\mathfrak{m}$ and $\mathfrak{n}$.
\item It should be $\QQ$-bilinear.
\item It should be continuous (i.e., its $\QQ$-bilinearity also
applies to infinite $\QQ$-linear combinations).
\end{itemize}

\item Well-definedness is pretty clear.

\item \textbf{Example.} $\left(x_2^2 x_4\right) \left.\succeq\right. \left(x_3^2 x_5\right) = 0$, but
$\left(x_2^2 x_4\right) \left.\succeq\right. \left(x_2^2 x_5\right) = x_2^4 x_4 x_5$.
}


\end{itemize}

\vspace{9cm}

\end{frame}


\begin{frame}
\fti{Dendriform structure on $\QSym$, part 2}

\begin{itemize}

\item We now have defined two binary operations $\left.\prec\right.$
and $\left.\succeq\right.$ on $\Powser$. They satisfy:
\begin{align*}
a\left.  \prec\right.  b+a \left.  \succeq\right.b  &  =ab;\\
\left(  a\left.  \prec\right.  b\right)  \left.  \prec\right.  c  &  =a\left.
\prec\right.  \left(  bc\right)  ;\\
\left(  a \left.  \succeq\right.  b\right)  \left.  \prec\right.  c
&  =a \left.  \succeq\right. \left(  b\left.  \prec\right.
c\right)  ;\\
a \left.  \succeq\right.  \left(  b \left.  \succeq\right.
c\right)   &  =\left(  ab\right)  \left.  \succeq\right.  c.
\end{align*}

\pause

\item This says that $\left(\Powser, \left.\prec\right., \left.\succeq\right.\right)$
is a \textit{dendriform algebra} in the sense of Loday
(see, e.g., {\red \href{http://arxiv.org/abs/1101.0267}{Zinbiel,
\textit{Encyclopedia of types of algebras 2010},
arXiv:1101.0267}}).

\pause

\item $\QSym$ is closed under both operations $\prec$ and $\succeq$.
Thus, $\QSym$ becomes a dendriform subalgebra of $\Powser$.

\end{itemize}

\vspace{9cm}

\end{frame}

\begin{frame}
\fti{The kernel criterion for left/right-shuffle-compatibility}

\begin{itemize}

\only<1>{
\item Recall the \textbf{Theorem:} The descent statistic $\st$ is
      shuffle-compatible if and only if $\calK_{\st}$ is an
      ideal of $\QSym$.
}
\pause
\only<2-5>{
\item Similarly, we have:
      \begin{itemize}
      \item \textbf{Theorem.} The descent statistic $\st$ is
      left-shuffle-compatible if and only if $\calK_{\st}$ is a
      $\prec$-ideal of $\QSym$ (that is: $\QSym \prec \calK_{\st}
      \subseteq \calK_{\st}$ and $\calK_{\st} \prec \QSym
      \subseteq \calK_{\st}$).
      \item \textbf{Theorem.} The descent statistic $\st$ is
      right-shuffle-compatible if and only if $\calK_{\st}$ is a
      $\succeq$-ideal of $\QSym$ (that is: $\QSym \succeq \calK_{\st}
      \subseteq \calK_{\st}$ and $\calK_{\st} \succeq \QSym
      \subseteq \calK_{\st}$).
      \end{itemize}

\pause

\item \textbf{Corollary.} Let $\st$ be a descent statistic.
      If $\st$ has $2$ of the $3$ properties
      ``shuffle-compatible'', ``left-shuffle-compatible'' and
      ``right-shuffle-compatible'', then it has all $3$.
      \\ (To prove this, recall $ab = a \prec b + a \succeq b$.)
}

\pause

\item \only<4>{
      \textbf{Question.} Are there non-shuffle-compatible but
      left-shuffle-compatible descent statistics?
      \\ (I don't know of any, but haven't looked far.)
      }
      \pause
      \only<5>{
      Okay, but how do we actually prove that $\calK_{\st}$
      is a $\prec$-ideal of $\QSym$ ?
      }
\end{itemize}

\vspace{10cm}

\end{frame}

\begin{frame}
\fti{The dendriform product formula for the $F_\alpha$}

\begin{itemize}

\item An analogue of the product formula for $F_{\Comp\pi} \cdot F_{\Comp\sigma}$:
      \textbf{Theorem.} Let $\pi$ and $\sigma$ be two disjoint nonempty
        permutations. Assume that%
        \[
        \text{the first entry of }\pi\text{ is greater than the first entry of }%
        \sigma\text{.}%
        \]
        Then,%
        \[
        F_{\operatorname*{Comp}\pi}\left.  \prec\right.  F_{\operatorname*{Comp}%
        \sigma}=\sum_{\tau\in S_{\prec}\left(  \pi,\sigma\right)  }%
        F_{\operatorname*{Comp}\tau}%
        \]
        and%
        \[
        F_{\operatorname*{Comp}\pi}\left.  \succeq\right.  F_{\operatorname*{Comp}%
        \sigma}=\sum_{\tau\in S_{\succ}\left(  \pi,\sigma\right)  }%
        F_{\operatorname*{Comp}\tau}.
        \]

\pause

\item This theorem yields that $\Des$ is left-shuffle-compatible
      and right-shuffle-compatible,
      just as the product formula showed that $\Des$ is
      shuffle-compatible.

\pause

\item Can we play the same game with $\Epk$, using our
      $K_{n,\Lambda}^{\mathcal{Z}}$ series instead of $F_\alpha$ ?
      \pause
      \\ Not to my knowledge: I don't know of an analogue
      of the above theorem. Instead, I use a different approach.

\end{itemize}
\vspace{10cm}
\end{frame}

\begin{frame}
\fti{The $\bel$ and $\tvi$ operations}

\begin{itemize}

\only<1-2>{
\item I need two other operations on quasisymmetric functions.
}

\only<1>{
\item We define a binary operation $\bel$ on the
$\QQ$-vector space $\Powser$ as follows:

\begin{itemize}
\item On monomials, it should be given by
\[
\mathfrak{m}\bel \mathfrak{n}=\left\{
\begin{array}
[c]{c}%
\mathfrak{m}\cdot\mathfrak{n},\ \ \ \ \ \ \ \ \ \ \text{if }\max\left(
\operatorname*{Supp}\mathfrak{m}\right)  \leq \min\left(  \operatorname*{Supp}%
\mathfrak{n}\right)  ;\\
0,\ \ \ \ \ \ \ \ \ \ \text{if }\max\left(  \operatorname*{Supp}%
\mathfrak{m}\right)  >\min\left(  \operatorname*{Supp}\mathfrak{n}\right)
\end{array}
\right.
\]
for any two monomials $\mathfrak{m}$ and $\mathfrak{n}$.
\item It should be $\QQ$-bilinear.
\item It should be continuous (i.e., its $\QQ$-bilinearity also
applies to infinite $\QQ$-linear combinations).
\end{itemize}

\item Well-definedness is pretty clear.

\item \textbf{Example.} $\left(x_2^2 x_4\right) \bel \left(x_4^2 x_5\right) = x_2^2 x_4^3 x_5$ and
$\left(x_2^2 x_4\right) \bel \left(x_3^2 x_5\right) = 0$.
}
\only<2>{
\item We define a binary operation $\tvi$ on the
$\QQ$-vector space $\Powser$ as follows:

\begin{itemize}
\item On monomials, it should be given by
\[
\mathfrak{m}\tvi  \mathfrak{n}=\left\{
\begin{array}
[c]{c}%
\mathfrak{m}\cdot\mathfrak{n},\ \ \ \ \ \ \ \ \ \ \text{if }\max\left(
\operatorname*{Supp}\mathfrak{m}\right) < \min\left(  \operatorname*{Supp}%
\mathfrak{n}\right)  ;\\
0,\ \ \ \ \ \ \ \ \ \ \text{if }\max\left(  \operatorname*{Supp}%
\mathfrak{m}\right) \geq \min\left(  \operatorname*{Supp}\mathfrak{n}\right)
\end{array}
\right.
\]
for any two monomials $\mathfrak{m}$ and $\mathfrak{n}$.
\item It should be $\QQ$-bilinear.
\item It should be continuous (i.e., its $\QQ$-bilinearity also
applies to infinite $\QQ$-linear combinations).
\end{itemize}

\item Well-definedness is pretty clear.

\item \textbf{Example.} $\left(x_2^2 x_4\right) \tvi \left(x_4^2 x_5\right) = 0$, but
$\left(x_2^2 x_4\right) \tvi \left(x_5^2 x_6\right) = x_2^2 x_4 x_5^2 x_6$.
}

\only<3->{
\item $\QSym$ is closed under both operations $\bel$ and $\tvi$.

\item Belgthor ($\bel$) and Tvimadur ($\tvi$) are two calendar runes
signifying two of the 19 years of the Metonic cycle.
I sought two (unused) symbols that (roughly) look like ``stacking one
thing (monomial) atop another'', allowing overlap ($\bel$) and
disallowing overlap ($\tvi$).
}

\end{itemize}

\vspace{9cm}

\end{frame}

\begin{frame}
\fti{A crucial identity}

\begin{itemize}

\item \textbf{Proposition.} For any $a \in \Powser$
and $b \in \QSym$, we have
\[
\sum_{\left(  b\right)  }\left(  S\left(  b_{\left(  1\right)  }\right)
\bel a\right)  b_{\left(  2\right)  }=a\left.  \prec\right.  b ,
\]
where we use the Hopf algebra structure on $\QSym$
\only<1>{ and the
following notations:
\begin{itemize}
\item $S$ for the antipode of $\QSym$;
\item Sweedler's notation $\sum\limits_{\left(
b\right)  }b_{\left(  1\right)  }\otimes b_{\left(  2\right)  }$ for
$\Delta\left(  b\right)  $.
\end{itemize}
}
\only<2->{.}

\pause

\item This proposition was important in my study of ``dual
      immaculate creation operators''; it is equally helpful
      here. \\
      \textbf{Corollary.} Let $M$ be an ideal of $\QSym$.
      If $\QSym \bel  M\subseteq M$, then $M\left.  \prec\right.  \QSym \subseteq M$.

\pause

\item A similar identity for $\tvi$ yields: \\
      \textbf{Corollary.} Let $M$ be an ideal of $\QSym$.
      If $\QSym \tvi  M\subseteq M$, then $\QSym \left.  \succeq\right.  M \subseteq M$.

\pause

\item \textbf{Corollary.} Let $M$ be an ideal of $\QSym$
      that is a left $\bel$-ideal (that is,
      $\QSym \bel  M\subseteq M$) and a left $\tvi$-ideal
      (that is, $\QSym \tvi  M\subseteq M$).
      Then, $M$ is a $\prec$-ideal and a $\succeq$-ideal
      of $\QSym$.
\end{itemize}

\vspace{9cm}

\end{frame}


\begin{frame}
\fti{``Runic calculus''}

\begin{itemize}

\item The operations $\bel$ and $\tvi$ are associative and unital (with unity $1$).
\pause

\only<2>{
\item For any two nonempty (i.e., $\neq \tup{}$) compositions $\alpha$ and $\beta$, we have
\begin{align*}
M_{\alpha}\bel  M_{\beta} &=M_{\left[  \alpha,\beta\right]  }+M_{\alpha
\odot\beta}; \\
M_{\alpha} \tvi  M_{\beta} &=M_{\left[  \alpha,\beta\right] } ; \\
F_{\alpha} \bel  F_{\beta} &=F_{\alpha\odot\beta} ; \\
F_{\alpha} \tvi  F_{\beta} &= F_{\left[  \alpha,\beta\right]  } ,
\end{align*}
where $\left[  \alpha,\beta\right]  $ and $\alpha\odot\beta$ are
two compositions defined by%
\begin{align*}
& \left[  \left(  \alpha_{1},\alpha_{2},\ldots,\alpha_{\ell}\right)  ,\left(
\beta_{1},\beta_{2},\ldots,\beta_{m}\right)  \right]  \\
 &  =\left(  \alpha
_{1},\alpha_{2},\ldots,\alpha_{\ell},\beta_{1},\beta_{2},\ldots,\beta
_{m}\right)
\end{align*}
and
\begin{align*}
&\left(  \alpha_{1},\alpha_{2},\ldots,\alpha_{\ell}\right)  \odot\left(
\beta_{1},\beta_{2},\ldots,\beta_{m}\right)   \\
&  =\left(  \alpha_{1}%
,\alpha_{2},\ldots,\alpha_{\ell-1},\alpha_{\ell}+\beta_{1},\beta_{2},\beta
_{3},\ldots,\beta_{m}\right)  .
\end{align*}
}

\only<3>{
\item They satisfy
\begin{align*}
\left(  a\left.
%TCIMACRO{\TeXButton{bel}{\textarm{\belgthor}}}%
%BeginExpansion
\textarm{\belgthor}%
%EndExpansion
\right.  b\right)  \left.
%TCIMACRO{\TeXButton{tvi}{\textarm{\tvimadur}}}%
%BeginExpansion
\textarm{\tvimadur}%
%EndExpansion
\right.  c-a\left.
%TCIMACRO{\TeXButton{bel}{\textarm{\belgthor}}}%
%BeginExpansion
\textarm{\belgthor}%
%EndExpansion
\right.  \left(  b\left.
%TCIMACRO{\TeXButton{tvi}{\textarm{\tvimadur}}}%
%BeginExpansion
\textarm{\tvimadur}%
%EndExpansion
\right.  c\right)   &  =\varepsilon\left(  b\right)  \left(  a\left.
%TCIMACRO{\TeXButton{tvi}{\textarm{\tvimadur}}}%
%BeginExpansion
\textarm{\tvimadur}%
%EndExpansion
\right.  c-a\left.
%TCIMACRO{\TeXButton{bel}{\textarm{\belgthor}}}%
%BeginExpansion
\textarm{\belgthor}%
%EndExpansion
\right.  c\right)  ;\\
\left(  a\left.
%TCIMACRO{\TeXButton{tvi}{\textarm{\tvimadur}}}%
%BeginExpansion
\textarm{\tvimadur}%
%EndExpansion
\right.  b\right)  \left.
%TCIMACRO{\TeXButton{bel}{\textarm{\belgthor}}}%
%BeginExpansion
\textarm{\belgthor}%
%EndExpansion
\right.  c-a\left.
%TCIMACRO{\TeXButton{tvi}{\textarm{\tvimadur}}}%
%BeginExpansion
\textarm{\tvimadur}%
%EndExpansion
\right.  \left(  b\left.
%TCIMACRO{\TeXButton{bel}{\textarm{\belgthor}}}%
%BeginExpansion
\textarm{\belgthor}%
%EndExpansion
\right.  c\right)   &  =\varepsilon\left(  b\right)  \left(  a\left.
%TCIMACRO{\TeXButton{bel}{\textarm{\belgthor}}}%
%BeginExpansion
\textarm{\belgthor}%
%EndExpansion
\right.  c-a\left.
%TCIMACRO{\TeXButton{tvi}{\textarm{\tvimadur}}}%
%BeginExpansion
\textarm{\tvimadur}%
%EndExpansion
\right.  c\right)  ,
\end{align*}
where $\varepsilon:\Powser\to\QQ$ sends $f$ to $f\left(0,0,0,\ldots\right)$.

\pause

\item As a consequence,
\[
\left(  a\left.
%TCIMACRO{\TeXButton{bel}{\textarm{\belgthor}}}%
%BeginExpansion
\textarm{\belgthor}%
%EndExpansion
\right.  b\right)  \left.
%TCIMACRO{\TeXButton{tvi}{\textarm{\tvimadur}}}%
%BeginExpansion
\textarm{\tvimadur}%
%EndExpansion
\right.  c+\left(  a\left.
%TCIMACRO{\TeXButton{tvi}{\textarm{\tvimadur}}}%
%BeginExpansion
\textarm{\tvimadur}%
%EndExpansion
\right.  b\right)  \left.
%TCIMACRO{\TeXButton{bel}{\textarm{\belgthor}}}%
%BeginExpansion
\textarm{\belgthor}%
%EndExpansion
\right.  c=a\left.
%TCIMACRO{\TeXButton{bel}{\textarm{\belgthor}}}%
%BeginExpansion
\textarm{\belgthor}%
%EndExpansion
\right.  \left(  b\left.
%TCIMACRO{\TeXButton{tvi}{\textarm{\tvimadur}}}%
%BeginExpansion
\textarm{\tvimadur}%
%EndExpansion
\right.  c\right)  +a\left.
%TCIMACRO{\TeXButton{tvi}{\textarm{\tvimadur}}}%
%BeginExpansion
\textarm{\tvimadur}%
%EndExpansion
\right.  \left(  b\left.
%TCIMACRO{\TeXButton{bel}{\textarm{\belgthor}}}%
%BeginExpansion
\textarm{\belgthor}%
%EndExpansion
\right.  c\right)  .
\]
This says that $\left(\QSym, \bel, \tvi\right)$ is a
$As^{\left\langle 2\right\rangle }$\textit{-algebra}
(in the sense of Loday).

\item \textbf{Question.} What other identities do
$\bel$, $\tvi$, $\prec$ and $\succeq$ satisfy?
}

\end{itemize}

\vspace{9cm}

\end{frame}


\begin{frame}
\fti{How to check left-/right-shuffle-compatible}

\begin{itemize}

\item Recall the \textbf{Corollary:} Let $M$ be an ideal of $\QSym$
      that is a left $\bel$-ideal (that is,
      $\QSym \bel  M\subseteq M$) and a left $\tvi$-ideal
      (that is, $\QSym \tvi  M\subseteq M$).
      Then, $M$ is a $\prec$-ideal and a $\succeq$-ideal
      of $\QSym$.

\pause

\item Given a shuffle-compatible
      descent statistic $\st$, we thus conclude
      that if $\calK_{\st}$ is a left $\bel$-ideal and a left
      $\tvi$-ideal, then $\st$ is left-shuffle-compatible
      and right-shuffle-compatible.

\pause

\item Fortunately, this is easy to apply: \\
      \textbf{Proposition.}
      Let $\operatorname*{st}$ be a descent statistic.
      
      \begin{itemize}
      \item $\calK_{\st}$ is a left $\bel$-ideal
of $\operatorname*{QSym}$ if and only if
\only<3>{
$\operatorname*{st}$ has the
following property: If $J$ and $K$ are two $\operatorname*{st}$-equivalent
nonempty compositions, and if $G$ is any nonempty composition, then $G\odot J$
and $G\odot K$ are $\operatorname*{st}$-equivalent.
}
\only<4>{ \\
$\st\tup{A \odot B}$ (for nonempty compositions $A$ and $B$)
is uniquely determined by $\abs{A}$, $\abs{B}$, $\st A$ and $\st B$.
}
      \item $\calK_{\st}$ is a left $\tvi$-ideal
of $\operatorname*{QSym}$ if and only if
\only<3>{
$\operatorname*{st}$ has the
following property: If $J$ and $K$ are two $\operatorname*{st}$-equivalent
nonempty compositions, and if $G$ is any nonempty composition, then $\left[
G,J\right]  $ and $\left[  G,K\right]  $ are $\operatorname*{st}$-equivalent.
}
\only<4>{ \\
$\st\tup{\ive{A, B}}$ (for nonempty compositions $A$ and $B$)
is uniquely determined by $\abs{A}$, $\abs{B}$, $\st A$ and $\st B$.
}
      \end{itemize}

\end{itemize}

\vspace{5cm}

\end{frame}

\begin{frame}
\fti{Why $\Epk$ is left- and right-shuffle-compatible}

\begin{itemize}

\item Thus, proving that $\Epk$ is left- and right-shuffle-compatible
      requires showing that $\Epk\tup{A \odot B}$ and $\Epk\tup{\ive{A, B}}$
      (for nonempty compositions $A$ and $B$) are uniquely determined
      by $\abs{A}$, $\abs{B}$, $\Epk A$ and $\Epk B$.

\pause

\item This is not hard:
\begin{align*}
\operatorname*{Epk}\left(  A\odot B\right)  &=\left(
\left(  \operatorname*{Epk}A\right)  \setminus\left\{  n\right\}  \right)
\cup\left(  \operatorname*{Epk}B+n\right)  ; \\
\operatorname*{Epk}\left(  \left[  A,B\right]  \right)
&=\left(  \operatorname*{Epk}A\right)  \cup\left(  \left(  \operatorname*{Epk}%
B+n\right)  \setminus\left\{  n+1\right\}  \right) ,
\end{align*}
where $n = \abs{A}$.

\pause

\item Similarly,
      \only<3>{
        \begin{itemize}
        \item $\Des$ is left- and right-shuffle-compatible (again);
        \item $\des$ is left- and right-shuffle-compatible;
        \item $\maj$ is \textbf{not} left- or right-shuffle-compatible
              ($\maj\tup{A \odot B}$ and $\maj\tup{\ive{A, B}}$ depend
              not just on $\abs{A}$, $\abs{B}$, $\maj A$ and $\maj B$,
              but also on $\des B$).
        \end{itemize}
      }
      \pause
      \only<4>{
        \begin{itemize}
        \item $\tup{\des, \maj}$ is left- and right-shuffle-compatible;
        \item $\Lpk$ is left- and right-shuffle-compatible;
        \item $\Rpk$ is \textbf{not} left- or right-shuffle-compatible;
        \item $\Pk$ is \textbf{not} left- or right-shuffle-compatible.
        \end{itemize}
      }

\item More statistics remain to be analyzed.

\end{itemize}

\vspace{10cm}

\end{frame}

\begin{frame}
\fti{Further questions}

\begin{itemize}

\item \textbf{Question (repeated).}
      Can a statistic be shuffle-compatible without being
      a descent statistic? \\
      (Would $\operatorname{FQSym}$ help in studying such
      statistics?)

\item \textbf{Question (repeated).}
      Can a statistic be left-shuffle-compatible without
      being shuffle-compatible?

\item \textbf{Question.}
      What mileage do we get out of $\calZ$-enriched
      $\tup{P,\gamma}$-partitions for other choices of
      $\calN$ and $\calZ$ ?

\item \textbf{Question (repeated).}
      Where do the $\Gamma_{\calZ}\tup{P,\gamma}$ live?

\item \textbf{Question.}
      Hsiao and Petersen have generalized enriched
      $\tup{P, \gamma}$-partitions to ``colored
      $\tup{P, \gamma}$-partitions'' (with $\set{+,-}$
      replaced by an $m$-element set).
      Does this generalize our results?

\end{itemize}

\end{frame}



% \begin{frame}
% \fti{}

% \begin{itemize}

% \item 
      

% \item 
      

% \item 
      

% \item 
      

% \item 
      

% \end{itemize}

% \end{frame}


\begin{frame}
\fti{Thanks}

\textbf{Thanks} to Ira Gessel and Yan Zhuang for initiating this direction
(and for helpful discussions), and to
Sara Billey for an invitation to Seattle. \\
And thanks to you for attending!

\vspace{3cm}

\textbf{slides: \red \url{http://www.cip.ifi.lmu.de/~grinberg/algebra/seattle18.pdf}} \\
\textbf{paper: \red \url{http://www.cip.ifi.lmu.de/~grinberg/algebra/gzshuf2.pdf}} \\
\textbf{project: \red \url{https://github.com/darijgr/gzshuf}}

\end{frame}

\end{document}
